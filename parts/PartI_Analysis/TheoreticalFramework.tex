All microscopic phenomena observed to date can be described in the framework of the Standard Model (SM). The SM is a renormalisable quantum field theory of the strong, electromagnetic and weak interactions. According to the SM, matter particles are represented by spin-$\frac{1}{2}$ Dirac fermions coming in three generations while interaction between matter fields is mediated by spin-1 gauge bosons. The SM is based on the combined local $\mathrm{SU}\left(3\right)_{C} \bigotimes \mathrm{SU}\left(2\right)_{L} \bigotimes \mathrm{U}\left(1\right)$ symmetry representing strong, weak and electromagnetic sectors, respectively. The fermions are distinguished by the quantum numbers corresponding to the gauge groups. Quark fields are endowed with electroweak and colour charges, while leptons (electron $e$, muon $\mu$, tau $\tau$ and three corresponding neutrinos $\nu_e, \nu_\mu, \nu_\tau$) carry only electroweak charge. As a consequence of exact local gauge symmetry, the mediators of the strong force, gluons, are massless quanta. In contrast to the strong interaction, three electroweak gauge bosons $W^\pm, Z^0$ acquire mass as a result of spontaneous $\mathrm{SU}_{L}\left(2\right) \bigotimes \mathrm{U}\left(1\right)$ symmetry breaking, while the photon $\gamma$ stays massless. According to the Higgs mechanism, elementary particles acquire mass due to coupling to the scalar field, quanta of which have been recently discovered at the LHC~\cite{pl:b716:1,pl:b716:30}.

In this work, interpretation of the data is based on the Standard Model picture of particle physics. Subsequent sections briefly overview the theoretical framework used for the description of the strong sector of the Standard Model and hard interactions at \hera in particular.

\section{Kinematics of DIS}
\label{sec:kindis}
As described in Section~\ref{sec:kindis}, deep inelastic scattering at \hera can be characterised by two independent variables, for example, $\qsq$ and $\y$. The values of the kinematic variables can be determined from the components of the four-momenta of the scattered electron or the hadronic system or from the combination of the two. Below the different approaches for the measurement of the DIS kinematics are described.
\subsection{Electron Method}
\label{subsec:em}
The electron method utilises the energy, \eefin, and the polar angle, \thetae, of the scattered electron only\footnote{It is assumed that the electron does not undergo initial- and/or final-state radiation that effectively reduces the electron energy.}. The kinematic quantities characterising an event are given by the following expressions (see Section~\ref{sec:kindis}):
\begin{align}
	Q^2_\text{el} &= 2 E_e E_e' \left( 1 + \cos \theta_e \right),			\label{eq:q2el}							\\
	y_\text{el}      &= 1 - \frac{E_e'}{2 E_e}\left( 1 - \cos \theta_e \right),			\label{eq:yel} \\
	x_\text{el}      &= \frac{Q^2_\text{el}}{s y_\text{el}},			\label{eq:xel}
\end{align}
where, $\eeini$, is the energy of the initial-state electron. It has been shown~\cite{nim:a426:583} that this method accurately reconstructs the event kinematic variables at high values of $\y$, while at low values of inelasticity it has poor resolution.
\subsection{Jacquet-Blondel Method}
\label{subsec:jb}
Transverse momentum conservation and the almost complete hermiticity of the \zeus calorimeter enables the reconstruction of the event kinematics based exclusively on the energy deposits attributed to the hadronic system. Jacquet and Blondel proposed a method~\cite{proc:epfacility:1979:391} for the kinematic reconstruction for CC DIS events in which the final-state neutrino escapes the detector and cannot be measured. The kinematic variables are obtained from the following expressions:
\begin{align}
	y_\text{JB}      &= \frac{\sum{ \left( E - P_{Z} \right) }}{2E_e},			\label{eq:ybj} \\
	Q^2_\text{JB} &= \frac{ \left( \sum{P_{X}} \right)^2 + \left( \sum{P_{Y}} \right)^2 }{1-y_{JB}},			\label{eq:q2jb}							\\
	x_\text{JB}      &= \frac{Q^2_\text{JB}}{s y_\text{JB}}.			\label{eq:xjb}
\end{align}
The sum in these expressions runs over the reconstructed final-state objects (see above) excluding those belonging to the scattered electron. It is assumed that the target remnants escaping the detector volume have small transverse momentum and are therefore suppressed in the expressions. The accuracy of this method is limited by the hadron-calorimeter energy resolution, the presence of dead material and backsplash/backscattering in the calorimeter. Since this method does not require the detection of the scattered electron, it is the only choice for photoproduction events\footnote{In photoproduction events the electron is scattered at small angles and escapes detection in the beam pipe.}.
\subsection{Double-Angle Method}
\label{subsec:da}
The Double-angle method~\cite{proc:hera:1991:23} benefits from exploiting the combined information on the scattered electron and the hadronic final state. The kinematic variables can be expressed in terms of the electron scattering angle, \thetae, and \gamha, which in the quark-parton model corresponds to the scattering angle of the struck quark and can be obtained from the following equation
\begin{equation}
\cos \gamma_{\text{had}} = \frac{ \left( \sum{P_{X}} \right)^2 + \left( \sum{P_{Y}} \right)^2 - (\sum{E - P_{Z}})^2 }{\left( \sum{P_{X}} \right)^2 + \left( \sum{P_{Y}} \right)^2 + (\sum{E - P_{Z}})^2},
\label{eq:cosgam}
\end{equation}
where the sum runs over the energy deposits attributed to the hadronic final state.

The kinematic variables are obtained as follows:
\begin{align}
	y_\text{DA}      &= \frac{ \sin \theta_e \left( 1 - \cos \gamma_\text{had} \right)}{ \sin \gamma_\text{had} + \sin \theta_e - \sin \left( \gamma_\text{had} + \theta_e \right) },			\label{eq:yda} \\
	Q^2_\text{DA} &= 4E_e^2\frac{ \sin \gamma_\text{had} \left( 1 + \cos \theta_e \right) }{ \sin \gamma_\text{had} + \sin \theta_e - \sin \left( \gamma_\text{had} + \theta_e \right) },			\label{eq:q2da}							\\
	x_\text{DA}      &= \frac{E_e}{E_p} \frac{\sin \gamma_\text{had} + \sin \theta_e + \sin \left( \gamma_\text{had} + \theta_e \right)}{\sin \gamma_\text{had} + \sin \theta_e - \sin \left( \gamma_\text{had} + \theta_e \right)},			\label{eq:xda}
\end{align}
where \epini, is the proton initial energy.

Using these relations an expression for the energy of the scattered electron can be derived as
\begin{equation}
E_\text{el}^\text{DA} = \frac{2E_e \sin \gamma_\text{had}}{\sin \gamma_\text{had} - \sin \theta_e - \sin \left( \gamma_\text{had} + \theta_e \right)}.
\label{eq:eeda}
\end{equation}

 The measurements of the scattering angles usually have better resolution and are approximately independent of the absolute energy-scale of the calorimeter. It has been shown in~\cite{thesis:behr:2010} that this method is optimal in the phase space of this measurement and therefore this was used as a default in this analysis.

%This formulae will be used later (see Section~\ref{sec:kinrec}) for the determination of the uncertainty due the absolute electromagnetic energy scale of the calorimeter.

\section{Quantum Chromodynamics}
All microscopic phenomena observed up to date can be described in the framework of the Standard Model (SM). The SM is a renormalisable quantum field theory of the strong, electromagnetic and weak interactions. According to the SM the matter particles are represented by spin-$\frac{1}{2}$ Dirac fermions coming in three generations while interaction between matter fields is mediated by spin-1 gauge bosons. The SM is based on combined local $\mathrm{SU}\left(3\right) \bigotimes \mathrm{SU}\left(2\right) \bigotimes \mathrm{U}\left(1\right)$ symmetry representing strong, weak and electromagnetic sectors, respectively. The fermions are distinguished by the quantum numbers corresponding to the gauge groups. Quark fields are endowed with electroweak and colour charges, while leptons (electron $e$, muon $\mu$, tau $\tau$ and three corresponding neutrinos $\nu_e, \nu_\mu, \nu_\tau$) carry only electroweak charge. As a consequence of exact local gauge symmetry the mediators of the strong force, gluons, are massless quanta. In contrast to the strong interaction three electroweak gauge bosons $W^\pm, Z^0$ acquire mass as a result of spontaneous $\mathrm{SU}\left(2\right) \bigotimes \mathrm{U}\left(1\right)$ symmetry breaking, while photon $\gamma$ stays massless. According to the Higgs mechanism elementary particles acquire mass due to coupling to the scalar field, quanta of which have been recently discovered at the LHC~\cite{higgs atlas cms}.

Subsequent sections briefly outline the theoretical framework used for the description of the strong sector of the Standard Model and hard interactions at HERA in particular.
\section{Quantum Chromodynamics}
Quantum Chromodynamcs emerged as a model for description of hadron spectra and absence of observations of free hadron constituents. The QCD Lagrangian density can by derived from local $\mathrm{SU}\left(3\right)$ invariance and reads as follows
\begin{align}
 \mathcal{L}\left( x\right)_{\mathrm{QCD}}  = & -\dfrac{1}{4}G_a^{\mu\nu}G_{\mu\nu}^a + i\sum_{j=1}^n{\overline{\psi}^{\alpha}_j \gamma_\mu \left( D^\mu \right)_{\alpha\beta} \psi_j^\beta } - \sum_{j=1}^n{m_j\overline{\psi}^{\alpha}_j\psi_{j,\alpha}} \\
					      & - \dfrac{1}{2\alpha_G} \partial^\mu \mathcal{A}^a_\mu \partial_\mu \mathcal{A}_a^\mu - \partial_\mu \overline{\varphi}_a D^\mu \varphi^a,
\end{align}
where $G_{\mu\nu}^a \equiv \partial_\mu \mathcal{A}^a_\nu - \partial_\nu \mathcal{A}^a_\mu + gf_{abc}\mathcal{A}^b_\mu \mathcal{A}^c_\nu $, $a=1 \ldots 8$ are the Yang-Mills field strenght tensor~\cite{C.N. Yang and R.L. Mills, Phys. Rev. 96 (1954) 191.} constructed from gluon fields $\mathcal{A}^a_\mu$ in adjoint representation of $\mathrm{SU}\left(3\right)$. Quarks of different flavours are described by $\psi_j$ fields in fundamental representation of $\mathrm{SU}\left(3\right)$ while $\varphi^a$ are eight anti-commuting scalar Faddeev-Popov ghost fields required in quantisation procedure~\cite{L.D. Faddeev and Y.N Popov, Phys. Lett. B 25 (1967) 29;
B. De Wit, Phys. Rev. Lett. 12 (1964) 742.}. Covariant derivative, $\left( D^\mu \right)_{\alpha\beta} = \delta_{\alpha\beta}\partial_\mu - ig\sum_a{\dfrac{1}{2}\lambda^a_{\alpha\beta}\mathcal{A}^a_\mu}$, defines infinitesimal transformation in colour space acting on quark fields. Gell-Mann $3\times 3$ matrices, $\lambda^a_{\alpha\beta}$, are the generators of $su\left(3\right)$ algebra and $f_{abc}$ are its real structure constants related by the following expression:
\begin{align}
 \left[T_a,T_b\right] = if_{abc}T_c,\qquad T_a = \dfrac{1}{2}\lambda_a.
\end{align}
An important feature that can be readily observed is a non-linear term, $gf_{abc}\mathcal{A}^b_\mu \mathcal{A}^c_\nu$, in the definition of the field strength tensor. This term is the result of non-abelian structure of the symmetry group and determines the self-interaction of the force carriers. In contrast to electromagnetic interaction gluons carry two color charges. 

Given the Lagrangian density the Feynman rules for QCD can be derived.  Feynman diagrams representing interaction vertices of the fundamental fields are depicted in Figure~\ref{fig:FundamentalQCDInteractions}.
\begin{figure}[t]
	\centering
		\includegraphics[width=\textwidth]{./Figures/source/FundamentalQCDInteractions.png}
	\caption{The interaction verices of the Feynman rules of QCD and schematic colour flow interpretation for quark-gluon, three-gluon and four-gluon vertices.}
	\label{fig:FundamentalQCDInteractions}
\end{figure}

Such relatively simple model is extremely successful in description of vast variety of experimental data collected up to now. In particular, strong interaction has two distinct features: at large energy scales\footnote{In the limit $\Lambda_\mathrm{QCD}/E\rightarrow 0$, where $\Lambda^{n_f=5}_{\overline{\mathrm{MS}}}=226$ MeV is QCD characteristic energy scale.} hadron constituents behave as free particles and strength of the coupling decreases (``asymptotic freedom''); conversely, at low energy scales strength of the coupling grows, bounding quarks and gluons inside hadrons (``confinement'').  
 
At current stage direct solution of Yang-Mills equations is an incapable task. Almost all quantitative QCD predictions are based on three first-principle approaches: perturbative QCD (pQCD), lattice QCD and effective theories. Perturbative approach exploits smallness of the strong coupling constant in high-energy regime and develops successive approximations to the solution. Next section summarises basic information about pQCD approach.



\subsection{Structure of Perturbative Calculations}
\label{subsec:pqcdstructure}
In perturbative QCD the predictions for a physical observable, $f$, is calculated order-by-order as a power series in small coupling $\alpha_s \equiv \dfrac{g_s^2}{4\pi} \ll 1$
\begin{equation}
   f=f_{0}\alpha^{k}_s + f_{1}\alpha^{k+1}_s + f_{2}\alpha^{k+2}_s + \ldots
	 \label{eq:pertseries}
\end{equation}
The perturbation series starts at some power of the expansion parameter and only few terms of the series are usually calculated. The expansion coefficients $f_i$ in the above series are usually calculated by summing up Feynman diagrams (or similar techniques). Number of diagrams grows as $\sim i!$ with increasing perturbative order, therefore such series is, in general, divergent and has to be treated as an asymptotic expansion~\cite{lipatovQCD}. However, it is commonly assumed that few first terms in the series provide a reasonable approximation to exact solution. 

?????????????????????

Calculation of higher order corrections requires consideration of more complex fields configurations which emerge due to quantum fluctuations. Such configurations can be formally divided in two classes according to the topology of the corresponding Feyman diagrams. An example of next-to-leading order diagrams contributing to the jet production in deep inelastic scattering as well as in direct photoproduction processes are demonstrated in Figure~\ref{fig:nlojetfeyn}. The real corrections (Figure~\ref{fig:nlojetfeyn}a) are characterised by increased number of legs with respect to lower order graph, while in virtual contributions (Figure~\ref{fig:nlojetfeyn}b) the fields form closed loops.
 
\begin{figure}[h]
 \includegraphics[width=\textwidth]{./Figures/DirectNLORealVirtualCorrections}
\caption{Next-to-leading order corrections to the jet production include (a) virtual (b) real contributions.}
\label{fig:nlojetfeyn}
\end{figure}

Intuitively, with decreasing the spatial scale at which the process is considered, one has to take into account more fluctuations of quantum fields. As a result, the field couplings or masses has to be interpreted as effective parameters of the theory, that take into account charge screening, self interaction etc. Nevertheless, it is unnatural that the fluctuations occurring at scales much smaller than that corresponding to typical energy scale of the process in question, should have a significant influence. Remarkably, QCD admits redefinition of the coupling, fields and masses that incorporates contributions from fields fluctuations occurring in the limit of infinite energy. Such a property of QCD is called renormalisability and is a specific feature of the gauge theories. 



\subsection{Renormalisation and Renormalisation Group Equation}
When the problem is treated perturbatively, the class of quantum fluctuations contributing to the process is naturally restricted. As mentioned above, beyond the tree level this results in the dependence of the calculations on the parameter $\mu$, which approximately represents the spatial scale beyond which the effect of quantum fluctuations are absorbed into the dependence of the theory parameters on the scale $\mu$. The value of the parameter can be chosen arbitrarily, but the physical quantity calculated in perturbation theory, e.g. the jet production cross section, cannot depend on arbitrary parameter. This requirement can be formulated as follows\footnote{The condition must hold up to terms proportional to $\as^{k+1}$ if the expansion of an observable $\mathcal{S}$ is known to $\mathcal{O}\left(\as^{k}\right)$}:
\begin{equation}
 \frac{d}{d\log{\mu^2}} \mathcal{S} \left( Q^2/\mu^2, \alpha_s \right) = \frac{\partial \mathcal{S} }{\partial \log{\mu^2}} + \frac{\partial \alpha_s }{\partial \log{\mu^2}}\frac{\partial \mathcal{S} }{\partial \alpha_s} \stackrel{!}{=} \mathcal{O}\left(\as^{k+1}\right),
\end{equation}
where for simplicity $\mathcal{S}$ is chosen to be a dimensionless observable. An explicit $\mu$ dependence of $\mathcal{S} \left( Q^2/\mu^2, \alpha_s \right)$ has to be compensated by that of the coupling. An equation for the scale dependence of the strong coupling can be derived (see~\cite{QCDrge:2014} and references therein):
 \begin{equation}
   \frac{d\as\left(\mu\right)}{d\ln{\mu^2}} = \beta\left(\as\left(\mu\right)\right),\qquad \beta\left(\as\left(\mu\right)\right) = -\as^2\left(\beta_0 + \beta_1\as + \beta_2\as^2 + \dots \right).
 \label{eq:asrunnig}
 \end{equation}
This equation is called the renormalisation group equation (RGE). The few first terms in the $\beta$-function were calculated in perturbation theory to be
\begin{align}
	\beta_0 &= \dfrac{11C_A-2n_f}{12\pi} = \dfrac{33 - 2n_f}{12\pi^2},\\
	\beta_1 &= \dfrac{17C_A^2-n_fT_R\left(C_A+6C_F\right)}{24\pi^2} = \frac{153-19n_f}{24\pi^2},\\
	\beta_2 &= \dfrac{2857-\dfrac{5033}{9}n_f+\dfrac{325}{27}n_f^2}{128\pi^3},
\end{align}
where $C_A$, $C_F$ are $SU\left(3\right)$ structure coefficients, while $T_R=\dfrac{1}{2}$ and $n_f$ is the number of active flavours\footnote{In general, the $\beta$-function coefficients, $\beta_i$ depend on the employed renormalisation scheme. Only $\beta_0$ and $\beta_1$ are scheme independent. The $\beta_2$ term specified here refers to the widely used $\overline{\mathrm{MS}}$ renormalisation scheme.}$^,$
\footnote{It is assumed that heavy quark flavours decouple from the theory below energy scales much smaller than the heavy quark mass $\mu \ll m_h$.}. 

The equation~\ref{eq:asrunnig} can be solved analytically. Taking into account only the first term involving $\beta_0$, the solution is:
\begin{equation}
 \as\left(\mu^2\right) = \dfrac{ \as\left(\mu_0^2\right) }{1+\beta_0\as\left(\mu_0^2\right)\ln{\mu^2/\mu_0^2}} = \dfrac{1}{\beta_0\ln{\mu/\Lambda^2}}
\label{eq:asrunnigsolution}
\end{equation}
The initial condition for the solution is specified by the value of the coupling at the starting scale $\as\left(\mu_0\right)$ or alternatively the integration constant $\Lambda$. The positivity of $\beta_0$ in the SM results in the coupling constant vanishing when the energy scale $\mu$ increases or correspondingly shorter time scales are considered. Quarks and gluons behave as non-interacting free particles in the high-energy limit. On the other hand, in processes characterised by long time intervals or equivalently, small momenta, the coupling grows. Eventually, the coupling becomes undefined near the pole $\mu = \Lambda$. In this region, the theory becomes essentially nonperturbative and series expansion is no longer valid. 



\subsection{Factorisation}
\label{subsec:factorisation}
Another remarkable feature of QCD is factorisation of short- and long-time-scale processes. For example, the cross section for jet production in DIS can be represented in factorised form~\cite{Collins:1989gx} as:
\begin{equation}
 \mathrm{d}\sigma_{\mathrm{jet}} = \int{f\left(x,\mu_f\right)}\, \mathrm{d}\sigma_{\mathrm{part}}\left(x,\mu_r, \mu_f, \alpha_s\left(\mu_r\right) \right).
\label{eq:disfactorisation}
\end{equation} 
In this expression $f\left(x,\mu_f\right)$ represents the nonperturbative proton parton distribution function and $\mathrm{d}\sigma_{\mathrm{part}}$ is the hard-scattering partonic cross section that is calculated in perturbation theory. A schematic illustration representing this equation is depicted in Figure~\ref{fig:Factorisation}. Technically, in the calculation an additional factorisation scale, $\mu_F$ is introduced. The parameter $\mu_F$ defines approximately the virtuality of the intermediate (virtual) states that contribute to the hard scattering while the long-distance physics is absorbed in universal nonperturbative parameters\footnote{For DIS, the processes with energy scale $Q^2 \gg \mu_F^2 $ are attributed to the perturbative part, while those with $Q^2 \ll \mu_F^2 $ are absorbed into PDFs.}. Factorisation leads to the calculations being usually performed in two steps. The perturbative part can be evaluated as a series expansion in the strong coupling constant, described above, while parton distributions have to be determined experimentally. In this procedure, singularities attributed to the long-distance processes e.g. soft or collinear radiation of partons, are absorbed into nonperturbative terms. The factorisation scale, $\mu_f$, serves as a reference point at which the subtraction of the singularities is performed. The subtraction scheme defines the prescription for reshuffling of finite terms between partonic cross section and PDFs. The employed factorisation scheme must be consistent with that used for renormalisation. In this analysis the modified variant of the minimal subtraction, $\overline{\mathrm{MS}}$, scheme~\cite{Bardeen:1978yd} was used.
\begin{figure}[t]
	\centering
	\begin{subfloat}[]{
		\includegraphics[width=0.45\linewidth]{./Figures/source/Factorisation.png}
		\label{fig:Factorisation}
	 }%
	\end{subfloat}
	\begin{subfloat}[]{
		\includegraphics[width=0.45\linewidth]{./Figures/source/DGLAPLadder.png}
		\label{fig:DGLAPLadder}
	}%
	\end{subfloat}
	\caption{(a) Schematic illustration of the factorisation of the hard \ep-process into non-perturbative proton PDFs and hard-scattering partonic cross sections. (b) An example diagram with strongly ordered parton emission contributing to the NC DIS process.}
	\label{fig:factorisationdglapladder}
\end{figure}
%\begin{figure}[t]
	%\centering
		%\includegraphics[width=0.5\textwidth]{./Figures/source/Factorisation.png}
		%\includegraphics[width=0.5\textwidth]{./Figures/source/DGLAPLadder.png}
	%\caption{Factorisation}
	%\label{fig:Factorisation}
%\end{figure}


\subsubsection{DGLAP Equations}
\label{subsubsec:dglapeq}
The factorisation scale dependence of the PDFs is governed by the DGLAP evolution equation~\ref{DGLAP}:
\begin{equation}
\frac{\mathrm{d}}{\mathrm{d}\log{\mu_F^2}}
 \begin{pmatrix}
	\mathbf{f_{q_i}}\left(x,\mu_F\right) \\
	f_g\left(x,\mu_F\right)
 \end{pmatrix} = 
\sum_j^{2n_f}{\int_x^1{\frac{\mathrm{d}z}{z}
 \begin{pmatrix}
  \mathbf{P_{q_i \leftarrow q_j}}\left(z\right) & \mathbf{P_{q_i \leftarrow  g}}\left(z\right) \\
  P_{g \leftarrow q_j}\left(z\right) & P_{g \leftarrow g}\left(z\right) \\
 \end{pmatrix}
 \begin{pmatrix}
	\mathbf{f_{q_j}}\left(x/z,\mu_F\right) \\
	f_g\left(x/z,\mu_F\right)
 \end{pmatrix}
}}
\end{equation}
In this equation the summation runs over the number of active quark and antiquark flavours. The kernels of these equations are splitting functions $P_{ab}$ representing the probability of splitting of one parton into several partons and can be calculated from the collinear singularity of any hard scattering process as a power series in $\as$:
\begin{equation}
P\left(z,\as\left(\mu_f\right)\right) = \as\left(\mu_f\right) P_0\left(z\right) + \as^2\left(\mu_f\right)P_1\left(z\right) + \as^3\left(\mu_f\right)P_2\left(z\right) + \ldots
\label{eq:splittingfunc}
\end{equation}
At the moment the splitting functions are known to next-to-next-to-leading order~\cite{nnlosplittingfinctions}. Intuitively, this system of equations states how sensitive to the low momentum partons becomes the probe as the resolution scale $\mu_f$ increases. %The universality of the splitting kernels allows determination of the PDFs from the global fits to experimental data. 

Evolution of the scale dependent parton distributions according to the DGLAP equations effectively resums the Feynman diagrams with parton emission strongly ordered in transverse momentum $ \mu_{F,0} \ll \ldots \ll k_{T,i} \ll k_{T,i+1} \ll \ldots \ll \mu_F$ as in Figure~\ref{fig:DGLAPLadder}. Each parton emission in such approximation is accompanied by a term $\as\cdot\ln{\mu^2_F/\mu^2_{F,0}}$ in the matrix element, therefore such resummation is also called ``leading log approximation''.



\subsection{Scale Choice}
The size of the unknown higher-order terms in the perturbative series Eq.~\eqref{eq:pertseries} is usually one of the dominant sources of uncertainty in theoretical predictions. These contributions can be estimated from the dependence of the perturbative expansion on the renormalisation and factorisation scales. Using the renormalisation group equation it can be demonstrated that the scale dependence of higher-order coefficients accompanied by logarithmic terms are fully determined by lower-order coefficients~\cite{Behnke:2013pga}. For example, in the case of a dimensionless observable with perturbative expansion of the form Eq.~\eqref{eq:pertseries}, the following expressions can be obtained:
\begin{align}
	f_1\left(\frac{\mu}{Q}\right) &= f_1\left(1\right) - k\beta_0f_0\log{\frac{\mu}{Q}}, \\
	f_2\left(\frac{\mu}{Q}\right) &= f_2\left(1\right) - \left[\left(k+1\right)\beta_0f_1\left(1\right) + k\beta_1f_0\right]\log{\frac{\mu}{Q}}, \\
																						&+ \frac{k\left(k+1\right)}{2}\beta_0^2f_0\log^2{\frac{\mu}{Q}},\\
																						& \ldots 
\end{align}
Thus, the $\mu$ variation in the $\mathcal{O}\left(\as^n\right)$ expression corresponds to the higher-order terms of the form:
\begin{equation}
 \as^{n+1}\left(\mu\right) \sum_{i=1}^{n+1-k}{\text{(know part)}\cdot\log^i{\frac{\mu}{Q}}} + \mathcal{O}\left(\as^{n+1}\right).
\label{eq:scalevariationerror}
\end{equation}
However, terms that are not accompanied by the logarithms e.g. $f_2\left(1\right)$, require explicit calculation. Therefore the reliability of an estimate of the size of the truncated terms depends on whether $f_i\left(1\right), i \ge 1$ are of similar order as $f_0$. Notably,
the leading-order coefficient $f_0$ is independent of the renormalisation scale, therefore the scale dependence of the LO approximation is completely governed by the scale dependence of the strong coupling. Therefore a realistic estimate of the size of unknown terms is possible starting at least at NLO.

Besides that, the sensitivity of perturbative predictions to the $\mu_f$ scale variation has to be taken into account. Although formally the DGLAP equations perform all-orders resummation of the ladder diagrams, the residual dependence on the factorisation scale of order $\mathcal{O}\left(\as^{k+1}\right)$ persists in the pQCD calculations. Similarly to the renormalisation-scale dependence, the dependence of the perturbative coefficients on $\mu_f$ can be recovered:
\begin{equation}
 f_1\left(\frac{\mu_f}{Q},\frac{\mu_r}{Q}\right) = f_1\left(1,\frac{\mu_r}{Q}\right) + P_0 \otimes f_0 \log{\frac{Q^2}{\mu_f^2}},
\label{eq:factorisationscaledep}
\end{equation}
where $P_0$ is the LO splitting function and the convolution symbol denotes
\begin{equation}
P_0 \otimes f_0 = \int{\frac{\mathrm{d}z}{z}\,P_0\left(x/z\right)f_0\left(z\right)}.
\end{equation}
For the higher-order coefficients, a pattern similar to the renormalisation case is obtained.

The DGLAP equations involve the strong coupling evaluated at the factorisation scale $\mu_f$. Whenever this scale differs from $\mu_r$, the RGE evolution from factorisation to renormalisation point has to be performed. Since the RGE is determined by incomplete series for the $\beta$-function, the strong coupling evolution can be unreliable if calculation over a wide interval of scales is required. Mathematically this can be formulated as follows. Considering a Taylor expansion of $\as\left(\mu_f\right)$ around $\as\left(\mu_r\right)$ and substituting the RGE expression for the derivatives of the strong coupling $\mathrm{d}\as/\mathrm{d}\log{\frac{\mu_f^2}{\mu_r^2}}$ gives~\cite{Botje:2010ay}:
{\small
\begin{align}
 \as\left(\mu_f\right) &= \as\left(\mu_r\right) - \beta_0\log{\frac{\mu_f^2}{\mu_r^2}}\as^2\left(\mu_r\right) - \left(\beta_1\log{\frac{\mu_f^2}{\mu_r^2}} - \beta_2^2\log^2{\frac{\mu_f^2}{\mu_r^2}}\right)\as^3\left(\mu_r\right) + \mathcal{O}\left(\as^4\right)\\ \nonumber
 \as^2\left(\mu_f\right) & = \as^2\left(\mu_r\right) - 2\beta_0\log{\frac{\mu_f^2}{\mu_r^2}}\as^3\left(\mu_r\right) + \mathcal{O}\left(\as^4\right)\\
\as^3\left(\mu_f\right) & = \as^3\left(\mu_r\right) + \mathcal{O}\left(\as^4\right).
\end{align}
}
Thus, to ensure convergence of such an expansion, the factorisation scale must be closely related to the renormalisation scale. In addition, as in the case of renormalisation, the $\mu_f$ scale must be much larger than $\Lambda_\mathrm{QCD}$ to justify the applicability of perturbation theory results for the PDF evolution.

There is no general method to estimate the size of the contribution from missing terms in perturbative series. However, it is widely assumed that the corresponding uncertainty can be estimated from the variation of the renormalisation and factorisation scales up and down by a factor of two. The resulting variation of the observable depends on the central values $\mu_{f}^{0},\,\mu_{r}^{0}$ around which the variation is performed. It is desirable to choose central values such that the difference between the nominal result and the one with scaled values of $\mu_f$ and $\mu_r$ is minimised i.e. $\partial\mathcal{S}/\partial\mu=0$. This method is called the ``principle of minimum sensitivity'' (PMS)~\cite{Stevenson:1980du}. However, straightforward application of this method can result in a very large inclusive-jet cross section at the lowest \qsq~and \etjet~values~\cite{thesis:britzger:2013}. Alternative prescriptions for the scale choice can be found in~\cite{Ioffe:2010zz}. The proposed methods emphasise different aspects of the perturbative expansion. However, it should be noted that all are related to the behaviour of logarithmically enhanced terms.

In this analysis, the traditional prescription of choosing the scale corresponding to the typical energy scale of the process was adopted. More details are provided in Chapter~\ref{ch:resultscs}.


\section{Jet Definition}
\label{sec:jetalgo}
%Jets are collimated ``spays'' of particles emerging from the hard interaction of partons. Large kinematic boost of the scattered partons results in focusing of QCD radiation in a narrow cone around the parton momentum vector.In order to determine the momentum of the parent-parton an appropriate assignment of particles to a jet has to be performed. 
As was mentioned, bare partons do not appear as free particles because of the nature of the strong interaction. However high-energy quarks and gluons manifest themselves as collections of hadrons with approximately collinear momenta. Such hadronic final states localised in the kinematic phase space are called jets. Investigation of jet production provides access to the details of the underlying hard interaction as well as to the parton dynamics and the mechanism of parton showering and hadronisation. Provided the kinematics of the final-state jets, important quantities describing the kinematics of the hard scattering can be estimated. For example the longitudinal momentum fraction of the struck parton, $\xi$, can be calculated using:
\begin{equation}
\xi = x\left(1+\frac{M^2}{\qsq}\right),
\label{eq:xidef}
\end{equation}
where $x$ is the Bjorken scaling variable defined in the Eq.~\eqref{eq:vardefinitxbj} and $M$ is the invariant mass of two or more identified jets. Jets are important objects allowing to test the predictions of perturbative QCD (see Section~\ref{subsec:nlojetcalc}).

In the leading-order picture, jets correspond to individual partons emerging in high-energy collisions. An example of the basic diagrams contributing to the jet production in DIS is demonstrated in Figure~\ref{fig:LOFeynmandiags}. Since the flavour of the struck parton cannot be distinguished in NC DIS reactions, formally two types of processes contributing at leading order in the strong coupling can be distinguished, namely, the boson-gluon fusion (BGF) Figure~\ref{fig:LOFeynmandiags}\subref{subfig:lobgf} and QCD Compton (QCDC) scattering Figure~\ref{fig:LOFeynmandiags}\subref{subfig:loqcdc}, with gluons and quarks in the initial state, respectively.
\begin{figure}
	\centering
	\begin{subfloat}[]{
		\includegraphics[width=0.45\textwidth]{Figures/source/jetsfeynmab/BGF}
		\label{subfig:lobgf}
		}%
		\end{subfloat}
		\begin{subfloat}[]{
		\includegraphics[width=0.45\textwidth]{Figures/source/jetsfeynmab/QCDC}
		\label{subfig:loqcdc}
		}%
		\end{subfloat}
	\caption{Leading-order Feynman diagrams contributing to the jet production cross section in NC DIS. (a) Boson-gluon fusion; (b) QCD-Compton scatering processes.}
	\label{fig:LOFeynmandiags}
\end{figure}

The interplay of these two processes allows the effects attributed to the strong coupling and various PDF components to be disentangled, a value of \asz to be extracted and the proton PDFs to be constrained.

In order to give a rigorous definition of the jet, an algorithm for assignment of the particles to a jet must be provided. The proper combination of the particles has to fulfil the following general conditions:
\begin{itemize}
	\item infrared and collinear safety (see Section~\ref{subsec:nlojetcalc});
	\item conservation of factorisation properties of the hard and soft processes;
	\item little sensitivity to the hadronisation effects;
	\item relative insensitivity to the soft interactions of the hadron remnant;
	\item invariance under longitudinal Lorentz boosts;
	\item easy implementation at the particle level in experimental analyses as well as at the parton/hadron level in perturbative theoretical calculations.
\end{itemize}
Among others the recombination-type generalised \kt-algorithm satisfies all mentioned requirements and is defined by the following iterative procedure\footnote{The input objects may refer to the energy deposits in the calorimeter cells; the set of partons in MC or fixed-order predictions or the set of stable hadrons appearing at the hadron level of MC simulations.} (see Figure~\ref{fig:jetcombinationalgorithm}).
\begin{enumerate}
	\item A distance measure, $d_{ij}$, quantifying the phase-space separation of two objects $i$ and $j$, is defined for each pair of particles:
	\begin{equation}
	  d_{ij} = \mathrm{\text{min}} \left( E_{\text{T},i}^{2n}, E_{\text{T},j}^{2n} \right) \dfrac{\Delta R_{ij}^2}{R_0^2},
		\label{eq:dij}
	\end{equation}
	where $\Delta R_{ij}^2 = \left( \eta_{i} - \eta_{j} \right)^2 + \left( \phi_{i} - \phi_{j} \right)^2$ is the angular separation between objects. The dimensionless parameter $R_0$ determines the jet radius.
	\item A quantity, $d_i$, defining the distance to the beam-axis is calculated for each object $i$:
		\begin{equation}
	  d_{i} = E_{\text{T},i}^{2n}.
		\label{eq:di}
	\end{equation}
	\item Two objects $i$ and $j$ are merged according to the Snowmass~\cite{proc:snowmass:1990:134} convention\footnote{Other conventions exist. The Snowmass prescription results in massless jets.}, whenever some $d_{ij}$ is minimal among all $d_{ij}$ and $d_{i}$:
	\begin{align}
		E_\text{T} = E_{\text{T},i} + E_{\text{T},j} & \qquad \eta = \frac{\eta_iE_{\text{T},i} + \eta_jE_{\text{T},j}}{E_\text{T}} & \qquad \phi = \frac{\phi_iE_{\text{T},i} + \phi_jE_{\text{T},j}}{E_\text{T}}.			 \label{eq:snowmass}
	\end{align}
	When $d_i$ is the smallest, the object is called jet and removed from the list.
	\item The algorithm is repeated until no objects remain in the list.
\end{enumerate}

The parameter $n$ in Eq.~\eqref{eq:dij} defines three types of algorithm:
\begin{itemize}
	\item \textsl{$n$ = -1}: the inclusive anti-\kt~algorithm~\cite{Cacciari:2008gp}, which is now extensively used at the LHC. This algorithm results in jets of circular shape. The recombination process is characterised by first assigning particles with largest $E_\text{T}$ to the jets;
	\item \textsl{$n$ = 0}: the Cambridge-Aachen~\cite{Dokshitzer:1997in} algorithm, which takes into account only angular separations between objects, was mostly used in $e^+e^-$ collider experiments;
	\item \textsl{$n$ = 1}: the inclusive \kt~algorithm~\cite{Catani:1993hr}, which produces jets of irregular shape and, in contrast to anti-\kt, recombines particles with small $E_\text{T}$ first.
\end{itemize}
 It has been shown that the \kt~and anti-\kt~have similar performance in photoproduction~\cite{np:b864:1} and DIS~\cite{Abramowicz:2010ke}. The study~\cite{pl:b649:12} has demonstrated that $R=1$ is the optimal choice of the radius parameter at \hera. Taking this into account, the choice of the \kt~algorithm with $R=1$ was adopted in this thesis. Taking advantage of the longitudinal invariance of the algorithm, the jet search was performed in the Breit frame, which is described below.

\begin{figure}
	\centering
		\includegraphics[width=\linewidth]{./Figures/jetcombinationalgorithm.png}
	\caption{Recombination-type jet algorithm flow char.}
	\label{fig:jetcombinationalgorithm}
\end{figure}


\subsection{Breit Frame}
\label{subsec:breitframe}
The Breit or brick-wall reference frame~\cite{feynman:1972:photon,zfp:c2:237} is defined such that the exchange boson collides with a proton without transverse momentum transfer. In this frame the momenta of the proton, $P$, and exchange boson, $q$, satisfy the equation:
\begin{equation}
2x\vec{P} + \vec{q} \stackrel{!}{=} 0.
\label{eq:breitframe}
\end{equation}
In this frame the boson momentum is aligned along the positive $Z$-direction and has only one space-like component \textit{i.e.} $q=\left( 0, 0, 0, -Q\right)$. The schematic illustration of the QPM and QCD Compton processes in the Breit frame is demonstrated in Figure~\ref{fig:breitframe}. The presence of non-zero transverse momentum in the Breit frame is a distinct feature of a QCD process that can be easily identified experimentally. As result the requirement of a jet in the Breit frame with sufficiently high transverse energy is related to a parton generated in the lowest $\mathcal{O}\left(\alpha\as\right)$ order QCD hard process.
\begin{figure}
	\centering
	\begin{subfigure}{.49\textwidth}
		\centering
		\includegraphics[width=0.9\linewidth]{./Figures/BreitFrame}
		\caption{}
		\label{fig:breitframeqpm}
	\end{subfigure}
	\begin{subfigure}{.49\textwidth}
		\includegraphics[width=0.9\linewidth]{./Figures/BreitFrameQCD}
		\caption{}
		\label{fig:breitframeqcd}
	\end{subfigure}
	\caption{Schematic illustration of the Born (a) and QCD Compton (b) processes in the Breit frame in ($p_T$, $p_Z$)-plane. In quark-parton model process incoming exchange boson and parton have collinear momenta. The contribution from QCD processes results in non-zero outgoing parton transverse momentum.}
\label{fig:breitframe}
\end{figure}

\subsection{Calculation of Next-to-Leading-Order Jet Cross Section}
\label{subsec:nlojetcalc}
As described in Section~\ref{subsec:factorisation}, the predictions for the jet-production cross sections in $\ep$ collisions have a factorised form (see Eq.~\eqref{eq:disfactorisation}). The partonic cross section is calculated perturbatively, as a power series in the strong coupling, \as. The predictions for the jet cross sections are finite at each order according to the KNL theorem\cite{Kinoshita:1962ur,Lee:1964is}, provided an infrared- and collinear-safe jet-algorithm is used. However, the parton configurations with soft or collinear radiation have divergent matrix elements; after dimensional regularisation soft and collinear (overlapping) divergences appear as $1/\epsilon$ ($1/\epsilon^2$) poles in the expressions. These divergences cancel exactly with those arising from the virtual contributions.

The differential jet cross section is calculated according to the expression~\cite{PDG:2014}:
\begin{equation}
\frac{\mathrm{d}\sigma}{\mathrm{d}X} = \frac{1}{\text{flux}}\, \sum_n{ \frac{1}{n!} \, \int{\mathrm{d}\Phi^{n}} \, \overline{\sum}{ \left| \mathcal{M}^{\left(n\right)}\left(p_i\right) \right|^2 } \delta\left( X - \mathcal{X}_n\left( p_i\right)\right)},
\label{eq:pqcdxs}
\end{equation}
where $\mathrm{d}\Phi^{n}=\prod_{i=1}^{n}\frac{\mathrm{d^3}p_i}{\left(2\pi\right)^32E_i}$ is an element of $n$-body phase space and $\mathcal{M}$ denotes the Lorentz-invariant matrix element. The first summation is performed over all n-parton final states, assuming that quarks, antiquarks and gluons are indistinguishable ($1/n!$ is a symmetrisation factor). The inner sum represents the averaging over possible colour and spin configurations. The jet-function $\mathcal{X}_n\left( p_i\right)$ of the momenta of $n$ partons represents the measurement observable e.g. $\etjetb,\,\etajetb$ etc. In order to ensure cancellation of real and virtual divergences, the jet algorithm must be independent of the number of soft and collinear partons in the final state. The cancellation of divergences holds only if the observable satisfies the following conditions:
\begin{equation}
\left.
\begin{aligned}
	&\mathcal{X}_{n+1}\left( p_1,\dots,\lambda p_n,\left(1-\lambda\right)p_{n+1}\right)\\
	&\mathcal{X}_{n+1}\left( p_1,\dots,\lambda p_n,0\right)
\end{aligned}
\right\} = \mathcal{X}_{n}\left( p_1,\dots, p_n\right),
\label{eq:}
\end{equation}
where $\lambda\in\left[0;1\right]$ is a parameter used to implement smooth transition from from $n+1$ to $n$-parton configuration. The jet-functions $\mathcal{X}_{n+1}$ and $\mathcal{X}_{n}$ must be equal in collinear and soft limits. The algorithm must produce identical results if a single particle is replaced by a pair of collinear particles carrying the same total momentum, or if the energy of one of the particles vanishes.

In this analysis the infrared- and collinear-safe \kt-jet-algorithm was used for the reconstruction of jets from the final state partons. Since fixed-order QCD predictions refer to the jets of partons while the measurements refer to hadronic jets, the calculations were corrected to the hadron level using Monte Carlo predictions (see Section~\ref{subsec:hadrcorr}).

Practical calculations, suitable for the comparison with experimental results involving cuts (e.g. phase-space restrictions or detector-acceptance limitations) utilise numerical techniques for the calculation of the phase space integrals. General schemes for the calculation of the jet production cross section at next-to-leading order, suitable for numerical calculations and independent of experimental requirements, exist. One such scheme~\cite{Catani:1996vz} is briefly described in the following. 

\subsection{Subtraction Scheme}
\label{subsec:subscheme}
The NLO $n$-jet partonic cross section is a sum:
\begin{equation}
\sigma = \sigma_\text{LO} + \sigma_\text{NLO} = \int_n{\mathrm{d}\sigma^B} + \left[ \int_{n+1}{\mathrm{d}\sigma^R} + \int_n{\mathrm{d}\sigma^V} \right],
\label{eq:sigmanjet}
\end{equation}
where $\sigma^B$ is the born-level cross section, $\sigma^R$ is the real-radiation correction and $\sigma^V$ is the virtual correction. In order to remove explicit divergences from the the real and virtual parts, specially constructed counter-terms are added and subtracted from Eq.~\ref{eq:sigmanjet}. The counter-term is an approximation to the real-radiation contribution in the region of the phase-space containing a singularity and has the same point-wise singular behaviour. Each singular parton configuration requires a corresponding counter-term. The real-radiation contribution with subtracted counter-term, $\sigma^A$, becomes a regular function that can be integrated in $D=4$ dimensions:
\begin{equation}
\int_{n+1}{\mathrm{d}\sigma^R} \rightarrow \left[ \int_{n+1}{\mathrm{d}\sigma^R} - \int_{n+1}{\mathrm{d}\sigma^A} \right].
%\label{eq:sigmanjet}
\end{equation}
The virtual contribution term is modified as follows:
\begin{equation}
	\begin{split}
		\int_{n}{\mathrm{d}\sigma^V} \rightarrow& \left[ \int_{n}{\mathrm{d}\sigma^V} + \int_{n+1}{\mathrm{d}\sigma^A} \right] = \\
                                                      & \left[ \int_{n}{\mathrm{d}\sigma^V + \int_{1}{\mathrm{d}\sigma^A}} \right]
	\end{split}
%\label{eq:sigmanjet}
\end{equation}
The divergence in the virtual contribution appear as a pole in $\epsilon$ but this pole is exactly cancelled by that resulting from one-parton phase-space analytic integration of the counter-term. After the cancellation, the integration of the virtual part can be carried out numerically in physical $D=4$ dimensions. Since the net effect of adding and subtracting counter-terms is zero, this scheme results only in reshuffling of the divergences.


%\subsection{Jet Production in the Breit frame}
%\label{subsec:jetsinbreit}
%Jets in breit frame


\section{Monte Carlo Models}
\label{sec:mcmodels}
A precise theoretical description of the final state of $\ep$~scattering from first principles is currently an intractable problem. It requires calculations in regions of phase space where perturbative techniques are not applicable or have to be performed to high orders. Phenomenological models were developed in order to describe such processes. Typically, such models are implemented in the form of event generators and utilise Monte Carlo calculations. Some models used for the description of DIS hadronic final state are described in the following.
\subsection{QCD Parton Showers}
\label{sec:qcdpartonshower}
The parton-shower approach is used to simulate higher-order perturbative QCD contributions when a complete calculation is infeasible or unknown. For example, the DGLAP approach can be utilised to describe initial-state and final-state radiation. The probability for a branching, $\mathcal{P}_{a\rightarrow bc}$, during the evolution is governed by the equation:
\begin{equation}
\frac{\mathrm{d}\mathcal{P}_{a\rightarrow bc}}{\mathrm{d}\qsq} = \int_0^1{\mathrm{d}z\frac{\as\left(\qsq\right)}{2\pi}P_{_{a\rightarrow bc}}}\left(z\right),
\end{equation}
where $P_{a\rightarrow bc}\left(z\right)$ are the Altarelli-Parisi splitting kernels (see Section~\ref{subsubsec:dglapeq}).

Such an approximation is usually used in general-purpose event generators where the successive radiation is simulated until the evolution parameter, e.g. virtuality of the daughter partons, reaches some low energy scale $\mathcal{O}\left(1\;\GeV\right)$. At this point the showering process is stopped and partons are recombined into colourless hadrons.

In order to improve the leading-logarithmic accuracy of the parton-shower approach, hard emissions are described using complete matrix elements. In this case an additional intermediate scale is introduced. At this energy scale regions dominated by parton shower or hard-scattering dynamics are matched. Nowadays most of the event generators are based on LO matrix elements. However, NLO calculations with matched parton showers are starting to appear~\cite{Frixione:2007vw,Frixione:2002ik}.

Another approximation for QCD radiation that is widely used to describe DIS-related processes is the \emph{colour dipole model} (CDM)~\cite{np:b306:746,Andersson:1988ee,Andersson:1990dp,Gustafson:1992uh,Andersson:1988gp,Gustafson:1986db}. It is assumed in this model that the quark--antiquark pairs form colour dipoles with a corresponding dipole radiation pattern. The gluons themselves are interpreted as pairs of colour charges that also build colour dipoles. The schematic illustration corresponding to the CDM picture is shown in Figure~\ref{fig:cdm}. 
\begin{figure}[t]
	\centering
		\includegraphics[width=0.65\textwidth,angle=0]{./Figures/source/MEPSradiation}
	\caption{Schematic demonstration of the matrix element + parton shower approach.}
\label{fig:meps}
\end{figure}
\begin{figure}[t]%
\centering
\includegraphics[width=0.95\textwidth,trim={20 550 20 0},clip]{./Figures/source/CDMradiation}%
\caption{The radiation pattern from the colour-dipole model.}%
\label{fig:cdm}%
\end{figure}
The radiation from each dipole is assumed to be independent. It proceeds iteratively until some stopping criterion is reached, for example the invariant mass of a dipole falls below some cut-off  value. The CDM is based on leading-order matrix elements in the soft gluon approximation. The cross section for the parton emission with transverse momentum $p_T$ and rapidity~$y$ (see Eq.~\eqref{eq:rapidity}) in CDM reads:
\begin{equation}
\mathrm{d}\sigma = \frac{n_c\as}{2\pi}\frac{\mathrm{d}p_T^2}{p_T^2}\mathrm{d}y.
\end{equation}
In contrast to the leading-logarithm DGLAP-based parton-shower algorithm there is no $k_T$-ordering for the gluon radiation. Emitted partons are rather uniformly distributed in $k_T$, thus the CDM approach is somewhat similar to the BFKL evolution.

Another important issue in the simulation of the parton showers is quantum-mechanical interference of the initial-state and final-state radiation or the interference between the partons emitted either in the initial or final state. These effects are naturally taken into account in the complete perturbative calculations, however special care must be taken in the resummed calculations like those based on DGLAP evolution, because they are based on a probabilistic description of the whole process in contrast to quantum-mechanical probability amplitudes.

\subsection{Fragmentation}
\label{subsec:fragmentation}
In order to be able to compare pQCD predictions to experimental results the calculations have to be defined in terms of experimentally observable quantities, which usually are functions of the momenta of the final-state hadrons. The formation of hadrons, called \emph{hadronisation}, is essentially a non-perturbative process and first-principle calculations are impossible. Therefore phenomenological hadronisation models are used to correct partonic predictions in order to obtain a consistent observable definitions. In practice, the transition from partonic quantities to those defined in terms of hadrons is usually modelled by means of general-purpose event generators. Two widely used hadronisation models are described below.
\begin{figure}[t]
	\centering
	\begin{subfloat}[]{
		\includegraphics[width=0.45\linewidth,angle=0,trim={50 500 0 0},clip]{./Figures/source/LundString}
		\label{fig:lund}
	 }%
	\end{subfloat}
	\begin{subfloat}[]{
		\includegraphics[width=0.45\linewidth,trim={50 500 0 0},clip]{./Figures/source/ClusterModel}
		\label{fig:cluster}
	}%
	\end{subfloat}
	\caption{Schematic illustration of the string fragmentation (a) and cluster fragmentation (b) model}
\label{fig:fragmentationmodels}
\end{figure}
\subsubsection{String Fragmentation Model}
It is assumed in the Lund string model~\cite{Andersson:1983ia} that the flux of the colour field between two quarks is confined within a tube of finite transverse size. This string-like object has a constant energy-density per unit length of $\mathcal{O}\left( \text{1 \GeV/fm}\right)$ and the potential energy of the string increases with increasing separation between the quarks. When the tension exceeds the quark--anti-quark production threshold, the $q\bar{q}$-pair is picked up from the vacuum and the string breaks up. Loose ends of the string are terminated by newly created $q$ and $\bar{q}$ and the process is iterated until the potential energy of the daughter strings fall below a cut-off $\mathcal{O}\left( \text{1 \GeV}\right)$. The gluons have two colours and are represented as a joint between two strings or a kink in the colour flux of the $q\bar{q}$-system in this model. 

A schematic illustration of the Lund picture of the hadronisation process is shown in Figure~\ref{fig:lund}.
\subsubsection{Cluster Fragmentation Model}
In the cluster model~\cite{Webber:1983if,Field:1982dg} all partons after the parton-shower step are combined into colourless objects. If the invariant mass of the cluster is large enough it can decay into lighter clusters, which subsequently decay into hadrons. The gluons in this model are converted into $q\bar{q}$-pairs and do not appear in the hadron formation process. This model was inspired by the ``preconfinement''~\cite{Amati:1979fg} idea according to which the colour-connected partons group in the phase space towards the end of perturbative evolution. The cluster model process is schematically depicted in Figure~\ref{fig:cluster}.

\subsection{General-Purpose Event Generators}
General-purpose event generators are indispensable tool in high-energy physics because they provide full access to the details of the event  final state. Using event generators and detector simulations, the detector performance can be investigated (see Chapter~\ref{ch:unfolding}) or effects related to the background contributions can be estimated (see Section~\ref{sec:eventsampletab}).

The generation of events proceeds through Monte Carlo sampling of the processes according to the probability of their occurrence. An ensemble of MC events must resemble the characteristic features of the data. These programs usually have several levels naturally corresponding to the processes separated by different time-scales. The simulation of the hard interaction, occurring over the shortest time intervals, is usually based on the leading-order contribution that can be relatively easy calculated in perturbation theory (see Section~\ref{subsec:pqcdstructure}). The higher perturbative orders in MC generators are approximated by parton-shower models, as was briefly described in Section~\ref{sec:qcdpartonshower}. The last step, corresponding to the formation of color-neutral hadrons is implemented in hadronisation models (see Section~\ref{subsec:fragmentation}), which use the result of the parton-shower stage as an input. The output of the event generators is usually provided in the form of a table containing list of particles and their four-momentum vector components. The output available after the parton-shower and hadronisation steps are called the parton and hadron levels, respectively. 

In this work, the NC DIS events were generated using the \heracles program~\cite{cpc:69:155} with the \djangoh~\cite{cpc:81:381} interface to the \lepto~\cite{Ingelman:1996mq} and \ariadne~\cite{cpc:71:15,Lonnblad:1994wk} parton-shower simulation programs. The \djangoh code implements higher-order QED corrections i.e. real- and virtual-photon radiation as well as two-photon exchange. As an input in the MC, the \cteqfived~\cite{pr:d51:4763} proton PDF sets were utilised. Basic information about \lepto and \ariadne generators is summarised below.

\subsubsection{\lepto}
The \lepto event generator combines the leading-order QCD matrix elements (ME) for the hard-scattering process together with the DGLAP parton shower (PS) for the soft-gluon emission. In order to ensure colour coherence during the showering process, angular ordering is imposed. The Lund string model as implemented in \jetset~\cite{cpc:43:367} is used to simulate the hadronisation process. This generator also includes the LO electroweak processes necessary for the description of high-\qsq~DIS. The higher-order QED effects are obtained through the interface to the \heracles program. The \lepto generator is also often called MEPS and is used as a reference MC generator in this analysis.
\subsubsection{\ariadne}
The colour-dipole pattern for QCD radiation is implemented in the \ariadne event generator. Since this model naturally includes only the QCD Compton scattering diagram, the BGF graph contribution was introduced in addition. The hadronisation is performed using the same \jetset interface as used for \lepto. This event generator was used in the analysis mainly to estimate systematic effects attributed to the choice of the parton-shower model.
