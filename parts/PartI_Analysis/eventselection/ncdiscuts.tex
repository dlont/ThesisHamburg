\section{Signal Characteristics}
\label{sec:signalchar}
As can be seen from the Eq.~\eqref{eq:q2el}, large values of \qsq\, in NC DIS events correspond to large scattering angles of the electron. The electron produces a well isolated electromagnetic shower in the calorimeter with a charged track pointing from the interaction vertex to the electromagnetic cluster (see Figure~\ref{fig:ncdiseventdisplay}). At low and medium values of $\qsq < 500\, \GeV^2$ the electron typically scatters into the RCAL. For larger values of $\qsq$ the electron can be found in BCAL and at very high-$\qsq$ the electron scatters to the FCAL. In NC DIS at \hera initial-state transverse momentum is zero, therefore final-state transverse momentum of the event must vanish according to momentum conservation:
\begin{equation}
P_{T,tot}^2 = P_{X,tot}^2+P_{Y,tot}^2 = \left(\sum_i{E_i\sin{\theta_i}\cos{\phi_i}}\right)^2 + \left(\sum_i{E_i\sin{\theta_i}\sin{\phi_i}}\right)^2 \approx 0\;\GeV^2.
\end{equation}
 In the above equation the summation runs over all energy deposits in the CAL. Momentum conservation implies, that the electron recoils from the hadronic system, which balances the scattered electron in the transverse-momentum plane.
\begin{figure}[htbp]
	\centering
	\includegraphics[width=\textwidth]{./Figures/event_display_53520_93222_jets} 
	\caption{A typical NC DIS event with $E-P_Z=51~\GeV$, $P_T=2~\GeV$ and \qsq = 965~$\GeV^2$~and two jets identified in the Breit frame with \etjetb> 8~\GeV~as recorded by the \zeus detector. Red blocks represents energy deposits in the calorimeter. The final-state objects: jets (wide energy clusters in the BCAL) and the scattered electron are indicated with arrows.}
	\label{fig:ncdiseventdisplay}
\end{figure}

In NC DIS processes the quantity $\delta$, defined as
\begin{equation}
\delta = \delta_h + \delta_e = \left( E_p - P_{z,p} \right) + \left( E_e - P_{z,e}\right) = 2E_e = 55\;\GeV,
\label{eq:deltadis}
\end{equation}
is also conserved, i.e. the same variable calculated from the final state energy deposits
\begin{equation}
\delta = \sum_i{\left(E_i - P_{Z,i}\right)} = \sum_i{\left(E_i - E_i\cos{\theta_i}\right)} \approx 55\;\GeV.
\end{equation}
is preserved. The deviations of $P_{T,tot}$ and $\delta$ from the nominal quantity can be caused by e.g. undetected particles escaping the \zeus detector and/or due to finite energy resolution of the calorimeter.

As was explained in Section~\ref{subsec:breitframe}, at leading-order in the Breit frame a process with hard QCD interaction always has at least two hadronic jets. In practice, however, one of the jets can be failing to pass selection criteria due to limited detector acceptance and/or resolution, therefore the signal events in this analysis were required to have at least one jet identified in the Breit frame within the acceptance of the apparatus and having transverse energy exceeding some minimum energy threshold.

\section{Characteristics of Background Processes}
\label{sec:bgchar}

\subsection{Photoproduction}
\label{subsec:photoprodbg}
The cross section of inelastic electron-proton scattering with $\qsq \approx 0\;\GeV^2$ is large and serves as potential source of backgound to the NC DIS analysis. At low \qsq, the scattered electron is deflected by a small angle and escapes in the beam pipe. Isolated electromagnetic clusters from e.g. $\pi^0$ decays can be misidentified as the final-state electron thus mimicking the signature of NC DIS processes. In photoproduction, as in NC DIS the total transverse momentum vanishes but the quantity $\delta$ calculated from the final-state energy deposits will be smaller than 55 \GeV~by approximately twice the energy of the escaped electron\footnote{The energy and momentum of the particles escaping in the forward direction is approximately cancelled.}. Besides restrictions on the longitudinal momentum balance, the photoproduction events can be removed by applying extra cuts on isolation of the electron candidate because electromagnetic clusters from hadron decays are often part of the hadronic final state and are accompanied by other hadrons.

\subsection{Beam-Gas Interactions. Cosmic and Halo Muons}
\label{subsec:beamgasfeatures}
The electron and proton beams can interact with the residual gas in the beam pipe. The proton-gas collisions are characterised by large hadronic activity and multiple tracks emerging at low polar angles upstream from the nominal interaction point (see Figure~\ref{fig:beamgaseventdisplay}). In such events, signals in the FCAL follow those from the RCAL and are separated by the time interval necessary for particles to reach the forward calorimeter. Using the information about particles arrival time in forward and rear parts of the CAL as well as signals from the specific iron-scintillating detector (Veto-Wall) located in the rear part of the \zeus, these events can be efficiently rejected. In addition, the suppression of beam-gas events can be achieved by imposing the restrictions on the longitudinal momentum balance, because such processes are characterised by lower value of $\delta$ than that for NC DIS. In contrast to the proton-gas collisions, the rate of the electron beam interactions was found in a previous study~\cite{thesis:moritz:2001} to be very small and therefore negligible.

\begin{figure}[ht]
	\centering
	\includegraphics[width=\textwidth]{./Figures/event_display_54628_99024_lessthicklines1} 
	\caption{An example of proton beam -- residual gas collision identified using pilot bunch crossing. Many tracks emerge from the interaction point in the rear part of the \zeus detector outside of the acceptance of the tracking system.}
	\label{fig:beamgaseventdisplay}
\end{figure}

\begin{figure}[htbp]
	\centering
	\includegraphics[width=\textwidth]{./Figures/event_display_54450_48590_cosmic_track_thicklines} 
	\caption{A high-energy cosmic muon traversing the the volume of the \zeus detector. Muon stubs are present in the upper and lower halves of the backing calorimeter and muon chambers. An energy cluster from the interaction of the muon with the material of the CAL can mimic signature of hadronic jets.}
	\label{fig:cosmiceventdisplay}
\end{figure}
Additional sources of background come from cosmic and beam-halo muons. Cosmic muons are produced in high-energy interactions of cosmic rays with Earth atmosphere. Passage of cosmic muons though the volume of the detector results in a time difference between the signals in the upper and lower parts of the CAL (see Figure~\ref{fig:cosmiceventdisplay}). Such a timing signature and a typically low track multiplicity were used to detect cosmic events. In contrast to cosmic rays, beam-halo muons, emerging from decays of pions produced in beam-gas interactions upstream of the detector, travel parallel to the beam axis and therefore the time difference between the signals in the RCAL and FCAL can be used to eliminate such processes, similarly to beam-gas interactions described above. Moreover, cosmic and beam-halo events have non-zero total transverse momentum due to asymmetry of the energy deposits in the transverse plane. This information can also be used for the suppression of these background contributions. 

\subsection{QED Compton scattering}
The reactions of the type $ep \rightarrow e' \gamma p$ or $ep \rightarrow e'\gamma X$ in which the initial or the final state electron radiates a high-energy photon are called elastic and inelastic QED Compton (QEDC) scattering, respectively. In case of inelastic reaction the proton breaks up leaving some hadronic activity in the forward direction while in elastic processes the proton escapes down the beam pipe staying intact. QEDC events can be misidentified with high-\qsq~NC DIS processes when the electromagnetic cluster from the photon is confused with the final-state electron leading to incorrect determination of event kinematics. The characteristic features of such events are the presence of two isolated electromagnetic energy deposits in the calorimeter with approximately equal transverse momentum and low or vanishing hadronic activity in the FCAL (see Figure~\ref{fig:qedceventdisplay}). As was demonstrated in the study~\cite{thesis:moritz:2001} inelastic QEDC processes are well described by the \djangoh MC while elastic scattering is not well reproduced. Therefore such events have to be removed from the sample.

\begin{figure}[htbp]
	\centering
	\includegraphics[width=\textwidth]{./Figures/event_display_52396_3509_elastic_compton_thicklines} 
	\caption{An elastic QED Compton scattering event as recorded by the \zeus detector. The electromagnetic clusters from the electron and photon (two oppositely charged tracks from $\gamma \rightarrow e^+e^-$ conversion) are detected in the RCAL. The particles balance each other in $P_T$ and there is no energy deposit in the forward direction.}
	\label{fig:qedceventdisplay}
\end{figure}

\section{Event selection.}
The pre-selection of the NC DIS events relevant for the analysis starts naturally during the data-taking when the the \zeus trigger system is used. Only those events for which positive trigger decision was taken are stored on tape and can be further analysed off-line. The following sections present the general information about analysed data and MC samples and selection criteria used to obtain restricted event samples.

\subsection{Data and MC sets.}
This analysis was performed using the data recorded during the 2004 -- 2007 running period. The $\ep$ centre-of-mass collision energy during this period was \sqs = 318 \GeV. Table~\ref{tab:selecteddatasample} summarises the luminosity and average polarisation values for analysed data-taking periods.
\begin{table}
	\centering
		\begin{tabular}[h]{|c|c|c|}
		  \hline
			Period & Integrated Luminosity & Average Polarisation \\
			\hline \hline
			2004 -- 2005  & 133.6 \invpb  & -0.07 \invpb \\
			2006             & 53.0 \invpb    &  0.09 \invpb \\
			2006 -- 2007  & 109.5 \invpb  & -0.07 \invpb \\
			\hline
		\end{tabular}
	\caption{Luminosity and average polarisation of the final data samples used in the analysis.}
	\label{tab:selecteddatasample}
\end{table}
The total integrated luminosity of the processed data sample amounts to 295~\invpb.

MCMCMMC

\section{On-line selection}
\label{sec:onlineselect}
As described in Section~\ref{sec:daqtrigger} the \zeus trigger has a three-level architecture. Below individual requirements imposed on the data at each trigger level are described. Technically, after the event was recorded on storage-tape it was marked with specific flags, the so called trigger slots or bits, according to event characteristic features, like total and/or regional energy sums or presence of electromagentic clusters etc. Any of the trigger bits listed below was required to be fired\footnote{Which corresponds to a logical \textbf{OR} between individual bits.} in order to keep the event for further analysis. The following subsections outline the cuts imposed by every trigger bit utilised in this work. 

\subsection{FLT trigger}
\label{subsec:fltcuts}
\begin{itemize}
	\item \textbf{FLT bit 40:} The total energy in the electromagnetic section of the calorimeter, $E_\text{EMC}^\text{CAL}$, exceeds 20 \GeV\, and a condition on the track multiplicity, the so called \textit{track veto} (see Section~\ref{sec:trkvetoeff}).
	\item \textbf{FLT bit 43:} The total transverse energy in the calorimeter, $E_\text{T}^\text{CAL}$, exceeds 20 \GeV\, and at least one good track\footnote{????????????????????????????} is required.
	\item \textbf{FLT bit 50:} The total energy, $E^\text{CAL}$, greater than $15\, \GeV$\, or $E_\text{EMC}^\text{CAL} > 10\, \GeV\,$ or $E_\text{EMC}^\text{BCAL} > 3.5\, \GeV$\, or $E_\text{EMC}^\text{RCAL} > 2\, \GeV$\, and $E_\text{T}^\text{CAL} > 1\, \GeV$\, and a good track is required.
\end{itemize}

\subsection{SLT trigger}
\label{subsec:sltcuts}
The second-level trigger was used to suppress further beam-related background. The information from the \zeus global tracking trigger was utilised at SLT for the reconstruction of the interaction vertex position and to reject events originating from background processes. Moreover, the calorimeter timing information was used extensively to suppress beam-gas and cosmic-ray background. For example, hadrons emerging in vicinity of the nominal interaction point will be characterised by approximately the same arrival time in different parts of the CAL, in contrast to cosmic-shower events for which the signals from the upper part of the BCAL will arrive earlier than from the bottom part. Thus, given the high timing resolution of the CAL, the background processes can be efficiently discriminated.

\subsubsection{TLT trigger}
\label{subsec:tltcuts}
The final stage of the trigger selection was based on the identification of the scattered DIS electron. The reduced read-out rate at the TLT allowed application of the complex reconstruction algorithms that were much closer related to those used in the off-line analysis. The TLT bit \textsf{DIS03} together with DST bit 12 imposing loose cuts on the electron candidate were used in the selection chain and are summarised in the Table~\ref{tab:TLTDSTreq}.
\begin{table}[ht!]
\centering
\begin{tabular}{|c|c|}
\hline scattered electron energy & $E'_{el}>4\,\GeV$ \\ 
\hline distance from the beam pipe  & $R_{el}^{CAL}>35\,\cm$ \\ 
\hline longitudinal momentum balance & $30<E-p_Z<100\,\GeV$\\
\hline 
\end{tabular} 
\caption{The requirements imposed on the events at the TLT.}
\label{tab:TLTDSTreq}
\end{table}

\section{Off-line selection}
\label{sec:offlineselect}

\subsection{Data-quality requirements.}
The operation of the detector components was continuously monitored during the data-taking. The status of each of the detector components was stored for every run in the so-called, ``EVTAKE'' flag. For this analysis the EVTAKE=1 was required indicating that all main detector components e.g. CTD, MVD, CAL were fully operational.

In addition to EVTAKE flag, the LPOLTAKE and TPOLTAKE status records were used in this analysis in order to verify the availability of the information from the corresponding lepton-beam polarisation detectors. At least one of the detectors was required to be in good status on in order to determine the polarisation value for particular runs.

Trigger-level event selection cannot fulfil signal to background discrimination requirements necessary for precision analysis because of limited processing time and significant complexity of the reconstruction algorithms for the objects such as secondary vertices, particle decays or jets. In addition, information about detector operating conditions such as the number of dead channels, changes in high voltage, CTD gas pressure, temperature, etc. are difficult or impossible to take into account at the trigger level. After the trigger-based pre-selection a set of cuts is applied off-line to ensure a low level of background and high purity of selected sample. Additional requirements were imposed to restrict the phase space of the measurement. 

\subsection{Electron selection}
\label{subsec:eleselect}

To ensure high purity and reliability of the reconstruction of the scattered electron, the following requirements were imposed on electron candidates.
\begin{itemize}
	\item \textbf{Probability:} The probability, as given by the SINISTRA algorithm, was required to be greater than $90\%$. If several electron candidates satisfied this criterion, the candidate with the highest probability was used for the reconstruction of event kinematics.
	\item \textbf{Energy:} The electron energy, \eefin was required to be greater than 10 \GeV, because low-energy electron candidates appear mostly from misidentified electromagnetic showers from $\pi^0\rightarrow\gamma\gamma,\, \eta\rightarrow\gamma\gamma$ decays. In addition, this cut restricts the electron-finding algorithm to the region with high electron-identification efficiency.
	\item \textbf{Isolation:} In order to remove events in which the electromagnetic shower of the electron candidate is contaminated by the energy deposits from hadronic tracks, the fraction of the energy within a cone of radius of 0.7 units in the pseudorapidity-azimuth plane, not associated to the electron, was required to be less than $10\%$. The cone axis was defined by the electron momentum direction.
	\item \textbf{Track Matching:} The angular resolution of the tracking detectors is typically much better than that of the calorimeter, therefore precision of the position of the electromagnetic shower can be improved, when the information from the tracking detectors is used. The charged track pointing to the electromagnetic shower was required to have the distance of closest approach between the track extrapolation point at front surface of the CAL and the energy-cluster position to be less than 10 cm. The track energy as measured by the tracking system had to be greater than 3 \GeV. In case the electon track was outside the acceptance region\footnote{The tracking system covers the region of polar angles restricted to $0.3 < \theta_\text{e} < 2.85$. } of the tracking detectors, the information from the calorimeter system was used.
	\item \textbf{Position:} To ensure full containment of the electromagnetic shower inside the fiducial volume of the calorimeter system and to avoid regions with insufficiently good description by the MC simulations, additional requirements on the position of the electromagnetic shower were imposed. The events in which the electron was found in the following regions were rejected:
	\begin{itemize}
		\item $ -104\, \cm < Z_\text{e} < -98.5\, \cm	$ and $ 164\, \cm < Z_\text{e} < 174\, \cm $ - the so called ``super-crack'' regions between the BCAL and the RCAL or between the BCAL and the FCAL;
		\item $\left| X_\text{e} \right|$ < 10 \cm\, and $Y_\text{e}$ > 80 \cm. In this region some of the calorimeter cells were removed to make room for the  pipes transporting  liquid helium to the superconducting solenoid;
		\item $ 36\, \cm < R^\text{RCAL}_\text{e} < 170\, \cm $, where $R^\text{RCAL}_\text{e}$ is the distance in the $X-Y$ plane from the centre of the beam-pipe to the electron. The leakage of the electromagnetic showers from the electrons hitting the RCAL close to the inner or outer edges is not well simulated, especially at the trigger level~\cite{januschek-p96}. These regions were therefore excluded. 
	\end{itemize}
\end{itemize}

\subsection{Primary vertex selection}
\label{subsec:vtxselect}
Proper identification of the interaction point is important for the reconstruction of the kinematic variables. The longitudinal extent of the interaction region is determined by the length of the interacting bunches and the time structure of the beam. Beam-gas or cosmic rays events are evenly distributed along the longitudinal coordinate, while \ep\, events appear with higher rate in the vicinity of the nominal beam-beam interaction point. Distribution of the longitudinal component of the position of the primary vertex has Gaussian-like shape (see section~\ref{sec:zvtxrew}). Selecting the events with the primary interaction vertex satisfying $\chi^2 < 10$ and $Z_\text{vtx}$ being within $\sim \pm 3\sigma$ of the width of the distributions suppresses the non-physics background contribution and ensures good understanding of the dependence of the acceptance of the calorimeter and tracking systems on the position of the interaction vertex. The mean value and the width of the $Z_\text{vtx}$ changes between different data-taking periods. The restrictions imposed on $Z_\text{vtx}$ are detailed in Table~\ref{tab:zvxcut}.
\begin{table}[htbp]
	\centering
		\begin{tabular}{|c|c|}
			\hline
			Data-taking perod & imposed cut \\
			\hline
			\hline
			2004-2005 $e^{-}$ & $-32\,\cm<Z_\text{vtx}<30.1\,\cm$ \\
			2006-2007 $e^{+}$ & $-28.5\,\cm<Z_\text{vtx}<26.7\,\cm$ \\
			\hline
		\end{tabular}
	\caption{Requirements imposed on the longitudinal cordinate of the position of the interaction vertex.}
	\label{tab:zvxcut}
\end{table}

\subsection{Longitudinal Momentum balance}
\label{subsec:empzcut}
As discussed previously in Section~\ref{subsubsec:beamgasfeatures}, longitudinal momentum balance can be used to discriminate against photoproduction and beam-gas background. The actual value of quantity $\delta=\sum_i{\left(E_i-p_{z,i}\right)}$ for a particular event may deviate from the nominal $\delta=55\;\GeV$ due to finite resolution as well as initial-state radiation effects. Therefore the following requirement was implied:
\begin{equation}
38 < \delta < 65\;\GeV.
\label{eq:epmzcut}
\end{equation}
The lower value was chosen in order to reject photoproduction events efficiently, while the upper cut is required for the suppression of beam-gas events.

\subsection{Transverse Momentum balance}
\label{subsec:empzcut}
Due to finite resolution, the total transverse momentum of an NC DIS event may be greater that zero. The energy resolution of the calorimeter scales approximately as $1/\sqrt{E}$. In order to suppress beam-gas-related, cosmic-ray and charged current processes with a misidentified electron, the following ratio was required to be small:
\begin{equation}
\frac{p_T}{\sqrt{E_T}} < 2.5\;\sqrt{\GeV}.
\label{eq:ptcut}
\end{equation}

\subsection{Event Inelasticity}
\label{subsec:yelcut}

The DST bit 12 has a requirement on inelasticity of the event. In order to have the offline selection consistent with the trigger requirements, an additional cut $y_{el} < 0.75$ was imposed. This requirement rejects photoproduction background but the effect of the cut was found to be very small.

\subsection{Elastic QED-Compton}
\label{subsec:elasticqedcut}
Elastic Compton scattering processes ($ep \rightarrow ep\gamma$) in which the proton remains intact are not well described by Monte Carlo simulations. The characteristic feature of such processes is the presence of two isolated electromagnetic clusters in the calorimeter. Therefore, events containing two electron candidates with transverse momentum vectors opposite to each other ($\Delta\phi > 3$), a ratio of transverse momenta between 0.8 and 1.2 and total energy sum measured in the calorimeter, excluding the energy of the two EM clusters from the electron candidates, below 3 \GeV~were rejected.

\subsection{Higher-order QED predictions}
\label{subsec:qedcorcut}
In order to ensure the validity of higher-order QED corrections applied to the data (see Section~\ref{sec:qedcorr}) the region of phase space given by the requirement $y_{JB}\cdot\left(1-x_{DA}\right)^2>0.004$ was excluded. Monte Carlo predictions were found to be unreliable in this region~\cite{cpc:81:381}.

\subsection{Hadronic Scattering Angle}
\label{subsec:gammahadcut}
In order to suppress events with large hadronic activity in the forward region, where the simulations were found to be inaccurate~\cite{thesis:jose:2003}, a cut on the projection of the hadronic scattering angle in the FCAL $R^\text{FCAL}_{\gamma^{had}} > 18\;\cm$ was implied. This constraint had only a marginal effect because of inelasticity requirements suppressing events with large hadronic activity.

\subsection{Track multiplicity}
\label{subsec:trackmultcut}
The residual beam-gas contamination was minimised by requiring the presence of at least one track, which crossed a minimum of three CTD superlayers and had transverse momentum $p_T>0.2\;\GeV$. This cut had only a minor effect.

\subsection{Phase space}
\label{subsec:phasespace}
The phase-space cuts are performed for selection of the kinematic region of the reaction in question. As was mentioned earlier, two variables completely determine the kinematics of deep inelastic scattering. In this analysis, the boson virtuality \qsq~and inelasticity \y~were used.
\begin{itemize}
	\item \textbf{Photon virtuality}, $125 < \qsq_{DA} < 20000\;\GeV^2$;  deep inelastic scattering processes with large four-momentum transfer were selected.
	\item \textbf{Inelasticity}, $0.2 < \y_{DA} < 0.6\;\GeV$; the lower cut was imposed to reject events with large hadronic activity in the forward direction for which hadronisation corrections were found to be inaccurate. The upper cut was applied to ensure reliability of the detector acceptance corrections.
\end{itemize}

\subsection{Jet selection}
\label{subsec:jetselect}
In order to identify jets, the \kt~clustering-algorithm in the longitudinally-invariant inclusive mode was used. The jet search was performed in the Breit frame, therefore the momentum vectors corresponding to the energy deposits were transformed accordingly. The momentum vectors were constructed from the energy and orientation of the calorimeter-cell centre with respect to the interaction point. By definition the constructed vectors are massless. All cells excluding those corresponding to the electron candidate and satisfying the following requirements were used for the jet reconstruction:
\begin{itemize}
	\item the signals from natural Uranium radioactivity were suppressed by requiring the minimum energy in the cell to be above a threshold;
	\item the cut on imbalance between two photomultiplier signals attributed to a single cell was imposed in order to reject signals due to spontaneous discharge of one of the PMT bases.
\end{itemize}

The phase space for jet production was limited by the following requirements:
\begin{itemize}
	\item \textbf{transverse energy} of the jets in the Breit frame was required to be greater than 8~\GeV.
	\item \textbf{pseudorapidity} of the jets had to be in region $-1<\etajetlab<2.5$.
\end{itemize}
The purity of the jet sample was enhanced by applying the following jet-cleaning cuts:

\begin{itemize}
	\item good isolation of the electron candidate from hadronic jets was ensured by requiring the distance between electron and jet in the pseudorapidity-azimuth plane to be $\Delta R > 1$;
	\item initial-state electron radiation hitting the RCAL was often reconstructed as a jet. Such events were removed from the sample if a jet with $\etjetb > 5\;\GeV$ and $\etajetlab < -1$ was identified;
	\item jets with low transverse energy in the laboratory frame have large relative energy-scale uncertainty, therefore jets with $\etjetlab < 3\;\GeV$ were excluded.
\end{itemize}

\section{Final event sample}
\label{sec:eventsampletab}
The imposed requirements restrict the kinematic phase space of the measurements to

\begin{gather}
125 < \qsq < 20000\;\GeV^2 \qquad 0.2 < y < 0.6, \\
\etjetb > 8\;\GeV \qquad -1 < \etajetlab < 2.5.
\end{gather}
%The information about selected event samples is summarised in Table~\ref{tab:numevents}.
%\begin{table}[htbp]
%	\centering
%		\begin{tabular}{|c|c|}
%			\hline
%			Sample & Number of events \\
%			\hline
%			\hline
%			Data 2004-2005 $e^{-}$ & ??? \\
%			Data 2006-2007 $e^{+}$ & ??? \\
%			Lepto MC $e^{+}$ & ??? \\
%			Ariadne MC $e^{+}$ & ??? \\
%			\hline
%		\end{tabular}
%	\caption{Number of selected events in data and MC simulations.}
%	\label{tab:numevents}
%\end{table}

In general, the data are very well described by the Monte Carlo simulations when renormalised to the size of the data sample. A comparison of MC predictions to the data is demonstrated in Figures~\ref{fig:cp_ari},~\ref{fig:cp_ariadne}. 

%LEPTO
% % % % % % % % % % % % % % % % EVENT
\begin{figure}[ht!]
\begin{center}
\begin{subfloat}[]{\includegraphics[width=.32\textwidth,trim={5 0 50 0},clip] {./Figures/control_plots_ev/lep/new/h_Zvtxcontro_plot_lepto}
   \label{fig:cplep_subfig1}
 }%
\end{subfloat}
 \begin{subfloat}[]{\includegraphics[width=.32\textwidth,trim={5 0 50 0},clip]{./Figures/control_plots_ev/lep/new/h_Ydacontro_plot_lepto}
   \label{fig:cplep_subfig2}
 }%
\end{subfloat}
\begin{subfloat}[]{\includegraphics[width=.32\textwidth,trim={5 0 50 0},clip] {./Figures/control_plots_ev/lep/new/h_Q2dacontro_plot_lepto}
   \label{fig:cplep_subfig3}
 }%
\end{subfloat}
\newline
 \begin{subfloat}[]{\includegraphics[width=.32\textwidth,trim={5 0 50 0},clip]{./Figures/control_plots_ev/lep/h_sienecontro_plot_lepto}
   \label{fig:cplep_subfig4}
 }%
\end{subfloat}
 \begin{subfloat}[]{\includegraphics[width=.32\textwidth,trim={5 0 50 0},clip]{./Figures/control_plots_ev/lep/h_siphcontro_plot_lepto}
   \label{fig:cplep_subfig5}
 }%
\end{subfloat}
 \begin{subfloat}[]{\includegraphics[width=.32\textwidth,trim={5 0 50 0},clip]{./Figures/control_plots_ev/lep/h_sithcontro_plot_lepto}
   \label{fig:cplep_subfig6}
 }%
\end{subfloat}
\newline
 \begin{subfloat}[]{\includegraphics[width=.32\textwidth,trim={5 0 50 0},clip]{./Figures/control_plots_ev/lep/new/h_empzcontro_plot_lepto}
   \label{fig:cplep_subfig7}
 }%
\end{subfloat}
 \begin{subfloat}[]{\includegraphics[width=.32\textwidth,trim={5 0 50 0},clip]{./Figures/control_plots_ev/lep/new/h_ptmcontro_plot_lepto}
   \label{fig:cplep_subfig8}
 }%
\end{subfloat}
 \begin{subfloat}[]{\includegraphics[width=.32\textwidth,trim={5 0 50 0},clip]{./Figures/control_plots_ev/lep/h_cosgammahadcontro_plot_lepto}
   \label{fig:cplep_subfig9}
 }%
\end{subfloat}
\caption{Comparison of corrected MC (\lepto) and data distributions for event variables after full inclusive-jet selection.}
\label{fig:cp_lepto}
\end{center}
\end{figure}

\newpage
%ARIADNE
\begin{figure}[ht!]
\begin{center}
\begin{subfloat}[]{\includegraphics[width=.32\textwidth,trim={5 0 50 0},clip] {./Figures/control_plots_ev/ari/new/h_Zvtxcontro_plot_ariadne}
   \label{fig:cpari_subfig1}
 }%
\end{subfloat}
 \begin{subfloat}[]{\includegraphics[width=.32\textwidth,trim={5 0 50 0},clip]{./Figures/control_plots_ev/ari/new/h_Ydacontro_plot_ariadne}
   \label{fig:cpari_subfig2}
 }%
\end{subfloat}
\begin{subfloat}[]{\includegraphics[width=.32\textwidth,trim={5 0 50 0},clip]{./Figures/control_plots_ev/ari/new/h_Q2dacontro_plot_ariadne}
   \label{fig:cpari_subfig3}
 }%
\end{subfloat}
\newline
 \begin{subfloat}[]{\includegraphics[width=.32\textwidth,trim={5 0 50 0},clip]{./Figures/control_plots_ev/ari/h_sienecontro_plot_ariadne}
   \label{fig:cpari_subfig4}
 }%
\end{subfloat}
 \begin{subfloat}[]{\includegraphics[width=.32\textwidth,trim={5 0 50 0},clip]{./Figures/control_plots_ev/ari/h_siphcontro_plot_ariadne}
   \label{fig:cpari_subfig5}
 }%
\end{subfloat}
 \begin{subfloat}[]{\includegraphics[width=.32\textwidth,trim={5 0 50 0},clip]{./Figures/control_plots_ev/ari/h_sithcontro_plot_ariadne}
   \label{fig:cpari_subfig6}
 }%
\end{subfloat}
\newline
 \begin{subfloat}[]{\includegraphics[width=.32\textwidth,trim={5 0 50 0},clip]{./Figures/control_plots_ev/ari/new/h_empzcontro_plot_ariadne}
   \label{fig:cpari_subfig7}
 }%
\end{subfloat}
 \begin{subfloat}[]{\includegraphics[width=.32\textwidth,trim={5 0 50 0},clip]{./Figures/control_plots_ev/ari/new/h_ptmcontro_plot_ariadne}
   \label{fig:cpari_subfig8}
 }%
\end{subfloat}
 \begin{subfloat}[]{\includegraphics[width=.32\textwidth,trim={5 0 50 0},clip]{./Figures/control_plots_ev/ari/h_cosgammahadcontro_plot_ariadne}
   \label{fig:cpari_subfig9}
 }%
\end{subfloat}
\caption{Comparison of corrected MC (\ariadne) and data distributions for event variables after full inclusive-jet selection.}
\label{fig:cp_ariadne}
\end{center}
\end{figure}

