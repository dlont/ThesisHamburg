\section{Signal Characteristics}
\label{sec:signalchar}
As can be seen from the Eq.~\eqref{eq:q2el}, large values of \qsq\, in NC DIS events correspond to large scattering angles of the electron. This produces a well isolated electromagnetic shower in the calorimeter with a charged track pointing from the interaction vertex to the electromagnetic cluster (see Figure~\ref{fig:ncdiseventdisplay}). At low and medium values of $\qsq < 500\, \GeV^2$, the electron typically scatters into the RCAL. For larger values of $\qsq$, the electron can be found in BCAL and at very high-$\qsq$ the electron scatters to the FCAL. In NC DIS at \hera the initial-state transverse momentum is zero, therefore final-state transverse momentum calculated from the sum over all energy deposits in the CAL must vanish:
\begin{equation}
P_{T,tot}^2 = P_{X,tot}^2+P_{Y,tot}^2 = \left(\sum_i{E_i\sin{\theta_i}\cos{\phi_i}}\right)^2 + \left(\sum_i{E_i\sin{\theta_i}\sin{\phi_i}}\right)^2 \approx 0\;\GeV^2.
\end{equation}
 Momentum conservation implies that the electron recoils from the hadronic system, which balances the scattered electron in the transverse-momentum plane.
\begin{figure}[htbp]
	\centering
	\includegraphics[width=\textwidth]{./Figures/event_display_53520_93222_jets} 
	\caption{A typical NC DIS event with $E-P_Z=51~\GeV$, $P_T=2~\GeV$ and \qsq = 965~$\GeV^2$~and two jets identified in the Breit frame with \etjetb> 8~\GeV. Red blocks represent energy deposits in the calorimeter. The final-state hadronic jets (wide energy clusters in the BCAL) and the scattered electron are indicated with arrows.}
	\label{fig:ncdiseventdisplay}
\end{figure}

Furthermore, in NC DIS processes the quantity $\delta$, defined as
\begin{equation}
\delta = \delta_h + \delta_e = \left( E_p - P_{z,p} \right) + \left( E_e - P_{z,e}\right) = 2E_e = 55\;\GeV,
\label{eq:deltadis}
\end{equation}
is also conserved\footnote{In the following this is referred as longitudinal momentum balance.}, i.e. the same variable calculated from the final-state energy deposits has to fulfill
\begin{equation}
\delta = \sum_i{\left(E_i - P_{Z,i}\right)} = \sum_i{\left(E_i - E_i\cos{\theta_i}\right)} \approx 55\;\GeV.
\end{equation}
In this equation the photon remnants provide vanishing contribution to $\delta$ since they move approximately parallel to $z$-axis in negative direction and 
\begin{equation}
E^{\gamma\text{-remnant}}-P_{Z}^{\gamma\text{-remnant}} \approx 0\,\GeV.
\end{equation}
Deviations of $P_{T,tot}$ and $\delta$ from the nominal quantity can be caused by e.g. undetected particles escaping the  detector volume and/or due to the finite energy and spatial resolution of the calorimeter.

As was explained in Section~\ref{subsec:breitframe}, at leading order $\left(\mathcal{O}\left(\as\right)\right)$ in the Breit frame a process with hard QCD interaction always has at least two hadronic jets. In experiment, however, one of the jets can fail to pass the selection criteria due to limited detector acceptance and/or resolution, therefore the signal events in this analysis were required to have at least one jet appearing in the Breit frame within the fiducial volume of the detector and having transverse energy exceeding some minimum energy threshold.

\section{Characteristics of Background Processes}
\label{sec:bgchar}

\subsection{Photoproduction}
\label{subsec:photoprodbg}
The cross section of inelastic electron-proton scattering with $\qsq \approx 0\;\GeV^2$ is large and serves as a potential source of background to the NC DIS analysis. At low \qsq, the scattered electron is deflected by a small angle and escapes in the beam pipe. Isolated electromagnetic clusters from e.g. $\pi^0$ decays can be misidentified as the final-state electron thus mimicking the signature of NC DIS processes. In photoproduction, as in NC DIS, the total transverse momentum vanishes but the quantity $\delta$ calculated from the final-state energy deposits will be smaller than 55 \GeV~by approximately twice the energy of the escaped electron.
%\footnote{The energy and momentum of the particles escaping in the forward direction is approximately cancelled.}. 
Besides restrictions on $\delta$, the photoproduction events can be removed by applying additional cuts on isolation of the electron candidate because electromagnetic clusters from hadron decays are often accompanied by other hadrons from the hadronic final state.

\subsection{Beam-Gas Interactions, Cosmics and Halo Muons}
\label{subsec:beamgasfeatures}
The electron and proton beams can interact with the residual gas within the volume enclosed by the beam pipe. The proton-gas collisions are characterised by large hadronic activity and multiple tracks emerging at low polar angles upstream from the nominal interaction point (see Figure~\ref{fig:beamgaseventdisplay}). In such events, signals in the FCAL follow those from the RCAL and are separated by a time interval. Using the information about particles arrival time in forward and rear parts of the CAL as well as signals from the specific iron-scintillator detector (Veto-Wall~\cite{zeus:1993:bluebook}) located in the rear part of the \zeus, these events can be efficiently rejected. In addition, the suppression of beam-gas events can be achieved by imposing restrictions on the longitudinal momentum balance, because such processes are characterised by a lower value of $\delta$ than that for NC DIS events. 

In contrast to the proton-gas collisions, the rate of the electron beam interactions was found in a previous study~\cite{thesis:moritz:2001} to be negligible.

\begin{figure}[ht]
	\centering
	\includegraphics[width=\textwidth]{./Figures/event_display_54628_99024_lessthicklines1} 
	\caption{An example of proton beam -- residual gas collision identified using pilot-bunch crossing number. Many tracks emerge from the interaction point in a $Z$-region corresponding to the rear part of the \zeus detector outside of the acceptance of the tracking system.}
	\label{fig:beamgaseventdisplay}
\end{figure}

\begin{figure}[htbp]
	\centering
	\includegraphics[width=\textwidth]{./Figures/event_display_54450_48590_cosmic_track_thicklines} 
	\caption{A high-energy cosmic muon traversing the the volume of the \zeus detector. Muon track segments are present in the upper and lower halves of the backing calorimeter and muon chambers. Energy clusters from the interaction of the muon with the material of the CAL can mimic signatures of hadronic jets.}
	\label{fig:cosmiceventdisplay}
\end{figure}

\begin{figure}[htbp!]
	\centering
	\includegraphics[width=\textwidth]{./Figures/event_display_54019_64617_beamhalo_overlay1} 
	\caption{A typical DIS event overlaid with a high-energy muon from the beam-gas interaction downstream from the interaction point. A beam halo muon traversing the detector volume in the lower part of the barrel calorimeter.}
	\label{fig:beamhaloeventdisplay}
\end{figure}

Additional sources of background come from cosmic and beam-halo muons. Cosmic muons are produced in high-energy interactions of cosmic rays with the earth's atmosphere. Passage of cosmic muons though the volume of the detector results in a time difference between the signals in the upper and lower parts of the CAL (see Figure~\ref{fig:cosmiceventdisplay}). Such a timing signature and a typically low track multiplicity were used to detect cosmic events. 

In contrast to cosmic rays, beam-halo muons (see Figure~\ref{fig:beamhaloeventdisplay}), emerge from decays of pions produced in beam-gas interactions upstream of the detector and travel parallel to the beam axis, therefore the time difference between the signals in the RCAL and FCAL can be used to eliminate such processes, similarly to beam-gas interactions described above. 

Moreover, cosmic and beam-halo events have non-zero total transverse momentum due to asymmetry of the energy deposits in the transverse plane. This information can also be used for the suppression of these background contributions. \marginpar{OB:Make a statement about final contribution from cosmics and beam halo.DL:I think I have it at the end.\ding{52}}

\subsection{QED Compton scattering}
The reactions of the type $ep \rightarrow e' \gamma p'$ or $ep \rightarrow e'\gamma X$ in which the initial or the final-state electron radiates a high-energy photon are called elastic and inelastic QED Compton (QEDC) scattering, respectively. In case of inelastic reaction the proton breaks up resulting in hadronic activity in the forward direction while in elastic processes the proton escapes down the beam pipe staying intact. 
\begin{figure}[htbp]
	\centering
	\includegraphics[width=\textwidth]{./Figures/event_display_52396_3509_elastic_compton_thicklines} 
	\caption{An elastic QED Compton scattering event as recorded by the \zeus detector. The electromagnetic clusters from the electron and photon (two oppositely charged tracks from $\gamma \rightarrow e^+e^-$ conversion) are detected in the RCAL. The particles balance each other in $P_T$ and there is no energy deposit in the forward direction.}
	\label{fig:qedceventdisplay}
\end{figure}

QEDC events can be misidentified as high-\qsq~NC DIS processes when the electromagnetic cluster from the photon is wrongly reconstructed as the final-state electron that leads to incorrect determination of event kinematics. The characteristic features of such events are the presence of two isolated electromagnetic energy deposits in the calorimeter with approximately equal transverse momentum and low or vanishing hadronic activity in the FCAL (see Figure~\ref{fig:qedceventdisplay}). As was demonstrated in the study~\cite{thesis:moritz:2001} inelastic QEDC processes are well described by the \djangoh MC~\cite{cpc:81:381} while elastic scattering is not well reproduced. Therefore such events have to be removed from the data sample by imposing dedicated cuts (see below).

\section{Event Selection}
The pre-selection of the NC DIS events relevant for the analysis starts naturally during the data-taking phase when the the \zeus trigger system is used. Only those events for which a positive trigger decision was taken are stored on tape and can be further analysed offline. In the following sections general information about analysed data and MC samples as well as selection criteria applied to event samples are described.

\subsection{Data and MC Sets.}
This analysis was performed using the data recorded during the 2004 -- 2007 \hera running period. The $\ep$ centre-of-mass collision energy during this period was \sqs = 318 \GeV. The total integrated luminosity of the processed data sample amounts to 295~\invpb. The luminosity values for the analysed data-taking periods are summarised in Table~\ref{tab:selecteddatasample}.

\begin{table}[htpb!]
	\centering
		\begin{tabular}[h]{c|c}
		  \hline
			Period & Integrated Luminosity \\
			\hline \hline
			2004 -- 2005  & 133.6 \invpb  \\
			2006             & 53.0 \invpb   \\
			2006 -- 2007  & 108.5 \invpb \\
			\hline
		\end{tabular}
	\caption{Luminosity values for the final data samples used for the jet cross section measurements.}
	\label{tab:selecteddatasample}
\end{table}

Monte Carlo simulated events were used for the estimation of detector effects and the size of various contributions not accounted for in the perturbative QCD calculations (see Chapters~\ref{ch:unfolding},~\ref{ch:resultscs}). The MC events for the mentioned data-taking periods were generated separately taking varying experimental conditions into account. The program packages \lepto~\cite{Ingelman:1996mq} and \ariadne~\cite{cpc:71:15,Lonnblad:1994wk}, described in Section~\ref{sec:mcmodels} were utilised for this purpose. In total, about 32 million NC DIS events with $\qsq > 100\;\GeV^2$ were generated, leading to statistical uncertainty in the MC distributions smaller than 1\% across the measured phase space. 

Additional samples of photoproduction events with $\qsq\approx 0~\GeV^2$ generated using \herwig~\cite{cpc:67:465, jhep01:2001:010} and \pythia~\cite{cpc:82:74} programs were also used. These samples were utilised in the jet-photoproduction analysis~\cite{np:b864:1} for the estimation of detector effects. Another sample of simulated events that almost fills the gap in \qsq between the photoproduction and high-\qsq regions was taken from the analysis of~\cite{thesis:perrey:2011} and covers the interval $4~\GeV<\qsq<100~\GeV$. The photoproduction and low-\qsq MC samples were used for the estimation of background contribution (see below).

\section{On-line Selection}
\label{sec:onlineselect}
As described in Section~\ref{sec:daqtrigger} the \zeus trigger has a three-level architecture. At each level, every event can be classified into different categories\footnote{A single event can be attributed to \textbf{several} categories.} according to event characteristics such as total and/or regional energy sums, presence of electromagnetic clusters, etc. Technically, when an event is recorded on storage-tape it is marked with specific flags, the so-called trigger slots or bits, which correspond to different event classes. The requirements imposed by specific trigger bits are listed below. Any of the specified trigger bits was required to be fired\footnote{Which corresponds to a logical \textbf{OR} between individual bits.} in order to keep the event for further analysis. 

\subsection{FLT Trigger}
\label{subsec:fltcuts}
The FLT selection is centred around the idea of having an event with large transverse momentum and significant electromagnetic energy deposit corresponding to the DIS electron. Besides that, every FLT bit imposes additional restrictions (not listed here) on the CAL timing and information from other detector components like Veto-Wall in order to suppress background processes.
\begin{itemize}
	\item \textbf{FLT bit 40:} The total energy in the electromagnetic section of the calorimeter, $E_\text{EMC}^\text{CAL}$, exceeds 20 \GeV\, and a condition on the track multiplicity, the so called \textit{track veto} (see Section~\ref{sec:trkvetoeff}).
	\item \textbf{FLT bit 43:} The total transverse energy in the calorimeter, $E_\text{T}^\text{CAL}$, exceeds 20 \GeV\, and at least one \textit{good track}\footnote{FLT Track class > 1 (see Section~\ref{sec:trkvetoeff}).} is required.
	\item \textbf{FLT bit 50:} The total energy, $E^\text{CAL}$, greater than $15\; \GeV$\, or $E_\text{EMC}^\text{CAL} > 10\; \GeV\,$ or $E_\text{EMC}^\text{BCAL} > 3.5\; \GeV$\, or $E_\text{EMC}^\text{RCAL} > 2\; \GeV$\, and $E_\text{T}^\text{CAL} > 1\; \GeV$\, and a good track is required.
\end{itemize}

\subsection{SLT Trigger}
\label{subsec:sltcuts}
The second-level trigger was used to suppress further beam-related background. The information from the \zeus global tracking trigger was utilised at the SLT for the reconstruction of the interaction vertex position and to reject events originating from background processes. Moreover, the calorimeter timing information was used extensively to suppress beam-gas and cosmic-ray background. For example, hadrons emerging in the vicinity of the nominal interaction point are characterised by approximately the same arrival time in different parts of the CAL, in contrast to cosmic-shower events for which the signals from the upper part of the BCAL will arrive earlier than those from the bottom part. Thus, given the high timing resolution of the CAL, the background processes can be efficiently discriminated at SLT.

The SLT bits DIS01, 04, 07 were used in the analysis. Besides the required FLT bits, the following restrictions were imposed:
\begin{itemize}
	\item \textbf{SLT bit DIS01:} The total energy in the electromagnetic sections of the calorimeter, $E_\text{EMC}^\text{RCAL} > 2.5\;\GeV$ or $E_\text{EMC}^\text{BCAL} > 2.5\;\GeV$~or $E_\text{EMC}^\text{FCAL} > 10\;\GeV$ or hadronic energy in the forward region $E_\text{HAC}^\text{FCAL} > 10\;\GeV$ and $E-P_Z > 29\;\GeV$;
	\item \textbf{SLT bit DIS04:} The total energy in the electromagnetic sections of the calorimeter, $E_\text{EMC}^\text{RCAL} > 2.5\;\GeV$ or $E_\text{EMC}^\text{BCAL} > 2.5\;\GeV$~or $E_\text{EMC}^\text{FCAL} > 10\;\GeV$ or hadronic energy in the forward region $E_\text{HAC}^\text{FCAL} > 10\;\GeV$ and $E-P_Z > 19\;\GeV$;
	\item \textbf{SLT bit DIS07:} The energy of the SLT electron candidate $E_e^{SLT}>5\;\GeV$.
\end{itemize}

\subsection{TLT Trigger}
\label{subsec:tltcuts}
The final stage of the trigger selection was based on the identification of the scattered DIS electron. The reduced read-out rate at the TLT allowed application of the complex reconstruction algorithms much more closely related to those used in the offline analysis. The TLT bit \textsf{DIS03} together with DST bit 12 imposing loose cuts on the electron candidate were used in the selection chain and are summarised in the Table~\ref{tab:TLTDSTreq}, where  $E'_{el}$ denotes the energy of the scattered electron candidate and $R_{el}^{CAL}$ denotes the distance of the centre-of-gravity of the electron cluster to the beam-line in the $XY$ plane.
\begin{table}[ht!]
\centering
\begin{tabular}{c|c}
\hline scattered electron energy & $E'_{el}>4\,\GeV$ \\ 
\hline \pbox{6cm}{distance from the origin of the \\ coordinate system}  & $R_{el}^{CAL}>35\,\cm$ \\ 
\hline longitudinal momentum balance & $30<E-p_Z<100\,\GeV$\\
\hline 
\end{tabular} 
\caption{The requirements imposed on the events at the TLT.}
\label{tab:TLTDSTreq}
\end{table}

\section{Offline Selection}
\label{sec:offlineselect}
The trigger-level event selection cannot fulfill the signal-to-background discrimination requirements necessary for precision analysis because of limited processing time and significant complexity of the reconstruction algorithms for the objects such as secondary vertices, particle decays or jets. In addition, information about detector operating conditions such as the number of dead channels, changes in high voltage, CTD gas pressure, temperature, etc. are difficult or impossible to take into account at the trigger level. Thus, after the trigger-based pre-selection, a set of cuts is applied offline to ensure a low level of background and a high signal purity of the selected sample. Additional requirements are typically imposed on a data sample to restrict the phase space of the measurement to a region of well understood detector acceptance and efficiency.

\subsection{Data-Quality Requirements.}
The offline selection starts with removing runs with inappropriate detector operation. For this purpose during the data-taking the functioning of the detector components was continuously monitored. The status of each of component was stored for every run in the so-called, ``EVTAKE'' flag. In this analysis EVTAKE=1 was required, indicating that all main detector components e.g. CTD, MVD, CAL were fully operational.

In addition to the EVTAKE flag, the LPOLTAKE and TPOLTAKE status records were used in order to verify the availability of the information from the corresponding lepton-beam polarisation detectors. At least one of the detectors was required to have good status in order to determine the polarisation value for particular runs.

\subsection{Electron Selection}
\label{subsec:eleselect}
The next key step in the selection procedure is the electron-based identification of the high-\qsq DIS event. To ensure high purity and reliability of the reconstruction of the scattered electron, the following requirements were imposed on electron candidates:
\begin{itemize}
	\item \textbf{Probability:} The electron identification probability, as given by the SINISTRA algorithm, was required to be greater than $90\%$. If several electron candidates satisfied this criterion, the candidate with the highest probability was used for the reconstruction of event kinematics.
	\item \textbf{Energy:} The electron energy, \eefin was required to be greater than 10 \GeV, to ensure the best electron-candidate reconstruction efficiency and high acceptance. Moreover, this cut helps to reduce the amount of wrongly identified low-energy electron candidates arising due to electromagnetic showers from $\pi^0\rightarrow\gamma\gamma,\, \eta\rightarrow\gamma\gamma$ decays.
	\item \textbf{Isolation:} In order to remove events in which the electromagnetic shower of the electron candidate is contaminated by the energy deposits from hadrons, the fraction of the energy within a cone of radius of 0.7 units in the pseudorapidity-azimuth plane, not associated to the electron, was required to be less than $10\%$. The cone axis was defined by the electron momentum direction.
	\item \textbf{Track Matching:} The tracking system can be used to validate the electron identification, because electromagnetic clusters within the acceptance of the tracking system\footnote{The tracking system covers the region of polar angles restricted to $0.3 < \theta_\text{e} < 2.85$.} that have no matching track are most likely photons. Moreover, the tracking information can be used to determine much more precisely the polar angle of the scattered electron than the determination from the electron cluster and primary vertex position only. Thus, if the electron candidate was within the tracking system acceptance region, the presence of a matched track was imposed. This track was required to 
	%The angular resolution of the tracking detectors is typically much better than that of the calorimeter, therefore precision of the position of the electromagnetic shower can be improved, when the information from the tracking detectors is used. The charged track pointing to the electromagnetic shower was required to
	have a distance of closest approach between the track extrapolation point at the front surface of the CAL and the cluster centre-of-gravity-position of less than 10 $\cm^2$. The track energy as measured by the tracking system had to be greater than 3 \GeV, taking into account energy losses by bremsstrahlung. In case the electron track was outside the acceptance region of the tracking detectors, the information from the calorimeter system was used to determine the position of the electron candidate.
	\item \textbf{Position:} To ensure full containment of the electromagnetic shower inside the fiducial volume of the calorimeter system and to avoid regions with insufficient description by the MC simulations, additional requirements on the position of the electromagnetic shower were imposed. The events in which the electron was found in the following regions were rejected:
	\begin{itemize}
		\item $ -104\, \cm < Z_\text{e} < -98.5\, \cm	$ and $ 164\, \cm < Z_\text{e} < 174\, \cm $, where $Z_\text{e}$ is the longitudinal coordinate of the centre of gravity of the of the electromagnetic cluster attributed to the electron --- the so called ``super-crack'' regions between the BCAL and the RCAL or between the BCAL and the FCAL;
		\item $\left| X_\text{e} \right|$ < 10 \cm\, and $Y_\text{e}$ > 80 \cm, where $X_\text{e}$ and $Y_\text{e}$ are the coordinates of the position of the electron energy deposit in the CAL. In this region some of the calorimeter cells were removed to make room for the  pipes transporting  liquid helium to the superconducting solenoid;
		\item $ 36\, \cm < R^\text{RCAL}_\text{e} < 170\, \cm $, where $R^\text{RCAL}_\text{e}$ is the distance in the $X-Y$ plane from the origin of the \zeus CS to the electron. The leakage of the electromagnetic showers from the electrons hitting the RCAL close to the inner or outer edges is not well simulated, especially at the trigger level~\cite{thesis:januschek:2011}. These regions were therefore excluded. 
	\end{itemize}
\end{itemize}

\subsection{Primary Vertex Selection}
\label{subsec:vtxselect}
Proper identification of the interaction point is important for the reconstruction of the kinematic variables. The longitudinal extent of the interaction region is determined by the length of the interacting bunches and the time structure of the beam. Beam-gas or cosmic-ray events are approximately evenly distributed along the longitudinal coordinate, while \ep\, events appear with higher rate in the vicinity of the nominal beam-beam interaction point. The distribution of the longitudinal component of the position of the primary vertex has a Gaussian-like shape (see Section~\ref{sec:zvtxrew}). Selecting the events with the primary interaction vertex fit satisfying $\chi^2 < 10$\marginpar{OB:explain $\chi^2$.\\DL:has to clarify.} and $Z_\text{vtx}$ being within $\sim \pm 3\sigma$ of the width of the distributions suppresses the non-physics background contribution and ensures good understanding of the dependence of the acceptance of the calorimeter and tracking systems on the position of the interaction vertex. The mean value and the width of the $Z_\text{vtx}$ changes between different data-taking periods. The final restrictions imposed on $Z_\text{vtx}$ are detailed in Table~\ref{tab:zvxcut}.
\begin{table}[htbp]
	\centering
		\begin{tabular}{|c|c|}
			\hline
			Data-taking perod & imposed cut \\
			\hline
			\hline
			2004-2005 $e^{-}$ & $-32\,\cm<Z_\text{vtx}<30.1\,\cm$ \\
			2006-2007 $e^{+}$ & $-28.5\,\cm<Z_\text{vtx}<26.7\,\cm$ \\
			\hline
		\end{tabular}
	\caption{Requirements imposed on the longitudinal cordinate of the position of the interaction vertex.}
	\label{tab:zvxcut}
\end{table}

\subsection{Longitudinal Momentum Balance}
\label{subsec:empzcut}
As discussed previously in Section~\ref{subsec:beamgasfeatures}, longitudinal momentum balance can be used to discriminate against photoproduction and beam-gas background. As mentioned, the actual value of quantity $\delta=\sum_i{\left(E_i-p_{z,i}\right)}$ for a particular event may deviate from the nominal $\delta=55\;\GeV$ due to finite energy resolution and/or ISR effects\footnote{ISR leads to smaller values of $\delta$.}. Therefore the following requirement was implied on $\delta$:
\begin{equation}
38 < \delta < 65\;\GeV.
\label{eq:epmzcut}
\end{equation}
The lower value was chosen in order to reject photoproduction and beam-gas events, while the upper cut is required for the suppression of events with significant energy deposits in the rear direction, e.g. due to backspash processes that are poorly simulated in MC.

\subsection{Transverse Momentum Balance}
\label{subsec:empzcut}
Due to finite resolution, the total transverse momentum of an NC DIS event may be greater that zero. The energy resolution of the calorimeter scales approximately as $1/\sqrt{E}$. In order to suppress beam-gas-related, cosmic-ray and charged current processes (see Section~\ref{sec:kindis}) with a misidentified electron, the following ratio was required to be small:
\begin{equation}
\frac{p_T}{\sqrt{E_T}} < 2.5\;\sqrt{\GeV}.
\label{eq:ptcut}
\end{equation}

\subsection{Event Inelasticity}
\label{subsec:yelcut}

The DST bit 12 has a requirement on the inelasticity of the event. As was mentioned, the online information at the trigger level can be less precise than that available offline. In order to have the offline selection consistent with the trigger requirements, an additional cut 
\begin{equation}
y_{el} < 0.75
\label{eq:yelcut}
\end{equation}
was imposed. Besides other restrictions listed in Table~\ref{tab:TLTDSTreq}, this particular requirement was included into the definition of the DST bit 12 in order to rejects photoproduction background more efficiently, but since the measurement is performed in a much more restricted phase space (see Section~\ref{subsec:phasespace}), the effect of this offline cut was found to be very small.

\subsection{Elastic QED-Compton}
\label{subsec:elasticqedcut}
As described above, elastic Compton scattering processes ($ep \rightarrow ep\gamma$) are not well described by Monte Carlo simulations. Therefore, events containing two electron candidates satisfying the requirements listed in Section~\ref{subsec:eleselect} and having transverse-momentum vectors opposite to each other, $\left|\phi^1 - \phi^2\right| > 3$, the ratio of transverse momenta, $0.8 < P_{T}^1/P_T^{2} < 1.2$, and the total event-energy sum, excluding the contribution from the two electron clusters, $E_T^{CAL} < 3 \GeV$, were rejected.

\subsection{Higher-Order QED Predictions}
\label{subsec:qedcorcut}
In order to ensure the validity of higher-order QED corrections (see Section~\ref{sec:qedcor}) the region of phase space given by the requirement 
\begin{equation}
y_{JB}\cdot\left(1-x_{DA}\right)^2>0.004
\end{equation}
was excluded because Monte Carlo predictions were found to be unreliable in this region~\cite{cpc:81:381}.

\subsection{Hadronic Scattering Angle}
\label{subsec:gammahadcut}
In order to suppress events with large hadronic activity in the forward region, where the simulations were found to be inaccurate~\cite{thesis:jose:2003}, a cut on the projection of the hadronic scattering angle in the FCAL $R^\text{FCAL}_{\gamma^{had}} > 18\;\cm$ was implied. This constraint had only a marginal effect because inelasticity requirements suppress events with large hadronic activity in the FCAL.

\subsection{Track Multiplicity}
\label{subsec:trackmultcut}
The residual beam-gas contamination was minimised by requiring the presence of at least one track, which crossed a minimum of three CTD superlayers and had transverse momentum $p_T>0.2\;\GeV$. It was found that this cut had only a minor effect on the signal acceptance, as expected, since events with jets composed solely of neutral particles are very unlikely.

\subsection{Phase Space}
\label{subsec:phasespace}
Phase-space cuts are performed to select the kinematic region of the reaction of interest. As mentioned earlier, two variables completely determine the kinematics of deep inelastic scattering. In this analysis, the exchanged boson virtuality \qsq~and inelasticity \y~were used for the NC DIS phase space definition.
\begin{itemize}
	\item \textbf{Photon virtuality}, $125 < \qsq_{DA} < 20000\;\GeV^2$;  deep inelastic scattering processes with large four-momentum transfer were selected.
	\item \textbf{Inelasticity}, $0.2 < \y_{DA} < 0.6$; the lower cut was imposed to reject events with large hadronic activity in the forward direction for which hadronisation corrections were found to be inaccurate. The upper cut was applied to ensure reliability of the detector acceptance corrections, since the region $\y_{DA} > 0.6$ was poorly described in the MC~\cite{thesis:behr:2010}.
\end{itemize}

\subsection{Jet Selection}
\label{subsec:jetselect}
In order to identify jets, the \kt~clustering-algorithm in the longitudinally invariant inclusive mode was used. The jet search was performed in the Breit frame, therefore the momentum vectors corresponding to the energy deposits in the CAL were transformed accordingly. 
%The momentum vectors were constructed from the energy and orientation of the calorimeter-cell centre with respect to the interaction point. By definition the constructed vectors are massless. All cells excluding those corresponding to the electron candidate and satisfying the following requirements were used for the jet reconstruction:
%\begin{itemize}
%	\item the signals from natural Uranium radioactivity were suppressed by requiring the minimum energy in the cell to be above a threshold;
%	\item the cut on imbalance between two photomultiplier signals attributed to a single cell was imposed in order to reject signals due to spontaneous discharge of one of the PMT bases.
%\end{itemize}

The phase space for jet production was limited by the following requirements:
\begin{itemize}
	\item \textbf{Transverse energy} of the jets in the Breit frame was required to be greater than 8~\GeV. Such a relatively high energy-threshold was chosen to ensure applicability of perturbative QCD, statistical significance of the event sample and high purity of the jet signal.
	\item \textbf{Pseudorapidity} of the jets had to be in the region $-1<\etajetlab<2.5$. The lower cut corresponds approximately to the transition region between the barrel and rear parts of the CAL; since the RCAL has only one HAC section (see Section~\ref{subsec:UCAL}), the high-energy hadronic showers cannot be fully absorbed, leading to the increase of the absolute energy-scale uncertainty of the jets in that region. The upper $\etajetlab$ cut was motivated by the trigger limitations since events with large hadronic activity in the forward direction occur at a rate that exceeds the capabilities of the trigger.
	
\end{itemize}
The purity of the jet sample was enhanced by applying the following jet-cleaning cuts:

\begin{itemize}
	\item good isolation of the electron candidate from hadronic jets was ensured by requiring the distance between electron and jet in the pseudorapidity-azimuth plane to be $\Delta R > 1$;
	\item initial-state electron radiation hitting the RCAL was often reconstructed as a jet. Such events were removed from the sample if a jet with $\etjetb > 5\;\GeV$ and $\etajetlab < -1$ was identified;
	\item jets with low transverse energy in the laboratory frame have large relative energy-scale uncertainty, therefore jets with $\etjetlab < 3\;\GeV$ were excluded.
\end{itemize}

\section{Final Event Sample}
\label{sec:eventsampletab}
Summarising, the imposed requirements restrict the kinematic phase space of the measurements to
\begin{gather}
125 < \qsq < 20000\;\GeV^2 \qquad 0.2 < y < 0.6, \label{eq:kinematicphasespacecuts}\\
\etjetb > 8\;\GeV \qquad -1 < \etajetlab < 2.5. \label{eq:jetphasespacecuts}
\end{gather}
%The information about selected event samples is summarised in Table~\ref{tab:numevents}.
%\begin{table}[htbp]
%	\centering
%		\begin{tabular}{|c|c|}
%			\hline
%			Sample & Number of events \\
%			\hline
%			\hline
%			Data 2004-2005 $e^{-}$ & ??? \\
%			Data 2006-2007 $e^{+}$ & ??? \\
%			Lepto MC $e^{+}$ & ??? \\
%			Ariadne MC $e^{+}$ & ??? \\
%			\hline
%		\end{tabular}
%	\caption{Number of selected events in data and MC simulations.}
%	\label{tab:numevents}
%\end{table}
A comparison of the data and \ariadne (\lepto) MC distributions for the most important event observables after full selection is demonstrated in Figures~\ref{fig:cp_lepto} and~\ref{fig:cp_ariadne}, respectively. The presented plots include the distribution of the reconstructed longitudinal coordinate of the primary vertex, $Z_\text{vtx}$; kinematic observables: $\y_\text{DA}$, $\qsq_\text{DA}$; energy, azimuthal and polar angle distributions for the NC DIS electron candidate as well as the polar angle of the hadronic final state, $\cos{\gamma_\text{had}}$; observables characterising momentum balance: $p_T/\sqrt{E_T}$, $\left(E-P_Z\right)$. In all figures the bin content of the MC distributions was rescaled to the number of events in the data sample by applying a constant factor:
\begin{equation}
\mathcal{R} = \frac{N_\text{data}}{N_\text{MC}},
\label{eq:mcrescale}
\end{equation}
where $N_\text{data}$ and $N_\text{MC}$ are the number of events in the corresponding samples. 

The primary vertex distribution (Figures~\ref{fig:cp_lepto}\subref{fig:cplep_subfig1},~\ref{fig:cp_ariadne}\subref{fig:cpari_subfig1}) has a proper Gaussian-like shape reflecting the charge distribution and the timing structure of colliding bunches, explained in Section~\ref{subsec:beamstruct}. Distributions of $\y_\text{DA}$and $\qsq_\text{DA}$ (Figures~\ref{fig:cp_lepto}\subref{fig:cplep_subfig2},\subref{fig:cplep_subfig3},~\ref{fig:cp_ariadne}\subref{fig:cpari_subfig2},\subref{fig:cpari_subfig3}) approximatively demonstrate the behaviour predicted by Eq.~\eqref{eq:ddifDIS}. In the phase space of the measurement, the total NC DIS cross section is dominated by the $F_2$ term, thus neglecting $F_{L,3}$ and, in addition, small $y^2F_2$ terms, an approximately linear dependence on $y$ is obtained; the steeply falling nature of the \qsq distribution is related to the $1/Q^2$ and $\left(Q^2 + M_Z^2\right)^{-1}$ nature of the photon and \zn propagators, respectively. Distributions of electron variables (Figures~\ref{fig:cp_lepto}\subref{fig:cplep_subfig4}--\subref{fig:cplep_subfig6},~\ref{fig:cp_ariadne}\subref{fig:cpari_subfig4}--\subref{fig:cpari_subfig6}) on one hand reflect event kinematics related to \qsq and \y distributions and on the other hand the performance of electron identification and reconstruction. In particular, the flatness of the $\phi_{ele}$ distribution is a manifestation of the azimuthal symmetry of NC DIS scattering and the drop around $\phi_{ele}=\pi/2$ is attributed to low efficiency of the uninstrumented region of the RCAL, described in Section~\ref{subsec:eleselect}. The peak of the $\left(E-P_Z\right)$ distribution (Figures~\ref{fig:cp_lepto}\subref{fig:cplep_subfig7},~\ref{fig:cp_ariadne}\subref{fig:cpari_subfig7}) is located at the value $\delta \approx 55 \GeV$ predicted by longitudinal momentum conservation (see Section~\ref{sec:signalchar}). The origin of the observed discrepancy in the description of the data by MC simulations is unclear and was taken into account in the systematic uncertainty (see Section~\ref{subsec:systunc}) The distributions of missing transverse momentum and $\cos{\gamma_\text{had}}$ (Figures~\ref{fig:cp_lepto}\subref{fig:cplep_subfig8},\subref{fig:cplep_subfig9} and \ref{fig:cp_ariadne}\subref{fig:cpari_subfig8},\subref{fig:cpari_subfig9}) are a convolution of event kinematics and detector effects.

In general, the observed distributions are very well described by the Monte Carlo simulations when all corrections and reweightings (see Chapter~\ref{ch:calibcorr}) are applied to the data and MC, thus verifying the accuracy of the determination of the detector acceptance (Section~\ref{sec:acccor}) and systematic effects (Section~\ref{subsec:systunc}) using the MC samples.

The residual background contribution from beam-gas interactions was estimated by applying the described selection procedure to events with a proton bunch crossing only. In this pilot-bunch sample, only a small number of events persisted after the selection. The visual inspection of such collisions revealed that selected events were typical NC DIS reactions with hard hadronic jets. It was concluded that for some runs the bunch crossing number was set incorrectly during the data-taking\footnote{No bunch crossing number requirements were applied for selections of the final sample.}. Similarly, full off-line selection chain was applied to events identified at the trigger level as cosmic. No events from this sample survived final selection, thus the conclusion, that the background due to cosmic showers can be neglected, was drawn.

The photoproduction and low-\qsq NC DIS contamination was investigated utilising MC samples prepared for dedicated studies of jet production at low and very low exchanged boson virtualities. The estimated contribution of these events to the final jet sample is illustrated in Figure~\ref{fig:cp_bg}. The photoproduction component estimate varies by a factor of two depending on the generator used. The \pythia MC predicts a cross section approximately twice as large as that given by \herwig for the kinematic region $\qsq < 1~\GeV^2$. Due to the steeply falling nature of the \qsq distribution, the NC DIS cross section obtained using \lepto for a much wider region $4< \qsq < 125~\GeV^2$ appeared to be approximately of the same magnitude as for the $\qsq < 1~\GeV^2$ region. As demonstrated in Figures~\ref{fig:cp_bg}(a, d) the background populates the low-\qsq and low-$E_T$ region of phase space and arises mostly due to incorrect identification of the DIS electron candidate. In total, the contamination from events with $\qsq < 125~\GeV^2$ to the measurement sample amounted to less than 0.2\% and was neglected.
%LEPTO
% % % % % % % % % % % % % % % % EVENT
\begin{figure}[ht!]
\begin{center}
\begin{subfloat}[]{\includegraphics[width=.32\textwidth,clip] {./Figures/bg/h_Q2da_bgr_contro_plot_cs_lepto}
   \label{fig:cpbg_subfig1}
 }%
\end{subfloat}
 \begin{subfloat}[]{\includegraphics[width=.32\textwidth,clip]{./Figures/bg/h_Yda_bgr_contro_plot_cs_lepto}
   \label{fig:cpbg_subfig2}
 }%
\end{subfloat}
\begin{subfloat}[]{\includegraphics[width=.32\textwidth,clip] {./Figures/bg/h_empz_bgr_contro_plot_cs_lepto}
   \label{fig:cpbg_subfig3}
 }%
\end{subfloat}
\newline
 \begin{subfloat}[]{\includegraphics[width=.32\textwidth,clip]{./Figures/bg/h_calet_bgr_contro_plot_cs_lepto}
   \label{fig:cpbg_subfig4}
 }%
\end{subfloat}
 \begin{subfloat}[]{\includegraphics[width=.32\textwidth,clip]{./Figures/bg/h_siene_bgr_contro_plot_cs_lepto}
   \label{fig:cpbg_subfig5}
 }%
\end{subfloat}
 \begin{subfloat}[]{\includegraphics[width=.32\textwidth,clip]{./Figures/bg/h_sith_bgr_contro_plot_cs_lepto}
   \label{fig:cpbg_subfig6}
 }%
\end{subfloat}

\caption{Comparison of corrected MC (\lepto) and data distributions for event variables after full inclusive-jet selection. Various background components estimated using MC simulations are also shown.}
\label{fig:cp_bg}
\end{center}
\end{figure}

%LEPTO
% % % % % % % % % % % % % % % % EVENT
\begin{figure}[ht!]
\begin{center}
\begin{subfloat}[]{\includegraphics[width=.32\textwidth,trim={5 0 50 0},clip] {./Figures/control_plots_ev/lep/new/h_Zvtxcontro_plot_lepto}
   \label{fig:cplep_subfig1}
 }%
\end{subfloat}
 \begin{subfloat}[]{\includegraphics[width=.32\textwidth,trim={5 0 50 0},clip]{./Figures/control_plots_ev/lep/new/h_Ydacontro_plot_lepto}
   \label{fig:cplep_subfig2}
 }%
\end{subfloat}
\begin{subfloat}[]{\includegraphics[width=.32\textwidth,trim={5 0 50 0},clip] {./Figures/control_plots_ev/lep/new/h_Q2dacontro_plot_lepto}
   \label{fig:cplep_subfig3}
 }%
\end{subfloat}
\newline
 \begin{subfloat}[]{\includegraphics[width=.32\textwidth,trim={5 0 50 0},clip]{./Figures/control_plots_ev/lep/h_sienecontro_plot_lepto}
   \label{fig:cplep_subfig4}
 }%
\end{subfloat}
 \begin{subfloat}[]{\includegraphics[width=.32\textwidth,trim={5 0 50 0},clip]{./Figures/control_plots_ev/lep/h_siphcontro_plot_lepto}
   \label{fig:cplep_subfig5}
 }%
\end{subfloat}
 \begin{subfloat}[]{\includegraphics[width=.32\textwidth,trim={5 0 50 0},clip]{./Figures/control_plots_ev/lep/h_sithcontro_plot_lepto}
   \label{fig:cplep_subfig6}
 }%
\end{subfloat}
\newline
 \begin{subfloat}[]{\includegraphics[width=.32\textwidth,trim={5 0 50 0},clip]{./Figures/control_plots_ev/lep/new/h_empzcontro_plot_lepto}
   \label{fig:cplep_subfig7}
 }%
\end{subfloat}
 \begin{subfloat}[]{\includegraphics[width=.32\textwidth,trim={5 0 50 0},clip]{./Figures/control_plots_ev/lep/new/h_ptmcontro_plot_lepto}
   \label{fig:cplep_subfig8}
 }%
\end{subfloat}
 \begin{subfloat}[]{\includegraphics[width=.32\textwidth,trim={5 0 50 0},clip]{./Figures/control_plots_ev/lep/h_cosgammahadcontro_plot_lepto}
   \label{fig:cplep_subfig9}
 }%
\end{subfloat}
\caption{Comparison of corrected MC (\lepto) and data distributions for event variables after full inclusive-jet selection.}
\label{fig:cp_lepto}
\end{center}
\end{figure}

\newpage
%ARIADNE
\begin{figure}[ht!]
\begin{center}
\begin{subfloat}[]{\includegraphics[width=.32\textwidth,trim={5 0 50 0},clip] {./Figures/control_plots_ev/ari/new/h_Zvtxcontro_plot_ariadne}
   \label{fig:cpari_subfig1}
 }%
\end{subfloat}
 \begin{subfloat}[]{\includegraphics[width=.32\textwidth,trim={5 0 50 0},clip]{./Figures/control_plots_ev/ari/new/h_Ydacontro_plot_ariadne}
   \label{fig:cpari_subfig2}
 }%
\end{subfloat}
\begin{subfloat}[]{\includegraphics[width=.32\textwidth,trim={5 0 50 0},clip]{./Figures/control_plots_ev/ari/new/h_Q2dacontro_plot_ariadne}
   \label{fig:cpari_subfig3}
 }%
\end{subfloat}
\newline
 \begin{subfloat}[]{\includegraphics[width=.32\textwidth,trim={5 0 50 0},clip]{./Figures/control_plots_ev/ari/h_sienecontro_plot_ariadne}
   \label{fig:cpari_subfig4}
 }%
\end{subfloat}
 \begin{subfloat}[]{\includegraphics[width=.32\textwidth,trim={5 0 50 0},clip]{./Figures/control_plots_ev/ari/h_siphcontro_plot_ariadne}
   \label{fig:cpari_subfig5}
 }%
\end{subfloat}
 \begin{subfloat}[]{\includegraphics[width=.32\textwidth,trim={5 0 50 0},clip]{./Figures/control_plots_ev/ari/h_sithcontro_plot_ariadne}
   \label{fig:cpari_subfig6}
 }%
\end{subfloat}
\newline
 \begin{subfloat}[]{\includegraphics[width=.32\textwidth,trim={5 0 50 0},clip]{./Figures/control_plots_ev/ari/new/h_empzcontro_plot_ariadne}
   \label{fig:cpari_subfig7}
 }%
\end{subfloat}
 \begin{subfloat}[]{\includegraphics[width=.32\textwidth,trim={5 0 50 0},clip]{./Figures/control_plots_ev/ari/new/h_ptmcontro_plot_ariadne}
   \label{fig:cpari_subfig8}
 }%
\end{subfloat}
 \begin{subfloat}[]{\includegraphics[width=.32\textwidth,trim={5 0 50 0},clip]{./Figures/control_plots_ev/ari/h_cosgammahadcontro_plot_ariadne}
   \label{fig:cpari_subfig9}
 }%
\end{subfloat}
\caption{Comparison of corrected MC (\ariadne) and data distributions for event variables after full inclusive-jet selection.}
\label{fig:cp_ariadne}
\end{center}
\end{figure}

