The extraction of signal events is based on high-\qsq\, NC DIS selection. It starts with online pre-selection using the three-level trigger system, then the selection proceeds with identifying well reconstructed high-\qsq\, NC DIS events. It finishes with identifying events with hard jets within the selected NC DIS sample. This section describes the characteristic features of the NC DIS events containing hard jets. The background sources and selection requirements, applied to the data and MC simulations, are detailed.

This analysis was performed using the data recorded during the 2004 -- 2007 running period. The $\ep$ centre-of-mass collision energy during this period was \sqs = 318 \GeV. The integrated luminosity of the measurement amounts to 295 \invpb.

\subsection{Signal Characteristics}
\label{subsec:signalchar}
As can be seen from the Eq.~\ref{eq:q2el}, large values of \qsq\, in NC DIS events correspond to large scattering angles of the electron. The electron produces a well isolated electromagnetic shower in the calorimeter with a charged track pointing to the interaction vertex (see Figure~\ref{fig:ncdiseventdisplay}). At low and medium values of $\qsq < 500\, \GeV^2$ the electron typically scatters to the RCAL. For larger values of $\qsq$ electron can be found in BCAL and at very high-$\qsq$ electron scatters to the FCAL. According to the momentum conservation the electron recoils from the hadronic system, which balances the scattered electron in the transverse plane.
\begin{figure}[htbp]
	\centering
	\includegraphics[width=\textwidth]{./Figures/dis} 
	\caption{A typical NC DIS event as recorded by the \zeus detector. (a) The x-y view. (b) The z-r view. Red blocks represents energy deposits in the calorimeter. The final state jets and the scattered electron are indicated with arrows.}
	\label{fig:ncdiseventdisplay}
\end{figure}
In NC DIS processes the quantity $\delta$ defined as
\begin{equation}
\delta = \delta_h + \delta_e = \left( E_p - P_{z,p} \right) - \left( E_e - P_{z,e}\right) = 2E_e = 55\;\GeV,
\label{eq:deltadis}
\end{equation}
must be conserved. Similarly, the total transverse momentum of the event must vanish, because the initial-state transverse momentum is zero.

\subsection{Background Characteristics}
\label{subsec:bgchar}

\subsubsection{Photoproduction}
\label{subsubsec:photoprodbg}
The cross section of inelastic electron-proton scattering with $\qsq \approx 0\;\GeV^2$ is large and serves as potential source of backgound to NC DIS analysis. At low \qsq the scattered electron is deflected by a small angle and escapes in the beampipe. Isolated electromagnetic clusters from $\pi^0$ decays can be misidentified as the final-state electron. The variable $\delta$ calculated from the final-state energy deposits will be smaller than 55 \GeV~by approximately two time energy of the escaped electron\footnote{The energy and momentum of the particles escaping in the forward direction is approximately cancelled.}.

\subsubsection{Beam-Gas interactions. Cosmic and Halo Muons}
\label{subsubsec:beamgasfeatures}
The electron and proton beams can interact with the residual gas in the beam pipe. The proton collisions are characterised by large hadronic activity and big number of tracks emerging at low polar angles downstream from the nominal interaction point. Using the calorimeter timing information as well as signals from Veto-Wall such events can be successfully rejected. Additionally, such processes are characterised by low values of $\delta$. The rate of the electron beam interactions was found in a previous study~\cite{thesis:moritz:2001} to be very small and can be ignored.

Additional source of background comes from cosmic and beam-halo muons. Cosmic muons are produced in high-energy cosmic rays. Passage of muons though the volume of the detector results in time difference between the signals in upper and lower parts of the CAL. Beam-halo muons, emerging from decays of pions produced in beam gas interactions upstream the detector, travel parallel to the beam axis. The time difference between signals in the RCAL and FCAL is used to eliminate such processes. In general, comic and beam-halo events have non-zero total transverse momentum due to imbalance of the energy deposits in the transverse plane. This information can used for background suppression. The detailed description of event selection is presented in the following sections.

\subsection{On-line selection}
\label{subsec:onlineselect}
The operation of the detector components was continuously monitored during the data-taking. The status of each of the detector components was stored for every run in the so-called, ``EVTAKE'' flag. For this analysis the EVTAKE=1 was required indicating that all main detector components e.g. CTD, MVD, CAL were fully operational.

In addition to EVTAKE flag, the LPOLTAKE and TPOLTAKE status records were used in this analysis in order to verify the availability of the information from the corresponding lepton-beam polarisation detectors. At least one of the detectors was required to be in good status on in order to determine the polarisation value for particular runs.
\subsubsection{FLT trigger}
\label{subsubsec:fltcuts}
\begin{itemize}
	\item \textbf{FLT bit 40:} A total energy in the electromagnetic section of the calorimeter, $E_\text{EMC}^\text{CAL}$, exceeds 20 \GeV\, and track veto.
	\item \textbf{FLT bit 43:} A total transverse energy in the calorimeter, $E_\text{T}^\text{CAL}$, exceeds 20 \GeV\, and good track.
	\item \textbf{FLT bit 50:} A total energy, $E^\text{CAL}$, greater than $15\, \GeV$\, or $E_\text{EMC}^\text{CAL} > 10\, \GeV\,$ or $E_\text{EMC}^\text{BCAL} > 3.5\, \GeV$\, or $E_\text{EMC}^\text{RCAL} > 2\, \GeV$\, and $E_\text{T}^\text{CAL} > 1\, \GeV$\, and good track.
\end{itemize}

\subsubsection{SLT trigger}
\label{subsubsec:sltcuts}
The second-level trigger was used to suppress further the beam-related background. The information from the \zeus global tracking trigger was utilised at SLT for the reconstruction of the interaction vertex position and to reject events originating from background processes. Moreover, the calorimeter timing information was used extensively to suppress beam-gas and cosmic-ray background. For example, hadrons emerging in vicinity of the nominal interaction point will be characterised by approximately the same arrival time in different parts of the CAL, in contrast to cosmic-shower events for which the signals from the upper part of the BCAL will arrive earlier than from the bottom part. Thus, given the high timing resolution of the CAL, the background processes can be efficiently discriminated.

\subsubsection{TLT trigger}
\label{subsubsec:tltcuts}
The final stage of the trigger selection was based on the identification of the scattered DIS electron. The reduced read-out rate at the TLT allowed application of the complex reconstruction algorithms that were much closer related to those used in the off-line analysis. The TLT bit \textsf{DIS03} together with DST bit 12 were used in the selection chain and impose loose cuts summarised in the Table~\ref{tab:TLTDSTreq}.
\begin{table}
\centering
\begin{tabular}{|c|c|}
\hline scattered electron energy & $E'_{el}>4\,\GeV$ \\ 
\hline distance from the beam pipe  & $R_{el}^{CAL}>35\,\cm$ \\ 
\hline longitudinal momentum balance & $30<E-p_Z<100\,\GeV$\\
\hline 
\end{tabular} 
\caption{The requirements imposed on the events at the TLT.}
\label{tab:TLTDSTreq}
\end{table}

\subsection{Off-line selection}
\label{subsec:offlineselect}

After the trigger-based pre-selection a set of cuts is applied off-line to ensure low level of background and high purity of selected sample. Additional requirements imposed to restrict the phase space of the measurement. 

\subsubsection{Phase space}
\label{subsubsec:phasespace}
The phase space cuts are performed for selection of the kinematic region of the reaction in question. As was mentioned earlier, two variables completely determine the kinematics of deep inelastic scattering. In this analysis exchange-boson virtuality \qsq and inelasticity \y were used.
\begin{itemize}
	\item \textbf{Photon virtuality}, $125 < \qsq_{DA} < 20000\;\GeV$;  deep inelastic scattering processes with large four-momentum transfer are selected.
	\item \textbf{Inelasticity}, $0.2 < \y_{DA} < 0.6\;\GeV$; the lower cut was imposed for rejection of events with large hadronic activity in the forward direction for which hadronisation corrections were found to be inaccurate. The upper cut was applied to ensure reliability of the detector acceptance corrections.
\end{itemize}

\subsubsection{Electron selection}
\label{subsubsec:eleselect}

To ensure high purity and reliability of the reconstruction of the scattered electron following requirements were imposed on electron candidates.
\begin{itemize}
	\item \textbf{Probability:} The probability, as given by the SINISTRA algorithm, was required to be greater than $90\%$. In case, when several electron candidates satisfied this criterion, the candidate with the highest probability was used for the reconstruction of event kinematics.
	\item \textbf{Energy:} The electron energy, \eefin was required to be greater than 10 \GeV, because low energy electron candidates appear mostly from misidentified electromagnetic showers from $\pi^0\rightarrow\gamma\gamma,\, \eta\rightarrow\gamma\gamma$ decays. In addition, this cut restricts the electron-finding algorithm to the region with high electron-identification efficiency.
	\item \textbf{Isolation:} In order to remove events in which electron-candidate electromagnetic shower is contaminated by the energy deposits from hadronic track, the fraction of the energy within a cone of radius of 0.7 units in pseudorapidity-azimuth plane, not associated to the electron, was required to be less than $10\%$. The cone axis was defined by the electron momentum direction.
	\item \textbf{Track Matching:} Angular resolution of the tracking detectors is typically much better than that of the calorimeter, therefore precision of the position of the electromagnetic shower can be improved, when the information from the tracking detectors is used. The charged track pointing to the electromagnetic shower was required to have the distance of closet approach between the track extrapolation point at front surface of the CAL and the energy-cluster position was required to be less than 10 cm. The track energy as measured by the tracking system had to be greater than 3 \GeV. In case, when the electon track was outside the acceptance region\footnote{The tracking system covers the region of polar angles restricted to $0.3 < \theta_\text{e} < 2.85$. } of the tracking detectors the information from the calorimeter system was used.
	\item \textbf{Position:} To ensure full containment of the electromagnetic shower inside the fiducial volume of the calorimeter system and to avoid regions with insufficiently good description by the MC simulations, additional requirements on the position of the electromagnetic shower were imposed. The events in which the electron was found in the following regions were rejected:
	\begin{itemize}
		\item $ -104\, \cm < Z_\text{e} < -98.5\, \cm	$ and $ 164\, \cm < Z_\text{e} < 174\, \cm $ - the so called ``super-crack'' regions between the BCAL and the RCAL or between the BCAL and the FCAL;
		\item $\left| X_\text{e} \right|$ < 10 \cm\, and $Y_\text{e}$ > 80 \cm. In this region some of the calorimeter cells were removed for the  pipes transporting  liquid helium to the superconducting solenoid;
		\item $ 36\, \cm < R^\text{RCAL}_\text{e} < 170\, \cm $, where $R^\text{RCAL}_\text{e}$ is the distance in the $X-Y$ plane from the centre of the beam-pipe to the electron. The leakage of the electromagnetic showers from the electrons hitting the RCAL close to the inner or outer edges is not well simulated, especially at the trigger level~\cite{januschek-p96}. These regions were therefore excluded. 
	\end{itemize}
\end{itemize}

\subsubsection{Primary vertex selection}
\label{subsubsec:vtxselect}
Proper identification of the interaction point is important for the reconstruction of the kinematic variables. The longitudinal extent of the interaction region is determined by the length of the interacting bunches and the time structure of the beam. Beam-gas or cosmic rays events are evenly distributed along the longitudinal coordinate, while \ep\, events appear with higher rate in the vicinity of the nominal beam-beam interaction point. Distribution of the longitudinal component of the position of the primary vertex has Gaussian-like shape (see section~\ref{sec:zvtxrew}). Selecting the events with the primary interaction vertex satisfying $\chi^2 < 10$ and $Z_\text{vtx}$ being within $\sim \pm 3\sigma$ of the width of the distributions suppresses the non-physics background contribution and ensures good understanding of the dependence of the acceptance of the calorimeter and tracking systems on the position of the interaction vertex. The mean value and the width of the $Z_\text{vtx}$ changes between different data-taking periods. The restrictions imposed on $Z_\text{vtx}$ are detailed in Table~\ref{tab:zvxcut}.
\begin{table}[htbp]
	\centering
		\begin{tabular}{|c|c|}
			\hline
			Data-taking perod & imposed cut \\
			\hline
			\hline
			2004-2005 $e^{-}$ & $-32\,\cm<Z_\text{vtx}<30.1\,\cm$ \\
			2006-2007 $e^{+}$ & $-28.5\,\cm<Z_\text{vtx}<26.7\,\cm$ \\
			\hline
		\end{tabular}
	\caption{Requirements imposed on the longitudinal cordinate of the position of the interaction vertex.}
	\label{tab:zvxcut}
\end{table}

\subsubsection{Longitudinal Momentum balance}
\label{subsubsec:empzcut}
As discussed previously in Section~\ref{subsubsec:beamgasfeatures} longitudinal momentum balance can be used to discriminate photoproduction and beam-gas background. The actual value of quantity $\delta=\sum_i{\left(E_i-p_{z,i}\right)}$ for particular event may deviate from the nominal $\delta=55\;\GeV$ due to finite resolution as well as initial-state radiation effects. Therefore the following requirement was implied:
\begin{equation}
38 < \delta < 65\;\GeV.
\label{eq:epmzcut}
\end{equation}
The lower value was chosen in order to reject photoproduction events efficiently, while the upper cut is required for beam-gas events suppression.

\subsubsection{Transverse Momentum balance}
\label{subsubsec:empzcut}
Due to finite resolution total transverse momentum of the NC DIS event may be greater that zero. The energy resolution of the calorimeter scales approximately as $1/\sqrt{E}$. In order to suppress beam-gas related, cosmic-ray and charged current processes with misidentified electron the following ratio was required to be small:
\begin{equation}
\frac{p_T}{\sqrt{E_T}} < 2.5\;\sqrt{\GeV}.
\label{eq:ptcut}
\end{equation}

\subsubsection{Event Inelasticity}
\label{subsubsec:yelcut}

The DST bit 12 has a requirement on inelasticity of the event. In order to have offline selection consistent with the trigger requirements an additional cut $y_{el} < 0.75$ was imposed. This requirement rejects photoproduction background but the effect of the cut was found to be very small.

\subsubsection{Elastic QED-Compton}
\label{subsubsec:elasticqedcut}
Elastic Compton scattering processed ($ep \rightarrow ep\gamma$) in which the proton stay intact, are not well described by Monte Carlo simulations. The characteristic feature of such processes is the presence of two isolated electromagnetic clusters in the calorimeter. Therefore the events containing two electron candidates with transverse momentum vectors opposite to each other ($\Delta\phi > 3$), a ratio of transverse momenta between 0.8 and 1.2 and total energy sum measured in the calorimeter excluding the energy of two EM clusters below 3 \GeV were rejected.

\subsubsection{Higher-order QED predictions}
\label{subsubsec:qedcorcut}
In order to ensure the validity of higher-order QED corrections applied to the data (see Section~\ref{sec:qedcorr}) the region of phase space given by the requirement $y_{JB}\cdot\left(1-x_{DA}\right)^2>0.004$ was excluded. In this region MC predictions were found to be unreliable~\cite{cpc:81:381}.

\subsubsection{Hadronic Scattering Angle}
\label{subsubsec:gammahadcut}
In order to suppress events with large hadronic activity in forward region, where the simulations were found to be inaccurate~\cite{thesis:jose:2003}, a cut on projection of the hadronic scattering angle in the FCAL $R^\text{FCAL}_{\gamma^{had}} > 18\;\cm$ was implied. This constraint had only marginal effect because of inelasticity requirements suppressing events with large hadronic activity.

\subsubsection{Track multiplicity}
\label{subsubsec:trackmultcut}
The residual beam-gas contamination was achieved by requiring at least one track, which crossed minimum three CTD superlayers and had transverse momentum $p_T>0.2\;\GeV$. This cut had only minor effect.

\subsubsection{Jet selection}
\label{subsubsec:jetselect}
In order to identify jets the \kt clustering-algorithm in longitudinally-invariant inclusive mode was used. Jet search was performed in the Breit frame, therefore the momentum vectors corresponding the the energy deposits were transformed accordingly. The momentum vectors were constructed from energy and orientation of the calorimeter-cell centre with respect to interaction point. By definition the constructed vectors are massless. All cells excluding those corresponding to the electron candidate and satisfying the following requirements were used for the jet reconstruction.
\begin{itemize}
	\item The signals from natural Uranium radioactivity were suppressed by requiring the minimum energy in the cell to be above a threshold;
	\item The cut on imbalance between two photomultiplier signals attributed to a single cell was imposed in order to reject signals due to spontaneous discharge of one of the PMTs.
\end{itemize}

The phase space for the jet production was limited by the following requirements:
\begin{itemize}
	\item \textbf{Transverse energy} of the jets was required to be greater than 8~\GeV.
	\item \textbf{Pseudorapidity} of the jets has to be in region $-1<\etajetlab<2.5$.
\end{itemize}
The purity of the jet sample was enhanced by applying the following jet-cleaning cuts:

\begin{itemize}
	\item Good isolation of the electron candidate from hadronic jets was ensured by requiring the distance between electron and jet in the pseudorapidity-azimuth plane to be $\Delta R > 1$.
	\item Initial-state electron radiation hitting the RCAL was often reconstructed as a jet. Such events were removed from the sample if a jet with $\etjetb > 5\;\GeV$ and $\etajetlab < -1$ were identified.
	\item The jets with low transverse energy in the laboratory frame have large energy scale uncertainty, therefore jets with $\etjetlab < 3\;\GeV$ were excluded.
\end{itemize}

\subsection{Final event sample}
\label{subsec:eventsampletab}
The imposed requirements restrict the kinematic phase space of the measurements to

\begin{gather}
125 < \qsq < 20000\;\GeV^2 \qquad 0.2 < y < 0.6, \\
\etjetb > 8\;\GeV \qquad -1 < \etajetlab < 2.5.
\end{gather}
%The information about selected event samples is summarised in Table~\ref{tab:numevents}.
%\begin{table}[htbp]
%	\centering
%		\begin{tabular}{|c|c|}
%			\hline
%			Sample & Number of events \\
%			\hline
%			\hline
%			Data 2004-2005 $e^{-}$ & ??? \\
%			Data 2006-2007 $e^{+}$ & ??? \\
%			Lepto MC $e^{+}$ & ??? \\
%			Ariadne MC $e^{+}$ & ??? \\
%			\hline
%		\end{tabular}
%	\caption{Number of selected events in data and MC simulations.}
%	\label{tab:numevents}
%\end{table}

In general, the data are very well described by the Monte Carlo simulations when renormalised to the size of the data sample. A comparison of MC predictions to the data is demonstrated in Figures~\ref{fig:cp_ari},~\ref{fig:cp_ariadne}. 

%LEPTO
% % % % % % % % % % % % % % % % EVENT
\begin{figure}[ht!]
\begin{center}
\begin{subfloat}[]{\includegraphics[width=.32\textwidth,trim={5 0 50 0},clip] {./Figures/control_plots_ev/lep/new/h_Zvtxcontro_plot_lepto}
   \label{fig:cplep_subfig1}
 }%
\end{subfloat}
 \begin{subfloat}[]{\includegraphics[width=.32\textwidth,trim={5 0 50 0},clip]{./Figures/control_plots_ev/lep/new/h_Ydacontro_plot_lepto}
   \label{fig:cplep_subfig2}
 }%
\end{subfloat}
\begin{subfloat}[]{\includegraphics[width=.32\textwidth,trim={5 0 50 0},clip] {./Figures/control_plots_ev/lep/new/h_Q2dacontro_plot_lepto}
   \label{fig:cplep_subfig3}
 }%
\end{subfloat}
\newline
 \begin{subfloat}[]{\includegraphics[width=.32\textwidth,trim={5 0 50 0},clip]{./Figures/control_plots_ev/lep/h_sienecontro_plot_lepto}
   \label{fig:cplep_subfig4}
 }%
\end{subfloat}
 \begin{subfloat}[]{\includegraphics[width=.32\textwidth,trim={5 0 50 0},clip]{./Figures/control_plots_ev/lep/h_siphcontro_plot_lepto}
   \label{fig:cplep_subfig5}
 }%
\end{subfloat}
 \begin{subfloat}[]{\includegraphics[width=.32\textwidth,trim={5 0 50 0},clip]{./Figures/control_plots_ev/lep/h_sithcontro_plot_lepto}
   \label{fig:cplep_subfig6}
 }%
\end{subfloat}
\newline
 \begin{subfloat}[]{\includegraphics[width=.32\textwidth,trim={5 0 50 0},clip]{./Figures/control_plots_ev/lep/new/h_empzcontro_plot_lepto}
   \label{fig:cplep_subfig7}
 }%
\end{subfloat}
 \begin{subfloat}[]{\includegraphics[width=.32\textwidth,trim={5 0 50 0},clip]{./Figures/control_plots_ev/lep/new/h_ptmcontro_plot_lepto}
   \label{fig:cplep_subfig8}
 }%
\end{subfloat}
 \begin{subfloat}[]{\includegraphics[width=.32\textwidth,trim={5 0 50 0},clip]{./Figures/control_plots_ev/lep/h_cosgammahadcontro_plot_lepto}
   \label{fig:cplep_subfig9}
 }%
\end{subfloat}
\caption{Comparison of corrected MC (\lepto) and data distributions for event variables after full inclusive-jet selection.}
\label{fig:cp_lepto}
\end{center}
\end{figure}

\newpage
%ARIADNE
\begin{figure}[ht!]
\begin{center}
\begin{subfloat}[]{\includegraphics[width=.32\textwidth,trim={5 0 50 0},clip] {./Figures/control_plots_ev/ari/new/h_Zvtxcontro_plot_ariadne}
   \label{fig:cpari_subfig1}
 }%
\end{subfloat}
 \begin{subfloat}[]{\includegraphics[width=.32\textwidth,trim={5 0 50 0},clip]{./Figures/control_plots_ev/ari/new/h_Ydacontro_plot_ariadne}
   \label{fig:cpari_subfig2}
 }%
\end{subfloat}
\begin{subfloat}[]{\includegraphics[width=.32\textwidth,trim={5 0 50 0},clip]{./Figures/control_plots_ev/ari/new/h_Q2dacontro_plot_ariadne}
   \label{fig:cpari_subfig3}
 }%
\end{subfloat}
\newline
 \begin{subfloat}[]{\includegraphics[width=.32\textwidth,trim={5 0 50 0},clip]{./Figures/control_plots_ev/ari/h_sienecontro_plot_ariadne}
   \label{fig:cpari_subfig4}
 }%
\end{subfloat}
 \begin{subfloat}[]{\includegraphics[width=.32\textwidth,trim={5 0 50 0},clip]{./Figures/control_plots_ev/ari/h_siphcontro_plot_ariadne}
   \label{fig:cpari_subfig5}
 }%
\end{subfloat}
 \begin{subfloat}[]{\includegraphics[width=.32\textwidth,trim={5 0 50 0},clip]{./Figures/control_plots_ev/ari/h_sithcontro_plot_ariadne}
   \label{fig:cpari_subfig6}
 }%
\end{subfloat}
\newline
 \begin{subfloat}[]{\includegraphics[width=.32\textwidth,trim={5 0 50 0},clip]{./Figures/control_plots_ev/ari/new/h_empzcontro_plot_ariadne}
   \label{fig:cpari_subfig7}
 }%
\end{subfloat}
 \begin{subfloat}[]{\includegraphics[width=.32\textwidth,trim={5 0 50 0},clip]{./Figures/control_plots_ev/ari/new/h_ptmcontro_plot_ariadne}
   \label{fig:cpari_subfig8}
 }%
\end{subfloat}
 \begin{subfloat}[]{\includegraphics[width=.32\textwidth,trim={5 0 50 0},clip]{./Figures/control_plots_ev/ari/h_cosgammahadcontro_plot_ariadne}
   \label{fig:cpari_subfig9}
 }%
\end{subfloat}
\caption{Comparison of corrected MC (\ariadne) and data distributions for event variables after full inclusive-jet selection.}
\label{fig:cp_ariadne}
\end{center}
\end{figure}

