As described in Section~\ref{sec:kindis}, deep inelastic scattering at \hera can be characterised by two independent variables, for example, $\qsq$ and $\y$. The values of the kinematic variables can be determined from the components of the four-momenta of the scattered electron or the hadronic system or from the combination of the two. Below the different approaches for the measurement of the DIS kinematics are described.
\subsection{Electron Method}
\label{subsec:em}
The electron method utilises the energy, \eefin, and the polar angle, \thetae, of the scattered electron only\footnote{It is assumed that the electron does not undergo initial- and/or final-state radiation that effectively reduces the electron energy.}. The kinematic quantities characterising an event are given by the following expressions:
\begin{align}
	Q^2_\text{el} &= 2 E_e E_e' \left( 1 + \cos \theta_e \right),			\label{eq:q2el}							\\
	y_\text{el}      &= 1 - \frac{E_e'}{2 E_e}\left( 1 - \cos \theta_e \right),			\label{eq:yel} \\
	x_\text{el}      &= \frac{Q^2_\text{el}}{s y_\text{el}},			\label{eq:xel}
\end{align}
where, $\eeini$, is the energy of the initial-state electron. It has been shown~\cite{nim:a426:583} that this method accurately reconstructs the event kinematic variables at high values of $\y$, while at low values of inelasticity it has poor resolution.
\subsection{Jacquet-Blondel Method}
\label{subsec:jb}
Transverse momentum conservation and the almost complete hermiticity of the \zeus calorimeter enables the reconstruction of the event kinematics based exclusively on the energy deposits attributed to the hadronic system. Jacquet and Blondel proposed a method~\cite{proc:epfacility:1979:391} for the kinematic reconstruction for CC DIS events in which the final-state neutrino escapes the detector and cannot be measured. The kinematic variables are obtained from the following expressions:
\begin{align}
	y_\text{JB}      &= \frac{\sum{ \left( E - P_{Z} \right) }}{2E_e},			\label{eq:ybj} \\
	Q^2_\text{JB} &= \frac{ \left( \sum{P_{X}} \right)^2 + \left( \sum{P_{Y}} \right)^2 }{1-y_{JB}},			\label{eq:q2jb}							\\
	x_\text{JB}      &= \frac{Q^2_\text{JB}}{s y_\text{JB}}.			\label{eq:xjb}
\end{align}\marginpar{OB:Needs derivation of the formulas.\\DL:I think it is unnecessary because can lookup by reference.}
The sum in these expressions runs over the reconstructed final-state objects (see above) excluding those belonging to the scattered electron. It is assumed that the target remnants escaping the detector volume have small transverse momentum and are therefore suppressed in the expressions. The accuracy of this method is limited by the hadron-calorimeter energy resolution, the presence of dead material and backsplash/backscattering in the calorimeter. Since this method does not require the detection of the scattered electron, it is the only choice for photoproduction events\footnote{In photoproduction events the electron is scattered at small angles and escapes detection in the beam pipe.}.
\subsection{Double-angle Method}
\label{subsec:da}
The Double-angle method~\cite{proc:hera:1991:23} benefits from exploiting the combined information on the scattered electron and the hadronic final state. The kinematic variables can be expressed in terms of the electron scattering angle, \thetae, and \gamha, which in the quark-parton model corresponds to the scattering angle of the struck quark and can be obtained from the following equation
\begin{equation}
\cos \gamma_{\text{had}} = \frac{ \left( \sum{P_{X}} \right)^2 + \left( \sum{P_{Y}} \right)^2 - (\sum{E - P_{Z}})^2 }{\left( \sum{P_{X}} \right)^2 + \left( \sum{P_{Y}} \right)^2 + (\sum{E - P_{Z}})^2},
\label{eq:cosgam}
\end{equation}
where the sum runs over the energy deposits attributed to the hadronic final state.

The kinematic variables are obtained as follows:
\begin{align}
	y_\text{DA}      &= \frac{ \sin \theta_e \left( 1 - \cos \gamma_\text{had} \right)}{ \sin \gamma_\text{had} + \sin \theta_e - \sin \left( \gamma_\text{had} + \theta_e \right) },			\label{eq:yda} \\
	Q^2_\text{DA} &= 4E_e^2\frac{ \sin \gamma_\text{had} \left( 1 + \cos \theta_e \right) }{ \sin \gamma_\text{had} + \sin \theta_e - \sin \left( \gamma_\text{had} + \theta_e \right) },			\label{eq:q2da}							\\
	x_\text{DA}      &= \frac{E_e}{E_p} \frac{\sin \gamma_\text{had} + \sin \theta_e + \sin \left( \gamma_\text{had} + \theta_e \right)}{\sin \gamma_\text{had} + \sin \theta_e - \sin \left( \gamma_\text{had} + \theta_e \right)},			\label{eq:xda}
\end{align}
where \epini, is the proton initial energy.

Using these relations an expression for the energy of the scattered electron can be derived as
\begin{equation}
E_\text{el}^\text{DA} = \frac{2E_e \sin \gamma_\text{had}}{\sin \gamma_\text{had} - \sin \theta_e - \sin \left( \gamma_\text{had} + \theta_e \right)}.
\label{eq:eeda}
\end{equation}\marginpar{OB:Needs derivation of the formulas, again.\\DL:I think it is unnecessary because can lookup by reference.}

 The measurements of the scattering angles usually have better resolution and are approximately independent of the absolute energy-scale of the calorimeter. It has been shown in~\cite{thesis:behr:2010} that this method is optimal in the phase space of this measurement and therefore this was used as a default in this analysis.

%This formulae will be used later (see Section~\ref{sec:kinrec}) for the determination of the uncertainty due the absolute electromagnetic energy scale of the calorimeter.