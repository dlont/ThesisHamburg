The hadronic final-state of the DIS process contains charged and neutral particles that are detected in the different components of the \zeus detector. The neutral particles can only be detected in the calorimeter while charged particles produce tracks in MVD and CTD and also give a signal in the CAL. In this analysis the properties of the hadronic final state were determined using calorimeter cells and islands. Both approaches are briefly described below.

\textbf{Calorimeter cells.}
The energy deposits in the elementary calorimeter units i.e. cells (see Section~\ref{subsec:UCAL}), were used as an input to the jet finder. In order to suppress noise signals, the minimum energy threshold $E_\text{EM}>50$ MeV and $E_\text{HAD}>100$ MeV was imposed on energy deposits in the electromagnetic and hadronic parts, respectively. Besides that, to avoid signals from spontaneous high-voltage discharge of the photomultipliers, the cells with a difference of the two photomultiplier signals exceeding 90\% were also excluded. Since the scattered DIS electron does not contribute to the hadronic final state the cells attributed to the electron were excluded from consideration. 

\textbf{Calorimeter islands.}
Since single particles can give rise to signals in more than one cell, the energy deposits from adjacent cells satisfying the afore-men\-tio\-ned requirements were combined, using a dedicated algorithm~\cite{upub:grosse-knetter:zn9739}, into so-called calorimeter ``islands''. The energy of the island was calculated from the energy sum of the corresponding cells $E_k$ and the position of the island $\vec{r}_{isl}$ was defined as the energy-weighted average of the coordinates $\vec{r}_i$ of the centres of individual cells belonging to an island:
\begin{align}
\vec{r}_{isl} &= \frac{\sum_i{w_i \vec{r}_i}}{\sum_i{w_i}}, \\
w_i &= \max\left(0,W_0+\ln{\frac{E_i}{\sum_k{E_k}}}\right),
\end{align}
where $w_i$ is a weight assigned to the energy deposit and $W_0$ is a tunable parameter defining the minimum fraction of the energy deposit contributing to the cell-island position. Given the energy of the calorimeter island or cell the four-momentum components of the vector corresponding to the energy deposit were determined using following equations:
\begin{align}
	E    &= E_\mathrm{cell (isl)}, \\
	P_X &= E \sin{\theta}\cos{\phi}, \\
	P_Y &= E \sin{\theta}\sin{\phi}, \\
	P_Z &= E \cos{\theta},
\end{align}
where $E_{cell \left( isl \right)}$ is the cell (island) energy; the angles $\theta, \phi$ are determined from the direction from the primary vertex to the centre of the calorimeter cell or cluster. The energy of the cell (island) was corrected for losses in the dead material in front of the calorimeter and/or in the inter-module gaps as well as for the ``backsplash'' effect. The backsplash process is characterised by particle energy deposits in locations different from their expected direction e.g. backscattering from the CAL surface or showering of the particles in front of the calorimeter. Non-uniformities of the response in different parts of the CAL were also taken into account. In this thesis the island information was mainly used for the reconstruction of kinematic quantities (see below).