As it was mentioned, in the \zeus detector tracks of the charged particles are identified by means of the tracking detectors i.e. CTD and MVD.

The track identification is initialised by the VCTRACK algorithm~\cite{upub:hartner:zn9858,upub:hartner:zn9764}. In the pattern recognition stage VCTRACK identifies 3 hits in the outermost layers of the CTD that may belong to a single track candidate. This allows estimation of the curvature and charge of the candidate. The algorithm proceeds inwards in the direction of production point identifying potential hit candidates within the search window defined by the track candidate parameters uncertainty. In order to improve the precision of the track parameter estimate and remove outliers the VCTRACK output hit candidates are supplied to the Kalman-filter~\cite{Klaman:1960} based algorithm RTFIT~\cite{upub:spiridonov:rtfit}. The output of the RTFIT contains the information necessary for determination of the track momentum and the point of origin\footnote{The track production point is defined as a point of closest approach to the beam-line.}.

The location of production vertex is determined by means of the same VCTRACK code. The set of tracks originating from vicinity of the proton beam are identified and the weighted average of the coordinates of the individual track production points is determined. The calculation is performed by minimising the corresponding $\chi^2$ function. In this process, tracks providing large contribution to the $\chi^2$ are typically discarded as outliers. The precision on the position of primary vertex is further improved by applying the iterative ``deterministic annealing filter'' (DAF)~\cite{Fruewirth:2010} algorithm and taking into account the beam-spot\footnote{Beam-spot is a volume defined by the size of the intersection electron and proton bunches. It amount approximately to $80 \times 20\; \mu m^{2} \times 8\; \cm$.} constraint. Secondary decay vertices were also identified but not used in this analysis.