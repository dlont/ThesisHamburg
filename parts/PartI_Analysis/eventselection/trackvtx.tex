As mentioned in Chapter~\ref{ch:expsetup}, in the \zeus detector tracks of charged particles are identified by means of the tracking detectors i.e. CTD and MVD. Track identification is initialised by the VCTRACK algorithm~\cite{upub:hartner:zn9858,upub:hartner:zn9764}. In the pattern recognition stage, VCTRACK identifies 3 hits in the outermost layers of the CTD that may belong to a single track candidate. This allows estimation of the curvature and charge of the candidate. The algorithm proceeds inwards in the direction of the production point, identifying potential hit candidates within the search window defined by the uncertainty of the parameters of the track candidate. In order to improve the precision of the track-parameter estimate and remove outliers, the VCTRACK output hit candidates are supplied to the Kalman-filter based~\cite{Klaman:1960} algorithm RTFIT~\cite{upub:spiridonov:rtfit}. The output of the RTFIT contains the information necessary for determination of the track momentum and the point of origin\footnote{The track production point is defined for a given run as the position of the primary (secondary) vertex for vertex-fitted tracks or otherwise the point of closest approach to the centre of the beam distribution.}.

The location of the production vertex is determined by means of the same VCTRACK code. The set of tracks originating from the vicinity of the proton beam is identified and the weighted average of the coordinates of the individual track production points is determined. The calculation is performed by minimising a corresponding $\chi^2$ function. In this process, tracks providing a large contribution to the $\chi^2$ are typically discarded as outliers. The precision on the position of primary vertex is further improved by applying the iterative ``deterministic annealing filter'' (DAF)~\cite{Fruewirth:2010} algorithm and taking into account the beam-spot\footnote{The beam-spot is a volume defined by the size of the intersection of the electron and proton bunches. The position of the beam-spot is defined as a centre of gravity of the coordinates of many primary vertices in a given run. The cross section of the beam-spot in the $XY$ plane amounts to approximately $80 \times 20\; \mu m^{2} \times 12\; \cm$.} constraint. Secondary decay vertices were also identified but not used in this analysis.