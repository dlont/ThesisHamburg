High-energy partons, emerging from the hard interaction, in the experiment appear as collimated ``spays'' of particles, called jets. In order to determine the parent-parton momentum an appropriate assignment of a particle to a jet has to be performed. Investigation of the jet production provides an access to the details of underlying hard interaction as well as to the parton dynamics and the mechanism of parton showering. The proper combination of the particles has to satisfy general conditions:
\begin{itemize}
	\item infrared and collinear safety;
	\item conservation of factorisation properties of the hard and soft processes;
	\item little sensitivity to the hadronisation effects;
	\item relative insensitivity to the hadron remnant;
	\item invariance under longitudinal Lorentz boosts;
	\item easy implementation at the particle level in experimental analyses as well as at the parton/hadron level in perturbative theoretical calculations.
\end{itemize}
 The jet algorithm that was used in this thesis in described in the following subsection.