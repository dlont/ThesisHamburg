The pre-processing with ORANGE/ PHANTOM libraries includes dead-material and non-uniformity corrections in the electron identification algorithms. Nevertheless a residual discrepancy between data and MC simulations was observed. 

The resolution and electromagnetic energy scale were investigated in data and MC by taking the ratio $E_\text{SI}/E_\text{DA}$, where $E_\text{SI}$ is the measured electron energy including all corrections and $E_\text{DA}$ the energy measured by the double-angle method. In each bin of $E_\text{DA}$ a Gaussian fit to $E_\text{SI}/E_\text{DA}$ distribution was performed. The mean value extracted from the fit was plotted as a function of $E_\text{DA}$. The results of the fits are shown in Figure~\ref{fig:ele_enescale_unc}. It was found that the difference between the absolute energy scales in data and Monte Carlo simulations is less than 2\%. This discrepancy was taken into account as a systematic uncertainty as described in Section~\ref{subsec:systunc}.
\begin{figure}
	\centering
		\includegraphics[width=0.7\textwidth]{./Figures/ele_enescale_unc.png}
	\caption{Difference between the electromagnetic energy scale in data and Monte Carlo simulations as a function of electron energy.}
	\label{fig:ele_enescale_unc}
\end{figure}
