Inclusive-jet cross sections are steeply falling functions of jet transverse energy. An uncertainty on hadronic energy scale affects strongly the precision with which the jet cross sections can be measured. The jet energy-scale uncertainty is usually the dominant source of systematic uncertainty. In recent ZEUS publications~\cite{epj:c70:965, np:b864:1} the jet energy-scale uncertainty was determined to be $\pm 1\%$ for jets with transverse energy, $\etjet>10\; \GeV$ and $\pm 3\%$ for jets with $3\; \GeV<\etjet < 10\; \GeV$, resulting in systematic uncertainty on the measured jet cross section of about $5-10\%$ depending on the region of phase space. In this section the study of the hadronic energy scale performed in this analysis is described in detail.

The response of the calorimeter to jets was investigated by comparing the measured jet transverse energy to the transverse energy of the scattered electron. The electron energy determined using the double-angle method (see sec.~\ref{damethod}), $E_T^{DA}$, is approximately independent of the absolute energy scale of the calorimeter and therefore was used as a reference scale for the comparison. The transverse energy of the jet must be equal to that of the final state electron according to momentum conservation\footnote{It is assumed that the transverse momentum of the beam remnants is negligibly small and the hadronic final state is attributed to a single jet.} and the following relation must be satisfied
\begin{equation}
r = \frac{E_T^{jet}}{E_T^{DA}} = 1.
\label{eq:etjetetelbalance}
\end{equation}

In fact, because of the energy of neutrinos and the difference in the energy loss of the jet and the electron the ratio $r$ is different from unity. The deviation of a double ratio calculated in data and Monte Carlo events
\begin{equation}
C_\text{scale} = \frac{r^{DATA}}{r^{MC}} \stackrel{!}{=} 1
\label{eq:cscale}
\end{equation}
would indicate the difference in the energy-scale of the calorimeter in data and simulations. This factor can be used to correct the relative difference between data and MC assuming that $C_\text{scale}$ is independent of the energy of jets. The procedure for extracting the correction factors is described in the following.

For accumulation of substantial statistic sample for the calibration purposes the modifications to the selection procedure were introduced:
\begin{itemize}
	\item in order to avoid the problem with incorrectly reconstructed Lorentz boost to the Breit frame, the jet search was performed in the laboratory frame;
	\item a single jet with $\etjetlab > 10\; \GeV$ and no other jets with $\etjetlab > 5\; \GeV$ were required in order to suppress further hadronic activity not related to the hard scattering;
	\item the requirement on the pseudorapidity and the transverse energy in the Breit frame was omitted;
	\item to enlarge the amount of events with hadronic activity in the forward direction the inelasticity cut on $y$ was removed.
\end{itemize}

Figure~\ref{fig:ratcalibcontrolplot} demonstrates the description of the quantity $r$ by the Monte Carlo simulations.
\begin{figure}[htbp]
	\centering
		\includegraphics{./Figures/blank.jpeg} 
	\caption{Ratio $r$ of the transverse energies of the jet and the electron measured with double-angle method.}
	\label{fig:ratcalibcontrolplot}
\end{figure}

