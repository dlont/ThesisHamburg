Unlike in photoproduction and CC DIS events, the presence of the scattered electron in the final state is a distinct feature of high-$\qsq$ NC DIS, therefore the identification of an electron candidate in an event is used to discriminate between different event classes. The kinematic variables characterising an event, like $\qsq$, $\y$ and $x$ can be determined from the measured electron quantities $\eefin$ and $\thetae$ (see Section~\ref{sec:kinrec}), therefore an unambiguous identification of the scattered electron as well as a precise and unbiased measurement of the electron variables is crucial. Two methods of electron identification based on different combinations of the information from various detector components were employed in this thesis. 

The SINISTRA algorithm~\cite{nim:a365:508} was used as a nominal electron-identification algorithm in this analysis while the EM algorithm~\cite{epj:c11:427,upub:Straub:url} was used for the estimation of the systematic uncertainty attributed to the electron identification procedure (see Section~\ref{subsec:systunc}). The former algorithm is based on a neural-network pattern-recognition technique~\cite{desy:95:54}. The cell islands, described above are used as an input to the electron finders. The information about the longitudinal and lateral distribution of the energy deposits in the CAL is used by SINISTRA as an input in order to discriminate between electromagnetic- and hadron-induced showers. All calorimeter clusters are ordered according to their probability of being of electromagnetic type ($P=1$ - electromagnetic; $P=0$ - hadronic). 

The EM algorithm, on the other hand, combines the information about the energy distribution in the calorimeter with the information from the tracking detectors as well as kinematic features of the NC DIS events in order to estimate the probability, that a particular calorimeter cluster is the true scattered electron. A detailed comparison of two algorithms can be found in~\cite{upub:schlenstedt:zn9977}.

The energy of the electron candidate is determined from the sum of the energy deposits in the calorimeter and corrected for non-uniformities and energy losses using a dedicated electron-energy correction procedure. When a matched track pointing to the electromagnetic cluster was found, the position of the candidate was replaced by the more precise information from the tracking system.