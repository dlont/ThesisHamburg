 In many high energy physics analyses, the estimation of detector effects is often based on Monte Carlo simulations. However, as it will be discussed in Chapter~\ref{ch:unfolding}, a reliable estimate of the influence of the detector response is only possible if the MC simulations accurately describe all relevant distributions. %\sout{Therefore a detailed description of the observed data distributions is one of the crucial steps in any analysis.} 
 The discrepancy between data and MC may originate from two basic sources: inadequacy of the modelling of the \ep~physics process up to the generation of the final-state hadrons or of the response of the detector to the generated particles.
 
Improvement in the simulations is achieved by using more accurate predictions for the physical process and employing better detector simulation algorithms. However, obtaining a detailed description of the complete physical process can be a formidable task, therefore, often, a more feasible approach is used. An improvement of the description of the data distribution is obtained by assigning weights to the MC events (\emph{reweighting method}). The weights are usually functions of the kinematic variables and are adjusted in such a way that the simulations reasonably describe the data. The size of the weights is usually determined from an empirical fit to the data. The reweighting procedure is required to be independent of the reconstruction, therefore it must be based on \emph{true} level information. In the case when several quantities are reweighted, the final weight factor applied to the MC event is a product of individual weights, $w = \prod_i w_i$.
 
 This chapter describes the corrections applied to the MC simulations. At the beginning, the details of the longitudinal-vertex-position reweighting are presented. Later the MC track-veto correction is described.

 \section{Reweighting of the Longitudinal Vertex Position}
 \label{sec:zvtxrew}
 The detection and reconstruction of the scattered electron depend on the longitudinal position of the primary vertex, $Z_\text{vtx}$. In particular the detector and trigger acceptances vary with $Z_\text{vtx}$. The shape of the distribution of the primary vertex position is determined by the length\footnote{The space-charge distribution within a bunch typically has a Gaussian shape.} of the interacting bunches and thus depends on machine conditions. In order to suppress non-$ep$ background, restrictions were applied on the primary vertex position (see Chapter~\ref{ch:selection}). Therefore, since the luminosity measurements refer to the whole $ep$ interaction region, any cuts on $Z_\text{vtx}$ have an effect on the overall normalisation. 

An accurate simulation of the $Z_\text{vtx}$ distribution in the MC samples is therefore very important. The $Z_\text{vtx}$-reweighting procedure adopted here was developed in~\cite{thesis:behr:2010} and consisted of the following steps:
\begin{itemize}
 \item in order to avoid a possible bias from the jet selection as well as from tracking restrictions at the trigger level, the FLT30 bit was required instead of the FLT40, 43, 50 bits that were used as standard in the analysis;
 \item selection cuts on the longitudinal position of the interaction vertex were removed;
 \item the $Z_\text{vtx}$ distributions in the data and MC were fitted to the sum of four Gaussian functions, 
\begin{equation}
f\left(\vec a\right)=\sum_{i=1}^4{a_N^{\left(i\right)}\exp{\left[-\left(Z_\text{vtx}-a_{\mu}^{\left(i\right)}\right)^2/\left(a_\sigma^{\left(i\right)}\right)^2\right]}}.
\label{eq:fourgauss}
\end{equation}
 The reweighting factors, $w_{Z}$, were determined from the fit of the ratio of the normalised data and MC distributions to the function 
\begin{equation}
w=f\left(\vec a_1\right)/f\left(\vec a_2\right)
\label{eq:zvtxweght}
\end{equation}
 using the parameters of individual fits $\vec a_{\mathrm{Data}}, \vec a_{\mathrm{MC}}$ as seed values;
 \item the weights were determined from the detector-level distributions as function of the reconstructed position, $Z_\text{vtx}$. They were then applied to the MC events as a function of the true position, $Z_\text{vtx}^\text{true}$. This substitution $Z_\text{vtx} \mapsto Z_\text{vtx}^\text{true}$ can be made because the migration effects for the $Z_\text{vtx}$ distribution were found to be very small and can be neglected.
\end{itemize}

The existing MC samples for the 2004--2005~$e^-$ and 2006~$e^-$ data taking periods describe the data very well in the whole interaction region. Only for the 2006/2007~$e^+$ running period did the MC not reproduce the data distribution reasonably\footnote{The primary vertex distribution for the \hera II running period was measured in a dedicated un-biased study of low-\qsq NC DIS events~\cite{upub:oliver:zn07008} and was implemented in the MC production software}. In particular, disagreement between the measured and simulated distributions in the satellite-bunch and transition regions ($\left|Z_{\text{vtx}}\right|>30\,\cm$) was observed. Therefore, a reweighting of the longitudinal position of the primary vertex was implemented only for this MC sample. The comparison between data and MC distributions for the 2006/2007~$e^+$ data taking period before the reweighting is shown in Figure~\ref{fig:zvtxrew}. Although, individual fits have relatively large $\chi^2/N_\text{df}$ values, which, in principle, may introduce a bias, the final fit to the ratio of the normalised distributions has $\chi^2/N_\text{df}\approx 1$, and justifies using the fit results in the reweighting. After correcting the MC distribution (see Figure~\ref{fig:zvtxrewaf}) good agreement between data and simulations was observed.\marginpar{OB:unclear which true distribution of zvtx was generated in MC?\\DL:added a footnote.}
\begin{figure}[t]
\begin{center}
 \hspace{-35pt}\includegraphics[width=1.1\textwidth]{./Figures/zvtxrew/h_Zvtxfit_ratio_p_lepto_fix}%
\end{center}
\caption{The $Z_\text{vtx}$ distributions in the data and \lepto MC before the reweighting (top left panel) together with the individual fits of the data (top right panel) and MC (bottom right panel) to the function Eq.~\eqref{eq:fourgauss}. The ratio DATA/MC and the fit to Eq.~\eqref{eq:zvtxweght} is also shown (bottom left panel).} 
\label{fig:zvtxrew}
\end{figure}

\begin{figure}[p]
\begin{center}
 \includegraphics[height=0.9\textheight]{./Figures/zvtxrew/h_Zvtxratio_p_lepto}
\end{center}
\caption{Comparison of the $Z_\text{vtx}$ in data and \lepto MC distribution after reweighting (top panel) and ratio of the two distributions (bottom panel).} 
\label{fig:zvtxrewaf}
\end{figure}
 
 \section{Track Veto Efficiency Correction}
 \label{sec:trkvetoeff}
 An accurate description of the trigger efficiency is an important ingredient in this analysis.

As it was described in Chapter~\ref{ch:detector} the \zeus trigger was used to reject non-$ep$ background and to enhance the acceptance for physics-related processes. At the FLT most of the trigger bits were utilising CTD information to veto the events characterised by specific combinations of all and vertex-fitted tracks. The definitions for the corresponding track classes is illustrated in Figure~\ref{fig:trackvetodefinition}. According to this information two track-veto types, relevant for this analysis, were identified (see Table~\ref{tab:trackveto}). 
\begin{table}[htpb]
 \centering
 \begin{tabular}{lc}
 \multicolumn{2}{c}{track-veto requirement} \\
  \hline
 ``semi-loose'' & track-class $\le$ 2 or (track class = 8 and track multiplicity $\ge$ 26) \\
 ``tight''      & track-class $\le$ 2 \\
 \end{tabular} 
\caption{The track-veto condition used in the first level trigger.}
\label{tab:trackveto}
\end{table} 
In order to check the description of the track-veto in MC simulations monitor trigger bit was used. The FLT30 was requiring an isolated electromagnetic cluster in the RCAL and therefore was independent of the CTD information. The track-veto efficiency, expressed as a ratio
\begin{equation}
 \epsilon_\mathrm{trk} = \frac{N\left(\text{track veto} \wedge \text{FLT30}\right)}{N\left(\text{FLT30}\right)},
\end{equation}
where $N\left(\text{FLT30}\right)$ is a number of events triggered by the FLT30 and $N\left(\text{track veto} \wedge \text{FLT30}\right)$ is a number of events in a subset satisfying additional track-veto requirements, was studied separately in data and MC for different data taking periods. To determine $N\left(\text{track veto} \wedge \text{FLT30}\right)$ the track-veto was simulated offline by imposing additional restrictions on track quantities available at the FLT.

The efficiency was investigated as a function of $y_{DA}$. This variable was strongly correlated with the size of hadronic activity and thus with track multiplicity. The corresponding ratios in the data and MC are shown in Figures~\ref{fig:tveffdatamc}~\subref{fig:tveffdatamc_subfig1}--\subref{fig:tveffdatamc_subfig4}. The discrepancy between the data and MC simulations was attributed to the bad description of the track-class distribution in MC. In order to compensate for higher efficiency in MC an additional correction was implemented. The ratio of efficiencies in the data and MC was fitted to the first order polynomial
\begin{equation} 
 f\left(y_{DA}\right)=a_0 + a_1 \cdot y_{DA}.
\end{equation}
Because the efficiency observed in MC was higher than that in the data it can be corrected by rejecting overabundant MC events. Therefore, for each MC event a uniformly distributed random number, $r$, was generated and the event was rejected if $r > f\left(y_{DA}\right)$. The correction was implemented in \lepto and \ariadne samples and for different data-taking periods separately. The size of the correction depends on the value of $y_{DA}$ and was typically less than 0.5\% for 2004--2005$e^-$ and 2006$e^-$ and less than 3\% for 2007$e^+$. It was observed that for the ``semi-loose'' track-veto the same correction as for ``tight'' track-veto can be applied. The comparison of the track-veto efficiencies in the data and MC after applying the correction is illustrated in Figures~\ref{fig:aftveffdatamc}~\subref{fig:aftveffdatamc_subfig1}--\subref{fig:aftveffdatamc_subfig4}. After the correction the data efficiency was very well described by MC. 

The systematic effects attributed to the MC track-veto correction was examined investigating trigger efficiency as a function of CTD-FLT track multiplicity. The results of these studies are detailed in Chapter~\ref{systematics}.
\begin{figure}[t]
  \begin{center}
    \includegraphics[width=0.65\textwidth,trim={0 120 0 120},clip]{../thesis/Figures/classes96}
  \end{center}
  \caption{The definition of track veto classes (taken from~\cite{Yamazaki site}).}
  \label{fig:trackvetodefinition}
\end{figure}

\begin{figure}[ht!]
\begin{center}
\begin{subfloat}[]{\includegraphics[width=.45\linewidth,trim={0 0 280 0},clip] {../thesis/Figures/tvrew/tvrew_lep07p_yda_ltv}
   \label{fig:tveffdatamc_subfig1}
 }%
\end{subfloat}
 \begin{subfloat}[]{\includegraphics[width=.45\linewidth,trim={0 0 280 0},clip]{../thesis/Figures/tvrew/tvrew_ari07p_yda_ltv}
   \label{fig:tveffdatamc_subfig2}
 }%
\end{subfloat}
\newline
\begin{subfloat}[]{\includegraphics[width=.45\linewidth,trim={0 0 280 0},clip] {../thesis/Figures/tvrew/ratio_tvrew_lep07p_yda_ltv}
   \label{fig:tveffdatamc_subfig3}
 }%
\end{subfloat}
 \begin{subfloat}[]{\includegraphics[width=.45\linewidth,trim={0 0 280 0},clip]{../thesis/Figures/tvrew/ratio_tvrew_ari07p_yda_ltv}
   \label{fig:tveffdatamc_subfig4}
 }%
\end{subfloat}
\end{center}
\caption{Loose track-veto efficiency as a function of $y_{DA}$ in data and \lepto (a), \ariadne (b) MC. The ratio of the track-veto efficiency in data and \lepto (c), \ariadne (d) MC and the result of the fit.}
\label{fig:tveffdatamc}
\end{figure}

%After correction
\begin{figure}[pht]
\begin{center}
\begin{subfloat}[]{\includegraphics[width=.45\linewidth,trim={0 0 280 0},clip] {../thesis/Figures/tvrew/checktvrew_lep07p_yda_ltv}
   \label{fig:aftveffdatamc_subfig1}
 }%
\end{subfloat}
 \begin{subfloat}[]{\includegraphics[width=.45\linewidth,trim={0 0 280 0},clip]{../thesis/Figures/tvrew/checktvrew_lep07p_yda_sltv}
   \label{fig:aftveffdatamc_subfig2}
 }%
\end{subfloat}
\newline
\begin{subfloat}[]{\includegraphics[width=.45\linewidth,trim={0 0 280 0},clip] {../thesis/Figures/tvrew/checktvrew_ari07p_yda_ltv}
   \label{fig:aftveffdatamc_subfig3}
 }%
\end{subfloat}
 \begin{subfloat}[]{\includegraphics[width=.45\linewidth,trim={0 0 280 0},clip]{../thesis/Figures/tvrew/checktvrew_ari07p_yda_sltv}
   \label{fig:aftveffdatamc_subfig4}
 }%
\end{subfloat}
\end{center}
\caption{Comparison between data and MC of the loose (a,c) and semi-loose (b,d) track-veto after applying the track veto correction.}
\label{fig:aftveffdatamc}
\end{figure}



\section{Virtuality and Jet Transverse Energy Reweighting}
\label{sec:q2etrew}
After inclusive-jet selection described in Chapter~\ref{ch:selection} and after applying the $Z_\text{vtx}$ and track-veto corrections explained above, the distributions for kinematic variables were still not very well described by MC. In order to obtain reliable estimate of the detector effects to be corrected for in the cross section determination procedure (see Chapter~\ref{ch:Unfolding}), further reweighing of the Monte Carlo distributions was employed. 

The kinematic distributions in \lepto and \ariadne featured different properties, thus, for example, an excess of events was observed in the high-\qsq region in the \ariadne sample (see Figure~\ref{fig:q2rew_ari}(?)), while the \lepto had a deficiency in this region (see Figure~\ref{fig:q2rewlepto}()). In addition, as can be seen from Figure~\ref{fig:q2rew_ari}\subref{fig:q2rew_ari_subfig5} the \ariadne MC predicted harder jet spectrum than in the data. In order to take into account differences between the MC samples, two different reweighting procedures were developed for \lepto and \ariadne.

\subsection{\lepto reweighting}
The reweighting as a function of \qsq was imposed on the events simulated using the \lepto event generator. Two iterations were necessary to achieve adequate description of the data distribution. In each iteration the ratio of the normalised \qsq distributions in the data and MC was fitted to empirical functions to determine the weights, $w\left( \qsq\right) $. In order to mitigate the influence of detector effects, the data were corrected to the generator level using the acceptance correction factors determined at each step. The ansatz for the reweighting functions was as follows:
\begin{equation}
w\left( \qsq\right) = 
\begin{cases}
a_1 + a_2 / \log_{10}{\left( \qsq/\GeV^2\right) }, & \text{(1st iteration)} \\
b_1 + b_2 \cdot \left( \qsq/\GeV^2\right) ,            & \text{(2nd iteration)}.
\end{cases}
\end{equation}
The product of the weights obtained in each iteration was used for the final reweighting of the \lepto sample. The effect of each iteration of the \qsq~spectrum reweighing is illustrated in Figures~\ref{fig:q2rewlepto}~\subref{fig:q2rewlepto_subfig1}--\subref{fig:q2rewlepto_subfig3}. Overall, a considerable improvement of the description of the \qsq~distribution was observed. The deviation between the data and MC in the last bin of \qsq~after the reweighing was statistically insignificant.

\subsection{\ariadne reweighting}
It was found that the reweighting as a function of \qsq~was insufficient for achieving reasonable agreement between jet spectra in the data and \ariadne MC, in particular the residual discrepancy in \etjetb~distribution was observed. A direct reweighting as a function of \etjetb~is impossible because, while the jet can be present at the reconstructed level, it can be lost due to, for example, phase space restrictions at the generator level. A dedicated simultaneous reweighting as a function of the transverse energy of the hardest jet, $E_{T,B}^{jet1}$, and the process virtuality \qsq~was employed. The reweighting procedure proceeds as follows:
\begin{itemize}
	\item the data and MC events were classified according to the jet multiplicity into three categories with a one, two and three or more jets, respectively;
	\item for each category the two dimensional \qsq~vs $E_{T,B}^{jet1}$ distribution in the data and MC was measured;
	\item the bin content in MC was multiplied by \\$w\left( \qsq, E_{T,B}^{jet1} \right)=a_1\,\exp{\left( -\alpha E_{T,B}^{jet1}\right) }\left(1- \exp{\left( -\beta\qsq\right) } + a_2 E_{T,B}^{jet1} \right) $, where $a_i, \alpha, \beta$ are free dimensionfull parameters, and a 2d-likelihood fit of the shape of the data distribution was performed;
\end{itemize}
The described procedure was iterated until reasonable agreement between data and MC was achieved. Besides the variables used in the reweighting procedure the description of other jet quantities was verified, e.g. the comparison of the \etajetb~and jet multiplicity distributions before and after reweighting is demonstrated in Figures~\ref{fig:q2rew_ari1}\subref{fig:q2rew_ari_subfig5}--\subref{fig:q2rew_ari_subfig8}
\begin{figure}[pht]
\begin{center}
\begin{subfloat}[]{\includegraphics[width=0.49\textwidth,trim={0 0 50 100},clip] {./Figures/q2rew/h_q2_CS_dcontro_plot_ariadne_befrew}
   \label{fig:q2rew_ari_subfig1}
 }%
\end{subfloat}
 \begin{subfloat}[]{\includegraphics[width=0.49\textwidth,trim={0 0 50 100},clip]{./Figures/q2rew/h_q2_CS_dcontro_plot_ariadne_afrew}
   \label{fig:q2rew_ari_subfig2}
 }%
\end{subfloat}
\newline
\begin{subfloat}[]{\includegraphics[width=0.49\textwidth,trim={0 0 50 100},clip] {./Figures/q2rew/h_breit_kt_det_injet_etcontro_plot_ariadne_befrew}
   \label{fig:q2rew_ari_subfig3}
 }%
\end{subfloat}
 \begin{subfloat}[]{\includegraphics[width=0.49\textwidth,trim={0 0 50 100},clip]{./Figures/q2rew/h_breit_kt_det_injet_etcontro_plot_ariadne_afrew}
   \label{fig:q2rew_ari_subfig4}
 }%
\end{subfloat}
\end{center}
\caption{Result of three iterations of the \qsq reweighting in the \ariadne MC sample.}
\label{fig:q2rew_ari}
\end{figure}

\begin{figure}[pht]
\begin{center}
\begin{subfloat}[]{\includegraphics[width=0.49\textwidth,trim={0 0 50 100},clip] {./Figures/q2rew/h_breit_kt_det_injet_multcontro_plot_ariadne_befrew}
   \label{fig:q2rew_ari_subfig5}
 }%
\end{subfloat}
 \begin{subfloat}[]{\includegraphics[width=0.49\textwidth,trim={0 0 50 100},clip]{./Figures/q2rew/h_jetmult_lead_detcontro_plot_ariadne_afrew}
   \label{fig:q2rew_ari_subfig6}
 }%
\end{subfloat}
\newline
\begin{subfloat}[]{\includegraphics[width=0.49\textwidth,trim={0 0 50 100},clip] {./Figures/q2rew/h_breit_kt_det_injet_etacontro_plot_ariadne_befrew}
   \label{fig:q2rew_ari_subfig7}
 }%
\end{subfloat}
 \begin{subfloat}[]{\includegraphics[width=0.49\textwidth,trim={0 0 50 100},clip]{./Figures/q2rew/h_breit_kt_det_injet_etacontro_plot_ariadne_afrew}
   \label{fig:q2rew_ari_subfig8}
 }%
\end{subfloat}
\end{center}
\caption{Result of three iterations of the \qsq reweighting in the \ariadne MC sample.}
\label{fig:q2rew_ari1}
\end{figure}

\begin{figure}[pht]
\begin{center}
\begin{subfloat}[]{\includegraphics[width=\linewidth,trim={0 0 0 0},clip] {./Figures/q2rew/q2rew_1stiter}
   \label{fig:fig:q2rew_lep_subfig1}
 }%
\end{subfloat}
\newline
 \begin{subfloat}[]{\includegraphics[width=\linewidth,trim={0 0 0 0},clip]{./Figures/q2rew/q2rew_2nditer}
   \label{fig:fig:q2rew_lep_subfig2}
 }%
\end{subfloat}
\newline
\begin{subfloat}[]{\includegraphics[width=\linewidth,trim={0 0 0 0},clip] {./Figures/q2rew/q2rew_check}
   \label{fig:fig:q2rew_lep_subfig3}
 }%
\end{subfloat}
\end{center}
\caption{Result of three iterations of the \qsq reweighting in the \lepto MC sample.}
\label{fig:q2rewlepto}
\end{figure}
 %
\newpage
\section{Electromagnetic Energy Scale}
\label{sec:eleenescale}
The pre-processing of the data with \orange/\PHANTOM libraries includes dead-material and non-uniformity corrections in the electron identification algorithms. Nevertheless a residual discrepancy between data and MC simulations was observed. In order to study this discrepancy in detail the data and MC events satisfying the requirements described in Chapter~\ref{ch:selection} were used.

The resolution and electromagnetic energy scale were investigated in data and MC by taking the ratio $E_\text{SI}/E_\text{DA}$, where $E_\text{SI}$ is the measured electron energy including all corrections and $E_\text{DA}$ the energy measured by the double-angle method (see Section~\ref{subsec:da}). As was explained, the electron energy determined using the double-angle method is approximately independent of the absolute energy scale of the calorimeter and therefore was used as a reference scale for the comparison. As shown in Figure~\ref{fig:ele_enescale}~\subref{fig:ele_enescale1} the distribution has a Gaussian-like shape with the full width at half maximum of the data and \lepto distributions of about 10\%. In general, the simulations adequately describe the shape of the data distribution, however a systematic shift of the mean value was observed.

To investigate this shift in more details in each bin of $E_\text{DA}$, a Gaussian fit to the $E_\text{SI}/E_\text{DA}$ distribution was performed. The mean value extracted from the fit was plotted as a function of $E_\text{DA}$ (see Figure~\ref{fig:ele_enescale}\subref{fig:ele_enescale_2}). It was found that the double ratio between the absolute $E_\text{SI}$ energy scales in data and Monte Carlo simulations deviated from unity less than 2\%. This discrepancy was taken into account as a systematic uncertainty as described in Section~\ref{subsec:systunc}.

\begin{figure}[p!]
\begin{center}
\begin{subfloat}[]{\includegraphics[height=0.35\textheight] {./Figures/eleenescale/ele_enescale_rat}
   \label{fig:ele_enescale1}
 }%
\end{subfloat}
\newline
\begin{subfloat}[]{\hspace{-100pt}\includegraphics[height=0.45\textheight]{./Figures/eleenescale/ele_enescale_unc}
   \label{fig:ele_enescale_2}
 }%
\end{subfloat}
\end{center}
\caption{(a) Comparison of $E_\text{SI}/E_\text{DA}$ in data and Monte Carlo. (b) Double ratio between the electromagnetic energy scale in data and Monte Carlo simulations as a function of electron energy.}
\label{fig:ele_enescale}
\end{figure}

%\begin{figure}
	%\centering
		%\includegraphics[width=0.45\textwidth]{./Figures/eleenescale/ele_enescale_unc}
	%\caption{Difference between the electromagnetic energy scale in data and Monte Carlo simulations as a function of electron energy.}
	%\label{fig:ele_enescale_unc}
%\end{figure}


 \section{Jet Corrections}
 \subsection{Jet Energy-Scale Calibration}
 Inclusive-jet cross sections are steeply falling functions of jet transverse energy. An uncertainty on hadronic energy scale affects strongly the precision with which the jet cross sections can be measured. The jet energy-scale uncertainty is usually the dominant source of systematic uncertainty. In recent ZEUS publications~\cite{epj:c70:965, np:b864:1} the jet energy-scale uncertainty was determined to be $\pm 1\%$ for jets with transverse energy, $\etjet>10\; \GeV$ and $\pm 3\%$ for jets with $3\; \GeV<\etjet < 10\; \GeV$, resulting in systematic uncertainty on the measured jet cross section of about $5-10\%$ depending on the region of phase space. In this section the study of the hadronic energy scale performed in this analysis is described in detail.

The response of the calorimeter to jets was investigated by comparing the measured jet transverse energy to the transverse energy of the scattered electron. The electron energy determined using the double-angle method (see sec.~\ref{damethod}), $E_T^{DA}$, is approximately independent of the absolute energy scale of the calorimeter and therefore was used as a reference scale for the comparison. The transverse energy of the jet must be equal to that of the final state electron according to momentum conservation\footnote{It is assumed that the transverse momentum of the beam remnants is negligibly small and the hadronic final state is attributed to a single jet.} and the following relation must be satisfied
\begin{equation}
r = \frac{E_T^{jet}}{E_T^{DA}} = 1.
\label{eq:etjetetelbalance}
\end{equation}

In fact, because of the energy of neutrinos and the difference in the energy loss of the jet and the electron the ratio $r$ is different from unity. The deviation of a double ratio calculated in data and Monte Carlo events
\begin{equation}
C_\text{scale} = \frac{r^{DATA}}{r^{MC}} \stackrel{!}{=} 1
\label{eq:cscale}
\end{equation}
would indicate the difference in the energy-scale of the calorimeter in data and simulations. This factor can be used to correct the relative difference between data and MC assuming that $C_\text{scale}$ is independent of the energy of jets. The procedure for extracting the correction factors is described in the following.

For accumulation of substantial statistic sample for the calibration purposes the modifications to the selection procedure were introduced:
\begin{itemize}
	\item in order to avoid the problem with incorrectly reconstructed Lorentz boost to the Breit frame, the jet search was performed in the laboratory frame;
	\item a single jet with $\etjetlab > 10\; \GeV$ and no other jets with $\etjetlab > 5\; \GeV$ were required in order to suppress further hadronic activity not related to the hard scattering;
	\item the requirement on the pseudorapidity and the transverse energy in the Breit frame was omitted;
	\item to enlarge the amount of events with hadronic activity in the forward direction the inelasticity cut on $y$ was removed.
\end{itemize}

Figure~\ref{fig:ratcalibcontrolplot} demonstrates the description of the quantity $r$ by the Monte Carlo simulations.
%\begin{figure}[htbp]
%	\centering
%		\includegraphics{./Figures/blank.jpeg} 
%	\caption{Ratio $r$ of the transverse energies of the jet and the electron measured with double-angle method.}
%	\label{fig:ratcalibcontrolplot}
%\end{figure}


  \subsection{Jet Energy-Scale Uncertainty}
  As  was mentioned previously, the precise determination of the jet energy-scale uncertainty is an important ingredient in many jet analyses, so the accuracy of the jet energy-scale corrections, described above, was investigated in a dedicated study. Assuming a valid description of the fraction of charged and neutral particles constituting a jet in MC simulations, the difference in hadronic energy scale in the data and MC was examined independently using the tracking information, after the jet energy-scale correction, described above, was implemented. 

For jets with transverse energy in the laboratory frame $\etjetlab > 10\,\GeV$, the tracks\footnote{A track was required to pass through at least three CTD superlayers and to be fitted to the primary vertex; transverse momentum of the track has to be $p_T > 300 \MeV$.} within the tracking system acceptance attributed to jets were identified according to their proximity to the jet axis
\begin{equation}
R^2 = \left(\eta_{\mathrm{track}} - \eta_{\mathrm{jet}}\right)^2 + \left(\phi_{\mathrm{track}} - \phi_{\mathrm{jet}}\right)^2 < 1.
\label{eq:rtrackjetcut}
\end{equation}
The ratio of transverse energy of the jet, \etjetlab, and the sum of traverse momenta, $p_{T,i}$, of matched tracks 
\begin{equation}
r_\text{tracks} = \frac{\etjetlab}{\sum\limits_{\text{tracks}}{p_{T,i}}},
\end{equation}
was compared in the data and MC in different regions of $\etajetlab$ and for different running periods (see Figures~\ref{fig:ratcalibcontrolplotunc}). Overall, the MC simulations provide a good description of the data in shape, however the discrepancy between the mean values of the data and MC distributions remains. The double ratio $\left<r_\text{tracks}^\text{DATA}\right>/\left<r_\text{tracks}^\text{MC}\right>$ measured separately for different data-taking periods is illustrated in Figure~\ref{fig:ratcalibcontrolplotunc1}. The relative difference between the hadronic energy scale in data and MC does not exceed 1\%. This discrepancy was therefore taken into account in the systematic uncertainty. Exploiting the transverse momentum conservation in \ep collisions at \hera (see Section~\ref{sec:signalchar}) it was demonstrated in~\cite{thesis:behr:2010,thesis:perrey:2011} that jets with pseudorapidity outside the tracking acceptance region also amounts to $\pm 1\%$. The energy-scale uncertainty for jets with transverse energy $3<\etjetlab<10\,\GeV$ was found~\cite{thesis:behr:2010} to be $\pm 3\%$.
\begin{figure}[h!]
	\centering
		\includegraphics[width=\textwidth]{./Figures/ensccorr/verification/c_contplot1_enscvar} 
	\caption{The distribution of the ratio $r_\text{tracks} = \frac{\etjetlab}{\sum\limits_{\text{tracks}}{p_{T,i}}}$ and of the mean values $\left\langle r_\text{tracks}\right\rangle$ in data and MC for different data-taking periods.}
	\label{fig:ratcalibcontrolplotunc}
\end{figure}

\begin{figure}[ht]
	\centering
		\includegraphics[width=0.95\textwidth]{./Figures/ensccorr/verification/c_contplot_enscvar} 
	\caption{The double ratio $\left\langle r^{DATA}_\text{tracks}\right\rangle/\left\langle r^{MC}_\text{tracks}\right\rangle$ for different data-taking periods. The hatched band indicates the attributed jet energy-scale uncertainty.}
	\label{fig:ratcalibcontrolplotunc1}
\end{figure}

 \subsection{Jet Energy Correction}
 The energy of hadronic jets reconstructed from energy deposits in the calorimeter is influenced by various effects e.g. particle absorption in uninstrumented material between the production point and the calorimeter, inhomogeneities of the detector, noise etc. Hadron jets in the \zeus detector typically loose 5-15\% of their energy in inactive media (superconducting solenoid, support structures etc.) in front of the calorimeter. This effect may lead to systematic migrations of jets to cross section bins with lower energy. In principle, such effects must be taken into account in the unfolding procedure (see Chapter~\ref{ch:unfolding}). Nevertheless, in order to minimise migrations and to avoid a possible bias from the energy loss in inactive detector media, a dedicated jet-energy correction was employed in this analysis.

Two approaches for correcting the jet energy loss exist:
\begin{itemize}
 \item the \emph{bottom-up} approach consists of correcting the energy of the input objects (i.e. calorimeter cells in this analysis) to compensate for the energy loss and then using the corrected objects as an input to the jet algorithm;
 \item in the \emph{top-down} approach the energy of identified jets is corrected directly.
\end{itemize}
In principle, with the former approach more precise correction can be achieved, because individual jet details can be take into account, while in the later, only an average correction is achieved. Nevertheless, the \emph{top-down} approach is much more simple and therefore was used in this thesis.

Monte Carlo simulations were used to estimate the energy loss because they provide the detailed information about hadron propagation in the detector volume. The measured jet energy, $E_T^{jet,det}$, depends approximately linearly on the 'true', $E_T^{jet}$, value, but the size of the energy loss depends on the thickness of the traversed material and therefore on the jet pseudorapidity in the laboratory frame. For this reason, the complete measurement region $-1<\etajetlab<2.5$ was divided into 14 equal size regions and the correction was determined separately in each $\etajetlab$ bin.

The procedure was as follows:
\begin{itemize}
 \item in the MC events accepted at the generated and reconstructed levels, a pair of jets was identified according to the distance between jets in the $\eta-\phi$ plane. A hadron-level jet was matched to the reconstructed jet if the distance 
\begin{equation}
r=\sqrt{\left[\left(\eta_{had}-\eta_{det}\right)^2 + \left(\phi_{had}-\phi_{det}\right)^2\right]}
\end{equation}
between the two was small, $r<0.7$, and no further jets were reconstructed within the cone\marginpar{OB:Would that be possible (Rcut <1).\\DL:I think, yes, because jets have irregular shape and R=1 is a maximum radius.}. In order to avoid any bias on the correction procedure introduced by the boundaries of the jet phase space, the cuts on the jet transverse energy were relaxed to $\etjetbhad>6\;\GeV$ and $\etjetbdet>3\;\GeV$ at the hadron and detector levels, respectively;
 \item in each bin $i$ of $\etajetlab$, a linear fit $\left\langle\etjetbdet\right\rangle = a_i + b_i\cdot\left\langle\etjetbhad\right\rangle$ was performed, where $\left\langle\etjetbdet\right\rangle$ and $\left\langle\etjetbhad\right\rangle$ were determined using the set of matched jets; the fits for the 2004--2005~$e^-$ running period are illustrated in Figure~\ref{fig:05e_lepto_coeffitients};
 \item the corrections were determined using the \ariadne and \lepto sample for each data-taking period separately. 
\end{itemize}
\begin{figure}[p]
\centering
\includegraphics[height=0.9\textheight]{./Figures/etcorr/new/coeff/05e_lepto_coeffitients}
\caption{The average measured detector-level jet transverse energy $\left\langle\etjetbdet\right\rangle$ as a function of $\left\langle\etjetbhad\right\rangle$ and the corresponding straight-line fits in different regions of $\etajetlab$ for the data-taking period 2004--2005~$e^-$ in the \lepto MC sample.}
\label{fig:05e_lepto_coeffitients}
\end{figure}

Given the extracted fit parameters, the components of the jet four-momentum were scaled such that the following relation was obtained:
 \begin{equation}
  E_{T,B}^{jet,det} \mapsto E_{T,B}^{jet,corr} = \frac{E_{T,B}^{jet,det} - a_i}{b_i}.
 \end{equation}
As can be seen in Figure~\ref{fig:05e_lepto_coeffitients}, the fractional energy loss, represented by the slope of the fitted straight line, varies as a function of jet pseudorapidity. The size of the correction decreases towards the forward region of the detector. Such a behaviour was attributed to the variation of the amount of material in front of the calorimeter, in particular, the presence of superconducting solenoid surrounding the tracking system. 

Since the analysis was performed in the Breit frame, the jet pseudorapidity in the laboratory frame was recalculated and the corresponding correction factors were applied. Assuming a valid description of the detector effects in the simulations, the correction was applied to both the data and MC jets. In the simulations the correction was applied on top of that introduced in Section~\ref{subsec:jetenescale}.

%thus introducing a dependence on the parton-shower simulation. \textcolor{blue}{Therefore in the unfolding procedure the respective data and MC samples were utilised.} 

 \subsection{Conclusion}
 %\newpage
 % % % % % % % % % % % % % % % % % % % % ARIADNE % % % % % % % % % % % % % % % % % %
% % % % % % % % % % % % % % % % % % % % JETS % % % % % % % % % % % % % % % % % % % %
\begin{figure}[ht!]
\begin{center}
\begin{subfloat}[]{\includegraphics[width=.32\textwidth,trim={5 0 50 0},clip] {./Figures/control_plots_ev/ari/h_jetmult_lead_detcontro_plot_ariadne_afrew}
   \label{fig:cplep_subfig1}
 }%
\end{subfloat}
 \begin{subfloat}[]{\includegraphics[width=.32\textwidth,trim={5 0 50 0},clip]{./Figures/control_plots_ev/ari/h_breit_kt_det_injet_multcontro_plot_ariadne}
   \label{fig:cplep_subfig2}
 }%
\end{subfloat}
\begin{subfloat}[]{\includegraphics[width=.32\textwidth,trim={5 0 50 0},clip] {./Figures/control_plots_ev/ari/h_breit_kt_det_injet_etcontro_plot_ariadne}
   \label{fig:cplep_subfig3}
 }%
\end{subfloat}
\newline
 \begin{subfloat}[]{\includegraphics[width=.32\textwidth,trim={5 0 50 0},clip]{./Figures/control_plots_ev/ari/h_breit_kt_det_injet_etacontro_plot_ariadne}
   \label{fig:cplep_subfig4}
 }%
\end{subfloat}
 \begin{subfloat}[]{\includegraphics[width=.32\textwidth,trim={5 0 50 0},clip]{./Figures/control_plots_ev/ari/h_breit2lab_kt_det_injet_etcontro_plot_ariadne}
   \label{fig:cplep_subfig5}
 }%
\end{subfloat}
 \begin{subfloat}[]{\includegraphics[width=.32\textwidth,trim={5 0 50 0},clip]{./Figures/control_plots_ev/ari/h_breit2lab_kt_det_injet_etacontro_plot_ariadne}
   \label{fig:cplep_subfig6}
 }%
\end{subfloat}

\label{fig:cp_arijets}
\end{center}
\end{figure}

% % % % % % % % % % % % % % % % % % % % LEPTO % % % % % % % % % % % % % % % % % %
% % % % % % % % % % % % % % % % % % % % JETS % % % % % % % % % % % % % % % % % % % %
\begin{figure}[ht!]
\begin{center}
\begin{subfloat}[]{\includegraphics[width=.32\textwidth,trim={5 0 50 0},clip] {./Figures/control_plots_ev/lep/h_jetmult_lead_detcontro_plot_lepto_afrew}
   \label{fig:cplep_subfig1}
 }%
\end{subfloat}
 \begin{subfloat}[]{\includegraphics[width=.32\textwidth,trim={5 0 50 0},clip]{./Figures/control_plots_ev/lep/h_breit_kt_det_injet_multcontro_plot_lepto}
   \label{fig:cplep_subfig2}
 }%
\end{subfloat}
\begin{subfloat}[]{\includegraphics[width=.32\textwidth,trim={5 0 50 0},clip] {./Figures/control_plots_ev/lep/h_breit_kt_det_injet_etcontro_plot_lepto}
   \label{fig:cplep_subfig3}
 }%
\end{subfloat}
\newline
 \begin{subfloat}[]{\includegraphics[width=.32\textwidth,trim={5 0 50 0},clip]{./Figures/control_plots_ev/lep/h_breit_kt_det_injet_etacontro_plot_lepto}
   \label{fig:cplep_subfig4}
 }%
\end{subfloat}
 \begin{subfloat}[]{\includegraphics[width=.32\textwidth,trim={5 0 50 0},clip]{./Figures/control_plots_ev/lep/h_breit2lab_kt_det_injet_etcontro_plot_lepto}
   \label{fig:cplep_subfig5}
 }%
\end{subfloat}
 \begin{subfloat}[]{\includegraphics[width=.32\textwidth,trim={5 0 50 0},clip]{./Figures/control_plots_ev/lep/h_breit2lab_kt_det_injet_etacontro_plot_lepto}
   \label{fig:cplep_subfig6}
 }%
\end{subfloat}

\label{fig:cp_leptojets}
\end{center}
\end{figure}
 