 In many high energy physics analyses the estimation of detector effects is often based on the Monte Carlo simulations. However, as it will be discussed in Chapter~\ref{ch:cs} a reliable estimate of the influence of the detector responce is only possible if the MC simulations accurately describe all relevant distributions. %\sout{Therefore a detailed description of the observed data distributions is one of the crucial steps in any analysis.} 
 The discrepancy between data and MC may originate from two basic sources: inadequacy of the modelling of the physics process or the particle transport through the detector volume.
 
 The improvement in the simulations is provided by using more accurate predictions for the physical observables and employing better simulation algorithms. However, obtaining the detailed description of the physical process can be a formidable task, therefore, often, a more feasible approach is used. An improvement of the description of the data distribution is obtained by assigning weights to the MC events (\emph{reweighting method}). The weights are usually the functions of the kinematic variables and the sum of the reweighting factors are adjusted in such a way that the simulations reasonably describe the data. The size of the weights is usually determined from the empirical fit to the data. The reweighting procedure is required to be independent of the reconstruction therefore it must be based on \emph{true} level information. In case when several quantities are reweighted, the final weight factor applied to the MC event is a product of individual weights, $w = \prod_i w_i$.
 
 This chapter describes the corrections applied to the MC simulations. At the beginning, the details of the longitudinal vertex position reweighting are presented. Later the MC track-veto correction is described.

 \section{Longitudinal vertex position reweighting}
 The detection and reconstruction of the scattered electron depend on the longitudinal position of the primary vertex, $Z_\text{vtx}$. In particular the detector and trigger acceptances are varying with $Z_\text{vtx}$. The shape of the primary vertex position distribution is determined by the length\footnote{The space-charge distribution within a bunch typically has a Gaussian shape.} of the interacting bunches and thus on machine and detector conditions. In order to suppress non-$ep$ background restrictions were applied on the primary vertex position. Therefore, since the luminosity measurements refer to the whole $ep$ interaction region, the cuts on $Z_\text{vtx}$ have an effect on the overall normalisation. 

Summarising, an accurate simulation of the $Z_\text{vtx}$ distribution in the MC samples is very important. The $Z_\text{vtx}$-reweighting procedure adopted here was developed in~\cite{joerg} and consisted of the following steps.
\begin{itemize}
 \item In order to avoid a possible bias from the jet selection as well as from tracking restrictions at the trigger level the FLT30 bit was required instead of FLT40, 43, 50 bits used as a standard in the analysis.
 \item Any restrictions on the longitudinal position of the interaction vertex were removed.
 \item The $Z_\text{vtx}$ distributions distributions in the data and MC were fitted to the sum of four Gaussian functions, $f\left(\vec a\right)=\sum_{i=1}^4{a_N^{\left(i\right)}\exp{\left[-\left(Z_\text{vtx}-a_{\mu}^{\left(i\right)}\right)^2/\left(a_\sigma^{\left(i\right)}\right)^2\right]}}$. The reweighting factors, $w_{Z}$, were determined from the fit of the ratio of the normalised data and MC distributions to the function $w=f\left(\vec a\right)/f\left(\vec a'\right)$ using the parameters of the first fit as seed values.
 \item The weights were determined from the detector level distributions as function of the reconstructed position, $Z_\text{vtx}$. They are then applied to the events as function of the true position, $Z_\text{vtx}^\text{true}$. This substitution $Z_\text{vtx} \mapsto Z_\text{vtx}^\text{true}$ can be made, because the migration effects for the $Z_\text{vtx}$ distribution were found to be very small and can be neglected.
\end{itemize}

The existing MC samples for the 2004--2005$e^-$ and 2006$e^-$ data taking periods describe the data very well in the whole interaction region. Only for the 2006/2007$e^+$ running period, the MC did not reproduce the data distribution reasonably. Therefore the reweighting of the longitudinal position of the primary vertex was implemented only for this MC sample. The comparison between data and MC distributions for the 2006/2007$e^+$ data taking period before and after reweighting is demonstrated in Figures~\ref{fig:zvtxrew} and~\ref{fig:zvtxrewaf}.
\begin{figure}[t]
\begin{center}
 \hspace{-35pt}\includegraphics[width=1.1\textwidth]{./Figures/zvtxrew/h_Zvtxfit_ratio_p_lepto}%
\end{center}
\caption{The $Z_\text{vtx}$ distributions in the data and \lepto MC before the reweighting together with the individual fits. In the lower left panel the ratio DATA/MC and the fit is demonstrated.} 
\label{fig:zvtxrew}
\end{figure}

\begin{figure}[p]
\begin{center}
 \includegraphics[height=0.9\textheight]{./Figures/zvtxrew/h_Zvtxratio_p_lepto}
\end{center}
\caption{Comparison of the $Z_\text{vtx}$ in data and \lepto MC distribution after reweighting (Top panel) and ratio of the two distributions (Bottom panel).} 
\label{fig:zvtxrewaf}
\end{figure}
 
 \section{Track veto efficiency correction}
 An accurate description of the trigger efficiency is an important ingredient in this analysis.

As described in Chapter~\ref{ch:expsetup}, the \zeus trigger was used to select hard \ep~collisions with high acceptance and to reject non-$\ep$ background. At the FLT, most of the trigger bits utilised CTD information to veto events characterised by specific combinations of all and vertex-fitted tracks. The definitions for the corresponding event classes is illustrated in Figure~\ref{fig:trackvetodefinition}. Events with large track-multiplicity and few tracks fitted to the primary vertex (corresponding classes are 1, 2, 8) originate typically from beam-gas collisions, thus such events have to be discarded. Cosmic-ray events with low track multiplicity (corresponding class is 3) are also typically excluded. Such kind of trigger requirements on event track multiplicity were generally called trigger track-veto.

 According to this information, two track-veto types relevant for this analysis were identified: ``semi-loose'' and ``tight'' (see Table~\ref{tab:trackveto}). \marginpar{OB:unclear track classes.}
\textcolor{blue}{
\begin{table}[htpb]
 \centering
 \begin{tabular}{lc}
 \multicolumn{2}{c}{track-veto requirement} \\
  \hline
 ``semi-loose'' & track-class $\le$ 2 or (track class = 8 and track multiplicity $\ge$ 26) \\
 ``tight''      & track-class $\le$ 2 \\
 \end{tabular} 
\caption{The track-veto condition used in the first level trigger.}
\label{tab:trackveto}
\end{table} 
}
In order to check the description of the track-veto in MC simulations, a monitor trigger was used. The FLT30 required an isolated electromagnetic cluster in the RCAL and therefore was independent of the CTD information. The track-veto efficiency, expressed as the ratio
\begin{equation}
 \epsilon_\mathrm{trk} = \frac{N\left(\text{track veto} \wedge \text{FLT30}\right)}{N\left(\text{FLT30}\right)},
\end{equation}
where $N\left(\text{FLT30}\right)$ is the number of events triggered by the trigger bit FLT30 and $N\left(\text{track veto} \wedge \text{FLT30}\right)$ is the number of events in a subset satisfying additional track-veto requirements, was studied separately in data and MC for different data-taking periods. To determine $N\left(\text{track veto} \wedge \text{FLT30}\right)$, the track veto was emulated offline by imposing additional restrictions on track quantities available at the FLT.

The efficiency was investigated as a function of $y_{DA}$ because this variable was strongly correlated with the amount of hadronic activity and thus with the track multiplicity. The corresponding ratios in the data and MC are shown in Figures~\ref{fig:tveffdatamc}~\subref{fig:tveffdatamc_subfig1}--\subref{fig:tveffdatamc_subfig4}. The discrepancy between data and MC simulations was attributed to a bad description of the track-class distribution in the MC. In order to compensate for higher efficiency in the MC, an additional correction was implemented. The ratio of efficiencies in the data and MC was fitted to a first order polynomial
\begin{equation} 
 f\left(y_{DA}\right)=a_0 + a_1 \cdot y_{DA}.
\end{equation}
For both MC generators reasonable fit quality was obtained, nevertheless the main figure of merit for the reweighting procedure was the quality of the description of the data after implementing the correction in MC.

As the efficiency observed in MC was higher than that in the data it can be corrected by rejecting excess MC events. Therefore, for each MC event a uniformly distributed random number, $r$, was generated and the event was rejected if $r > f\left(y_{DA}\right)$. The correction was implemented in the \lepto and \ariadne samples, separately for different data-taking periods. The size of the correction depends approximately linearly on the value of $y_{DA}$ and, on average, was typically less than 0.5\% for 2004--2005~$e^-$ and 2006~$e^-$ samples and less than 3\% for the 2006-2007~$e^+$ sample. It was observed that for the ``semi-loose'' track-veto, the same correction as for ``tight'' track-veto can be applied. The comparison of the track-veto efficiencies in the data and MC after applying the correction is illustrated in Figures~\ref{fig:aftveffdatamc}~\subref{fig:aftveffdatamc_subfig1}--\subref{fig:aftveffdatamc_subfig4}. After the correction the data efficiency was very well described by the MC. 

The systematic effects attributed to the MC track-veto correction were examined by investigating the trigger efficiency as a function of the CTD-FLT track multiplicity. The results of these studies are detailed in Section~\ref{subsec:systunc}.
\begin{figure}[t]
  \begin{center}
    \includegraphics[width=0.65\textwidth,trim={0 120 0 120},clip]{./Figures/classes96}
  \end{center}
  \caption{The definition of track veto classes (taken from~\protect\cite{YamazakiSite}).}
  \label{fig:trackvetodefinition}
\end{figure}

\begin{figure}[ht!]
\begin{center}
\begin{subfloat}[]{\includegraphics[width=.45\textwidth,trim={0 0 0 0},clip,angle=-90] {./Figures/tvrew/tvrew_lep07p_yda_ltv}
   \label{fig:tveffdatamc_subfig1}
 }%
\end{subfloat}
 \begin{subfloat}[]{\includegraphics[width=.45\textwidth,trim={0 0 0 0},clip,angle=-90]{./Figures/tvrew/tvrew_ari07p_yda_ltv}
   \label{fig:tveffdatamc_subfig2}
 }%
\end{subfloat}
\newline
\begin{subfloat}[]{\includegraphics[width=.45\textwidth,trim={0 0 0 0},clip,angle=-90] {./Figures/tvrew/ratio_tvrew_lep07p_yda_ltv}
   \label{fig:tveffdatamc_subfig3}
 }%
\end{subfloat}
 \begin{subfloat}[]{\includegraphics[width=.45\textwidth,trim={0 0 0 0},clip,angle=-90]{./Figures/tvrew/ratio_tvrew_ari07p_yda_ltv}
   \label{fig:tveffdatamc_subfig4}
 }%
\end{subfloat}
\end{center}
\caption{Loose track-veto efficiency as a function of $y_{DA}$ in the data and \lepto MC (a) and \ariadne MC (b). Distributions of the ratio of the track-veto efficiency in data and \lepto MC (c), and data and \ariadne MC (d) and the results of the straight-line fits.}
\label{fig:tveffdatamc}
\end{figure}\marginpar{OB:Lacks info about fit quality.\\DL:Added in text.}

%After correction
\begin{figure}[ht!]
\begin{center}
\begin{subfloat}[]{\hspace{10pt}\includegraphics[width=.45\linewidth,trim={0 0 0 0},clip,angle=-90] {./Figures/tvrew/checktvrew_lep07p_yda_ltv}
   \label{fig:aftveffdatamc_subfig1}
 }%
\end{subfloat}
 \begin{subfloat}[]{\includegraphics[width=.45\linewidth,trim={0 0 0 0},clip,angle=-90]{./Figures/tvrew/checktvrew_lep07p_yda_sltv}
   \label{fig:aftveffdatamc_subfig2}
 }%
\end{subfloat}
\newline
\begin{subfloat}[]{\includegraphics[width=.45\linewidth,trim={0 0 0 0},clip,angle=-90] {./Figures/tvrew/checktvrew_ari07p_yda_ltv}
   \label{fig:aftveffdatamc_subfig3}
 }%
\end{subfloat}
 \begin{subfloat}[]{\includegraphics[width=.45\linewidth,trim={0 0 0 0},clip,angle=-90]{./Figures/tvrew/checktvrew_ari07p_yda_sltv}
   \label{fig:aftveffdatamc_subfig4}
 }%
\end{subfloat}
\end{center}
\caption{Comparison of the loose (a,c) and semi-loose (b,d) track-veto efficiency in data and MC after applying the track-veto correction.}
\label{fig:aftveffdatamc}
\end{figure}



 % 
 \newpage
 \section{Jet corrections}
 \subsection{Hadronic energy scale}
 \subsection{Jet energy correction}
% The energy of hadronic jets reconstructed from energy deposits in the calorimeter is influenced by various effects e.g. particle absorption in uninstrumented material between the production point and the calorimeter, inhomogeneities of the detector, noise etc. Hadron jets loose typically 5-15\% of their energy in inactive media (superconducting solenoid, support structures etc.) in front of the calorimeter. This effect may lead to the systematic migrations of jets to cross section bins with lower energy. In principle, such effects must be taken into account in the unfolding procedure. Nevertheless, in order to minimise jet migrations and avoid possible bias from the energy loss in inactive detector media, a dedicated jet-energy correction was employed in this analysis.

Two common approaches for correcting the jet energy loss exist:
\begin{itemize}
 \item the \emph{bottom-up} approach consists of correcting the energy of the input objects (i.e. calorimeter cells in this analysis) to compensate for the energy loss and then using the corrected objects as an input to the jet algorithm;
 \item in the \textcolor{blue}{\emph{top-bottom}} approach the energy of identified jets is corrected directly.
\end{itemize}
In general, the two approaches are equivalent if the unfolding of the measured cross sections is performed. In this thesis, the \textcolor{blue}{top-bottom} approach was adopted.

Monte Carlo simulations were used to estimate the energy loss because they provide the detailed information about hadron propagation in the detector volume. The measured jet energy, $E_T^{jet,det}$, depends approximately linearly on the 'true', $E_T^{jet}$, value, but the size of the energy loss depends on the thickness of the traversed material and therefore on the jet pseudorapidity in the laboratory frame. For this reason, the complete $-1<\etajetlab<2.5$ measurement region was divided into 14 equal regions and the correction was determined separately in each $\etajetlab$ bin.

The procedure proceeds as follows:
\begin{itemize}
 \item in the MC events accepted at the generated and reconstructed levels, a pair of jets was identified according to the distance between jets in the $\eta-\phi$ plane. A hadron-level jet was matched to the reconstructed jet if the distance $r=\sqrt{\left[\left(\eta_{had}-\eta_{det}\right)^2 + \left(\phi_{had}-\phi_{det}\right)^2\right]}$ between the two was small, $r<0.7$, and no further jets were reconstructed within the cone. In order to avoid any bias on the correction procedure introduced by the boundaries of the jet phase space, the cuts on the jet transverse energy were relaxed to $\etjetbhad>6\,\GeV$ and $\etjetbdet>3\,\GeV$ at the hadron and detector levels, respectively;
 \item in each bin $i$ of $\etajetlab$, a linear fit $\left\langle\etjetbdet\right\rangle = a_i + b_i\cdot\left\langle\etjetbhad\right\rangle$ was performed, where $\left\langle\etjetbdet\right\rangle$ and $\left\langle\etjetbhad\right\rangle$ were determined using the matched jet sample;
 \item the corrections were determined using the \ariadne and \lepto sample for each data taking period separately. 
\end{itemize}
The fits are illustrated in Figure~\ref{fig:05e_lepto_coeffitients}.
\begin{figure}[p]
\centering
\includegraphics[width=\linewidth]{./Figures/etcorr/coeff/05e_lepto_coeffitients}
\caption{The average measured detector-level jet transverse energy $\left\langle\etjetbdet\right\rangle$ as a function of $\left\langle\etjetbhad\right\rangle$ and the corresponding straight-line fits in different regions of $\etajetlab$ for the data-taking period 2004--2005~$e^-$ in the \lepto MC sample.}
\label{fig:05e_lepto_coeffitients}
\end{figure}

Given the extracted fit parameters, the components of the jet four-momentum were scaled such that the following relation was obtained:
 \begin{equation}
  E_{T,B}^{jet,det} \mapsto E_{T,B}^{jet,corr} = \frac{E_{T,B}^{jet,det} - a_i}{b_i}.
 \end{equation}
Because the analysis in this thesis was performed in the Breit frame, the jet pseudorapidity in the laboratory frame was recalculated and the corresponding correction factors were applied. Assuming a valid description of the detector effects in the simulations, the correction was applied to both the data and MC jets, thus introducing a dependence on the parton-shower simulation. \textcolor{blue}{Therefore in the unfolding procedure the respective data and MC samples were utilised.} In the MC simulations the correction was applied in addition to that introduced in Section~\ref{sec:jetcalib}.

\begin{figure}[p]
\centering
\includegraphics[width=\linewidth]{./Figures/etcorr/coeff/05e_lepto_coeffitients}
\caption{Correction lines}
\label{fig:05e_lepto_coeffitients}
\end{figure}

 \subsection{Conclusion}
\newpage

%ARIADNE
\begin{figure}[ht!]
\begin{center}
\begin{subfloat}[]{\includegraphics[width=.32\textwidth,trim={5 0 50 0},clip] {./Figures/control_plots_ev/ari/h_Zvtxcontro_plot_ariadne}
   \label{fig:cpari_subfig1}
 }%
\end{subfloat}
 \begin{subfloat}[]{\includegraphics[width=.32\textwidth,trim={5 0 50 0},clip]{./Figures/control_plots_ev/ari/h_Ydacontro_plot_ariadne}
   \label{fig:cpari_subfig2}
 }%
\end{subfloat}
\begin{subfloat}[]{\includegraphics[width=.32\textwidth,trim={5 0 50 0},clip]{./Figures/control_plots_ev/ari/h_Q2dacontro_plot_ariadne}
   \label{fig:cpari_subfig3}
 }%
\end{subfloat}
\newline
 \begin{subfloat}[]{\includegraphics[width=.32\textwidth,trim={5 0 50 0},clip]{./Figures/control_plots_ev/ari/h_sienecontro_plot_ariadne}
   \label{fig:cpari_subfig4}
 }%
\end{subfloat}
 \begin{subfloat}[]{\includegraphics[width=.32\textwidth,trim={5 0 50 0},clip]{./Figures/control_plots_ev/ari/h_siphcontro_plot_ariadne}
   \label{fig:cpari_subfig5}
 }%
\end{subfloat}
 \begin{subfloat}[]{\includegraphics[width=.32\textwidth,trim={5 0 50 0},clip]{./Figures/control_plots_ev/ari/h_sithcontro_plot_ariadne}
   \label{fig:cpari_subfig6}
 }%
\end{subfloat}
\newline
 \begin{subfloat}[]{\includegraphics[width=.32\textwidth,trim={5 0 50 0},clip]{./Figures/control_plots_ev/ari/h_ptmcontro_plot_ariadne}
   \label{fig:cpari_subfig7}
 }%
\end{subfloat}
 \begin{subfloat}[]{\includegraphics[width=.32\textwidth,trim={5 0 50 0},clip]{./Figures/control_plots_ev/ari/h_empzcontro_plot_ariadne}
   \label{fig:cpari_subfig8}
 }%
\end{subfloat}
 \begin{subfloat}[]{\includegraphics[width=.32\textwidth,trim={5 0 50 0},clip]{./Figures/control_plots_ev/ari/h_cosgammahadcontro_plot_ariadne}
   \label{fig:cpari_subfig9}
 }%
\end{subfloat}
\label{fig:cp_ariadne}
\end{center}
\end{figure}
\newpage
% % % % % % % % % % % % % % % % % % % % JETS % % % % % % % % % % % % % % % % % % % %
\begin{figure}[ht!]
\begin{center}
\begin{subfloat}[]{\includegraphics[width=.32\textwidth,trim={5 0 50 0},clip] {./Figures/control_plots_ev/ari/h_jetmult_lead_detcontro_plot_ariadne_afrew}
   \label{fig:cplep_subfig1}
 }%
\end{subfloat}
 \begin{subfloat}[]{\includegraphics[width=.32\textwidth,trim={5 0 50 0},clip]{./Figures/control_plots_ev/ari/h_breit_kt_det_injet_multcontro_plot_ariadne}
   \label{fig:cplep_subfig2}
 }%
\end{subfloat}
\begin{subfloat}[]{\includegraphics[width=.32\textwidth,trim={5 0 50 0},clip] {./Figures/control_plots_ev/ari/h_breit_kt_det_injet_etcontro_plot_ariadne}
   \label{fig:cplep_subfig3}
 }%
\end{subfloat}
\newline
 \begin{subfloat}[]{\includegraphics[width=.32\textwidth,trim={5 0 50 0},clip]{./Figures/control_plots_ev/ari/h_breit_kt_det_injet_etacontro_plot_ariadne}
   \label{fig:cplep_subfig4}
 }%
\end{subfloat}
 \begin{subfloat}[]{\includegraphics[width=.32\textwidth,trim={5 0 50 0},clip]{./Figures/control_plots_ev/ari/h_breit2lab_kt_det_injet_etcontro_plot_ariadne}
   \label{fig:cplep_subfig5}
 }%
\end{subfloat}
 \begin{subfloat}[]{\includegraphics[width=.32\textwidth,trim={5 0 50 0},clip]{./Figures/control_plots_ev/ari/h_breit2lab_kt_det_injet_etacontro_plot_ariadne}
   \label{fig:cplep_subfig6}
 }%
\end{subfloat}

\label{fig:cp_lepto}
\end{center}
\end{figure}

\newpage

%LEPTO
% % % % % % % % % % % % % % % % EVENT
\begin{figure}[ht!]
\begin{center}
\begin{subfloat}[]{\includegraphics[width=.32\textwidth,trim={5 0 50 0},clip] {./Figures/control_plots_ev/lep/h_Zvtxcontro_plot_lepto}
   \label{fig:cplep_subfig1}
 }%
\end{subfloat}
 \begin{subfloat}[]{\includegraphics[width=.32\textwidth,trim={5 0 50 0},clip]{./Figures/control_plots_ev/lep/h_Ydacontro_plot_lepto}
   \label{fig:cplep_subfig2}
 }%
\end{subfloat}
\begin{subfloat}[]{\includegraphics[width=.32\textwidth,trim={5 0 50 0},clip] {./Figures/control_plots_ev/lep/h_Q2dacontro_plot_lepto}
   \label{fig:cplep_subfig3}
 }%
\end{subfloat}
\newline
 \begin{subfloat}[]{\includegraphics[width=.32\textwidth,trim={5 0 50 0},clip]{./Figures/control_plots_ev/lep/h_sienecontro_plot_lepto}
   \label{fig:cplep_subfig4}
 }%
\end{subfloat}
 \begin{subfloat}[]{\includegraphics[width=.32\textwidth,trim={5 0 50 0},clip]{./Figures/control_plots_ev/lep/h_siphcontro_plot_lepto}
   \label{fig:cplep_subfig5}
 }%
\end{subfloat}
 \begin{subfloat}[]{\includegraphics[width=.32\textwidth,trim={5 0 50 0},clip]{./Figures/control_plots_ev/lep/h_sithcontro_plot_lepto}
   \label{fig:cplep_subfig6}
 }%
\end{subfloat}
\newline
 \begin{subfloat}[]{\includegraphics[width=.32\textwidth,trim={5 0 50 0},clip]{./Figures/control_plots_ev/lep/h_ptmcontro_plot_lepto}
   \label{fig:cplep_subfig7}
 }%
\end{subfloat}
 \begin{subfloat}[]{\includegraphics[width=.32\textwidth,trim={5 0 50 0},clip]{./Figures/control_plots_ev/lep/h_empzcontro_plot_lepto}
   \label{fig:cplep_subfig8}
 }%
\end{subfloat}
 \begin{subfloat}[]{\includegraphics[width=.32\textwidth,trim={5 0 50 0},clip]{./Figures/control_plots_ev/lep/h_cosgammahadcontro_plot_lepto}
   \label{fig:cplep_subfig9}
 }%
\end{subfloat}
\label{fig:cp_lepto}
\end{center}
\end{figure}
\newpage
% % % % % % % % % % % % % % % % % % % % JETS % % % % % % % % % % % % % % % % % % % %
\begin{figure}[ht!]
\begin{center}
\begin{subfloat}[]{\includegraphics[width=.32\textwidth,trim={5 0 50 0},clip] {./Figures/control_plots_ev/lep/h_jetmult_lead_detcontro_plot_lepto_afrew}
   \label{fig:cplep_subfig1}
 }%
\end{subfloat}
 \begin{subfloat}[]{\includegraphics[width=.32\textwidth,trim={5 0 50 0},clip]{./Figures/control_plots_ev/lep/h_breit_kt_det_injet_multcontro_plot_lepto}
   \label{fig:cplep_subfig2}
 }%
\end{subfloat}
\begin{subfloat}[]{\includegraphics[width=.32\textwidth,trim={5 0 50 0},clip] {./Figures/control_plots_ev/lep/h_breit_kt_det_injet_etcontro_plot_lepto}
   \label{fig:cplep_subfig3}
 }%
\end{subfloat}
\newline
 \begin{subfloat}[]{\includegraphics[width=.32\textwidth,trim={5 0 50 0},clip]{./Figures/control_plots_ev/lep/h_breit_kt_det_injet_etacontro_plot_lepto}
   \label{fig:cplep_subfig4}
 }%
\end{subfloat}
 \begin{subfloat}[]{\includegraphics[width=.32\textwidth,trim={5 0 50 0},clip]{./Figures/control_plots_ev/lep/h_breit2lab_kt_det_injet_etcontro_plot_lepto}
   \label{fig:cplep_subfig5}
 }%
\end{subfloat}
 \begin{subfloat}[]{\includegraphics[width=.32\textwidth,trim={5 0 50 0},clip]{./Figures/control_plots_ev/lep/h_breit2lab_kt_det_injet_etacontro_plot_lepto}
   \label{fig:cplep_subfig6}
 }%
\end{subfloat}

\label{fig:cp_lepto}
\end{center}
\end{figure}
