Since ancient times humanity spent an enormous effort trying to identify the basic building blocks of Nature and find guiding principles that govern all observed phenomena. In this endeavour a deeper insight was usually provided by new experimental observations; such a trend is still to be seen today.

Beginning from the early 20th century, scattering experiments have played an increasingly important role in revealing the microscopic structure of matter. Thus, for example, pioneering studies of the scattering of $\alpha$-particles on Gold led Rutherford~\cite{pm:21:669} to the discovery of the atomic nucleus. Soon it was realised that the nucleus is composed of protons and neutrons~\cite{Chadwick:1932ma}, which were generically named ``nucleons''. Approximately half a century after Rutherford's experiment, the investigation of high energy inelastic electron-nucleon scattering in a series of MIT-SLAC experiments~\cite{Panofsky:1968pb, Bloom:1969kc, Briedenbach:1969, Taylor:1991ew, Kendall:1991np} provided key evidence for nucleon substructure. Studies of the internal structure of the nucleon culminated in the high-precision determination of the proton content performed at \hera.

The proton constituents: \emph{quarks} and \emph{gluons}, which are generally called \emph{partons}, do not appear as free particles in experiment but are tightly bound inside hadrons. However, they manifest themselves in high-energy scattering experiments as 'sprays' of hadrons, called \emph{jets}. At \hera the production of jets can be investigated in a wide kinematic phase space offering a unique opportunity to constrain the proton parton density functions (PDFs).

In recent decades, all experimental and theoretical findings about the interaction of the elementary constituents of matter were unified in the modern concept of the \emph{Standard Model}~\cite{PDG:2014} of particle physics. The investigation of the processes involving jets can be regarded as a lab for testing the least understood sector of the Standard Model --- Quantum chromodynamics (QCD).

At \hera, jet production has been measured in wide variety of reactions including neutral current~\cite{epj:c19:289,pl:b547:164,pl:b551:226,np:b765:1,pl:b649:12,epj:c65:363,epj:c67:1,pl:b507:70,epj:c23:13,pl:b515:17,epj:c44:183,pr:d85:052008}, charged current~\cite{epj:c31:149,pr:d78:032004} and photoproduction~\cite{pl:b560:7,epj:c29:497,epj:c11:35,epj:c23:615,pl:b531:9,epj:c25:13,pl:b639:21,pr:d76:072011,pl:b443:394,np:b792:1} processes. The data have been used for the extraction of the strong coupling as well as for constraining the proton PDFs~\cite{epj:c42:1}.

The content of this thesis is organised as follows. An outline of the theoretical framework for the jet production and the kinematics of the deep inelastic scattering are described in Chapter~\ref{ch:theory}. The \hera machine as well as the \zeus detector are introduced in Chapter~\ref{ch:expsetup}. The details of the final-state reconstruction including the event and jet selection criteria are presented in Chapter~\ref{ch:selectionreco}. The corrections and reweightings applied to the data and Monte Carlo are described in Chapter~\ref{ch:calibcorr}. The comparison of the MC simulations with the data are also given there. In Chapter~\ref{ch:unfolding}, the cross section determination procedure is explained. Finally, in Chapters~\ref{ch:resultscs},~\ref{ch:resultsqcdfit} the results of the measurements of inclusive-jet cross sections and QCD analysis of the data are discussed. The thesis conduces with the summary and possible directions for future studies.