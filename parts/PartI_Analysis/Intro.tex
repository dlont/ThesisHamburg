Since ancient times humanity spent an enormous effort trying to identify the basic building blocks of Nature and find guiding principles that govern all observed phenomena. Beginning from early 20th century scattering experiments play a majour role in revealing microscopic structure of matter. Thus, for example, pioneering studies of the scattering of $\alpha$-particles on Gold led Rutherford~\cite{rutherford} to the discovery of atomic nucleus. Soon it was realised that the nucleus is composed of protons and neutrons, generally called nucleons. Approximately half a century after Rutherford's experiment the investigation of high energy inelastic electron-nucleon scattering in  the  series of SLAC experiments~\cite{slac} provided key evidence for the nucleon substructure. The studies of internal structure of matter culminated in high-precision experiments conducted at HERA.

In this attempt all experimental and theoretical findings accumulated over the centuries were unified in modern concept of the \emph{Standard Model} of elementary particles.