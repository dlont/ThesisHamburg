Deep inelastic scattering is process in which a high-energy lepton ($l$) scatters on a nucleon\footnote{The proton in this thesis.} or a nuclei ($h$) with large momentum transfer. The formal equation for this reactions reads:
\begin{equation}
l\left( k \right) + h\left( P \right) \rightarrow l'\left( k' \right) + X\left( P' \right),
\label{eq:DISreactions}
\end{equation}
where $X$ denotes the hadronic final state. The leading order Feymnan diagram for this process is illustrated in Figure~\ref{fig:DISgraph}.
\begin{figure}
	\centering
		\includegraphics[width=\textwidth]{./Figures/DISgraph.png}
	\caption{The leading order Feynman diagram for the deep inelastic scattering process.}
	\label{fig:DISgraph}
\end{figure}
At leading order the interaction between lepton and proton is mediated by electroweak bosons. In case of virtual $\gamma$ or $\zn$ exchange the process is called Neutral Current (NC) DIS, while in case of $W^\pm$ exchange the process is called Charged Current (CC) DIS. At HERA the CC DIS process is characterised by transformation of the initial-state electron (positron) into final-state (anti-) neutrino.

Because of fixed centre-of-mass energy, $\sqrt{s}$, at HERA only two independent variables are enough to describe the scattering process. The following Lorentz-invariant quantities\footnote{In the equations the masses of initial-state lepton and hadron are ignored} are used:
\begin{align}
\qsq &= -q^2 = -\left( k - k' \right)^2,\\
     x &= \frac{\qsq}{2p\cdot q},\\
		 y &= \frac{p\cdot q}{p\cdot k} = \frac{\qsq}{sx},
\label{eq:vardefinit}
\end{align}
where \qsq\, is the negative square of the four-momentum transfer or the virtuality of the exchange boson. At the ZEUS experiment two kinematic regions were formally distinguished: $\qsq < 1\GeV$, typically $\qsq \approx 0$, called photoproduction region; $\qsq > 1\GeV$, called deep inelastic scattering regime. The scaling variable, $x$, introduced by Bjorken, in Quark-Parton Model can be interpreted as a longitudinal momentum-fraction of the proton taking part in the hard scattering. The variable $y$ represents the lepton energy fraction transferred to the proton in the hadron rest frame. The following equation relating introduced variables holds:
\begin{equation}
\qsq = sxy.
\label{eq:qsxy}
\end{equation}

Choosing $\qsq$ and $x$ as independent variables, the deep inelastic scattering cross section can be written in terms of the proton structure functions $F_i\left(x,\qsq\right)$:
\begin{equation}
\frac{d^2\sigma\left( e^\pm p \right) }{dxd\qsq} = \frac{4\pi\alpha^2}{x\qsq^4}\left[ Y_+F_2\left(x,\qsq\right) - y^2F_L\left(x,\qsq\right) \mp Y_-xF_3\left(x,\qsq\right) \right],
\label{eq:ddifDIS}
\end{equation}
where $Y_\pm = 1 \pm \left( 1 - y \right)^2$. The dominant contribution to the scattering cross section is given by $F_2\left(x,\qsq\right)$, which in Quark-Parton model is directly related to the quark content of the proton:
\begin{equation}
F_2\left(x\right) = \sum_i{e_i^2xf_i\left(x\right)}.
\label{eq:f2pdf}
\end{equation}
In this equation $e_i$ is a fractional charge of the quark and $f_i\left(x\right)$ is the proton parton density function (PDF) describing the valence and sea quark content of the nucleon. The longitudinal structure function, $F_L\left(x,\qsq\right)$, has significant contribution to the cross section only at high values of $y$ and can be related to the cross section, $\sigma_L$, for the absorption of the longitudinaly polarised virtual photons:
\begin{equation}
F_L = \frac{Q^2}{4\pi^2\alpha}\cdot \sigma_L.
\label{eq:sigmal}
\end{equation}
 The $F_3\left(x,\qsq\right)$ arises from \zn exchange and $\gamma\zn$-interference and has significant size only for $\qsq \gg M^2_Z$.

The proton PDFs are universal and independent of the process under consideration. At current stage PDFs can not be predicted reliably from the first principles and have to be determined in experiment. The measurement of inclusive DIS cross section at HERA provides a direct access to the proton PDFs. Investigation of various sub-processes contributing to the inclusive DIS cross section provides more details about the hard scattering.
