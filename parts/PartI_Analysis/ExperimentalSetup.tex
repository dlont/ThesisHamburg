This chapter introduces the most important features of the  experimental setup that was utilised in this thesis. At the beginning the relevant properties of the accelerator complex is briefly introduced. Later the relevant components of the \zeus detector are discussed.

\section{\hera Machine}
\label{sec:hera}
The Hadron-Electron Ring Accelerator (\hera), the only $\ep$ collider in the world, was built in Hamburg, Germany at the national accelerator research centre \desy. At \hera, electrons\footnote{In what follows, the term ``electron'' is used for both electrons and positrons, unless otherwise mentioned.} or positrons of energy $27.5$ \GeV\, collided with protons of energy up to 920 \GeV, resulting in a centre-of-mass energy up to $\sqs = 319\; \GeV$. Four experiments took data at different interaction points along the \hera ring. The \zeus and \hone experiments, devoted to the study of the internal structure of the proton and searches for phenomena beyond the Standard Model, were operating with colliding beams. The \hermes experiment, dedicated to the investigation of the spin structure of nucleons, was a fixed-target experiment utilising the electron beam only, whereas \hera-B used only the proton beam, aiming at the measurement of CP-violation in the $B\bar{B}$-system. The \hera machine operated during the period 1996--2007 with a shut-down in 2000--2002. This shut-down marks the separation between the so called \hera I and \hera II data-taking periods.

The acceleration of electrons to their nominal energies was achieved in several stages. A schematic view of the acceleration chains is presented in Figure~\ref{fig:hera_acceleration}. Electrons were initially accelerated in \linac I/II to 200 \MeV. After injection into the \desy II synchrotron the electron energy was increased up to 7.5 \GeV. Then, after reaching 12 \GeV\, in \petra, electrons were finally transported to \hera. The positron beam was obtained by pair production from bremsstrahlung emission of electrons. 

The proton beam was obtained in several steps from a $\text{H}^-$ ion source. At the first stage 50 \MeV\, ions from a \linac were transported to \desy III, where they underwent acceleration to 7.5 \GeV\, and stripping off the electrons. Later, after achieving an energy of 40 \GeV\, in the \petra ring, the protons were finally injected into \hera.

\begin{figure}[htpb]
	\centering
	\begin{subfloat}[]{\includegraphics[height=0.49\textheight]{./Figures/hera_acceleration_large.png}
			\label{fig:hera_acceleration_large}
	 }%
	\end{subfloat}
	\begin{subfloat}[]{\includegraphics[height=0.49\textheight]{./Figures/hera_acceleration_zoom.png}
			\label{fig:hera_acceleration_zoom}
	 }%
	\end{subfloat}
	\caption{Schematic view of electron and proton acceleration chains.}
\label{fig:hera_acceleration}
\end{figure}

\subsection{Beam Structure}
\label{subsec:beamstruct}
The usage of Radio-Frequency acceleration cavities at \hera lead to a disctinct time structure of the beams. Protons and electrons were grouped into bunches separated by $\sim 28.8$ m, which corresponds to $96$ ns time intervals. Not all bunches were filled. The so called pilot bunches, for which either the electron or proton ``bucket'' was not filled, were used for the study of the interaction of the beam with residual gas in the beam vacuum pipe. Bunches in which both proton and electron ``buckets'' were empty were used for the study of the cosmic event rate and other non-$\ep$ background.

\begin{figure}[t]
	\centering
		\includegraphics[width=.5\textwidth]{./Figures/lumi00_0.png}
	\caption{\hera delivered luminosity}
	\label{fig:lumi00_0}
\end{figure}

\subsection{Luminosity}
\label{subsec:luminosity}
The crucial parameter of the collider that determines the rate of the collisions is the luminosity. It is related to the rate, $R$, of a process via the following expression:
\begin{equation}
	R = \mathcal{L}\sigma,
\end{equation}
where $\mathcal{L}$ is the instantaneous luminosity and $\sigma$ is the cross section. The luminosity is related to the parameters of the colliding beams:
\begin{equation}
	\mathcal{L} = f\frac{n_1n_2}{4\pi\sigma_x\sigma_y},
\end{equation}
where $f$ is the bunch-crossing rate, $n_1,\,n_2$ are the numbers of particles in the bunches and $\sigma_x$, $\sigma_y$ the width parameters for beams with Gaussian profiles. An increase of the luminosity~\cite{hera-98-05} at \hera II was achieved mainly by reducing the transverse size of the beams by installing additional focusing magnets close to the interaction points. 

To relate the number of events, $N$, to the reaction rate, the instantaneous luminosity has to be integrated:
\begin{equation}
	N = \int{R\,\mathrm{dt}} = \sigma\int{\mathcal{L}\,\mathrm{dt}} = \sigma L,
\end{equation}
where $L$ is called the integrated luminosity and is often used to denote the amount of collected data. A comparison of the increase of the delivered integrated luminosity during the \hera I and \hera II running periods is presented in Figure~\ref{fig:lumi00_0}. A typical fraction of 60\% of the delivered integrated luminosity was available for the data analysis. The other 40\% were lost due to various reasons: not fully operational detector components, inefficiencies of the data-acquisition system, specific trigger problems etc. A summary of the values of delivered and gated (recorded physical events) luminosity for the data-taking periods relevant for this analysis is given in Table~\ref{tab:heraruns}.

\begin{table}
	\centering
		\begin{tabular}[h]{|c|c|c|c|}
		  \hline
			Period & Lepton Type & Delivered Luminosity & Gated Luminosity \\
			\hline \hline
			2004 -- 2005 & $e^{-}$  & 204.8 \invpb  & 152.26 \invpb \T\B\\
			2006         & $e^{-}$  & 86.1 \invpb  &  61.23 \invpb \T\B\\
			2006 -- 2007 & $e^{+}$  & 180.54 \invpb  & 145.9 \invpb \T\B\\
			\hline
		\end{tabular}
	\caption{Information about \hera running periods used in the analysis.}
	\label{tab:heraruns}
\end{table}

\subsection{Polarisation}
\label{subsec:polarisation}
The spin of the electron is naturally transversely polarised at storage rings like \hera due to the Sokolov-Ternov effect~\cite{Sokolov:1963zn,Baier:1969hw}. Between the \hera~I and \hera~II data-taking periods the accelerator was upgraded to provide longitudinally polarised lepton beams at the \zeus and \hone experiments. The characteristic polarisation build-up time for \hera was about 40 min. To obtain longitudinally polarised beams at the interaction points, a chain of horizontal and vertical dipole magnets (spin rotators~\cite{Barber:1994ew}) were installed on either side of \zeus and \hone. A typical longitudinal polarisation value of 30\% -- 40\% was achieved.

 The most important parameters of the upgraded \hera storage ring are summarised in Table~\ref{tab:HERAParameters}.
\begin{table}[htbp]
	\centering
		\begin{tabular}[h]{|c|c|c|}
			\hline
			Parameter       & Electron beam   & Proton Beam \\
			\hline \hline
			Energy            &     $27.5$ \GeV  & $920$ \GeV \\
			Beam Current  &     $60$ mA       & $160$ mA \\
			Particle per bunch & $3.5\times 10^{10}$ & $10^{11}$ \\
			Maximum number of bunches & 210 & 210 \\
			Bunch length &  $7.8$ \mm & $110$ -- $150$ \mm \\
			Beam size     &  $112\times 30$ $\mm^2$ & $112\times 30$ $\mm^2$ \\
			Polarisation time & $30$ min & -- \\
			Maximum instantaneous luminosity & $5\times 10^{31}$ cm$^{-2}$s$^{-1}$ & -- \\
			\hline
		\end{tabular}
	\caption{The \hera storage-ring parameters.}
	\label{tab:HERAParameters}
\end{table}




\section{The \zeus detector}
\label{subsec:zeusdet}
The data for this analysis were collected by the \zeus detector, which was a general-purpose $4\pi$-detector, designed for the measurement of the dynamics of $\ep$ interactions. The \zeus detector consisted of the tracking system, the calorimeter system and muon chambers; the decision to store the data for further off-line analysis was taken by the three-level trigger system. In this section the main characteristics of the detector components relevant for this analysis are briefly described. A detailed description of the detector can be found in~\cite{zeus:1993:bluebook}.
The schematic layout of the \zeus detector is presented in Figure~\ref{fig:zeus2d1}. A right-handed Cartesian coordinate system (CS) with the origin at the nominal interaction point was adopted in \zeus (see Figure~\ref{fig:zeus_coordsyst}). The incoming proton momentum vector defines the positive Z direction. It was also called the ``forward'' direction. The positive direction of the X-axis pointed towards the centre of the \hera ring, while the positive Y-axis pointed upwards. The azimuthal angle, $\phi$, was defined in the transverse X--Y plane, while the polar angle was measured with respect to the +Z axis. 
\begin{landscape}
\begin{figure}[htpb]
	\centering
		\includegraphics[angle=-90,width=\linewidth]{./Figures/zeus2d1.png}
	\caption{Z-Y cut of the \zeus detector}
	\label{fig:zeus2d1}
\end{figure}
\end{landscape}

The pseudorapidity was an important quantity used in this analysis and defined by the following expression:
\begin{equation}
\eta = -\ln \tan \dfrac{\theta}{2}.
\end{equation}
The pseudorapidity has simple transformation properties under Lorentz boosts. The difference $\Delta \eta$ is a Lorentz invariant. For massless particles the numerical value of the pseudorapidity coincides with the normal rapidity.
\begin{figure}[htpb]
	\centering
		\includegraphics[width=0.5\textwidth]{./Figures/zeus_coordsyst.jpg}
	\caption{The \zeus coordinate system.}
	\label{fig:zeus_coordsyst}
\end{figure}


\subsection{Tracking detectors}
\label{subsec:trackdet}
The main parts of the tracking system of the \zeus detector were of the silicon microvertex detector (MVD) and the central-tracking detector (CTD). These tracking detectors were used for the measurements of momenta and positions of charged particles as well as for the identification of the interaction and secondary decay vertices.

\subsubsection{Microvertex Detector}
\label{subsubsec:mvd}
The MVD was a silicon-strip detector located in the vicinity of the beam-pipe for achieving an excellent resolution for secondary vertex tagging. It was installed during the 2000--2001 shut-down before the HERAII running period. The microvertex detector was divided in forward (FMVD), barrel (BMVD) parts (see Figure~\ref{fig:MVD_artistic}). In the barrel part the silicon sensors were arranged in ladder structures grouped in three cylindrical layers surrounding the beam-pipe. In the forward direction the sensors were assembled into four circular discs (wheels) oriented perpendicularly to the beam direction. Large number of read-out channels and high hit resolution allowed reliable separation of tracks emerging from hadronic jets. The main MVD characteristics are summarised in Table~\ref{tab:mvdgeomparameters}. 

\begin{figure}[htbp]
	\centering
		\includegraphics[angle=0,width=\textwidth]{./Figures/panorama_MVD_top.jpg}
	\caption{Silicon microvertex detector.}
	\label{fig:MVD_artistic}
\end{figure}

\begin{table}[htbp]
	\centering
\begin{tabular}{ | c | c | }
     \hline
      Parameter & Value \\
			\hline
			\hline
			Polar-angle coverage & $7\degree \mbox{--} 160\degree$ \\ \hline
      Read-out pitch & $120\,\micron$  \\ \hline
			Single-hit resolution & $24\,\micron$ \\ \hline 
			Two-track separation & $200\,\micron$ \\
      \hline
     \end{tabular}
	\caption{Silicon Microvertex detector parameters}
	\label{tab:mvdgeomparameters}
\end{table}

Other details of the sensor characteristics and performance can be found in~\cite{tech:mvd:prc9701}.

\subsubsection{Central Tracking Detector}
\label{subsubsec:ctd}
The CTD was a multi-wire cylindrical drift chamber used for the determination of charged particle positions and momenta. The operation principle of CTD was based on detection of the ionisation of a gas mixture by the charged particles traversing the volume of CTD. The transverse momentum of the particle was determined from the curvature of the track in the solenoid magnetic field (see below). The dependence of energy losses within the volume of CTD on particle mass was used for identification of the particle type.
\begin{figure}[htpb]
	\centering
		\includegraphics[width=0.7\textwidth]{./Figures/ctd}
	\caption{Layout of the CTD octant.}
	\label{fig:ctd}
\end{figure}
In Figure~\ref{fig:ctd} one CTD octant is demonstrated. The wires were organised in nine superlayers (SL). Using different orientation of wires with respect to the CTD axis in odd and even superlayers made it possible to accurately determine the Z position of the hit. The inclination angle in even superlayers was about $\pm 5\%$. For fast determination of the Z coordinate at the trigger level the Z-by-timing technique was used. For this purpose 8 sense wires were installed in each of the SL3 and SL5. The precision of the determination of the Z position, achieved by this method, was about 3 \cm.

Both CTD and MVD operated in a $1.43$~Tesla magnetic field parallel to the Z-axis and produced by a thin superconducting solenoid surrounding the drfit chamber. For CTD-MVD tracks that pass through all nine CTD superlayers, the momentum resolution was $\sigma(p_{T})/p_{T} = 0.0029 p_{T} \oplus 0.0081 \oplus
0.0012/p_{T}$, with $p_{T}$ in \GeV. Other parameters characterising the CTD are collected in Table~\ref{tab:ctdgeomparameters}.

\begin{table}[htbp]
	\centering
\begin{tabular}{ | c | c | }
     \hline
      Parameter & Value \\
			\hline
			\hline
			Inner radius & $16.2\,\cm$ \\ \hline
      Outer radius & $85.0\,\cm$  \\ \hline
			Length & $241\,\cm$ \\ \hline 
			Polar-angle coverage & $11.3\degree \mbox{--} 168.2\degree$ \\ \hline
			Position resolution & $100-120\,\micron$ \\ \hline
			Z resolution & $1.4\,\mm$ (stereo)/$30\,\mm$ (timing) \\ \hline
			Two track resolution & < $2.5\,\mm$ \\
      \hline
     \end{tabular}
	\caption{Central tracking detector parameters}
	\label{tab:ctdgeomparameters}
\end{table}


\subsection{The Uranium Calorimeter}
\label{subsec:UCAL}
The uranium-scintillator compensating calorimeter (CAL) was used for the measurement of the energy of the scattered electrons and positrons and of hadronic jets. The CAL covered 99.7\% of the solid angle and consisted of the forward (FCAL), barrel (BCAL) and rear (RCAL) parts. The boost of the hadronic system in the proton direction determined the depth of the calorimeter necessary for the absorption of particles of different maximum energy in various parts of the detector.
\begin{figure}[h]
	\centering
		\includegraphics[width=0.8\textwidth]{./Figures/cal.jpg}
	\caption{Schematic view of the CAL along the beam axis.}
	\label{fig:cal}
\end{figure}

The requirement to absorb a maximum energy of about 800~\GeV\, resulted in $\sim 7 \lambda$ depth of the FCAL section, where $\lambda$ is the hadronic interaction length. The FCAL was longitudinally segmented into towers, each consisting of a single electromagnetic (EMC) and two hadronic (HAC) sections with the front-surface of $20 \times 5$~\cm$^{2}$ and $20 \times 20$~\cm$^{2}$, respectively. The towers were grouped into 23 modules forming the volume of the FCAL. The HAC and EMC sections, also called cells, consisted of the interleaved layers of depleted uranium (98.1\% U$^{238}$, 1.7\% Nb and 0.2\% U$^{235}$) and scintillator (SCSN28) of 3.3~\mm\, and 2.6~\mm\, thickness, respectively. Such thickness of the absorber and active plates were chosen in order to achieve equal response of the calorimeter to electromagnetic and hardronic showers of the same energy. The FCAL covered the polar angle range $2.2\degree < \theta < 39.9\degree$.

The BCAL had a very similar structure as the FCAL, but with 32 wedge-shaped modules. Due to lower hadronic activity in the BCAL its thickness was $\sim 5 \lambda$. The BCAL covered the angular range $36.7\degree< \theta <129.1\degree$.

The rear part of the calorimeter (RCAL) consisted of 23 modules of $\sim 4 \lambda$ depth. Because of much lower activity in the electron direction the RCAL had only one HAC section.

In total, the CAL had 5918 cells. Each cell was read out by two photomultipliers on the opposite sides of the cell. The light from the scintillator plates was guided to the photomultipliers through wavelength shifters. Usage of two photomultipliers helped to avoid events with spontaneous discharge of one of the photomultipliers and minimised the amount of dead cells if one of the PMTs was not functioning.
The natural radioactivity of the uranium made it possible to perform a channel-by-channel calibration of the calorimeter on a daily basis by providing a stable reference signals. The CAL energy (response) was calibrated down to $\pm 1\%$ using this technique. The calibration of the electronic readout was performed using test pulses simulating photomultipliers signal. A timing resolution of 1~ns was achieved for energy deposits $> 4.5$ \GeV.
The energy resolution of the CAL measured under test beam conditions was 
\begin{equation}
	\frac{\sigma \left(E\right)}{E} = \frac{18\%}{\sqrt{E}} \oplus 1\% \quad \text{for electrons}
\end{equation}
and 
\begin{equation}
	\frac{\sigma \left(E\right)}{E} = \frac{35\%}{\sqrt{E}} \oplus 1\% \quad \text{for hadrons},
\end{equation}
where E was the incident particle energy. The 1~ns time resolution of the calorimeter was utilised for rejection of the non-ep background by providing fast signals to the trigger system. The angles were measured with about 10~mrad precision.

Other characteristics of the \zeus calorimeter are summarised in Table~\ref{tab:calparams}.

\begin{table}[htbp]
	\centering
	{\small
		\begin{tabular}{|c|c|c|c|}
			     \hline
      &FCAL & BCAL & RCAL \\
			\hline
			\hline
			Polar angle coverage & $2.2\degree < \theta < 36.7\degree$ & $36.7\degree < \theta < 129.1\degree$ & $129.1\degree < \theta < 176.2\degree$ \\ \hline
			Pseudorapidity coverage & $4.0 > \eta > 1.1$ & $1.1 > \eta > -0.74$ & $-0.74 > \eta > -3.4$ \\ \hline
			EMC section depth & $25.9 X_0$ & $22.7 X_0$ & $25.9 X_0$ \\ \hline
			Total module depth & $7.14 \lambda$ & $5.1 \lambda$ & $3.99 \lambda$ \\ \hline		
		\end{tabular}
	}
	\caption{Geometric dimensions of the calorimeter modules, here $X_0$ and $\lambda$ are the radiation and interaction lengths, respectively.}
	\label{tab:calparams}
\end{table}


\section{Luminosity Measurement System}
\label{sec:lumimeas}
At \zeus for the determination of the instantaneous luminosity the $\ep$-brems-strahlung process~\cite{Bethe:1934za}
\begin{equation}
	e + p \rightarrow e' + \gamma + p
\end{equation}
was used. The large cross section of this process allows rapid accumulation of large event samples in a relatively short time. Furthermore, the theoretical predictions for this reaction have a precision of better than $0.5\%$. The detection of the photon or the electron emerging from the interaction was used as an experimental signature of this process. The schematic layout of the luminosity monitor is shown in Figure~\ref{fig:lumi_monitor_layout}.
\begin{figure}
	\centering
		\includegraphics{./Figures/lumi_monitor_layout.png}
	\caption{Layout of the \zeus luminosity monitor}
	\label{fig:lumi_monitor_layout}
\end{figure}
The photons from the reaction are emitted at small angles $\theta \lesssim 0.5$ mrad and leave the beam-pipe through a thin window located at $92.5$ m distance from the interaction point. Approximately $9\%$ of the photons were converting into $e^+e^-$ pairs inside the window medium. Electrons and positrons were deflected by a dipole magnet into the luminosity spectrometer (SPEC)~\cite{physics-0512153}, which was installed at 104 m distance downstream. The remaining photons were detected in the photon calorimeter (PCAL) ~\cite{desy-92-066,zfp:c63:391,acpp:b32:2025} located at 107 m.

The instantaneous luminosity was determined from the formula:
\begin{equation}
\mathcal{L} = \frac{R_{\ep\rightarrow \ep\gamma}}{ \sigma_{\ep\rightarrow \ep\gamma}}.
\end{equation}
The background from the interaction of electrons with residual beam gas was estimated using electron pilot bunches and subtracted in the calculations.

The information from SPEC was not available for all runs, while the PCAL functioned continuously, so the luminosity value determined using the PCAL was used on a run-by-run basis. The systematic uncertainty on the measurement obtained with SPEC was lower than that for PCAL. It amounts to $1.8\%$ and was used as the resulting precision of the luminosity value.



\section{Polarisation Measurement System}
\label{sec:polarmeas}
As described in Section~\ref{subsec:polarisation}, at the interaction point longitudinally polarised lepton beam is obtained. In order to determine the lepton beam polarisation, two independent detectors LPOL~\cite{nim:a479:334} and TPOL~\cite{nim:a329:79} were used.
The dependency of the Compton cross section on the orientation of the electron spin was utilised. In both cases a laser system was used as a source of incident photons.

Circularly polarised continuous green argon-ion laser light was used for the measurement of the transverse polarisation. The photons from the laser beam collided with the transversely polarised electrons. The scattered photons were detected in a dedicated calorimeter. The asymmetry of the photon-scattering-angle distribution was used to determine the polarisation value.

A similar measurement was performed with longitudinally polarised electrons. A Nd:YAG laser pulse was transported to the collision region, where circularly polarised photons backscatter from the electron beam. Switching the circular polarisation of the photon beam from left-handed to right-handed, the asymmetry in the total energy of the scattered photon was determined and the electron beam polarisation value was derived.


\section{Data Quality and Trigger System}
\label{sec:daqtrigger}
Due to limited data processing speed and storage capabilities, not every bunch crossing could be analysed. Furthermore, the total event sample is dominated by non-$\ep$ background from beam-gas interactions or cosmic-ray showers. In order to reduce the event rate to an acceptable level and efficiently reject the background, a sophisticated three-level trigger system~\cite{Smith:1992im,nim:a379:542,Carlin:1995rv} was used. The architecture of the \zeus trigger system is shown in Figure~\ref{fig:trigarch}. 

\subsection{First Level Trigger}
\label{subsec:flt} Each detector component was coupled to its own hardware first-level trigger operating with general information such as regional energy sums (CAL)~\cite{nim:a355:278}, track multiplicity (CTD)~\cite{nim:a315:431} or muon tracks. The information from 26 consecutive bunch crossings (2.5 $\mu$s) was stored in a 46-event-deep pipe-line and was analysed in parallel streams. The combined information from each detector component was sent to the programmable Global First Level Trigger (GFLT), which selected the events that should be kept for the consideration at the second level. The decision was taken during 1.9 $\mu$s, which corresponds to 20 bunch crossings. If an event was accepted, the analogue information from different detector components was digitised and transferred from the pipelines to the data buffers. The GFLT had 64 bits so called ``slots'' corresponding to different event categories (see Section~\ref{subsec:fltcuts} for the description of the FLT slots used in the analysis). By using the FLT the event rate was reduced from approximately 10 MHz to $\sim$1 kHz. The time interval during which the FLT was processing detector information and therefore was not operational amounted to approximately 1\% and was automatically accounted by disabling the luminosity monitors when the trigger was busy.

\subsection{Second Level Trigger}
\label{subsec:slt}
The second-level trigger~\cite{Allfrey:2007zz} had more time to process information because events were arriving at a reduced rate. The information from each detector component was combined at the GSLT~\cite{upub:abbiendi:zn99063,upub:chlebana:zn94102,Uijterwaal:1992xc}, which was based on a reconfigurable network of transputers. The additional time available to SLT allowed a better estimation of the position of the primary vertex, identification of  calorimeter clusters and reconstruction of tracks. The timing information from the calorimeter system was used to efficiently reject non-$\ep$ background events at the SLT. The output rate of the second level was in range 50 -- 100 Hz. The full information from accepted events was sent further to the event builder.

\subsection{Third Level Trigger}
\label{subsec:tlt}
The TLT~\cite{Bailey:1992iq,Bhadra:1989kz} was a cluster of computer servers running complex algorithms for the vertex reconstruction, electron identification and reconstruction of event kinematic variables. The highly configurable architecture of the TLT made possible utilisation of up-to-date calibration information as well as fine tuning of the selection algorithms. The output rate of the TLT of about 5 Hz was compatible with the storage capabilities, thus the information from the event builder was converted in ADAMO format~\cite{Hart:1990dz} and written on magnetic tape.

\begin{figure}[h]
	\centering
		\includegraphics[height=0.9\textheight]{./Figures/ZEUStrigger_syst.jpg}
	\caption{The \zeus trigger-system architecture.}
	\label{fig:trigarch}
\end{figure}
\newpage


\section{Detector Simulation}
\label{sec:detsim}
The detector response to the particles traversing the detector volume was simulated using the \mozart program (based on \geant 3.21~\cite{tech:cern-dd-ee-84-1}). In \mozart the propagation of particles in the detector volume including motion in the magnetic field, multiple scattering, energy losses, particle decays and particle showers was implemented. The four-momentum components of the initial- and final-state particles used as an input were generated by one of the general-purpose event generators e.g. \lepto or \ariadne (see Section~\ref{sec:mcmodels}). These and other event generators had a common front-end interface program called \amadeus. The behaviour of the trigger system was simulated with the \zgana program. The \zgana program keeps the event even if it was not accepted by the trigger chain, thus allowing estimation of the trigger efficiency using the simulated events. The output of \zgana was compatible with the format for the raw data from the detector and can be subsequently analysed by the the same programs. The reconstruction of the simulated and real events was provided by the \zephyr program. The output of \zephyr was stored in \adamo format. The information specific to each program was supplied through the set of steering and \gaf files containing, for example, the magnetic field map or the shape of the distribution of the longitudinal component of the interaction-vertex position. All programs keep track of the generated particles at each step of the simulation process, thus providing access to the generator level.

The high-level generic routines for the event reconstruction relevant in most of the analysis were collected in the \orange/\phantom library. The analysis ntuples in \paw or \rootpaw format containing the variables necessary for the analysis can be produced using the \orange or EAZE jobs. These ntuples can be subsequently analysed by means of user-specific C++ or FORTRAN codes. 

In this thesis the v08b version of the Common Ntuples was used for both data and MC. The diagram of the data flow in the simulation process is demonstrated in Figure~\ref{fig:detectorsimulation}.

\begin{figure}[p]
	\centering
		\includegraphics[width=\linewidth]{./Figures/detectorsimulation.png}
	\caption{The \zeus detector simulation data flow.}
	\label{fig:detectorsimulation}
\end{figure}

