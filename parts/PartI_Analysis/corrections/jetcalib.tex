In most jet analyses, the jet energy-scale uncertainty is usually the dominant source of systematic error. Inclusive-jet cross sections are steeply falling functions of jet transverse energy. An uncertainty on the hadronic energy scale affects strongly the precision with which the jet cross sections can be measured. In recent \zeus publications~\cite{epj:c70:965, np:b864:1} the jet energy-scale uncertainty was determined to be $\pm 1\%$ for jets with transverse energy $\etjet>10\; \GeV$ and $\pm 3\%$ for jets with $3\; \GeV<\etjet < 10\; \GeV$, resulting in a systematic uncertainty on the measured jet cross section of about $5-10\%$, depending on the region of phase space. In this section, the study of the hadronic energy scale performed in this analysis is described in detail.

The response of the calorimeter to jets was investigated by comparing the measured jet transverse energy to the transverse energy of the scattered electron. The transverse energy of the jet must be equal to that of the final-state electron according to momentum conservation\footnote{It is assumed that the transverse momentum of the beam remnants is negligibly small and the hadronic final state is attributed to a single jet.} and the following relation must be satisfied
\begin{equation}
r = \frac{E_T^{jet}}{E_T^{DA}}.
\label{eq:etjetetelbalance}
\end{equation}
The deviation of a double ratio calculated in data and Monte Carlo events
\begin{equation}
C_\text{scale} = \frac{r^{DATA}}{r^{MC}} 
\label{eq:cscale}
\end{equation}
indicates a difference in the hadronic energy-scale of the calorimeter in data and simulations. This factor can be used to correct the relative difference between data and MC assuming that $C_\text{scale}$ is independent of jet energy. 

The procedure for the extraction of the relative correction factors $C_\text{scale}$ is based on the single-jet event sample. Therefore the selection requirements described in Section~\ref{ch:selection} were modified to conform to the assumptions of the method. The required modifications are outlined below: 
\begin{itemize}
	\item in order to avoid problems with imprecisely reconstructed Lorentz boosts to the Breit frame, the jet search was performed in the laboratory frame; moreover, the calorimeter energy scale is related to the distribution of the material within the detector volume, thus the laboratory frame is more natural for this study;
	\item a single jet with $\etjetlab > 10\; \GeV$ and no other jets with $\etjetlab > 5\; \GeV$ were required in order to suppress further hadronic activity not related to the hard scattering;
	\item the requirement on the pseudorapidity and the transverse energy in the Breit frame was omitted (see Eq.~\eqref{eq:jetphasespacecuts});
	\item to enlarge the amount of events with hadronic activity in the forward direction, the inelasticity cut on $y$ was removed  (see Eq.~\eqref{eq:kinematicphasespacecuts}).
\end{itemize}

Figure~\ref{fig:ratcalibcontrolplot} demonstrates the description of the quantity $r$ by the Monte Carlo simulations in different intervals of \etajetlab. In general, MC describes the shape of the data well, however a discrepancy between the mean values of the order of 3\% was observed in some bins of $\etajetlab$. The ratio of the mean values in data and MC is illustrated in Figure~\ref*{fig:doubleratcalib}\subref{fig:doubleratcalib_1}. These values were used to correct the discrepancy in the hadronic energy scale between data and MC. The transverse energy of the jets in the MC was multiplied such that:
\begin{equation}
 \etjetb \mapsto \etjetb' = \etjetb \cdot C_\text{scale}.
\end{equation}
\begin{figure}[htbp]
	\centering
		\includegraphics[width=\textwidth]{./Figures/ensccorr/c_contplot1_05e_befcor} 
	\caption{The ratio $r=\frac{\etjet}{E_{T,DA}^e}$ of the transverse energies of the jet and the electron measured with the double-angle method in different regions of jet pseudorapidity \etajetlab and the mean value $\left\langle r\right\rangle$ as a function of $\etajetlab$ (bottom right panel) in data and MC for the 2004--2005~$e^-$ data-taking period.}
	\label{fig:ratcalibcontrolplot}
\end{figure}
The result of such a relative shift was verified and is demonstrated in Figure~\ref{fig:doubleratcalib}\subref{fig:doubleratcalib_2}.
\begin{figure}[htbp]
	\begin{center}
	\begin{subfloat}[]{\includegraphics[width=0.49\textwidth] {./Figures/ensccorr/c_contplot_05e_befcor}
   \label{fig:doubleratcalib_1}
 }%
\end{subfloat}
\begin{subfloat}[]{\includegraphics[width=0.49\textwidth]{./Figures/ensccorr/c_contplot_05e_afcor}
   \label{fig:doubleratcalib_2}
 }%
\end{subfloat}
\end{center}
	\caption{Double ratio $\left\langle r^{DATA}\right\rangle/\left\langle r^{MC}\right\rangle$ of the transverse energies of the jet and the electron measured with the double-angle method as a function of the jet pseudorapidity for the 2004--2005~$e^-$ data-taking period before (left panel) and after (right panel) jet energy-scale correction.}
	\label{fig:doubleratcalib}
\end{figure}

