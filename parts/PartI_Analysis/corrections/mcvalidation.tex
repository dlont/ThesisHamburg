As was demonstrated in Chapter~\ref{ch:selection} both MC models after the reweightings describe the observed distributions reasonably well and are consistent with each other at the detector level. However, as can be seen in Figure~\ref{fig:epa} the acceptance correction factors estimated using \ariadne and \lepto MC models can differ by 10\%, implying substantial difference at the hadron level. In order to investigate the influence of the choice of MC model on the cross section values, additional study has been performed.

The observed cross section value in particular bin can be regarded as a Gaussian-distributed random variable $x = \mathcal{N}\left(\mu,w\right)$ with the mean value, $\mu$, provided by the \textit{true} physical cross section and the variance, $w$, given by statistical uncertainty\footnote{Without loss of generality, experimental uncertainty can neglected for simplicity.}. Therefore the variable, called pull,
\begin{equation}
p=\frac{x-\mu}{w}
\label{eq:pulldef}
\end{equation}
must follow standard Gaussian distribution with zero mean and unit variance. 

\begin{figure}[p!]
\begin{center}
\begin{subfloat}[]{\includegraphics[width=.45\textwidth] {Figures/pulls/Lepto_q2_sum}
   \label{fig:pulls_subfig1}
 }%
\end{subfloat}
 \begin{subfloat}[]{\includegraphics[width=.45\textwidth]{Figures/pulls/Ariadne_q2_sum}
   \label{fig:pulls_subfig2}
 }%
\end{subfloat}
\newline
\begin{subfloat}[]{\includegraphics[width=.45\textwidth] {Figures/pulls/Lepto_et_sum}
   \label{fig:pulls_subfig3}
 }%
\end{subfloat}
 \begin{subfloat}[]{\includegraphics[width=.45\textwidth]{Figures/pulls/Ariadne_et_sum}
   \label{fig:pulls_subfig4}
 }%
\end{subfloat}
\newline
 \begin{subfloat}[]{\includegraphics[width=.45\textwidth]{Figures/pulls/Lepto_eta_sum}
   \label{fig:pulls_subfig5}
 }%
\end{subfloat}
 \begin{subfloat}[]{\includegraphics[width=.45\textwidth]{Figures/pulls/Ariadne_eta_sum}
   \label{fig:pulls_subfig6}
 }%
\end{subfloat}
\caption{Pull distributions for (a,b) \dsdqsq (c,d) \dsdetjetb (e,f) \dsdetajetb cross sections in \lepto and \ariadne MC.}
\label{fig:pulls_sum}
\end{center}
\end{figure}
