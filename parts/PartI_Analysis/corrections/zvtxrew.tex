The reconstruction of the scattered electron kinematics as well as the direction of the jet momentum depends on the longitudinal position of the primary vertex, $Z_\text{vtx}$. Moreover the detector and trigger acceptance also depend on $Z_\text{vtx}$. The shape of the primary vertex position distribution is determined by the length\footnote{The space-charge distribution within a bunch typically has a Gaussian shape.} of the interacting bunches and thus on machine and detector conditions. In order to suppress non-$ep$ background the restrictions on the primary vertex position were applied in this analysis. Therefore, since the luminosity measurements refer to the whole $ep$ interaction region, the cuts on $Z_\text{vtx}$ have an effect on the overall normalisation. 

Summarising, an accurate simulation of the $Z_\text{vtx}$ distribution in the MC samples is very important. The $Z_\text{vtx}$-reweighting procedure was developed in~\cite{joerg}. The main steps of the procedure are summarised below.
\begin{itemize}
 \item In order to avoid possible bias from the jet selection as well as from tracking restrictions at the trigger level the FLT30 bit was required instead of FLT40, 43, 50 bits used in the analysis.
 \item Any restrictions on the longitudinal position of the interaction vertex were removed.
 \item The $Z_\text{vtx}$ distributions distributions in the data and MC were fitted to the sum of four Gauss functions, $f\left(\vec a\right)=\sum_{i=1}^4{a_N^{\left(i\right)}\exp{\left[-\left(Z_\text{vtx}-a_{\mu}^{\left(i\right)}\right)^2/\left(a_\sigma^{\left(i\right)}\right)^2\right]}}$. The reweighting factors, $w_{Z}$, were determined from the fit of the ratio of the normalised data and MC distributions to the function $w=f\left(\vec a\right)/\left(\vec a'\right)$ using the parameters of the first fit as seed values.
 \item Nevertheless the weights were determined from the detector level distributions, the substitution $Z_\text{vtx} \hookrightarrow Z_\text{vtx}^\text{true}$ can be made, because the migration effects for the $Z_\text{vtx}$ distribution were found to be very small and can be neglected.
\end{itemize}

The existing MC samples for the 2004--2005$e^-$ and 2006$e^-$ data taking periods describe data very well in the whole interaction region. Only for the 2006/2007$e^+$ running period, the MC did not reproduce the data distribution reasonably. Therefore the reweighting of the longitudinal position of the primary vertex was implemented for this MC sample only. The comparison between data and MC distributions for the 2006/2007$e^+$ data taking period before and after reweighting is demonstrated in Figures~\ref{fig:zvtxrew},~\ref{fig:zvtxrewaf}.
\begin{figure}[t]
\begin{center}
 \hspace{-35pt}\includegraphics[width=1.1\textwidth]{./Figures/zvtxrew/h_Zvtxfit_ratio_p_lepto}%
\end{center}
\caption{The $Z_\text{vtx}$ distributions in the data and \lepto MC before the reweighting together with the individual fits. In the lower left panel the ratio DATA/MC and the fit is demonstrated.} 
\label{fig:zvtxrew}
\end{figure}

\begin{figure}[p]
\begin{center}
 \includegraphics[height=0.9\textheight]{./Figures/zvtxrew/h_Zvtxratio_p_lepto}
\end{center}
\caption{Comparison between the data and \lepto MC of the $Z_\text{vtx}$ distribution after reweighting.} 
\label{fig:zvtxrewaf}
\end{figure}