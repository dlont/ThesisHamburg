After inclusive-jet selection described in Chapter~\ref{ch:selection} and after applying the $Z_\text{vtx}$ and track-veto corrections explained above, the distributions for kinematic variables were still not very well described by MC. In order to obtain reliable estimate of the detector effects to be corrected for in the cross section determination procedure (see Chapter~\ref{ch:Unfolding}), further reweighing of the Monte Carlo distributions was employed. 

The kinematic distributions in \lepto and \ariadne featured different properties, thus, for example, an excess of events was observed in the high-\qsq region in the \ariadne sample (see Figure~\ref{fig:q2rew_ari}(?)), while the \lepto had a deficiency in this region (see Figure~\ref{fig:q2rewlepto}()). In addition, as can be seen from Figure~\ref{fig:q2rew_ari}\subref{fig:q2rew_ari_subfig5} the \ariadne MC predicted harder jet spectrum than in the data. In order to take into account differences between the MC samples, two different reweighting procedures were developed for \lepto and \ariadne.

\subsection{\lepto reweighting}
The reweighting as a function of \qsq was imposed on the events simulated using the \lepto event generator. Two iterations were necessary to achieve adequate description of the data distribution. In each iteration the ratio of the normalised \qsq distributions in the data and MC was fitted to empirical functions to determine the weights, $w\left( \qsq\right) $. In order to mitigate the influence of detector effects, the data were corrected to the generator level using the acceptance correction factors determined at each step. The ansatz for the reweighting functions was as follows:
\begin{equation}
w\left( \qsq\right) = 
\begin{cases}
a_1 + a_2 / \log_{10}{\left( \qsq/\GeV^2\right) }, & \text{(1st iteration)} \\
b_1 + b_2 \cdot \left( \qsq/\GeV^2\right) ,            & \text{(2nd iteration)}.
\end{cases}
\end{equation}
The product of the weights obtained in each iteration was used for the final reweighting of the \lepto sample. The effect of each iteration of the \qsq~spectrum reweighing is illustrated in Figures~\ref{fig:q2rewlepto}~\subref{fig:q2rewlepto_subfig1}--\subref{fig:q2rewlepto_subfig3}. Overall, a considerable improvement of the description of the \qsq~distribution was observed. The deviation between the data and MC in the last bin of \qsq~after the reweighing was statistically insignificant.

\subsection{\ariadne reweighting}
It was found that the reweighting as a function of \qsq~was insufficient for achieving reasonable agreement between jet spectra in the data and \ariadne MC, in particular the residual discrepancy in \etjetb~distribution was observed. A direct reweighting as a function of \etjetb~is impossible because, while the jet can be present at the reconstructed level, it can be lost due to, for example, phase space restrictions at the generator level. A dedicated simultaneous reweighting as a function of the transverse energy of the hardest jet, $E_{T,B}^{jet1}$, and the process virtuality \qsq~was employed. The reweighting procedure proceeds as follows:
\begin{itemize}
	\item the data and MC events were classified according to the jet multiplicity into three categories with a one, two and three or more jets, respectively;
	\item for each category the two dimensional \qsq~vs $E_{T,B}^{jet1}$ distribution in the data and MC was measured;
	\item the bin content in MC was multiplied by \\$w\left( \qsq, E_{T,B}^{jet1} \right)=a_1\,\exp{\left( -\alpha E_{T,B}^{jet1}\right) }\left(1- \exp{\left( -\beta\qsq\right) } + a_2 E_{T,B}^{jet1} \right) $, where $a_i, \alpha, \beta$ are free dimensionfull parameters, and a 2d-likelihood fit of the shape of the data distribution was performed;
\end{itemize}
The described procedure was iterated until reasonable agreement between data and MC was achieved. Besides the variables used in the reweighting procedure the description of other jet quantities was verified, e.g. the comparison of the \etajetb~and jet multiplicity distributions before and after reweighting is demonstrated in Figures~\ref{fig:q2rew_ari1}\subref{fig:q2rew_ari_subfig5}--\subref{fig:q2rew_ari_subfig8}
\begin{figure}[pht]
\begin{center}
\begin{subfloat}{\includegraphics[width=0.49\textwidth,trim={0 0 50 100},clip] {./Figures/q2rew/h_q2_CS_dcontro_plot_ariadne_befrew}
   \label{fig:q2rew_ari_subfig1}
 }%
\end{subfloat}
 \begin{subfloat}{\includegraphics[width=0.49\textwidth,trim={0 0 50 100},clip]{./Figures/q2rew/h_q2_CS_dcontro_plot_ariadne_afrew}
   \label{fig:q2rew_ari_subfig2}
 }%
\end{subfloat}
\newline
\begin{subfloat}{\includegraphics[width=0.49\textwidth,trim={0 0 50 100},clip] {./Figures/q2rew/h_breit_kt_det_injet_etcontro_plot_ariadne_befrew}
   \label{fig:q2rew_ari_subfig3}
 }%
\end{subfloat}
 \begin{subfloat}{\includegraphics[width=0.49\textwidth,trim={0 0 50 100},clip]{./Figures/q2rew/h_breit_kt_det_injet_etcontro_plot_ariadne_afrew}
   \label{fig:q2rew_ari_subfig4}
 }%
\end{subfloat}
\end{center}
\caption{Result of three iterations of the \qsq reweighting in the \ariadne MC sample.}
\label{fig:q2rew_ari}
\end{figure}

\begin{figure}[pht]
\begin{center}
\begin{subfloat}{\includegraphics[width=0.49\textwidth,trim={0 0 50 100},clip] {./Figures/q2rew/h_breit_kt_det_injet_multcontro_plot_ariadne_befrew}
   \label{fig:q2rew_ari_subfig5}
 }%
\end{subfloat}
 \begin{subfloat}{\includegraphics[width=0.49\textwidth,trim={0 0 50 100},clip]{./Figures/q2rew/h_jetmult_lead_detcontro_plot_ariadne_afrew}
   \label{fig:q2rew_ari_subfig6}
 }%
\end{subfloat}
\newline
\begin{subfloat}{\includegraphics[width=0.49\textwidth,trim={0 0 50 100},clip] {./Figures/q2rew/h_breit_kt_det_injet_etacontro_plot_ariadne_befrew}
   \label{fig:q2rew_ari_subfig7}
 }%
\end{subfloat}
 \begin{subfloat}{\includegraphics[width=0.49\textwidth,trim={0 0 50 100},clip]{./Figures/q2rew/h_breit_kt_det_injet_etacontro_plot_ariadne_afrew}
   \label{fig:q2rew_ari_subfig8}
 }%
\end{subfloat}
\end{center}
\caption{Result of three iterations of the \qsq reweighting in the \ariadne MC sample.}
\label{fig:q2rew_ari1}
\end{figure}

\begin{figure}[pht]
\begin{center}
\begin{subfloat}{\includegraphics[width=\linewidth,trim={0 0 0 0},clip] {./Figures/q2rew/q2rew_1stiter}
   \label{fig:fig:q2rew_lep_subfig1}
 }%
\end{subfloat}
\newline
 \begin{subfloat}{\includegraphics[width=\linewidth,trim={0 0 0 0},clip]{./Figures/q2rew/q2rew_2nditer}
   \label{fig:fig:q2rew_lep_subfig2}
 }%
\end{subfloat}
\newline
\begin{subfloat}{\includegraphics[width=\linewidth,trim={0 0 0 0},clip] {./Figures/q2rew/q2rew_check}
   \label{fig:fig:q2rew_lep_subfig3}
 }%
\end{subfloat}
\end{center}
\caption{Result of three iterations of the \qsq reweighting in the \lepto MC sample.}
\label{fig:q2rewlepto}
\end{figure}