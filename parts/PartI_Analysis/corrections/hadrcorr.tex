In order to compare directly the NLO QCD predictions with the data, the calculations has to be corrected for hadronisation effects, because the measurements refer to the jets of hadrons while the predictions to that of partons. To estimated the influence of the parton-shower modelling and hadronisation process on the jet production the prediction from \ariadne and \lepto event generators were utilised. The hadronisation correction was determined from the ratio 
\begin{equation}
 \mathcal{C}^\text{hadr}_i = \frac{\sigma_i^\text{hadr}}{\sigma_i^\text{part}}
 \label{eq:hadrcor}
\end{equation}
of the jet cross sections at the hadron, $\sigma_i^\text{hadr}$, and parton $\sigma_i^\text{part}$ levels, respectively. The parton level cross section was determined using the partons available as an input to hadronisation model after the parton shower simulation step. The hadron level refer to the 'stable'\footnote{According to the \zeus convention, all particles with the lifetime $\tau > 10$ ns.} particles available in the MC event record. An average of the correction factors determined from \ariadne and \lepto was used to correct for hadronisation effectss.

Figures~\ref{fig:hadrcorr} illustrate the hadronisation corrections as functions of \etjetb, \etajetb and \qsq. In general, the correction

\begin{figure}[htp!]
\begin{center}
\begin{subfloat}[]{\includegraphics[width=.48\textwidth,trim={5 0 50 0},clip] {./Figures/hadrcorr/h_etajetb_CS_h_h2phadr_corr_fac}
   \label{fig:hadrcor_subfig1}
 }%
\end{subfloat}
 \begin{subfloat}[]{\includegraphics[width=.48\textwidth,trim={5 0 50 0},clip]{./Figures/hadrcorr/h_etjetb_CS_h_h2phadr_corr_fac}
   \label{fig:hadrcor_subfig2}
 }%
\end{subfloat}
\newline
\begin{subfloat}[]{\includegraphics[width=.48\textwidth,trim={5 0 50 0},clip] {./Figures/hadrcorr/h_q2_CS_h_h2phadr_corr_fac}
   \label{fig:hadrcor_subfig3}
 }%
\end{subfloat}
\label{fig:hadrcor}
\end{center}
\end{figure}


