The theoretical predictions for the measured cross sections obtained using the \nlojet program include only the leading-order QED contribution, while the measurements were influenced by higher-order processes like the running of the electromagnetic coupling, initial- and final-state EM radiation etc. To maintain the consistency between the data and theoretical predictions, the measured cross sections were corrected to the Born level using the MC predictions. A multiplicative factor applied to the data was determined using two \lepto MC samples with higher-order QED processes switched on and off. The corresponding jet cross sections are denoted by $\sigma^\text{QED}$ and $\sigma^\text{BORN}$, respectively. The correction factor is equal to the ratio:
\begin{equation}
 \mathcal{C}^\text{QED}_i = \frac{\sigma_i^\text{BORN}}{\sigma_i^\text{QED}}.
 \label{eq:eqdcorr}
\end{equation}

Figure~\ref{fig:qedcorr} illustrates the QED corrections determined in different kinematic bins. In general, the correction is approximately independent of \etjetb and increases with increasing \qsq. It is about 3\% in the lowest \qsq range and reaches about 10\% for $5000 < \qsq < 20000\;\GeV^2$.
 
 %\begin{figure}[ht]
%\begin{center}
%\begin{subfloat}{\includegraphics[width=0.45\linewidth,trim={0 0 0 0},clip] {./Figures/qedcorr/h_etjetb_CS_h_h2pqed_corr_fac}
   %\label{fig:qedcorr_subfig1}
 %}%
%\end{subfloat}
 %\begin{subfloat}{\includegraphics[width=0.45\linewidth,trim={0 0 0 0},clip]{./Figures/qedcorr/h_etajetb_CS_h_h2pqed_corr_fac}
   %\label{fig:qedcorr_subfig2}
 %}%
%\end{subfloat}
%\begin{subfloat}{\includegraphics[width=0.45\linewidth,trim={0 0 0 0},clip] {./Figures/qedcorr/h_q2_CS_h_h2pqed_corr_fac.pdf}
   %\label{fig:qedcorr_subfig3}
 %}%
%\end{subfloat}
%\end{center}
%\caption{QED multiplicative correction factors for inclusive-jet cross sections as functions of \etjetb, \etajetb and \qsq applied to pQCD predictions.}
%\label{fig:qedcorr}
%\end{figure}

% % % % % % % % % % % % % % % % % % % % HADRONISATION CORRECTIONS % % % % % % % % % % % % % % % % % %

\begin{figure}[ht!]
\begin{center}
\begin{subfloat}[]{\includegraphics[width=.32\textwidth] {./Figures/qedcorr/h_etjetbinq2_CS_h_h2p0qed_corr_fac}
   \label{fig:qedcor_subfig1}
 }%
\end{subfloat}
 \begin{subfloat}[]{\includegraphics[width=.32\textwidth]{./Figures/qedcorr/h_etjetbinq2_CS_h_h2p1qed_corr_fac}
   \label{fig:qedcor_subfig2}
 }%
\end{subfloat}
\begin{subfloat}[]{\includegraphics[width=.32\textwidth] {./Figures/qedcorr/h_etjetbinq2_CS_h_h2p2qed_corr_fac}
   \label{fig:qedcor_subfig3}
 }%
\end{subfloat}
\newline
 \begin{subfloat}[]{\includegraphics[width=.32\textwidth]{./Figures/qedcorr/h_etjetbinq2_CS_h_h2p3qed_corr_fac}
   \label{fig:qedcor_subfig4}
 }%
\end{subfloat}
 \begin{subfloat}[]{\includegraphics[width=.32\textwidth]{./Figures/qedcorr/h_etjetbinq2_CS_h_h2p4qed_corr_fac}
   \label{fig:qedcor_subfig5}
 }%
\end{subfloat}
 \begin{subfloat}[]{\includegraphics[width=.32\textwidth]{./Figures/qedcorr/h_etjetbinq2_CS_h_h2p5qed_corr_fac}
   \label{fig:qedcor_subfig6}
 }%
\end{subfloat}
\caption{QED multiplicative correction factors for inclusive-jet cross sections as functions of \etjetb in bins of \qsq.}
\label{fig:qedcorr}
\end{center}
\end{figure}
