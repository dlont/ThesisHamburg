An accurate description of the trigger efficiency is an important ingredient in this analysis. As described in Chapter~\ref{ch:expsetup}, the \zeus trigger was used to select hard \ep~collisions with high acceptance and to reject non-$\ep$ background. At the FLT, most of the trigger bits utilised CTD information to veto events characterised by specific combinations of all and vertex-fitted tracks. For example, events with large track-multiplicity and few tracks fitted to the primary vertex (corresponding event classes are 1, 2, 8) originate typically from beam-gas collisions and thus have to be discarded. Cosmic-ray events with low track multiplicity (corresponding to event class 3) are also typically excluded. The definitions of all event classes is illustrated in Figure~\ref{fig:trackvetodefinition}. This kind of trigger track-multiplicity requirements were generally called trigger track-veto.

 Two track-veto types were relevant for this analysis: ``semi-loose'' defined by the conditions: track-class $\le$ 2 or track class = 8 and track multiplicity $\ge$ 26, and ``tight'' track-veto requiring track-class $\le$ 2. 

In order to check the description of the track-veto in MC simulations, a monitor trigger was used. The FLT30 required an isolated electromagnetic cluster in the RCAL and therefore was independent of the CTD information. The track-veto efficiency, expressed as the ratio
\begin{equation}
 \epsilon_\mathrm{trk} = \frac{N\left(\text{track veto} \wedge \text{FLT30}\right)}{N\left(\text{FLT30}\right)},
\end{equation}
where $N\left(\text{FLT30}\right)$ is the number of events triggered by the trigger bit FLT30 and $N\left(\text{track veto} \wedge \text{FLT30}\right)$ is the number of events in a subset satisfying additional track-veto requirements, was studied separately in data and MC for different data-taking periods. To determine $N\left(\text{track veto} \wedge \text{FLT30}\right)$, the track veto was simulated offline by imposing additional restrictions on track quantities available at the FLT. The efficiency was investigated as a function of $y_{DA}$ because this variable was strongly correlated with the amount of hadronic activity and thus with the track multiplicity. Only for the 2006/2007~$e^+$ period was a significant discrepancy observed (see Figures~\ref{fig:tveffdatamc}\subref{fig:tveffdatamc_subfig1}--\subref{fig:tveffdatamc_subfig2}). The corresponding ratios in the data and MC are shown in Figures~\ref{fig:tveffdatamc}\subref{fig:tveffdatamc_subfig3}--\subref{fig:tveffdatamc_subfig4}. The disagreement between data and MC simulations was attributed to a bad description of the track-class distribution in the MC. In order to compensate for higher efficiency in the simulations, an additional correction was implemented. As the efficiency observed in MC was higher than that in the data it can be corrected by rejecting excess MC events. Therefore, for each MC event a uniformly distributed random number, $r$, was generated and the event was rejected if $r > f\left(y_{DA}\right)$. The quantity $f\left(y_{DA}\right)$ was obtained from a fit of the ratio of efficiencies in the data and MC to the function
\begin{equation} 
 f\left(y_{DA}\right)=a_0 + a_1 \cdot y_{DA}.
\end{equation}
For both MC generators, reasonable fit quality was obtained (see Figures~\ref{fig:tveffdatamc}\subref{fig:tveffdatamc_subfig3}--\subref{fig:tveffdatamc_subfig4}), however the quality of the description of the data after implementing the correction was more important.

 The correction was implemented in the \lepto and \ariadne samples, separately for different data-taking periods. As shown by the fits, the size of the correction depends approximately linearly on the value of $y_{DA}$ and, on average, was typically less than 0.5\% for 2004--2005~$e^-$ and 2006~$e^-$ samples and less than 3\% for the 2006-2007~$e^+$ sample. It was observed that for the ``semi-loose'' track-veto, the same correction as for ``tight'' track-veto can be applied. The comparison of the track-veto efficiencies in the data and simulations after applying the correction is illustrated in Figures~\ref{fig:aftveffdatamc}~\subref{fig:aftveffdatamc_subfig1}--\subref{fig:aftveffdatamc_subfig4}. After the correction the data efficiency was very well described by the MC. 

The systematic effects attributed to the MC track-veto correction were examined by investigating the trigger efficiency as a function of the CTD-FLT track multiplicity. The results of these studies are detailed in Section~\ref{subsec:systunc}.
\begin{figure}[t]
  \begin{center}
    \includegraphics[width=0.65\textwidth,trim={0 120 0 120},clip]{./Figures/classes96}
  \end{center}
  \caption{The definition of track veto classes (taken from~\protect\cite{YamazakiSite}).}
  \label{fig:trackvetodefinition}
\end{figure}

\begin{figure}[p!]
\begin{center}
\begin{subfloat}[]{\includegraphics[width=.45\textwidth,trim={0 0 0 0},clip,angle=-90] {./Figures/tvrew/tvrew_lep07p_yda_ltv}
   \label{fig:tveffdatamc_subfig1}
 }%
\end{subfloat}
 \begin{subfloat}[]{\includegraphics[width=.45\textwidth,trim={0 0 0 0},clip,angle=-90]{./Figures/tvrew/tvrew_ari07p_yda_ltv}
   \label{fig:tveffdatamc_subfig2}
 }%
\end{subfloat}
\newline
\begin{subfloat}[]{\includegraphics[width=.45\textwidth,trim={0 0 0 0},clip,angle=-90] {./Figures/tvrew/ratio_tvrew_lep07p_yda_ltv}
   \label{fig:tveffdatamc_subfig3}
 }%
\end{subfloat}
 \begin{subfloat}[]{\includegraphics[width=.45\textwidth,trim={0 0 0 0},clip,angle=-90]{./Figures/tvrew/ratio_tvrew_ari07p_yda_ltv}
   \label{fig:tveffdatamc_subfig4}
 }%
\end{subfloat}
\end{center}
\caption{Loose track-veto efficiency as a function of $y_{DA}$ in the data and \lepto MC (a) and \ariadne MC (b). Distributions of the ratio of the track-veto efficiency in data and \lepto MC (c), and data and \ariadne MC (d) and the results of the straight-line fits.}
\label{fig:tveffdatamc}
\end{figure}

%After correction
\begin{figure}[p!]
\begin{center}
\begin{subfloat}[]{\hspace{10pt}\includegraphics[width=.45\linewidth,trim={0 0 0 0},clip,angle=-90] {./Figures/tvrew/checktvrew_lep07p_yda_ltv}
   \label{fig:aftveffdatamc_subfig1}
 }%
\end{subfloat}
 \begin{subfloat}[]{\includegraphics[width=.45\linewidth,trim={0 0 0 0},clip,angle=-90]{./Figures/tvrew/checktvrew_lep07p_yda_sltv}
   \label{fig:aftveffdatamc_subfig2}
 }%
\end{subfloat}
\newline
\begin{subfloat}[]{\includegraphics[width=.45\linewidth,trim={0 0 0 0},clip,angle=-90] {./Figures/tvrew/checktvrew_ari07p_yda_ltv}
   \label{fig:aftveffdatamc_subfig3}
 }%
\end{subfloat}
 \begin{subfloat}[]{\includegraphics[width=.45\linewidth,trim={0 0 0 0},clip,angle=-90]{./Figures/tvrew/checktvrew_ari07p_yda_sltv}
   \label{fig:aftveffdatamc_subfig4}
 }%
\end{subfloat}
\end{center}
\caption{Comparison of the loose (a,c) and semi-loose (b,d) track-veto efficiency in data and MC after applying the track-veto correction.}
\label{fig:aftveffdatamc}
\end{figure}

