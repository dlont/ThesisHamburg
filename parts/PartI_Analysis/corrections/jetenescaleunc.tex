As  was mentioned previously, the precise determination of the jet energy-scale uncertainty is an important ingredient in many jet analyses, so the accuracy of the jet energy-scale corrections, described above, was investigated in a dedicated study. Assuming a valid description of the fraction of charged tracks inside a jet by MC simulations, the difference in hadronic energy scale in the data and MC was examined independently using the tracking information after the jet energy-scale correction, described above, was implemented. 

For jets with transverse energy in the laboratory frame $\etjetlab > 10\,\GeV$, the tracks in the tracking detectors acceptance region attributed to jets were identified according to their proximity to the jet axis. The ratio of transverse energy of the jet and the sum of traverse momenta of matched tracks 
\begin{equation}
r_\text{tracks} = \frac{\etjetlab}{\sum\limits_{\text{tracks}}{p_{T,i}}}
\end{equation}
was compared in the data and MC in different regions of $\etajetlab$ and for different running periods (see Figure~\ref{fig:ratcalibcontrolplotunc1}). Overall, the MC simulations provide a good description of the data in shape and central value, however the discrepancy between the mean values of the data and MC distributions remains. The double ratio $r_\text{tracks}^\text{DATA}/r_\text{tracks}^\text{MC}$ measured separately for different data-taking periods is illustrated in Figure~\ref{fig:ratcalibcontrolplotunc1}. The difference between the hadronic energy scale in data and MC does not exceed 1\%. This discrepancy was therefore taken into account in the systematic uncertainty. It was demonstrated in~\cite{joerg, php jets presentation unpublished} that \textcolor{blue}{jets with pseudorapidity outside the tracking acceptance region also amounts to $\pm 1\%$}. The energy-scale uncertainty for jets with transverse energy $3<\etjetlab<10\,\GeV$ was found~\cite{joerg} to be at least $\pm 3\%$.
\begin{figure}[h!]
	\centering
		\includegraphics[width=\textwidth]{./Figures/ensccorr/verification/c_contplot1_enscvar} 
	\caption{The ratio $r_\text{tracks} = \frac{\etjetlab}{\sum\limits_{\text{tracks}}{p_{T,i}}}$ and the mean values $\left\langle r_\text{tracks}\right\rangle$ in data and MC for different data-taking periods.}
	\label{fig:ratcalibcontrolplotunc}
\end{figure}

\begin{figure}[ht]
	\centering
		\includegraphics[width=0.95\textwidth]{./Figures/ensccorr/verification/c_contplot_enscvar} 
	\caption{The double ratio $\left\langle r^{DATA}_\text{tracks}\right\rangle/\left\langle r^{MC}_\text{tracks}\right\rangle$ for different data-taking periods. The hatched band indicates the attributed jet energy-scale uncertainty.}
	\label{fig:ratcalibcontrolplotunc1}
\end{figure}
