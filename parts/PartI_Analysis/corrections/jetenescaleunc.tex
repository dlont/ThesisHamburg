As mentioned previously, the precise determination of the jet energy-scale uncertainty is an important ingredient in many jet analyses, so the accuracy of the jet energy-scale corrections, described above, was investigated in a dedicated study. Assuming a valid description of the fraction of charged and neutral particles constituting a jet in MC simulations, the difference in hadronic energy scale in the data and MC was examined independently using the tracking information, after the jet energy-scale correction, described above, was implemented. 

For jets with transverse energy in the laboratory frame $\etjetlab > 10\,\GeV$, the tracks\footnote{A track was required to pass through at least three CTD superlayers and to be fitted to the primary vertex; transverse momentum of the track has to be $p_T > 300 \MeV$.} within the tracking-system acceptance attributed to jets were identified according to their proximity to the jet axis
\begin{equation}
R^2 = \left(\eta_{\mathrm{track}} - \eta_{\mathrm{jet}}\right)^2 + \left(\phi_{\mathrm{track}} - \phi_{\mathrm{jet}}\right)^2 < 1.
\label{eq:rtrackjetcut}
\end{equation}
The ratio of transverse energy of the jet, \etjetlab, and the sum of traverse momenta, $p_{T,i}$, of matched tracks 
\begin{equation}
r_\text{tracks} = \frac{\etjetlab}{\sum\limits_{\text{tracks}}{p_{T,i}}},
\end{equation}
was compared in the data and MC in different regions of $\etajetlab$ and for different running periods (see Figures~\ref{fig:ratcalibcontrolplotunc}). Overall, the MC simulations provide a good description of the data in shape, however the discrepancy between the mean values of the data and MC distributions remains. The double ratio $\left<r_\text{tracks}^\text{DATA}\right>/\left<r_\text{tracks}^\text{MC}\right>$ measured separately for different data-taking periods is illustrated in Figure~\ref{fig:ratcalibcontrolplotunc1}. The relative difference between the hadronic energy scale in data and MC does not exceed 1\%. This discrepancy was therefore taken into account in the systematic uncertainty. Exploiting the transverse momentum conservation in \ep collisions at \hera (see Section~\ref{sec:signalchar}), it was demonstrated in~\cite{thesis:behr:2010,thesis:perrey:2011} that jets with pseudorapidity outside the tracking acceptance region also contribute $\pm 1\%$. The energy-scale uncertainty for jets with transverse energy $3<\etjetlab<10\,\GeV$ was found~\cite{thesis:behr:2010} to be $\pm 3\%$.
\begin{figure}[h!]
	\centering
		\includegraphics[width=\textwidth]{./Figures/ensccorr/verification/c_contplot1_enscvar} 
	\caption{The distribution of the ratio $r_\text{tracks} = \frac{\etjetlab}{\sum\limits_{\text{tracks}}{p_{T,i}}}$ and of the mean values $\left\langle r_\text{tracks}\right\rangle$ in data and MC for different data-taking periods.}
	\label{fig:ratcalibcontrolplotunc}
\end{figure}

\begin{figure}[ht]
	\centering
		\includegraphics[width=0.95\textwidth]{./Figures/ensccorr/verification/c_contplot_enscvar} 
	\caption{The double ratio $\left\langle r^{DATA}_\text{tracks}\right\rangle/\left\langle r^{MC}_\text{tracks}\right\rangle$ for different data-taking periods. The hatched band indicates the attributed jet energy-scale uncertainty.}
	\label{fig:ratcalibcontrolplotunc1}
\end{figure}
