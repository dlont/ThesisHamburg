\begin{figure}[h!]
\begin{center}
\begin{subfloat}[]{\includegraphics[width=0.45\linewidth,trim={0 0 0 0},clip,angle=-90] {./Figures/eleenescale/ele_enescale_rat}
   \label{fig:ele_enescale1}
 }%
\end{subfloat}
\begin{subfloat}[]{\includegraphics[width=0.45\linewidth,trim={0 0 0 0},clip,angle=-90]{./Figures/eleenescale/ele_enescale_unc}
   \label{fig:ele_enescale_2}
 }%
\end{subfloat}
\end{center}
\caption{(a) Comparison of $E_\text{SI}/E_\text{DA}$ in data and Monte Carlo. (b) Difference between the electromagnetic energy scale in data and Monte Carlo simulations as a function of electron energy.}
\label{fig:ele_enescale}
\end{figure}

The pre-processing of the data with ORANGE/ PHANTOM libraries includes dead-material and non-uniformity corrections in the electron identification algorithms. Nevertheless a residual discrepancy between data and MC simulations was observed. In order to study this discrepancy in detail the samples data and MC events satisfying the requirements described in Chapter~\ref{selection} were used.

The resolution and electromagnetic energy scale were investigated in data and MC by taking the ratio $E_\text{SI}/E_\text{DA}$, where $E_\text{SI}$ is the measured electron energy including all corrections and $E_\text{DA}$ the energy measured by the double-angle method. 

In each bin of $E_\text{DA}$ a Gaussian fit to the $E_\text{SI}/E_\text{DA}$ distribution was performed. The mean value extracted from the fit was plotted as a function of $E_\text{DA}$. The results of the fits are shown in Figure~\ref{fig:ele_enescale_unc}. It was found that the difference between the absolute energy scales in data and Monte Carlo simulations is less than 2\%. This discrepancy was taken into account as a systematic uncertainty as described in Section~\ref{subsec:systunc}.

%\begin{figure}
	%\centering
		%\includegraphics[width=0.45\textwidth]{./Figures/eleenescale/ele_enescale_unc}
	%\caption{Difference between the electromagnetic energy scale in data and Monte Carlo simulations as a function of electron energy.}
	%\label{fig:ele_enescale_unc}
%\end{figure}
