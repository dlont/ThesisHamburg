The pre-processing of the data with \orange/\PHANTOM libraries includes dead-material and non-uniformity corrections in the electron identification algorithms. Nevertheless a residual discrepancy between data and MC simulations was observed. In order to study this discrepancy in detail the data and MC events satisfying the requirements described in Chapter~\ref{ch:selection} were used.

The resolution and electromagnetic energy scale were investigated in data and MC by taking the ratio $E_\text{SI}/E_\text{DA}$, where $E_\text{SI}$ is the measured electron energy including all corrections and $E_\text{DA}$ the energy measured by the double-angle method (see Section~\ref{subsec:da}). As was explained, the electron energy determined using the double-angle method is approximately independent of the absolute energy scale of the calorimeter and therefore was used as a reference scale for the comparison. As shown in Figure~\ref{fig:ele_enescale}~\subref{fig:ele_enescale1} the distribution has a Gaussian-like shape with the full width at half maximum of the data and \lepto distributions of about 10\%. In general, the simulations adequately describe the shape of the data distribution, however a systematic shift of the mean value was observed.

To investigate this shift in more details in each bin of $E_\text{DA}$, a Gaussian fit to the $E_\text{SI}/E_\text{DA}$ distribution was performed. The mean value extracted from the fit was plotted as a function of $E_\text{DA}$ (see Figure~\ref{fig:ele_enescale}\subref{fig:ele_enescale_2}). It was found that the double ratio between the absolute $E_\text{SI}$ energy scales in data and Monte Carlo simulations deviated from unity by less than 2\%. This discrepancy was taken into account as a systematic uncertainty as described in Section~\ref{subsec:systunc}.

\begin{figure}[p!]
\begin{center}
\begin{subfloat}[]{\includegraphics[height=0.35\textheight] {./Figures/eleenescale/ele_enescale_rat}
   \label{fig:ele_enescale1}
 }%
\end{subfloat}
\newline
\begin{subfloat}[]{\hspace{-100pt}\includegraphics[height=0.45\textheight]{./Figures/eleenescale/ele_enescale_unc}
   \label{fig:ele_enescale_2}
 }%
\end{subfloat}
\end{center}
\caption{Comparison of $E_\text{SI}/E_\text{DA}$ in data and Monte Carlo (a). Double ratio between the electromagnetic energy scale in data and Monte Carlo simulations as a function of electron energy (b).}
\label{fig:ele_enescale}
\end{figure}

%\begin{figure}
	%\centering
		%\includegraphics[width=0.45\textwidth]{./Figures/eleenescale/ele_enescale_unc}
	%\caption{Difference between the electromagnetic energy scale in data and Monte Carlo simulations as a function of electron energy.}
	%\label{fig:ele_enescale_unc}
%\end{figure}
