\label{sec:polcor}
\begin{figure}[h]
 \begin{center}
 \includegraphics[width=0.33\textwidth,bb= 0 0 567 544]{./Figures/polrew/05e}
 % 05e.eps: 0x0 pixel, 300dpi, 0.00x0.00 cm, bb= 0 0 567 544
\end{center}
\caption{The polarisation reweighting factors for 2004/2005$e^-$ data taking period determined using the HECTOR program. The red curve represents the spline interpolation.}
\label{fig:polcor05e}
\end{figure} 
The MC samples used in this analysis were generated assuming vanishing polarisation of the lepton beam, $P_e = 0$. In order to take non-zero polarisation of the electrons into account, the MC samples were reweighted using theoretical predictions. For this purpose the HECTOR program~\cite{HECTOR} interfaced to BASES~\cite{bases} with the CTEQ5D PDFs~\cite{CTEQ5} was used. The reweighting factors were determined from the ratio of predictions for the unpolarised inclusive DIS cross sections and those for the lepton beam polarisation corresponding to particular data-taking period. The polarisation correction was implemented as weight assigned to each MC event according to the \qsq~of the scattering process:
\begin{equation}
 w_p\left(\qsq\right) = w_p = \frac{\sigma_\mathrm{pol}}{\sigma_\mathrm{unpol}}.
\end{equation}

The average polarisation for different data-taking periods is summarised in Table~\ref{tab:polvalues}.
\begin{table}[h]
 \centering
 \begin{tabular}{lc}
 Data-taking period & Average polarisation, $P_e$ \\
\hline
 2004/2005$e^-$   & -0.06184 \\
 2006$e^-$   & 0.09386  \\
 2006/2007$e^+$ & -0.06857
\end{tabular} 
\caption{The average polarisation values for the data samples used in the analysis.}
\label{tab:polvalues}
\end{table}
The obtained correction factors as a function of \qsq~and a spline interpolation is illustrated in Figure~\ref{fig:polcor05e}. The size of the correction increases with increasing \qsq~but nowhere exceeds 3\% and typicaly is below 1\%. The sign of the correction depends on the helicity of the lepton beam.

Besides the polarisation correction applied to MC events, an inverse of determined factors was applied to the data in order to obtain jet cross sections corresponding to unpolarised lepton scattering.