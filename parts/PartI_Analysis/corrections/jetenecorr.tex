Hadrons loose on average up to 20\% of their energy in uninstrumented material before reaching the CAL therefore the jet energy measured by summing the associated energy deposits in the calorimeter cells usually differs from the original jet energy. This effect may lead to the systematic migrations of jets to the cross section bins with lower energy. In principle, such effects must be taken into account in the unfolding procedure. Nevertheless, in order to minimise jet migrations and possible bias from the energy loss in inactive detector medium a dedicated jet energy correction was employed in this analysis.

Two common approaches for correcting the jet energy loss exist:
\begin{itemize}
 \item the \emph{bottom-up} approach consists of correcting the energy of the input objects (i.e. calorimeter cells in this analysis) to compensate for the energy loss and then use the corrected objects as an input to the jet algorithm.
 \item in the \emph{top-bottom} approach the energy of identified jets is corrected directly.
\end{itemize}
The two approaches are equivalent if the unfolding of the measured cross sections is performed afterwards. In this thesis the top-bottom approach was adopted.

In order to estimate the amount of energy loss, the MC simulations were used because they provide the detailed information about hadron propagation in the detector volume. The measured jet energy, $E_T^{jet,det}$, depends approximately linearly of the 'true', $E_T^{jet}$, value, however the size of the energy loss depends on the thickness of the traversed material and therefore on the jet pseudorapidity in the laboratory frame. For this reason the complete $-1<\etajetlab<2.5$ measurement region was divided into 14 equidistant regions and the correction was determined separately in each $\etajetlab$ bin.

The procedure proceeds as follows:
\begin{itemize}
 \item In the MC events accepted at the generated and reconstructed levels the pairs of jets were identified according to the distance between jets in the $\eta-\phi$ plane. A hadron level jet was matched to the reconstructed jet if the distance $r=\sqrt{\left[\left(\eta_{had}-\eta_{det}\right)^2 + \left(\phi_{had}-\phi_{det}\right)^2\right]}$ between the two was small $r<0.7$ and no further jets were reconstructed within the cone. In order to avoid any bias on the correction procedure introduced by the boundaries of the jet phase space, the cuts on the jet transverse energy were loosen to $\etjetbhad>6\,\GeV$ and $\etjetbdet>3\,\GeV$ at the hadron and detector levels, respectively.
 \item In each bin $i$ of $\etajetlab$ a linear fit $\left\langle\etjetbdet\right\rangle = a_i + b_i\cdot\left\langle\etjetbhad\right\rangle$ was performed, where $\left\langle\etjetbdet\right\rangle$ and $\left\langle\etjetbhad\right\rangle$ were determined using the matched jet sample.
 \item The corrects were determined using the \ariadne and \lepto sample and for each data taking period separately. 
\end{itemize}
The fits are illustrated in Figure~\ref{fig:05e_lepto_coeffitients}.
\begin{figure}[p]
\centering
\includegraphics[width=\linewidth]{./Figures/etcorr/coeff/05e_lepto_coeffitients}
\caption{The average measured detector level jet transverse energy $\left\langle\etjetbdet\right\rangle$ as a function of $\left\langle\etjetbhad\right\rangle$ and the corresponding straight-line fits in different regions of $etajetlab$ for the data-taking period 2004--2005$e^-$ in the \lepto MC sample.}
\label{fig:05e_lepto_coeffitients}
\end{figure}

Given the extracted fit parameters the components of jets four-momentum were scaled such that the following relation was obtained:
 \begin{equation}
  E_{T,B}^{jet,det} \mapsto E_{T,B}^{jet,corr} = \frac{E_{T,B}^{jet,det} - a_i}{b_i}.
 \end{equation}
Because the analysis was performed in the Breit frame the jet pseudorapidity in the laboratory frame was recalculated and the corresponding correction factors were applied. Assuming valid description of the detector effects in the simulations, the correction was applied to both the data and MC jets thus introducing the dependence on the parton shower simulation approach. Therefore later in the unfolding procedure the respective data and MC samples were utilised. In the MC simulations the correction was applied on top of that introduced in Section~\ref{sec:jetcalib}. The described correction on average improved the correlation between the reconstructed and generated level quantities.
