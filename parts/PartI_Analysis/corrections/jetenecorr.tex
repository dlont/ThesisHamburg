In general hadrons loose on average upto 20\% of their energy in uninstrumented material in front of the CAL. Thus the energy of hadronic jets measured by summing the associated energy deposits in the calorimeter cells usualy differs from the jet orinal energy. In principle, such effects must be taken into account in the unfolding procedure. However, in order to minimise possible bias from the energy loss in inactive detector medium a dedicated jet energy correction was employed. 

For the correction of the energy loss of the jets two common approaches exist:
\begin{itemize}
 \item the \emph{bottom-up} approach consists of correcting the energy of the input objects (i.e. calorimeter cells energy in this analysis) to compensate for energy loss and then use the corrected objects as an input to the jet algorithm.
 \item in the \emph{top-bottom} approach the jet energy is directly corrected to minimise the difference between the measured and 'true' energy.
\end{itemize}
The second approach was accepted in this thesis. The MC simulations provide the details of hadron propagation in the detector volume, hence the MC samples were used to determine neccessary correction factors 

The size of the correction depends on the thickness of traversed material and the therefore on the pseudorapidity of the jet.