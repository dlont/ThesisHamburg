As described in Section~\ref{subsec:polarisation}, at the interaction point the lepton beam is longitudinally polarised. In order to determine the lepton beam polarisation, two independent detectors LPOL~\cite{nim:a479:334} and TPOL~\cite{nim:a329:79} were used.
The dependence of the Compton cross section on the orientation of the electron spin was utilised. In both cases a laser system was used as a source of incident photons.

Circularly polarised continuous green argon-ion laser light was used for the measurement of the transverse polarisation. The photons from the laser beam collided with the transversely polarised electrons. The scattered photons were detected in a dedicated calorimeter. The asymmetry of the photon-scattering-angle distribution was used to determine the polarisation value.

A similar measurement was performed with longitudinally polarised electrons. A Nd:YAG laser pulse was transported to the collision region, where circularly polarised photons backscatter from the electron beam. Switching the circular polarisation of the photon beam from left-handed to right-handed, the asymmetry in the total energy of the scattered photon was determined and the electron beam polarisation value was derived.
