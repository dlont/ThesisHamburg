At ZEUS for the determination of instantaneous luminosity the $\ep$-bremsstrahlung process 
\begin{equation}
	e + p \rightarrow e' + \gamma + p
\end{equation}
was used. Large cross section of this process allows rapid accumulation of large event sample in a relatively short time. Furthermore, the predictions for this reaction have better than $0.5\%$ precision. The detection of the photon or the electron emerging from the interaction was used as an experimental signature of this process. The schematic layout of the luminosity monitor is shown in Figure~\ref{fig:lumi_monitor_layout}.
\begin{figure}
	\centering
		\includegraphics{./Figures/lumi_monitor_layout.png}
	\caption{Layout of the ZEUS luminosity monitor}
	\label{fig:lumi_monitor_layout}
\end{figure}
The photons from the reaction are emitted at small angles $\theta \lesssim 0.5$ mrad and leave the beam-pipe through a thin window located at $92.5$ m distance from the interaction point. Approximately $9\%$ of the photons were converting into $e^+e^-$ pairs. Electrons and positrons were deflected by a dipole magnet into the luminosity spectrometer (SPEC), which was installed at 104 m distance downstream. The remaining photons were detected in the photon calorimeter (PCAL) located at 107 m.

The instantaneous luminosity was determined from the formula:
\begin{equation}
\mathcal{L} = \frac{R_{\ep\rightarrow \ep\gamma}}{ \sigma_{\ep\rightarrow \ep\gamma}}.
\end{equation}
The background from the interaction of electrons with residual beam gas was estimated using electron pilot bunches and subtracted in the calculations.

The information from SPEC was not available for all runs, while the PCAL was functioning continuously, therefore the luminosity value determined using the PCAL on a run-by-run basis was used. The systematic uncertainty on the measurement obtained with SPEC was lower than that for PCAL. It amounts to $1.8\%$ and was used as the resulting precision of the luminosity value.

