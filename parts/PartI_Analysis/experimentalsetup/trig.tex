Due to limited data processing speed and storage capabilities, not every $\ep$ collision could be recorded. Furthermore, the total event sample is dominated by non-$\ep$ background from beam-gas interactions or cosmic-ray showers. In order to reduce the event rate to an acceptable level and efficiently reject the background, a sophisticated three-level trigger system~\cite{Smith:1992im,nim:a379:542,Carlin:1995rv} was used. The architecture of the \zeus trigger system is shown in Figure~\ref{fig:trigarch}. 

\subsection{First Level Trigger}
\label{subsec:flt} Each detector component was coupled to its own hardware first-level trigger operating with general information such as regional energy sums (CAL)~\cite{nim:a355:278}, track multiplicity (CTD)~\cite{nim:a315:431} or muon tracks. The information from 26 consecutive bunch crossings (2.5 $\mu$s) was stored in a 46-event-deep pipe-line and was analysed in parallel streams. The combined information from each detector component was sent to the programmable Global First Level Trigger (GFLT), which selected the events that should be kept for the consideration at the second level. The decision was taken within 1.9 $\mu$s, which corresponds to 20 bunch crossings. If an event was accepted, the analogue information from different detector components was digitised and transferred from the pipelines to the data buffers. The GFLT had 64 bits, so-called ``slots'', corresponding to different event categories (see Section~\ref{subsec:fltcuts} for the description of the FLT slots used in the analysis). By using the FLT the event rate was reduced to approximately 1 kHz. The time interval during which data taking was disabled while the FLT was processing detector information and therefore was not operational amounted to approximately 1\% and was automatically accounted for by disabling the luminosity monitors when the trigger was busy.

\subsection{Second Level Trigger}
\label{subsec:slt}
The second-level trigger~\cite{Allfrey:2007zz} had more time to process information because events were arriving at a reduced rate. The information from each detector component was combined at the GSLT~\cite{upub:abbiendi:zn99063,upub:chlebana:zn94102,Uijterwaal:1992xc}, which was based on a reconfigurable network of transputers. The additional time available for the SLT allowed a better estimation of the position of the primary vertex, identification of  calorimeter clusters and reconstruction of tracks. The timing information from the calorimeter system was used to reject non-$\ep$ background events efficiently at the SLT. The output rate of the second level was in range 50 -- 100 Hz. The full information for accepted events was sent on to the event builder.

\subsection{Third Level Trigger}
\label{subsec:tlt}
The TLT~\cite{Bailey:1992iq,Bhadra:1989kz} was a cluster of computer servers running complex algorithms for the vertex reconstruction, electron identification and reconstruction of event kinematic variables. The highly configurable architecture of the TLT made it possible to utilise of up-to-date calibration information as well as fine tuning of the selection algorithms. The output rate of the TLT of about 5 Hz was compatible with the storage capabilities, thus the information from the event builder was converted in ADAMO format~\cite{Hart:1990dz} and written on magnetic tape.

\begin{figure}[h]
	\centering
		\includegraphics[height=0.9\textheight]{./Figures/ZEUStrigger_syst.jpg}
	\caption{The \zeus trigger-system architecture.}
	\label{fig:trigarch}
\end{figure}
\newpage
