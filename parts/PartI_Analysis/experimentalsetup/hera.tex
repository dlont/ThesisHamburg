The Hadron-Electron Ring Accelerator (HERA), the only $\ep$ collider in the world, was built in Hamburg, Germany at the national accelerator research centre DESY. At HERA electrons\footnote{In what follows, the term electron is used for interchangeably for both electrons and positrons, unless otherwise mentioned.} or positrons of energy $27.5$ \GeV\, collided with protons of energy up to 920 \GeV, resulting in a centre-of-mass energy up to $\sqs = 319\, \GeV$. Four experiments took data at different interaction points along the HERA ring. The ZEUS and H1 experiments, devoted to the study of internal structure of the proton and searches for phenomena beyond the Standard Model, were operating with colliding beams. The HERMES experiment, dedicated to the investigation of the spin structure of nucleons, was a fixed-target experiment utilising the electron beam only, whereas HERA-B was using the proton beam aiming at the measurement of CP-violation in the $B\bar{B}$-system. The HERA machine was operating during 96--2007 years with a shut-down in 2000--2001. This shut-down marks the separation between the, so called, HERA I and HERA II data-taking periods.

The acceleration of electrons to their nominal energies was achieved in several stages. A schematic view of the acceleration chains is presented in Figure~\ref{fig:hera_acceleration}. Electrons were initially accelerated in LINAC I/II to 200 \MeV. After injection into the DESY II synchrotron the electron energy was increased up to 7.5 \GeV. Then, after reaching 12 \GeV\, in PETRA, electrons were finally transported to HERA. The positron beam was obtained from the conversion of the bremsstrahlung emission of electrons. 

The proton beam was obtained in several steps from a $\text{H}^-$ ion source. At the first stage 50 \MeV\, ions from a LINAC were transported to DESY III, where they undergo acceleration to 7.5 \GeV\, and stripping off the electrons. Later, after achieving energy of 40 \GeV\, in the PERTRA ring, the protons were injected into HERA.

\begin{figure}[htpb]
	\centering
	\begin{subfloat}[]{\includegraphics[width=0.49\linewidth]{./Figures/hera_acceleration_large.png}
			\label{fig:hera_acceleration_large}
	 }%
	\end{subfloat}
	\begin{subfloat}[]{\includegraphics[width=0.9\linewidth]{./Figures/hera_acceleration_zoom.png}
			\label{fig:hera_acceleration_zoom}
	 }%
	\end{subfloat}
	\caption{Schematic view of electrons and protons acceleration chains.}
\label{fig:hera_acceleration}
\end{figure}

\subsection{Beam Structure}
\label{subsec:beamstruct}
The usage of Radio-Frequency acceleration cavities at HERA lead to a disctinct time structure of the beams. Protons and electrons were grouped into bunches separated by $\sim 28.8$ m, which corresponds to $96$ ns time interval. Not all bunches were filled. The so called pilot bunches, for which either electron or proton ``bucket'' have not been filled, were used for the study of the interaction of the beam with residual gas in the vacuum beam pipe. The empty bunches with both positrons and electrons ``buckets'' empty, were used for the study of the cosmic event rate and other non-$\ep$ background.

\begin{figure}[t]
	\centering
		\includegraphics[width=.5\textwidth]{./Figures/lumi00_0.png}
	\caption{HERA delivered luminosity}
	\label{fig:lumi00_0}
\end{figure}

\subsection{Luminosity}
\label{subsec:luminosity}
The crucial parameter of the collider that determines the rate of the collisions is the luminosity. It is related to the cross section, $\sigma$, of a process via the following expression:
\begin{equation}
	R = \mathcal{L}\sigma,
\end{equation}
where $\mathcal{L}$ is the instantaneous luminosity and $R$ is the reaction rate. The luminosity is related to the parameters of the colliding beams:
\begin{equation}
	\mathcal{L} = f\frac{n_1n_2}{4\pi\sigma_x\sigma_y},
\end{equation}
where $f$ is the bunch crossing rate, $n_1,\,n_2$ are the numbers of particles in the bunches and $\sigma_x$, $\sigma_y$ - the width parameters for beams with Gaussian profiles. An increase of the luminosity at HERA was achieved during the shut-down mainly by reducing the transverse size of the beams installing additional magnets close to the interaction points. 

To relate the number of events, $N$, to the reaction rate the instantaneous luminosity has to be integrated:
\begin{equation}
	N = \int{R\,\mathrm{dt}} = \sigma\int{\mathcal{L}\,\mathrm{dt}} = \sigma L,
\end{equation}
where $L$ is called integrated luminosity and is often used to denote the amount of collected data. The comparison of increase of the integrated luminosity during HERA I and HERA II running periods is presented in Figure~\ref{fig:lumi00_0}. A summary of the values of delivered luminosity for the data-taking periods relevant for this analysis is given in Table~\ref{tab:heraruns}.

\begin{table}
	\centering
		\begin{tabular}[h]{|c|c|c|c|}
		  \hline
			Period & Lepton Type & Delivered Luminosity & Analysed Luminosity \\
			\hline \hline
			2003 -- 2004 & $e^{+}$  &   & \\
			2005           & $e^{-}$   &   & \\
			2005           & $e^{-}$   &   & \\
			2006 -- 2007 & $e^{+}$  &   & \\
			\hline
		\end{tabular}
	\caption{Information about HERA running periods used in the analysis}
	\label{tab:heraruns}
\end{table}

\subsection{Polarisation}
\label{subsec:polarisation}
The spin of the electron is naturally transversely polarised at the storage rings like HERA due to Sokolov-Ternov effect. Between the HERA I and HERA II data-taking periods the accelerator was upgraded to provide longitudinally polarised lepton beams at the ZEUS and H1 experiments. The characteristic polarisation set-up time for HERA is given in Table~\ref{tab:HERAParameters}. To obtain longitudinaly polarised beams at the interaction points, a chain of horizontal and vertical dipole magnets (spin rotators) were installed on either side of ZEUS and H1. A typical longitudinal polarisation value of 30\% -- 40\% was achieved.

 The most important parameters of the upgraded HERA storage ring are summarised in Table~\ref{tab:HERAParameters}.
\begin{table}[htbp]
	\centering
		\begin{tabular}[h]{|c|c|c|}
			\hline
			Parameter       & Electron beam   & Proton Beam \\
			\hline \hline
			Energy            &     $27.5$ \GeV  & $920$ \GeV \\
			Beam Current  &     $60$ mA       & $160$ mA \\
			Particle per bunch & $3.5\times 10^{10}$ & $10^{11}$ \\
			Maximum number of bunches & 210 & 210 \\
			Bunch length &  $7.8$ \mm & $110$ -- $150$ \mm \\
			Beam size     &  $112\times 30$ $\mm^2$ & $112\times 30$ $\mm^2$ \\
			Polarisation time & $30$ min & -- \\
			\hline
		\end{tabular}
	\caption{The HERA storage ring parameters.}
	\label{tab:HERAParameters}
\end{table}


