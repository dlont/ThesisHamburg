The Hadron-Electron Ring Accelerator (\hera), the only $\ep$ collider in the world, was built in Hamburg, Germany at the national accelerator research centre \desy. At \hera, electrons\footnote{In what follows, the term ``electron'' is used for both electrons and positrons, unless otherwise mentioned.} or positrons of energy $27.5$ \GeV\, collided with protons of energy up to 920 \GeV, resulting in a centre-of-mass energy up to $\sqs = 319\; \GeV$. Four experiments took data at different interaction points along the \hera ring. The \zeus and \hone experiments, devoted to the study of the internal structure of the proton and searches for phenomena beyond the Standard Model, were operating with colliding beams. The \hermes experiment, dedicated to the investigation of the spin structure of nucleons, was a fixed-target experiment utilising the electron beam only, whereas \hera-B used only the proton beam, aiming at the measurement of CP-violation in the $B\bar{B}$-system. The \hera machine operated during the period 1996--2007 with a shut-down in 2000--2002. This shut-down marks the separation between the so called \hera I and \hera II data-taking periods.

The acceleration of electrons to their nominal energies was achieved in several stages. A schematic view of the acceleration chains is presented in Figure~\ref{fig:hera_acceleration}. Electrons were initially accelerated in \linac I/II to 200 \MeV. After injection into the \desy II synchrotron the electron energy was increased up to 7.5 \GeV. Then, after reaching 12 \GeV\, in \petra, electrons were finally transported to \hera. The positron beam was obtained by pair production from bremsstrahlung emission of electrons. 

The proton beam was obtained in several steps from a $\text{H}^-$ ion source. At the first stage 50 \MeV\, ions from a \linac were transported to \desy III, where they underwent acceleration to 7.5 \GeV\, and stripping off the electrons. Later, after achieving an energy of 40 \GeV\, in the \petra ring, the protons were finally injected into \hera.

\begin{figure}[htpb]
	\centering
	\begin{subfloat}[]{\includegraphics[height=0.49\textheight]{./Figures/hera_acceleration_large.png}
			\label{fig:hera_acceleration_large}
	 }%
	\end{subfloat}
	\begin{subfloat}[]{\includegraphics[height=0.49\textheight]{./Figures/hera_acceleration_zoom.png}
			\label{fig:hera_acceleration_zoom}
	 }%
	\end{subfloat}
	\caption{Schematic view of electron and proton acceleration chains.}
\label{fig:hera_acceleration}
\end{figure}

\subsection{Beam Structure}
\label{subsec:beamstruct}
The usage of Radio-Frequency acceleration cavities at \hera lead to a disctinct time structure of the beams. Protons and electrons were grouped into bunches separated by $\sim 28.8$ m, which corresponds to $96$ ns time intervals. Not all bunches were filled. The so called pilot bunches, for which either the electron or proton ``bucket'' was not filled, were used for the study of the interaction of the beam with residual gas in the beam vacuum pipe. Bunches in which both proton and electron ``buckets'' were empty were used for the study of the cosmic event rate and other non-$\ep$ background.

\begin{figure}[t]
	\centering
		\includegraphics[width=.5\textwidth]{./Figures/lumi00_0.png}
	\caption{\hera delivered luminosity}
	\label{fig:lumi00_0}
\end{figure}

\subsection{Luminosity}
\label{subsec:luminosity}
The crucial parameter of the collider that determines the rate of the collisions is the luminosity. It is related to the rate, $R$, of a process via the following expression:
\begin{equation}
	R = \mathcal{L}\sigma,
\end{equation}
where $\mathcal{L}$ is the instantaneous luminosity and $\sigma$ is the cross section. The luminosity is related to the parameters of the colliding beams:
\begin{equation}
	\mathcal{L} = f\frac{n_1n_2}{4\pi\sigma_x\sigma_y},
\end{equation}
where $f$ is the bunch-crossing rate, $n_1,\,n_2$ are the numbers of particles in the bunches and $\sigma_x$, $\sigma_y$ the width parameters for beams with Gaussian profiles. An increase of the luminosity~\cite{hera-98-05} at \hera II was achieved mainly by reducing the transverse size of the beams by installing additional focusing magnets close to the interaction points. 

To relate the number of events, $N$, to the reaction rate, the instantaneous luminosity has to be integrated:
\begin{equation}
	N = \int{R\,\mathrm{dt}} = \sigma\int{\mathcal{L}\,\mathrm{dt}} = \sigma L,
\end{equation}
where $L$ is called the integrated luminosity and is often used to denote the amount of collected data. A comparison of the increase of the delivered integrated luminosity during the \hera I and \hera II running periods is presented in Figure~\ref{fig:lumi00_0}. A typical fraction of 60\% of the delivered integrated luminosity was available for the data analysis. The other 40\% were lost due to various reasons: not fully operational detector components, inefficiencies of the data-acquisition system, specific trigger problems etc. A summary of the values of delivered and gated (recorded physical events) luminosity for the data-taking periods relevant for this analysis is given in Table~\ref{tab:heraruns}.

\begin{table}
	\centering
		\begin{tabular}[h]{|c|c|c|c|}
		  \hline
			Period & Lepton Type & Delivered Luminosity & Gated Luminosity \\
			\hline \hline
			2004 -- 2005 & $e^{-}$  & 204.8 \invpb  & 152.26 \invpb \T\B\\
			2006         & $e^{-}$  & 86.1 \invpb  &  61.23 \invpb \T\B\\
			2006 -- 2007 & $e^{+}$  & 180.54 \invpb  & 145.9 \invpb \T\B\\
			\hline
		\end{tabular}
	\caption{Information about \hera running periods used in the analysis.}
	\label{tab:heraruns}
\end{table}

\subsection{Polarisation}
\label{subsec:polarisation}
The spin of the electron is naturally transversely polarised at storage rings like \hera due to the Sokolov-Ternov effect~\cite{Sokolov:1963zn,Baier:1969hw}. Between the \hera~I and \hera~II data-taking periods the accelerator was upgraded to provide longitudinally polarised lepton beams at the \zeus and \hone experiments. The characteristic polarisation build-up time for \hera was about 40 min. To obtain longitudinally polarised beams at the interaction points, a chain of horizontal and vertical dipole magnets (spin rotators~\cite{Barber:1994ew}) were installed on either side of \zeus and \hone. A typical longitudinal polarisation value of 30\% -- 40\% was achieved.

 The most important parameters of the upgraded \hera storage ring are summarised in Table~\ref{tab:HERAParameters}.
\begin{table}[htbp]
	\centering
		\begin{tabular}[h]{|c|c|c|}
			\hline
			Parameter       & Electron beam   & Proton Beam \\
			\hline \hline
			Energy            &     $27.5$ \GeV  & $920$ \GeV \\
			Beam Current  &     $60$ mA       & $160$ mA \\
			Particle per bunch & $3.5\times 10^{10}$ & $10^{11}$ \\
			Maximum number of bunches & 210 & 210 \\
			Bunch length &  $7.8$ \mm & $110$ -- $150$ \mm \\
			Beam size     &  $112\times 30$ $\mm^2$ & $112\times 30$ $\mm^2$ \\
			Polarisation time & $30$ min & -- \\
			Maximum instantaneous luminosity & $5\times 10^{31}$ cm$^{-2}$s$^{-1}$ & -- \\
			\hline
		\end{tabular}
	\caption{The \hera storage ring parameters.}
	\label{tab:HERAParameters}
\end{table}


