The schematic layout of the ZEUS detector is presentd in Figure~\ref{fig:zeus2d1}. A right-handed Cartesian coordinate system with the origin at the nominal interaction point was adopted in ZEUS. The incoming proton momentum vector defines the positive Z direction. It was also called ``forward'' direction. The positive direction of the X-axis was pointing towards the centre of the HERA ring, while the positive Y-axis was pointing upwards. The azimuthal angle, $\phi$, was defined in the transverse X--Y plane, while the polar angle was measured with respect to the +Z axis. 
\begin{landscape}
\begin{figure}[htpb]
	\centering
		\includegraphics[angle=-90,width=\linewidth]{./Figures/zeus2d1.png}
	\caption{Z-Y cut of the ZEUS detector}
	\label{fig:zeus2d1}
\end{figure}
\end{landscape}

The pseudorapidity was an important quantity used in this analysis and defined by the following expression:
\begin{equation}
\eta = -\ln \tan \dfrac{\theta}{2}.
\end{equation}
The pseudorapidity has simple transformation properties under Lorentz boost. The difference $\Delta \eta$ is a Lorentz-invariant. The numerical value of pseudorapidity coincides with normal rapidity for massless particles.
\begin{figure}[htpb]
	\centering
		\includegraphics[width=0.5\textwidth]{./Figures/zeus_coordsyst.jpg}
	\caption{The ZEUS coordinate system.}
	\label{fig:zeus_coordsyst}
\end{figure}
