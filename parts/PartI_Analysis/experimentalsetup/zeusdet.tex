The schematic layout of the \zeus detector is presented in Figure~\ref{fig:zeus2d1}. A right-handed Cartesian coordinate system (CS) with the origin at the nominal interaction point was adopted in \zeus (see Figure~\ref{fig:zeus_coordsyst}). The incoming proton momentum vector defines the positive Z direction. It was also called the ``forward'' direction. The positive direction of the X-axis pointed towards the centre of the \hera ring, while the positive Y-axis pointed upwards. The azimuthal angle, $\phi$, was defined in the transverse X--Y plane, while the polar angle was measured with respect to the +Z axis. 
\begin{landscape}
\begin{figure}[htpb]
	\centering
		\includegraphics[angle=90,width=\linewidth]{./Figures/ZEUS_2000_topview.png}
	\caption{Top view of the \zeus detector}
	\label{fig:zeus2d1}
\end{figure}
\end{landscape}

The pseudorapidity was an important quantity used in this analysis and defined by the following expression:
\begin{equation}
\eta = -\ln \tan \dfrac{\theta}{2}.
\end{equation}
The pseudorapidity has simple transformation properties under Lorentz boosts. The difference $\Delta \eta$ is a Lorentz invariant. For massless particles the numerical value of the pseudorapidity coincides with the normal rapidity defined as 
\begin{equation}
y=\lim_{m\rightarrow 0}\frac{1}{2}\ln{\left(\frac{E+p_Z}{E-p_Z}\right)}\approx \lim_{m\rightarrow 0}\frac{1}{2}\ln{\frac{\cos^2{\theta/2}+m^2/4p^2}{\sin^2{\theta/2}+m^2/4p^2}} = -\ln \tan \dfrac{\theta}{2} = \eta.
\end{equation}
	
\begin{figure}[htpb]
	\centering
		\includegraphics[width=0.5\textwidth]{./Figures/zeus_coordsyst.jpg}
	\caption{The \zeus coordinate system.}
	\label{fig:zeus_coordsyst}
\end{figure}
