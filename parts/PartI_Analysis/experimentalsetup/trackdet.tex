The main parts of the tracking system of the ZEUS detector were of the silicon microvertex detector (MVD) and central-tracking detector (CTD). These tracking detectors were used for the measurements of momenta and positions of charged particles as well as for the identification of the interaction and secondaty decay vertices.

\subsubsection{Microvertex Detector}
\label{subsubsec:mvd}
The MVD was a silicon-strip detector located in the vicinity of the beam-pipe for achieving an excellent resolution for secondary vertex tagging. It was installed during the 2000--2001 shut-down before the HERAII running period. The microvertex detector was divided in forward (FMVD), barrel (BMVD) parts (see Figure~\ref{fig:MVD_artistic}). In the barrel part the silicon sensors were arranged in ladder structures grouped in three cylindrical layers surrounding the beam-pipe. In the forward direction the sensors were assembled into four circular discs (wheels) oriented perpendicularly to the beam direction. Large number of read-out channels and high hit resolution allowed reliable separation of tracks emerging from hadronic jets. The main MVD characteristics are summarised in Table~\ref{tab:mvdgeomparameters}. 

\begin{figure}[htbp]
	\centering
		\includegraphics[angle=0,width=\textwidth]{./Figures/panorama_MVD_top.jpg}
	\caption{Silicon microvertex detector.}
	\label{fig:MVD_artistic}
\end{figure}

\begin{table}[htbp]
	%\centering
\begin{tabular}{ | c | c | }
     \hline
      Parameter & Value \\
			\hline
			\hline
			Polar-angle coverage & $7\degree \mbox{--} 160\degree$ \\ \hline
      Read-out pitch & $120\,\micron$  \\ \hline
			Single-hit resolution & $24\,\micron$ \\ \hline 
			Two-track separation & $200\,\micron$ \\
      \hline
     \end{tabular}
	\caption{Silicon Microvertex detector parameters}
	\label{tab:mvdgeomparameters}
\end{table}

Other details of the sensor characteristics and performance can be found in~\cite{tech:mvd:prc9701}.

\subsubsection{Central Tracking Detector}
\label{subsubsec:ctd}
The CTD was a multi-wire cylindrical drift chamber used for the determination of charged particle positions and momenta. The operation principle of CTD was based on detection of the ionisation of a gas mixture by the charged particle traversing the volume of CTD. The momentum of the particle was determined from the curvature of the track. The dependence of energy losses within the volume of CTD on particle mass was used for the identification of particle type.
\begin{figure}[htpb]
	\centering
		\includegraphics[width=0.7\textwidth]{./Figures/ctd}
	\caption{Layout of the CTD octant.}
	\label{fig:ctd}
\end{figure}
In Figure~\ref{fig:ctd} one CTD octant is demonstrated. The wires were organised in nine superlayers (SL). Using different orientation of wires with respect to the CTD axis in odd and even superlayers made possible accurate determination of the Z position of the hit. The inclination angle in even superlayers was about $\pm 5\%$. For the fast determination of the Z coordinate at the trigger level the Z-by-timing technique was used. For this purpose 8 sense wires were installed in each of the SL3 and SL5. The precision of the determination of the Z position, achieved by this method, was about 3 \cm.

Both CTD and MVD operated in $1.4$~T magnetic field of the thin superconducting solenoid surrounding the drfit chamber. For CTD-MVD tracks that pass through all nine CTD superlayers, the momentum resolution was $\sigma(p_{T})/p_{T} = 0.0029 p_{T} \oplus 0.0081 \oplus
0.0012/p_{T}$, with $p_{T}$ in \GeV. Other parameters characterising CTD are collected in Table~\ref{tab:ctdgeomparameters}.

\begin{table}[htbp]
	%\centering
\begin{tabular}{ | c | c | }
     \hline
      Parameter & Value \\
			\hline
			\hline
			Inner radius & $16.2\,\cm$ \\ \hline
      Outer radius & $85.0\,\cm$  \\ \hline
			Length & $241\,\cm$ \\ \hline 
			Polar-angle coverage & $11.3\degree \mbox{--} 168.2\degree$ \\ \hline
			Position resolution & $100-120\,\micron$ \\ \hline
			Z resolution & $1.4\,\mm$ (stereo)/$30\,\mm$ (timing) \\ \hline
			Two track resolution & < $2.5\,\mm$ \\
      \hline
     \end{tabular}
	\caption{Central tracking detector parameters}
	\label{tab:ctdgeomparameters}
\end{table}
