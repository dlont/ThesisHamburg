The uranium-scintillator compensating calorimeter (CAL)~\cite{thesis:kruger:1992,nim:a309:77,nim:a309:101,nim:a321:356,nim:a336:23} was used for the measurement of the energy of the scattered electrons and positrons and of hadronic jets. The CAL covered 99.7\% of the solid angle and consisted of the forward (FCAL), barrel (BCAL) and rear (RCAL) parts. The boost of the hadronic system determined the depth of the calorimeter necessary for the absorption of particles of different maximum energy in various parts of the detector.
\begin{figure}[h]
	\centering
		\includegraphics[width=0.8\textwidth]{./Figures/cal.jpg}
	\caption{Schematic view of the CAL along the beam axis.}
	\label{fig:cal}
\end{figure}

The requirement to absorb a maximum energy of about 800~\GeV\, resulted in $\sim 7 \lambda$ depth of the FCAL section, where $\lambda$ is the hadronic interaction length. The FCAL was longitudinally segmented into towers, each consisting of a single electromagnetic (EMC) and two hadronic (HAC) sections with a front-surface of $20 \times 5$~\cm$^{2}$ and $20 \times 20$~\cm$^{2}$, respectively. The FCAL towers were grouped into 23 modules. The EMC and HAC sections, also called cells, consisted of interleaved layers of depleted uranium (98.1\% U$^{238}$, 1.7\% Nb and 0.2\% U$^{235}$) and scintillator (SCSN28) of 2.6~\mm\, and 3.3~\mm\, thickness, respectively. The thickness of the absorber and active plates was chosen in order to achieve equal response of the calorimeter to electromagnetic and hadronic showers of the same energy. The FCAL covered the polar angle range $2.2\degree < \theta < 39.9\degree$.

The BCAL had a very similar internal structure as the FCAL, but was composed of 32 wedge-shaped modules forming a cylindrical barrel surrounding the tracking detectors and superconducting coil. Due to lower hadronic activity in the BCAL, its thickness was $\sim 5 \lambda$. The BCAL covered the angular range $36.7\degree< \theta <129.1\degree$.

The rear part of the calorimeter (RCAL) consisted of 23 modules of $\sim 4 \lambda$ depth. Because of much lower energy in the electron direction, the RCAL had only one HAC section.

All three parts of the calorimeter had a symmetric layout with respect to the beam axis and covered complete azimuthal range.

In total the CAL had 5918 cells. Each cell was read out by two photomultipliers. The light from the scintillator plates was guided to the photomultipliers through wavelength shifters attached to the opposite sides of a cell. Usage of two photomultipliers helped to avoid events with spontaneous discharge of one of the photomultipliers and minimised the amount of dead cells if one of the PMTs was not functioning.

The natural radioactivity of the uranium made it possible to perform a calibration of individual channels of the calorimeter on a daily basis by providing a stable reference signal. The CAL energy response was calibrated to $\pm 1\%$ using this technique. The calibration of the electronic readout was performed using test pulses simulating photomultiplier signals. A timing resolution of 1~ns was achieved for energy deposits $> 4.5$ \GeV.
The energy resolution of the CAL measured under test beam conditions was 
\begin{equation}
	\frac{\sigma \left(E\right)}{E} = \frac{18\%}{\sqrt{E}} \oplus 1\% \quad \text{for electrons}
\end{equation}
and 
\begin{equation}
	\frac{\sigma \left(E\right)}{E} = \frac{35\%}{\sqrt{E}} \oplus 1\% \quad \text{for hadrons},
\end{equation}
where E was the incident particle energy. The 1~ns time resolution of the calorimeter was utilised for rejection of non-\ep~background by providing fast signals to the trigger system. The particle incident angles, determined by the direction of the shower in the CAL with respect to the primary vertex position, were measured with about 10~mrad precision.

Other characteristics of the \zeus calorimeter are summarised in Table~\ref{tab:calparams}.

\begin{table}[htbp]
	\centering
	{\small
		\begin{tabular}{|c|c|c|c|}
			     \hline
      &FCAL & BCAL & RCAL \\
			\hline
			\hline
			Polar angle coverage & $2.2\degree < \theta < 36.7\degree$ & $36.7\degree < \theta < 129.1\degree$ & $129.1\degree < \theta < 176.2\degree$ \\ \hline
			Pseudorapidity coverage & $4.0 > \eta > 1.1$ & $1.1 > \eta > -0.74$ & $-0.74 > \eta > -3.4$ \\ \hline
			EMC section depth & $25.9 X_0$ & $22.7 X_0$ & $25.9 X_0$ \\ \hline
			Total module depth & $7.14 \lambda$ & $5.1 \lambda$ & $3.99 \lambda$ \\ \hline		
		\end{tabular}
	}
	\caption{Geometric dimensions of the calorimeter modules, here $X_0$ and $\lambda$ are the radiation and interaction lengths, respectively.}
	\label{tab:calparams}
\end{table}

\subsection{Backing Calorimeter}
\label{subsec:bac}
The high-energy hadronic showers, that cannot be fully contained within the volume of the UCAL, deposited the remaining energy in the backing calorimeter (BAC)~\cite{nim:a300:480} consisting of proportional gaseous tube detectors located within the volume of the magnet iron yoke.