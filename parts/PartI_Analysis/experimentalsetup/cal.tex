The uranium-scintillator compensating calorimeter (CAL) was used for the measurement of the energy of the scattered electrons and positrons and of hadronic jets. The CAL covered 99.7\% of the solid angle and consisted of the forward (FCAL), barrel (BCAL) and rear (RCAL) parts. The boost of the hadronic system in the proton direction determined the depth of the calorimeter necessary for the absorption of particles of different maximum energy in various parts of the detector.
\begin{figure}[h]
	\centering
		\includegraphics[width=0.8\textwidth]{./Figures/cal.jpg}
	\caption{Schematic view of the CAL along the beam axis.}
	\label{fig:cal}
\end{figure}

The requirement to absorb a maximum energy of about 800~\GeV\, resulted in $\sim 7 \lambda$ depth of the FCAL section, where $\lambda$ is the hadronic interaction length. The FCAL was longitudinally segmented into towers, each consisting of a single electromagnetic (EMC) and two hadronic (HAC) sections with the front-surface of $20 \times 5$~\cm$^{2}$ and $20 \times 20$~\cm$^{2}$, respectively. The towers were grouped into 23 modules forming the volume of the FCAL. The HAC and EMC sections, also called cells, consisted of the interleaved layers of depleted uranium (98.1\% U$^{238}$, 1.7\% Nb and 0.2\% U$^{235}$) and scintillator (SCSN28) of 3.3~\mm\, and 2.6~\mm\, thickness, respectively. Such thickness of the absorber and active plates were chosen in order to achieve equal response of the calorimeter to electromagnetic and hardronic showers of the same energy. The FCAL covered the polar angle range $2.2\degree < \theta < 39.9\degree$.

The BCAL had a very similar structure as the FCAL, but with 32 wedge-shaped modules. Due to lower hadronic activity in the BCAL its thickness was $\sim 5 \lambda$. The BCAL covered the angular range $36.7\degree< \theta <129.1\degree$.

The rear part of the calorimeter (RCAL) consisted of 23 modules of $\sim 4 \lambda$ depth. Because of much lower activity in the electron direction the RCAL had only one HAC section.

In total, the CAL had 5918 cells. Each cell was read out by two photomultipliers on the opposite sides of the cell. The light from the scintillator plates was guided to the photomultipliers through wavelength shifters. Usage of two photomultipliers helped to avoid events with spontaneous discharge of one of the photomultipliers and minimised the amount of dead cells if one of the PMTs was not functioning.
The natural radioactivity of the uranium made it possible to perform a channel-by-channel calibration of the calorimeter on a daily basis by providing a stable reference signals. The CAL energy (response) was calibrated down to $\pm 1\%$ using this technique. The calibration of the electronic readout was performed using test pulses simulating photomultipliers signal. A timing resolution of 1~ns was achieved for energy deposits $> 4.5$ \GeV.
The energy resolution of the CAL measured under test beam conditions was 
\begin{equation}
	\frac{\sigma \left(E\right)}{E} = \frac{18\%}{\sqrt{E}} \oplus 1\% \quad \text{for electrons}
\end{equation}
and 
\begin{equation}
	\frac{\sigma \left(E\right)}{E} = \frac{35\%}{\sqrt{E}} \oplus 1\% \quad \text{for hadrons},
\end{equation}
where E was the incident particle energy. The 1~ns time resolution of the calorimeter was utilised for rejection of the non-ep background by providing fast signals to the trigger system. The angles were measured with about 10~mrad precision.

Other characteristics of the \zeus calorimeter are summarised in Table~\ref{tab:calparams}.

\begin{table}[htbp]
	\centering
	{\small
		\begin{tabular}{|c|c|c|c|}
			     \hline
      &FCAL & BCAL & RCAL \\
			\hline
			\hline
			Polar angle coverage & $2.2\degree < \theta < 36.7\degree$ & $36.7\degree < \theta < 129.1\degree$ & $129.1\degree < \theta < 176.2\degree$ \\ \hline
			Pseudorapidity coverage & $4.0 > \eta > 1.1$ & $1.1 > \eta > -0.74$ & $-0.74 > \eta > -3.4$ \\ \hline
			EMC section depth & $25.9 X_0$ & $22.7 X_0$ & $25.9 X_0$ \\ \hline
			Total module depth & $7.14 \lambda$ & $5.1 \lambda$ & $3.99 \lambda$ \\ \hline		
		\end{tabular}
	}
	\caption{Geometric dimensions of the calorimeter modules, here $X_0$ and $\lambda$ are the radiation and interaction lengths, respectively.}
	\label{tab:calparams}
\end{table}
