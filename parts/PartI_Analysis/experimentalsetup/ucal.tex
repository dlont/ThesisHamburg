The ZEUS calorimeter was a sampling uranium calorimeter (CAL) used for the energy determination of charged and neutral particles. 
The hadronic showers are characterized by significant fraction of long-lived neutral particles like neutrons, neutrinos or pi0 and therefore usually have much smaller light yield within the active volume of the calorimeter compared to electromagnetic showers of the same energy. In order to achieve equal response of the detector to hadronic and electromagnetic showers, the special design of the calorimeter was used. The elementary readout unit of CAL, called cell, consisted of alternating 3.3 mm thick plates of depleted uranium and 2.6 mm thick plates of SCSN38 scintillator. The uranium and scintillator layers served as absorbing and active medium, respectively. The corresponding thicknesses of the plates were chosen in order to achieve compensation (equal response to hadronic and electromagnetic showers of the same energy).

\begin{figure}[tpb]
	\centering
		\includegraphics[width=0.7\textwidth]{../../Figures/cal.jpg}
	\caption{A schematic diagram of the CAL sections in the X -- Z plane.}
	\label{fig:cal}
\end{figure}

The light from each cell was transferred via the wavelenghtshifters to the photomultipliers. Each cell was equipped by two photomultipliers, attached to the opposite sides of the cell. Utilising the balance of signals from two photomultipliers the particle hit position can be estimated. Spatial resolution in the incident point position, determined in the test-beam studies, was $\sim 3.3\,\cm/\sqrt{E}$ and $\sim 6.6\,\cm/\sqrt{E}$ for electrons and hadrons, respectively. Furthermore, the events with spontaneous discharge of one of the photomultipliers can be identified. Usage of plastic scintillator with a short decay time and fast read-out electronics resulted in timing resolution $\sigma_t = 1.5/\sqrt{E} \oplus 0.5$ ns. For typical shower energies detected in the calorimeter the resulting time resolution was better than 1 ns. Fast signals from the calorimeter were utilised by triggers for the non-$\ep$ background suppression. 

In total the calorimeter had 5819 cells organised in three sections: forward (FCAL), barrel (BCAL) and rear (RCAL) calorimeters. A schematic view of the calorimeter in presented in Figure~\ref{fig:cal}.

The cells in FCAL were assembled in 23 vertically aligned modules, each having one electromagnetic (EMC) and two hadronic (HAC) cells with the front surface of $20 \times 5$ $\cm^2$ and $20 \times 20$ $\cm^2$, respectively. The BCAL had 32 wedge-shaped modules distributed radially around the beam axis. The rear part of the calorimeter had only one electromagnetic and one hadronic cell in each tower, but otherwise had the same structure as FCAL.

In the test beam studies the energy resolution of the calorimeter was determined to be $\sigma\left( E \right ) / E = 18\%/\sqrt{E} \oplus 1\%$ for electromagnetic showers and $\sigma\left( E \right ) / E = 35\%/\sqrt{E} \oplus 2\%$ for hadronic showers. Other characteristics of the ZEUS calorimeter are summarised in Table~\ref{tab:calparams}.

\begin{table}[htbp]
	\centering
	{\small
		\begin{tabular}{|c|c|c|c|}
			     \hline
      &FCAL & BCAL & RCAL \\
			\hline
			\hline
			Polar angle coverage & $2.2\degree < \theta < 36.7\degree$ & $36.7\degree < \theta < 129.1\degree$ & $129.1\degree < \theta < 176.2\degree$ \\ \hline
			Pseudorapidity coverage & $4.0 > \eta > 1.1$ & $1.1 > \eta > -0.74$ & $-0.74 > \eta > -3.4$ \\ \hline
			EMC section depth & $25.9 X_0$ & $22.7 X_0$ & $25.9 X_0$ \\ \hline
			Total module depth & $7.14 \lambda$ & $5.1 \lambda$ & $3.99 \lambda$ \\ \hline		
		\end{tabular}
	}
	\caption{Geometric dimensions of the calorimeter modules, here $X_0$ and $\lambda$ are the radiation and interaction lengths, respectively.}
	\label{tab:calparams}
\end{table}

Natural uranium radioactivity provided a stable and time-independent reference signal used for continuous monitoring of the calorimeter response. The calibration of the absolute energy-scale of the calorimeter was performed on a daily basis. The precision typically better than $1\%$ was achieved.
