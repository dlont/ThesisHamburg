The detailed description of systematic uncertainties on the measured jet cross sections was presented in chapter~\ref{ch:results}. The uncertainties attributed to the inclusive DIS measurements are summarised in~\cite{Aaron:2009aa}. Besides experimental uncertainties on the measured values additional sources appear due to indeterminacy of external input used in theoretical predictions. In general, the absolute magnitude of the uncertainties on the cross section values must be accompanied by information about mutual correlations between the measured values in different cross section bins. The correlations are taken into account in the $\chi^2$, as it was described above. Typically, an estimate of the correlation of systematic uncertainties is not know because the evaluation procedure can be very difficult or even not existing. For example, to author knowledge, no commonly accepted prescription for estimation of correlations attributed to the truncation of perturbative series exist. However, information about correlations is important and some assumptions have to be made\footnote{The following simple example can be considered. In general, the least-squares method for the problem with single fit-parameter e.g.~\asz, can be interpreted as weighted averaging of \asz values extracted from individual cross section bins with weights that are inversely proportional to experimental uncertainty. For illustration purpose, all sources of uncertainty except for luminosity are ignored. Relative error due to luminosity is equal for all bins, thus, if treated as uncorrelated, the resulting uncertainty $\Delta\asz$ scales $\propto 1/\sqrt{\text{N}_\text{bins}}$ as for $\text{N}_\text{bins}$ independent \asz measurements, while in case of fully correlated treatment, $\Delta\asz$ will not improve, when extra measurements are included, because the cross sections are assumed to shift up or down in all bins simultaneously.}. Nevertheless, positive statistical correlations in inclusive-jet measurements emerging due to fact that more than one jet can appear in the same event, are relatively easy to assess. In such a case the correlations are represented by statistical covariance matrix $C_{ij}^{\text{stat.}}$ that was described in Section~\ref{sec:statcorrjets}. In general, positive correlations between individual measurements result in larger uncertainty when propagated to the fit-parameters. The treatment of individual systematic sources is summarised below and is motivated by the following argumentation: 
\begin{itemize}
 \item \textbf{correlated systematic sources:}
 \begin{itemize}
 \item the uncertainty on the cross section normalisation due to the luminosity measurement $\delta_\text{lumi}$ in treated as fully correlated across all measurement points because according to Eq.~\eqref{eq:csdef} the integrated luminosity $\mathcal{L}$ enters the cross section as a constant factor common for all measurement bins;
 \item the absolute jet-energy scale uncertainty $\delta_\text{JES}$ is treated as fully correlated. This error account for the precision of jet-energy calibration. The calibration constants were determined as functions of the jet pseudorapidity in laboratory frame. It is assumed the the extracted values deviate from the true ones in correlated manner;
 \end{itemize}
 \item \textbf{uncorrelated systematic sources:}
 \begin{itemize}
 \item the model uncertainty was treated as uncorrelated. Such an approach is supported by the observation that the individual systematic variations performed for this source (see Chapter~\ref{ch:resultscs}) do not result in correlated deviation of the extracted cross sections with respect to the central value.
 \item the normalisation uncertainty due to simulation of the track-veto efficiency in MC $\delta_\text{TV}$ was treated as uncorrelated according to the same argument;
 \item although, conceptually, the uncertainty due to absolute electron energy scale $\delta_\text{EES}$ has to be treated in the same way as jet-energy scale error, $\delta_\text{EES}$ was assumed to be uncorrelated across measurement bins;
 \item the uncertainty due to electron identification $\delta_\text{eID}$ and due to $\left(E-p_Z\right)$-cut variation $\delta_\text{E-Pz}$ were also assumed to be uncorrelated;
 \end{itemize}
\end{itemize}
The uncertainties from uncorrelated sources except for statistical error were added in quadrature and included as a single uncorrelated source into the $\chi^2$-function. The experimental errors on the measured jet cross sections were propagated to \asz values by means of ``Hesse''-method implemented in \minuit. A study the impact of experimental precion on \asz extraction is explained in Section~\ref{subsec:assysunc}.
\begin{table}[t]
\centering
\begin{tabular}{|l|c|c|}
\hline
 systematic source & correlated & uncorrelated \\
\hline
\multicolumn{3}{c}{experimental}\\
\hline
 luminosity & $\times$ & \\
 jet energy scale & $\times$ & \\
\hline
 electron energy scale & & $\times$  \\
 electron identification & & $\times$ \\
 track-veto reweighting & & $\times$ \\
 MC model & & $\times$ \\
 $\left(E-p_Z\right)$-cut & & $\times$ \\
\hline
\multicolumn{3}{c}{theoretical}\\
\hline
 renormalisation scale & $\times$ & \\
 factorisation scale & $\times$ & \\
 hadronisation corrections & & $\times$ \\
\hline
\end{tabular}
\label{tab:correlsyst} 
\end{table}

In this analysis theoretical uncertainties related to the variation of renormalisation and factorisation scales were treated as correlated. The detailed discussion of the effect of different approaches for the treatment of scale variations is presented in Section~\ref{} together with discussion of the sensitivity of the fit procedure to \asz and PDF assumptions. Table~\ref{tab:correlsyst} contains an outline of systematic uncertainty treatment.