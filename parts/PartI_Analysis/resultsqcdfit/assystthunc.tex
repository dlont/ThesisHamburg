Additional sources of uncertainty on \as arise from the precision of theoretical calculations. In this section the results of a study of the sensitivity of the \asz fits to external theoretical assumptions is presented. In general, uncertainties attributed to assumptions are difficult to quantify or even impossible to define due to their non-stochastic nature. Approaches used for the treatment of particular error sources are typically tailored for each problem and statistical interpretation of the stability of the results if often doubtful. In this section the errors due to the truncation of the perturbative series, limited precision of the proton PDFs and hadronisation corrections are described in detail.

\subsubsection{Uncertainties due to missing higher orders}
\label{subsec:asscalevar}
A natural approach that can be used to estimate the uncertainties attributed to theoretical assumptions is to repeat the parameter determination under different assumptions and treat the difference between the results as an uncertainty on the extracted parameter value. If the value of \asz obtained under different theoretical assumption is denoted by $\alpha_s^{\ast}$ and the central value is $\alpha_s^0$, then the uncertainty is:
\begin{equation}
 \Delta\alpha_s = \alpha_s^{\ast} - \alpha_s^0.
\end{equation}
Such an approach is called the 'offset method'~\cite{amanda}. In principle, it takes into account non-linear effects in the error propagation. In case, when underlying assumption can be parametrised by a single continuous parameter, this method can be used for linear error propagation employing a numerical approximation for the first derivative:
\begin{equation}
 \frac{\partial \as}{ \partial a} \approx \frac{\Delta\alpha_s}{\Delta a},
\end{equation}
where $\Delta a$ is an infinitesimal change of the parameter in question.

By default, the offset method was used in this analysis for the determination of the uncertainty attributed to the truncation of the perturbation expansion for the jet cross sections. As it was described in Chapter~\ref{ch:theory}, the size of the missing higher order terms can be estimated by varying the renormalisation and factorisation scales. For this purpose the \asz fit was repeated using theoretical predictions evaluated with modified definitions of the renormalisation and factorisation scales. The commonly accepted definition of a '1$\sigma$ confidence interval' for a scale variation uncertainty~\cite{soper:1997} was adopted in this work. Two variants of the scale variation were considered:
\begin{itemize}
 \item \textbf{Independent variation of $\mu_R$ and $\mu_F$}. The theoretical predictions for the double-differential jet cross sections were recalculated with the scaled $\mu_R$ and $\mu_F$ parameters:
\begin{align}
 \mu_R \rightarrow 2\mu_R;&\qquad \mu_R \rightarrow 1/2\mu_R;\\
 \mu_F \rightarrow 2\mu_F;&\qquad \mu_F \rightarrow 1/2\mu_F;
\end{align}
performing each variation one at a time. In general, the fits were characterised by larger values of $\chi^2/N_\text{DF}$. Only the fit with $\mu_F \rightarrow 2\mu_F$ had smaller value of $\chi^2/N_\text{DF}$ than the central \asz-fit. The resulting variation of \asz was:
\begin{equation}
 \asz = 0.1218\;^{+0.0057}_{-0.0050}\left(\mu_R\right)\;^{+0.0003}_{-0.0003}\left(\mu_F\right)
\end{equation}
\item \textbf{Simultaneous variation of $\mu_R$ and $\mu_F$}. Because the analytic expressions for the renormalisation and factorisation scales are related, alternatively both scales were varied up and down simultaneously. The obtained variations are:
\begin{equation}
 \asz = 0.1218\;^{+0.0065}_{-0.0053}\left(\text{scales}\right)
\end{equation}
Such variant results in a more conservative estimates of the uncertainty. 
\end{itemize}
It can be seen that the fits are much more sensitive to the renormalisation scale variation and result in an asymmetric uncertainty of the \asz value. The comparison of the fit results for different scale variations is illustrated in Figure~\ref{fig:asthunc_dd} and denoted as $\delta_1$.

Other approaches for the treatment of the scale variation uncertainty were also investigated. As an alternative, theoretical uncertainties were included into the covariance matrix assuming that errors attributed to the scale variations are completely uncorrelated across the measurements points. In this case the total error attributed to the theoretical and experimental uncertainties was propagated to the uncertainty on \asz value using the Hesse method. The uncertainty contribution on \as from the scale variations was calculated from the total error $\Delta_\text{tot}$ according to:
\begin{equation}
 \Delta_\text{scales} = \sqrt{\left(\Delta_\text{tot}\right)^2 - \left(\Delta_\text{exp}\right)^2}
 \label{eq:asuncscalecontrib}
\end{equation}
Another approach was also tested. The uncertainties due to missing higher-order contributions were included into the $\chi^2$ using the nuisance parameters, similar to jet-energy scale or luminosity uncertainty. In this case the systematic shifts attributed to the theoretical uncertainty are treated as correlated across different bins and propagated to the result using the Hesse method. Overall, fits with Hesse treatment of the scale variation uncertainties resulted in substantial reduction of \asz error in comparison to the offset method. The comparison of different approaches is provided in Table~\ref{tab:scaleuncvariants}.
\begin{table}[h]
\centering
\begin{tabular}{|l|c|}
 \hline
 Treatment approach & $\Delta\as$ \\
 \hline
 \hline
  Offset & $\phantom{x}^{+0.0065}_{-0.0053}$\\
  Hesse (nuissance) & 0.0026 \\
  Hesse (uncorrelated) & 0.0015 \\
 \hline
\end{tabular}
\caption{Comparison of the uncertainty on extracted \asz value due to missing higher orders obtained using different error treatment approaches. The uncertainties due to renormalisation and factorisation scale variations are combined in a single contribution.}
\label{tab:scaleuncvariants}
\end{table}

The reduction of the \asz uncertainty can be explained as follows. If the simultaneous least-squares fit to all measured points of the double-differential cross sections is interpreted as the averaging procedure for the \asz values from individual measurements, then the offset method would correspond to the averaging of \asz values determined with the scaled definition for the renormalisation and factorisation parameters and assuming equal weight for every extraction. On the other hand, in the Hesse approach, the extractions characterised by smaller theoretical uncertainty acquire larger weight in the averaging procedure and therefore the uncertainty of the average \asz is smaller. In case of nuisance parameter treatment, the numerical value for the systematic shifts $b_{\mu_R}=\ln{\mu_R/\mu_R^0}$, $b_{\mu_F}=\ln{\mu_F/\mu_F^0}$, where $\mu_R^0$ and $\mu_F^0$ denote the default scale, represent the rescaling factor that has to be applied to the renormalisation or factorisation parameters in order to obtain a better fit. This treatment assumes full correlation of missing higher-orders contributions across the phase space, which is, in principle, difficult to justify. As demonstrated, simultaneous variation of scales results in a more conservative estimate of uncertainty on \as, therefore the errors on theoretical predictions estimated from simultaneous rescaling of $\mu_R$ and $\mu_F$ were used in the fit and a single parameter $b=b_\text{scale}$ was introduced instead of $b_{\mu_R}$ and $b_{\mu_F}$. The systematic shift obtained from the fit was $b_\text{scale}=0.11\pm 0.34$. Which indicates that within the uncertainty on $b_\text{scale}$ an $\mu_R=\sqrt{\qsq+\etjetb^2}$ and $\mu_F=Q$ cannot be distinguished from the 'optimal' scale choice.

Fits with alternative definitions of the renormalisation and factorisation scales were also performed. Scale choices such as $\mu_R=\etjetb$, $\mu_R=Q$ and $\mu_F=\etjetb$, $\mu_F=\sqrt{\qsq+\etjetb^2}$ were investigated. In general, alternative fits had comparable $\chi^2/N_\text{DF}$, but resulted in lower values of \asz. Only the fit with $\mu_R=\mu_F=\sqrt{\qsq+\etjetb^2}$ gave rise to a somewhat larger value of the strong coupling. The discrepancy between the results of different fit variants was smaller than the scale variation uncertainty estimated using the offset-method and therefore is assumed to be covered by the scale variation uncertainty.

\subsubsection{Sensitivity to the pPDF sets}
\label{subsec:aspdfassump}
Because the proton PDFs have limited precision, the stability of the \asz-fits with respect to PDF assumptions has to be investigated. Several effects contributing to the proton PDF uncertainty can be identified. All of them can be roughly attributed to one of the following categories:
\begin{itemize}
 \item \textbf{Limited PDFs precision due to data uncertainty.} Because PDFs are extracted in QCD fits to data, the uncertainty on the measurements propagates to the error on the proton PDFs, which in turn affects the precision of the \asz determination.
 \item \textbf{QCD-fit setup differences.} As explained above, a procedure involving simultaneous PDF fits of all individual parton species is not well defined problem and additional assumptions have to be imposed in order to obtain reasonable fit results. However, the choice of the data and fit settings is usually ambiguous. Typically, different fitting groups use partially different datasets to constrain PDF parameters. Furthermore, they can have different choices of PDF parametertrisation at the starting evolution scale, distinct approaches to the treatment of heavy quarks, alternative $\chi^2$-functions and/or error definitions, etc. This leads to different results for PDFs obtained by different groups.
 \item \textbf{PDF and \asz correlation.} In many high-energy processes, the strong coupling and the proton PDFs are coupled, i.e. the predictions are proportional to the product $\as\cdot f_i\left(x,\mu\right)$. In particular, the gluon component is always accompanied by a factor \as in the matrix elements, therefore, a value of \asz has to be assumed in the PDF fit, which introduces an implicit dependence of the results on \asz. In order to enable propagation of this type of uncertainty into pQCD predictions, different PDF fitting groups provide series of PDFs extracted assuming different values of \asz.
\end{itemize}

% \subsubsection{PDF uncertainty propagation}
In the \herapdf1.5 set, which was used as a default in the calculations of the inclusive-jet cross sections, 34 systematic variations were performed to estimate of the PDF uncertainty and were classified into three groups roughly corresponding to the categories listed above. 

The uncertainty due to the experimental precision of the data is propagated to the PDF parameters and is represented in the covariance matrix form. In order to simplify propagation of the uncertainties attributed to the proton PDFs, the matrix is transformed to diagonal form with eigenvectors representing the uncertainty on the corresponding linear combinations of the PDF parameters; $+1\sigma$ parameter variation along the eigenvectors translates into a 68\% confidence interval for the pQCD predictions around their central values according to the formula~\cite{Campbell:2006wx}
\begin{equation}
 \delta^{\pm}\mathbf{t} = \sqrt{\sum_{i=1}^{N_{eig}}{\left(\mbox{$\delta_i^{\pm}\mathbf{t}$}\right)^2}},
 \label{eq:mastereq}
\end{equation}
where the positive (negative) uncertainty, $\delta^{\pm}\mathbf{t}$, is calculated from the quadratic sum of positive (negative) variations of the predictions with respect to those based on the central PDF, $\mbox{$\delta_i^{\pm}\mathbf{t}$}^2$. The summation runs over eigenvectors, the number of which is equal to the number of free PDF parameters\footnote{In total, the \herapdf1.5 set provides 10 up and 10 down eigenvector variations.}. When propagated to the strong coupling, PDF eigenvector variation results in an asymmetric uncertainty on the \as value. A symmetric estimate of $\delta\as$ was defined as:
\begin{equation}
 \delta\as = \sqrt{\sum_{i=1}^{N_{eig}}{\frac{\left(\mbox{$\delta_i^{+}\as$} - \mbox{$\delta_i^{-}\as$}\right)^2}{2}}}.
 \label{eq:mastereqsym}
\end{equation}
It provides a reasonable average of the asymmetric uncertainties and can be considered as a modification of the offset method. In order to estimate the uncertainty on \asz, the \as fits were repeated for every individual up-and down-variation of the PDF eigenvectors and compared to the central value given in Eq.~\ref{eq:asdoubledifval}. The results of these fits are summarised in Figure~\ref{fig:asthunc_dd} and denoted by $\delta_2$.
\begin{landscape}
\begin{figure}[p]
 \centering
 \caption{The results of \asz extraction from double-differential cross sections. The individual systematic variations are combined into groups $\delta_1-\delta_5$ (see text).}
 \label{fig:asthunc_dd}
 \includegraphics[height=0.8\textheight,angle=180,trim=0 0 0 0,clip]{Figures/alphas/alphas_systematics_dd.png}
 % alphas_systematics.png: 1796x572 pixel, 72dpi, 63.36x20.18 cm, bb=0 0 1796 572
\end{figure}
\end{landscape}

The model uncertainty on the PDFs, attributed to the variation of the assumptions such as the fraction of the strange quarks in the $d$-type sea PDFs, mass of the charm and bottom quarks, etc. were also treated using the offset approach. The detailed description of the nature of individual variations are presented in~\cite{upub:herapdf1.5}. The uncertainty on \asz was estimated  by taking the difference between the variation and the central value, and then adding in quadrature all positive (negative) differences to obtain the positive (negative) model error (see Eq.~\eqref{eq:mastereq}). Corresponding results of the \as-fits are combined into $\delta_3$ group in Figure~\ref{fig:asthunc_dd}. 

Variations of the functional form of the PDF parameterisation (see~\cite{upub:herapdf1.5}) were also treated using the offset-method. To form the parametrisation envelope, the largest positive (negative) difference between the variation and the central value is taken as the positive (negative) parametrisation error. The outcome of the resulting variations of \asz are illustrated in Figure~\ref{fig:asthunc_dd} as $\delta_4$.

The uncertainty attributed to the assumption imposed on the magnitude of the strong coupling in the PDF extraction procedure was estimated from two PDF sets with \asz=0.1156 and \asz=0.1196. The difference between the two results, (see Figure\ref{fig:asthunc_dd}~($\delta_5$)), was scaled to the current uncertainty on the world average~\cite{Bethke:2012jm}.

In order to obtain the total error on the \asz-value due to the proton PDF, all mentioned sources were added in quadrature, which resulted in the final value:
\begin{equation}
 \asz = 0.1218\;^{+0.0029}_{-0.0018}\left(\text{PDF}\right)
\end{equation}

\begin{table}[t]
 \begin{center}
 \begin{tabular}{|c|c|c|}
 \hline
 PDF set & \asz-value & $\chi^2/n_\text{DF}$\\
 \hline
 \hline
 \herapdf1.5 & $0.1218^{+0.0029}_{-0.0018}$ & 0.626 \\
 NNPDF23 & 0.1221 & 1.00\\   
 MSTW08 & 0.1232 & 0.884\\ 
 CT10 & 0.1240 & 0.831\\ 
 \hline
 \end{tabular}
 \end{center}
 \caption{The results of the \asz extractions based on different PDF sets.}
 \label{tab:asdifferentPDFs}
 \end{table}

\asz fits with alternative predictions based on PDF sets provided by different fitting-groups were performed in order to investigate the stability of the \asz determination procedure. The considered set of PDFs comprises the extractions by the MSTW~\cite{}, NNPDF~\cite{} and CT~\cite{} groups. The obtained results were compared to the reference fit based on \herapdf1.5, described above. All fits were performed using the so-called 'central' PDF set determined with the \asz value recommended in the corresponding PDF-fit. Analysing the obtained results, several observations can be made:
\begin{itemize}
 \item All fits with alternative PDFs have larger $\chi^2$ than the reference one, however in all cases the obtained values of $\chi^2/n_\text{DF}$ were $\lesssim 1$. Figure~\ref{fig:chi2scanpdf} illustrates the scan of the $\chi^2$ as a function of \asz for different PDF sets. One possible explanation for the difference in the fit quality is the difference in the gluon distribution of different PDF sets, especially in the high-$x$ region, relevant for inclusive-jet production (see Figure~\ref{fig:pdf_gluon_comp}).
\begin{figure}[t]
 \centering
 \includegraphics[width=0.8\textwidth,angle=-90]{Figures/alphas/PDFsets_chi2_scan.pdf}
 % PDFsets_chi2_scan.: 1418x772 pixel, 72dpi, 50.02x27.23 cm, bb=0 0 1418 772
 \caption{The results of the scan of $\chi^2\left(\asz\right)$ as a function of \asz obtained using different PDF sets. The vertical dashed lines represent the PDF uncertainty from the \herapdf1.5 error analysis described in text.}
 \label{fig:chi2scanpdf}
\end{figure}
\begin{figure}[ht]
 \centering
 \includegraphics[width=0.8\textwidth]{Figures/alphas/pdf_comparison.pdf}
 % pdf_comparison.: 896x972 pixel, 72dpi, 31.61x34.29 cm, bb=0 0 896 972
 \caption{Comparison of the gluon distribution in the high-$\x$ region for different PDF sets at \qsq=100~\GeV$^2$.}
 \label{fig:pdf_gluon_comp}
\end{figure}
 %\item It is interesting to note that in the HERAPDF1.5 PDF-fit only inclusive NC and CC DIS data from HERA were used, while, for example, MSTW08 already includes HERAI inclusive-jet data. The preference of the HERAPDF set in the fit was unexplained and we believe that the 'true' PDF is somewhere between the considered sets.
 \item The obtained values of the strong coupling were larger than that extracted using \herapdf1.5, but the difference was always within the uncertainty attributed to the \herapdf1.5 set. A summary of the fit results based on different PDF sets is compiled in Table~\ref{tab:asdifferentPDFs}
\end{itemize}

In accord with the PDF4LHC recommendations, in recent analyses~\ref{Britzger} an additional uncertainty was attributed to the difference between results obtained with different PDF sets. In contrast to that, the checks, described above, reveal the consistency of the considered PDF sets in the phase space of the inclusive-jet measurement and support the adequacy of the \herapdf1.5 error analysis. Therefore no additional uncertainties due to PDFs were assigned to the \asz-value determined in this thesis. 
\begin{figure}[t]
 \centering
\begin{subfloat}[]{\includegraphics[width=0.45\textwidth,trim=20 10 0 10,clip,angle=-90]{Figures/alphas/alphas_asPDF_scan.pdf}
   \label{fig:aschi2scanpdf_a}
 }%
\end{subfloat}
\begin{subfloat}[]{\includegraphics[width=0.45\textwidth,trim=20 10 10 10,clip,angle=-90]{Figures/alphas/alphas_asPDFChi2_scan.pdf}
   \label{fig:aschi2scanpdf_b}
 }%
\end{subfloat}
 % PDFsets_chi2_scan.: 1418x772 pixel, 72dpi, 50.02x27.23 cm, bb=0 0 1418 772
 \caption{The dependence of the fitted value of \asz extracted from inclusive-jet data using PDF sets with different assumed \asz-values (left). Quality of the fit as a function of \asz-value assumed in the PDFs.}
 \label{fig:aschi2scanpdf}
\end{figure}
\subsubsection{Sensitivity to the \asz Assumption}
\label{subsec:asassump}
As was mentioned above, the strong coupling enters pQCD predictions for hadron-induced processes through the matrix elements and the DGLAP splitting functions $P\left(z\right)=\sum_{i=1}^{N}{\as^i\cdot P_i\left(z\right)}$ (see Chapter~\ref{ch:theory}). Therefore, in order to determine PDFs in global fits to the data, a value of \asz has to be either taken from other analyses or fitted together with PDFs. Due to the coupled nature of perturbative hard coefficients and PDFs ($c\left(\as\right)\cdot f\left(x,\mu\right)$) the decrease of one quantity leads to the increase of the other to maintain a constant cross section. Therefore the question of the influence of the \asz assumption in the PDF extraction procedure has to be addressed. Figure~\ref{fig:aschi2scanpdf_a} shows the results of the \asz fits to the double-differential inclusive-jet cross sections with PDF sets from several fitting groups, extracted under different assumptions on the magnitude of the strong coupling. It can be seen that, apart from MSTW08 sets, available for a wide range of assumed \asz values, the extracted \asz increases with increasing \asz assumed in the PDFs. Only for MSTW08 and NNPDF23 do the input and output values of \asz coincide at relatively large value of \as. Qualitatively, the correlation of the input and output values can be explained as follows: in the global fit, PDFs decrease as the assumed value of the strong coupling increases, therefore to compensate the reduction of $f\left(x,\mu\right)$ in the fit to inclusive-jet data, the determined \asz value increases.

As can be seen in Figure~\ref{fig:aschi2scanpdf_b}, the quality of the fit deteriorates with increasing assumed value of the strong coupling and the fits based on PDFs with the smallest \asz value yield the smallest values of $\chi^2/n_\text{DF}$. Nevertheless, for MSTW08, CT10 and \herapdf1.5, the dependence of $\chi^2/n_\text{DF}$ on the assumed value is very mild in the vicinity of the minimum. The resulting values of \as for the fits with minimal value of $\chi^2/n_\text{DF}$ are consistent with the central value within the experimental uncertainty.

\subsubsection{Uncertainties due to hadronisation corrections}
\label{subsec:ashadrunc}
The uncertainty on \as due to the that on the hadronisation corrections applied to the NLO pQCD predictions were estimated by error propagation utilising the Hesse method. The uncertainty on the correction factors was defined as a half-difference between the predictions obtained using the \ariadne or \lepto models and the average of those, which was used a default value (see Section~\ref{subsec:hadrcorr}).

\as fits were performed adding the uncertainty on the hadronisation correction factors quadratically to the uncorrelated systematic uncertainty. The effect of the additional source of error on the extracted \asz value was determined using the Eq.~\eqref{eq:asuncscalecontrib}, where in this case, the $\Delta_\text{tot}$ represented the combined uncertainty on \as due to hadronisation corrections and experimental errors.

The uncertainties on the predictions due to hadronisation correction factors was $\lessapprox$ 2\% but typically less than 0.5\% in most of the cross section bins. The \as fits were characterised by comparable quality of the fit and resulted in \asz very similar to the central value. The uncertainty on \asz due hadronisation corrections was smaller than 0.1\% and therefore was neglected.