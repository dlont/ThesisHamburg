The inclusive DIS double-differential reduced cross-sections $\mathrm{d}\tilde\sigma^2/\mathrm{d}x\mathrm{d}\qsq$ measured at \hera constitute the core of all modern proton parton-density extractions~\cite{Lai:2010vv,Martin:2009iq,Alekhin:2012ig,Ball:2011uy,Aaron:2009aa}. The combined dataset~\cite{Aaron:2009aa} used in all these analyses comprises measurements of inclusive NC and CC DIS cross sections by the \hone and \zeus collaborations. A summary of the data samples is provided in~\cite{Aaron:2009aa}. The kinematic phase space of the combined NC DIS cross sections is $6\cdot 10^{-7} \le x \le 0.65$ and $0.045 \le \qsq \le 30000$ \GeV$^2$ for values of inelasticity in the range $0.005 \le y \le 0.95$. The kinematic range of the CC DIS data is $1.3\cdot 10^{-2} \le x \le 0.40$ and $300 \le \qsq \le 30000$ \GeV$^2$ for $y$ in the region $0.037 \le y \le 0.76$. It is important to note that the jet measurements provided in this work are based on a statistically independent data sample.

In practice, datasets for different reactions have different sensitivity to various PDF components. For example, NC and CC DIS cross sections are very sensitive to the quark content of the proton, because electroweak bosons couple to the quarks. In contrast, gluons do not interact directly with $\gamma,\zn$ or $W^{\pm}$ but via $g\rightarrow q\bar q$ processes with one of the quarks coupling to the electroweak boson. Therefore, in the predictions for inclusive \ep scattering, the gluon density is always accompanied by a factor \as, which results in a strong correlation between the strong coupling and the gluon parton density in QCD fits to inclusive DIS data. An approximation~\cite{Prytz:1993vr} for the scaling violation of the structure function $F_2\left(x,\qsq \right)$ illustrates this property:
\begin{equation}
 \frac{\partial F_2\left(x,\qsq \right)}{\partial \ln \qsq} \approx \sum_i^{n_f}e_{q_i}^2 \frac{\as}{\pi} xg\left(2x\right) \int_0^1 P_{q\leftarrow g}\left(z\right)\mathrm{d}z, 
\end{equation}
where $P_{q\leftarrow g}\left(z\right)$ is the $g\rightarrow q\bar q$ splitting function. An independent source of information constraining the magnitude of the strong coupling or the gluon density is necessary for an unbiased determination of both quantities. It was demonstrated in earlier studies~\cite{Chekanov:2005nn,upub:herapdf1.7}, that including jet data into the fit mitigates the problem of large correlation between \as and $xg\left(x\right)$ and significantly improves the precision of a simultaneous $xg\left(x\right)$ and \as extraction. Figure~\ref{fig:pdfcontributions} shows Feynman diagrams for different powers of the strong coupling constant, illustrating the processes with quarks and gluons in the initial state.
\begin{figure}[htp]
 \centering
 \begin{center}
 \includegraphics[width=\textwidth]{Figures/deltacontrib}
 % deltacontrib.eps: 0x0 pixel, 300dpi, 0.00x0.00 cm, bb=71 431 536 735
\end{center}
 \caption{Example Feynman diagrams illustrating the interplay of different PDF components and \as for various DIS processes.}
 \label{fig:pdfcontributions}
\end{figure}
For example, the gluon density $xg\left(x\right)$ provides the dominant contribution\footnote{Due to infrared divergences, the gluon splitting probability $\mathcal{P}_{gg}$ tends to infinity as $E_g \rightarrow 0$. This leads to a rapid proliferation of the number of soft gluons constituting the proton at low longitudinal momentum $x$ and, in principle, leads to violation of unitarity. However, at large gluon densities non-perturbative recombination processes have to be taken into account and unitarity is restored.}\marginpar{OB:Why dominant?\\DL:Because $P_{gg}$ has the most singular behaviour as $x\rightarrow 0$.???} to the NC DIS $\ep$-scattering cross section at low and medium $\x$ at high $\qsq$, although it appears for the first time at NLO, while jet production in the Breit-frame (BF) is already sensitive to $xg\left(x\right)$ at leading order. The sum over all quark and anti-quark densities weighted with respective electric charges $x\Delta\left(x\right)=x\sum_{a}{e_a^2\left(q_a\left(x\right)+\bar{q}\left(x\right)\right)}$ is the next most important contribution to inclusive DIS and jet processes and appears in all orders of perturbation theory.
