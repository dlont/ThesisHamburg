The inclusive DIS reduced cross sections measured at \hera constitute the backbone of all modern proton parton density extractions~\cite{Lai:2010vv,Martin:2009iq,Alekhin:2012ig,Ball:2011uy,Aaron:2009aa}. The combined dataset~\cite{Aaron:2009aa} used in all mentioned analyses comprises measurements of inclusive NC and CC DIS data by the \hone and \zeus collaborations. The summary of the data samples is provided in~\cite{Aaron:2009aa}. In general, the phase space of the combined NC DIS cross sections is $6\cdot 10^{-7} \le x \le 0.065$ and $0.045 \le \qsq \le 30000 \GeV^2$ for values of inelasticity in range $0.005 \le y \le 0.95$. The kinematic range of the CC DIS data is $1.3\cdot 10^{-2} \le x \le 0.40$ and $300 \le \qsq \le 30000 \GeV^2$ and $y$ in region $0.037 \le y \le 0.76$.

In practice, different dataset provide different sensitivity to various PDF components. At leading order in DIS the coupling of exchanged boson is prortional to the electroweak charge of the quarks, therefore NC and CC DIS cross section are very sensitive to the quark content of the proton. In contrast to quarks, gluons, do not interact directly with the $\gamma,\zn$ or $W^{\pm}$ but strongly via $g\rightarrow q\bar q$ process with one of the quarks coupling to electroweak boson. Therefore, in the predictions of for inclusive \ep scattering, the gluon density is always acompanied by the \as, which results in a strong correlation between \as and the gluon parton density in QCD fits to inclusive DIS data alone~\cite{}.

It is important to note that the jet measurements provided in this work are based on statistically independent data sample.