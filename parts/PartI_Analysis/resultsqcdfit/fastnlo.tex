The \fastnlo~\cite{thesis:wobisch:2001,Kluge:2006,Wobisch:2011,Britzger:2012} approach is a based on the idea of calculating the LO and NLO weights, $\mathcal{C}_{n,a,i,j}$, which are independent of the \as and PDFs, on a 2d-grid in convolution variable $x$ and factorisation scale $\mu_F$. The cross section can be, then, obtained from a simple sum:
\begin{equation}
\sigma = \sum_{n,a,i,j}{ \as^n\left( \mu_{F,j} \right) F_a\left( x_i, \mu_{F,j}\right)\mathcal{C}_{n,a,i,j} },
\end{equation}
where the magnitude of the strong coupling and PDFs can be chosen independently, without repeating the time-consuming MC integration. Because certain parton configurations lead to the identical final-state, only weights for the linear combinations of PDFs $a=\left( \Delta, \Sigma, g\right) $ are stored. The weights $\mathcal{C}_{n,a,i,j}$ are organised in form of a table and can be effectively processed leading to the jet cross sections recalculation time of the order $\mathcal{O}\left( ms\right)$ sufficient for the usage in the fitting procedure.

The predictions for the inclusive-jet cross sections were calculated using the \nlojet program with the \fastnlo interface. The setting for the NLO pQCD calculations were described in detail in Section~\ref{sec:nlopredictions} and are briefly summarised in the Table~\ref{tab:nlosettings}.
\begin{table}[h]
\centering
\begin{tabular}{l|c}
Parameter  & Default setup \\ 
\hline \hline proton PDF set & \herapdf1.5 (NLO) \\
\hline renormaliastion scale & $\mu_R^2=\qsq + \etjetb^2$ \\ 
\hline factorisation scale          & $\mu_F^2=\qsq $ \\ 
\hline number of active flavours    & $n_f = 5 $ \\ 
\end{tabular} 
\caption{Summary of the theory settings used for the calculations for the \as determination.}
\label{tab:nlosettings}
\end{table}
The NLO predictions were corrected for electroweak and non-perturbative effects using the MC models as described in Sections~\ref{electroweak hadronisation}.
