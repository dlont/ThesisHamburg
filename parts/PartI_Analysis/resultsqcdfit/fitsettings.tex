The predictions for the inclusive-jet cross sections were calculated using the \nlojet program with the \fastnlo interface. The setting for the NLO pQCD calculations were described in detail in Section~\ref*{sec:nlopredictions} and are briefly summarised in the Table~\ref{tab:nlosettings}.
\begin{table}[h]
\centering
\begin{tabular}{l|c}
Parameter  & Default setup \\ 
\hline \hline proton PDF set & \herapdf1.5 (NLO) \\
\hline renormaliastion scale & $\mu_R^2=\qsq + \etjetb^2$ \\ 
\hline factorisation scale          & $\mu_F^2=\qsq $ \\ 
\hline number of active flavours    & $n_f = 5 $ \\ 
\end{tabular} 
\caption{Summary of the theory settings used for the calculations for the \as determination.}
\label{tab:nlosettings}
\end{table}
The NLO predictions were corrected for electroweak and non-perturbative effects using the MC models as described in Sections~\ref{electroweak hadronisation}.

These setting define the, so-called ``central-fit''. The value of the strong coupling determined with this setup is used as a reference for the studies of the sensitivity of the extracted \asz to the variation of the theory parameters that will be discussed in a section devoted to the treatment of systematic uncertainties (see Section~\ref{subsec:assystematics}).
