Agreement between data and theory is estimated using the following ansatz for the $\chi^2$-function~\cite{Aaron:2009aa}:
%\begin{equation}
% \chi^2\left( \mathbf{m},\mathbf{b}\right) = \sum_{i}{\frac{\left[ m^i-\Sigma_j{\gamma^i_jm^ib_j-\mu^i}\right]^2 }{\left( \delta_{i,\text{stat}}\mu^i\right)^2 + \left( \delta_{i,\text{uncor}}\mu^i\right)^2 }} + \sum_{j}{b_j^2}
% \label{eq:chi2uncor}
%\end{equation}
\begin{equation}
 \chi^2\left( \mathbf{m},\mathbf{b}\right) = \sum_{ij}{\left( m^i -\sum_{l}{\Gamma_l^i\left( m^i\right)b_l - \mu^i }\right) C^{-1}_{ij,\text{stat}} \left( m^j -\sum_{l}{\Gamma_l^j\left( m^j\right)b_l - \mu^j }\right) } + \sum_{j}{b_j^2},
 \label{eq:chi2corr}
\end{equation}
where $\mu^i$ is the measured central value of the cross section in bin $i$, the $\Gamma_i^l$ quantifies the sensitivity of the measurement $i$ to the correlated systematic source of uncertainty $l$, the $C^{-1}_{ij,\text{stat}}$ is covariance matrix for the measured data points. The parameter vectors $\mathbf{m}$ and $\mathbf{b}$ represent the theoretical predictions for the point $i$ and systematic uncertainties, respectively. The predictions $\mathbf{m}\left( \mathbf{a},\asz\right)$ are usually treated as functions of the proton PDF parameters, $\mathbf{a}$,  and \asz. The determination procedure of the parametrisation of the pPDFs and \as that best fit the data is based on the numerical minimisation of the $\chi^2$ function with respect to the free parameters. The nuisance parameters $\mathbf{b}$ introduced in the $\chi^2$ definition are fitted together with other parameters in order to consistently take into account systematic uncertainties correlated across measurement points. The last term in Eq.~\eqref{eq:chi2corr} represents the penalty for these parameters.

An efficient code~\cite{herafitter} for QCD analysis...  The \herafitter program is a tool originally developed for least-square fits of the proton PDFs. The other details concerning the $\chi^2$-function definition can be found in the \herafitter manual~\cite{herafitter:2014:manual}. The \herafitter~\cite{Aaron:2009aa,Aaron:2009kv} program package was used for the \as extraction. The main features of the \herafitter framework are outlined below.

The predictions for the inclusive-jet cross sections were calculated using the \nlojet program with the \fastnlo interface. The setting for the NLO pQCD calculations were described in detail in Section~\ref{sec:nlopredictions} and are briefly summarised in the Table~\ref{tab:nlosettings}.
\begin{table}[h]
\centering
\begin{tabular}{l|c}
Parameter  & Default setup \\ 
\hline \hline proton PDF set & \herapdf1.5 (NLO) \\
\hline renormaliastion scale & $\mu_R^2=\qsq + \etjetb^2$ \\ 
\hline factorisation scale          & $\mu_F^2=\qsq $ \\ 
\hline number of active flavours    & $n_f = 5 $ \\ 
\end{tabular} 
\caption{Summary of the theory settings used for the calculations for the \as determination.}
\label{tab:nlosettings}
\end{table}
The NLO predictions were corrected for electroweak and non-perturbative effects using the MC models as described in Sections~\ref{electroweak hadronisation}.

These setting define the, so-called ``central-fit''. The value of the strong coupling determined with this setup is used as a reference for the studies of the sensitivity of the extracted \asz to the variation of the theory parameters that will be discussed in a section devoted to the treatment of systematic uncertainties (see Section~\ref{subsec:assystematics}).
