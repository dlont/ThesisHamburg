QCD has become the accepted theory of the strong interaction. If one ignores the masses of quarks in the QCD Lagrangian~\eqref{eq:qcdlagrangian}\footnote{Such an approximation is valid at sufficiently high energy $m_q \ll E$.} the success of the theory with a single free parameter ($\as=\frac{g^2_s}{4\pi}$) is striking. Nevertheless, precision tests of QCD are ongoing. The value of the strong coupling, extracted from a wide variety of experimental data, can be regarded as a figure of merit unifying diverse approaches for testing QCD. This section contains a brief summary of \as determinations from jet measurements at \hera and \lhc.

Previous determinations of the strong coupling at \hera include extractions from NC DIS inclusive-jet data at high-\qsq~\cite{pl:b649:12,Aktas:2007aa}, low-\qsq~\cite{Aaron:2010ac} as well as from photoproduction data~\cite{pl:b560:7}. The extractions from inclusive-jet data in NC DIS performed at \hera before 2007 were combined, yielding a \hera average~\cite{upub:zp07125:hp07132}. More recent determinations include extractions from normalised inclusive-jet~\cite{epj:c65:363} and multijet~\cite{epj:c75:65} cross sections as well as from the ratio of three-to-two jet rate, $R_{3/2}$,~\cite{thesis:behr:2010} and photoproduction~\cite{np:b864:1}. A preliminary analysis of the data set very similar to the one presented in the thesis was reported in~\cite{upub:zp10002}.

Overall, the result obtained in this work, is in good agreement with previous analyses performed in different kinematic regions at \zeus and with early extractions by \hone. When compared to \hera I results~\cite{pl:b649:12,Aktas:2007aa,pl:b560:7}, this determination features increased statistical precision, especially in the high-\etjetb and high-\qsq region and has experimental precision comparable to \hera (2007) average. The presented extraction is also consistent within experimental uncertainties with more recent determinations from $R_{3/2}$~\cite{thesis:behr:2010} and photoproduction~\cite{np:b864:1} at \zeus. 

Comparing to~\cite{thesis:behr:2010} this determination benefits from the fact that an inclusive analysis gains larger event sample, due to excluding the requirement that both jets have to be within the detector acceptance\footnote{At leading order in pQCD an NC DIS event always has exactly two partons (see Section~\ref{sec:jetalgo}).}. On the other hand, in the ratio $R_{3/2}$ significant amount of systematic effects cancel. 

In comparison to the photoproduction result~\cite{np:b864:1} both determinations have very similar experimental uncertainties. Besides most the same sources of theoretical uncertainty the photoproduction analysis includes an additional source attributed to the knowledge of the photon PDF. Nevertheless, the photoproduction result has smaller theoretical uncertainty when compared to the values obtained in this thesis with the offset method\footnote{In the photoproduction analysis an offset method was used.}. In both cases the dominant source of theoretical uncertainty is the one due to terms beyond NLO, however, better precision of the analysis~\cite{np:b864:1} can be explained by the existence of a unique hard scale, $\etjetlab$, and typically larger jet transverse energy $\left(\etjetlab > 21\GeV\;\right)$.

This \asz determination is consistent with the preliminary result, obtained in an independent inclusive-jet study~\cite{upub:zp10002}, which was also based on \hera II data but performed within somewhat different phase space. That determination has slightly larger experimental uncertainty, while the theoretical error of that analysis benefits from using the reduced dataset with lower sensitivity to the variations in pQCD predictions and has to be compared to the uncertainty treatment, performed using the Hesse method, where the varying sensitivity of the data to pQCD predictions was taken into account. 

Normalised inclusive-jet~\cite{epj:c65:363} and multijet~\cite{epj:c75:65} cross sections from \hone experiment are characterised by small experimental uncertainty and are typically lower than that from \zeus. High experimental precision in these determinations was achieved by utilising normalised jet cross sections in which systematic effects cancel. Besides that different sensitivity of various jet data sets to pQCD predictions (e.g. NLO pQCD predictions for trijet cross sections are $\mathcal{O}\left(\as^3\right)$) was exploited. %Thus, slight tension ($2-3\;\sigma_\text{exp.}$) between current determinations and mentioned \hone results may potentially arise after reanalysis of the jet measurements with more precise theoretical predictions\footnote{As mentioned before, the main limitation on the precision of \asz, extracted from jet measurements, is due to the unknown contribution of the truncated terms of the perturbative series. As demonstrated in~\cite{Currie:2013dwa}, scale variation uncertainty for the inclusive-jet production in $pp$-collisions for $80<p_T^{jet}<97$ \GeV~is of the order $\pm2\%$. Thus, assuming similar uncertainty for jet cross sections in NC DIS, the resulting scale variation uncertainty on \asz will be of the order $\pm2\%$. It is also assumed, the central \asz values shift in a correlated way for all extractions.}.

The extractions of \asz from \hera are complementary to the determinations from jet cross sections in $pp$ collisions at the \lhc. When compared to \cms results~\cite{Chatrchyan:2012bja,Chatrchyan:2013txa,CMS:2014mna}, the value of \asz obtained in this work features similar theoretical uncertainty and somewhat larger experimental uncertainty. The \cms values are systematically lower then that presented in this thesis.

All discussed determinations including world average~\cite{PDG:2014} are summarised in Figure~\ref{fig:as_php_dis}.
\begin{figure}[htpb]
	\centering
		\includegraphics[width=\textwidth]{Figures/alphas/as_php_dis}
	\caption{Comparison of the values of the strong coupling constant determined at \hera by \hone and \zeus collaborations as well as those from CMS. The green band indicates the uncertainty on the world average \as value. Solid and dashed lines correspond to combined experimental and theoretical uncertainties, respectively.}
	\label{fig:as_php_dis}
\end{figure}
