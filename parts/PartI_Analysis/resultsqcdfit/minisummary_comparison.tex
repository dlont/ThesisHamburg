%\setstcolor{blue}
%\begin{itemize} \itemsep-5pt \parskip0pt
	%\item Introduction. Why various measurements of \as.
	%\begin{itemize} \itemsep-5pt \parskip0pt
		%\item Strong coupling from the jet measurements.
	%\end{itemize}
	%\item \st{Comparison} of \as extractions from jets at \hera.
	%\begin{itemize} \itemsep-5pt \parskip0pt
		%\item Comparison to HERAI (stress strong and weak points)
		%\item ... to low and high-\qsq data, to photoproduction.
		%\item ... to \hone
		%\item ... to \lhc measurements.
	%\end{itemize}
	%\item Conclusions?
%\end{itemize}
QCD has proven to be the theory of the strong interaction. If one ignores the masses of quarks in the QCD Lagrangian~\eqref{eq:qcdlagrangian}\footnote{Such an approximation is valid at sufficiently high energy $m_q \ll E$.} the success of the theory with a single free parameter ($\as=\frac{g^2_s}{4\pi}$) is striking. Nevertheless, precision tests of QCD are ongoing. The value of the strong coupling, extracted from a wide variety of experimental data, can be regarded as a figure of merit unifying diverse approaches for testing QCD. This section contains a brief summary of \as determinations from jet measurements at \hera and \lhc.

Previous determinations of the strong coupling at \hera include extractions from NC DIS inclusive-jet data at high-\qsq~\cite{pl:b649:12,Aktas:2007aa}, low-\qsq~\cite{Aaron:2010ac} as well as from photoproduction data~\cite{pl:b560:7}. The extractions from inclusive-jet data in NC DIS performed at \hera before 2007 were combined, yielding \hera average~\cite{upub:zp07125:hp07132}. More recent determinations include extractions from normalised inclusive-jet~\cite{epj:c65:363} and multijet~\cite{epj:c75:65} cross sections as well as from the ratio of three-to-two jet rate, $R_{3/2}$,~\cite{thesis:behr:2010} and photoproduction~\cite{np:b864:1}. A preliminary analysis of the data set very similar to the one presented in the thesis was reported in~\cite{upub:zp10002}.

Overall, the result, obtained in this work, is in good agreement with previous analyses performed in different kinematic regions at \zeus and with early \hone extraction. When compared to \hera I results~\cite{pl:b649:12,Aktas:2007aa,pl:b560:7}, this determination features increased statistical precision, especially in the high-\etjetb and high-\qsq region and have experimental precision comparable to \hera (2007) average. The presented extraction is also consistent within experimental uncertainties with more recent determinations from $R_{3/2}$~\cite{thesis:behr:2010} and photoproduction~\cite{np:b864:1} at \zeus. Moreover, this \asz determination is consistent with the preliminary result, obtained in independent inclusive-jet study~\cite{upub:zp10002}, which, as presented determination, was also based on \hera II data. %Although, that analysis benefits from using the reduced dataset featuring lower sensitivity to the variations in pQCD predictions. 

Normalised inclusive-jet~\cite{epj:c65:363} and multijet~\cite{epj:c75:65} cross sections from \hone experiment are characterised by small experimental uncertainty. Thus, slight tension ($2-3\;\sigma_\text{exp.}$) between current determinations and mentioned \hone results may potentially arise after reanalysis of the jet measurements with more precise theoretical predictions\footnote{As mentioned before, the main limitation on the precision of \asz, extracted from jet measurements, is due to the unknown contribution of the truncated terms of the perturbative series. As demonstrated in~\cite{Currie:2013dwa}, scale variation uncertainty for the inclusive-jet production in $pp$-collisions for $80<p_T^{jet}<97$ \GeV~is of the order $\pm2\%$. Thus, assuming similar uncertainty for jet cross sections in NC DIS, the resulting scale variation uncertainty on \asz will be of the order $\pm2\%$. It is also assumed, the central \asz values shift in a correlated way for all extractions.}.

The extractions of \asz from \hera are complementary to the determinations from jet cross sections in $pp$ collisions at the \lhc. When compared to the the \cms results~\cite{Chatrchyan:2012bja,Chatrchyan:2013txa,CMS:2014mna}, the value of \asz obtained in this work, features similar theoretical uncertainty and somewhat larger experimental uncertainty. The \cms values are systematically lower then the presented one.

All discussed determinations including world average~\cite{PDG:2014} are summarised in Figure~\ref{fig:as_php_dis}.
\begin{figure}[htpb]
	\centering
		\includegraphics[width=\textwidth]{Figures/alphas/as_php_dis}
	\caption{Comparison of the values of the strong coupling constant determined at \hera by \hone and \zeus collaborations as well as those from CMS. The green band indicates the uncertainty on world average \as value. Solid and dashed lines correspond to combined experimental and theoretical uncertainties, respectively.}
	\label{fig:as_php_dis}
\end{figure}
