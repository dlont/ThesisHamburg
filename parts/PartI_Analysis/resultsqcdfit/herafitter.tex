The \herafitter program is a tool originally developed for least-square fits of the proton PDFs. Nowadays, it is an efficient code for QCD analysis of high-energy collider data. In this analysis the \herafitter~\cite{Aaron:2009aa,Aaron:2009kv} program package was used for \as extraction as well as for simultaneous \as+ PDF fits.

The minimisation of the $\chi^2$-function is performed employing the iterative \migrad algorithm from the \minuit package~\cite{James:1975dr}. The minimisation procedure requires repeated recalculation of the $\chi^2$-function. Hence, the NLO pQCD predictions for the jet observables, when included in the fit, have to be calculated repeatedly in each iteration of the QCD-fit. However, evaluation of the predictions for the jet production in an iterative procedure using the techniques described in Section~\ref{sec:nlojet} has two main limitations: 
\begin{itemize}
 \item due to stochastic nature of the Monte Carlo approach for calculation of the phase space integrals, the $\chi^2$ minimisation procedure,  in general, may not converge;
 \item the computing time for a single iteration (i.e. evaluation of the $\chi^2$ function for a single fit parameter value) is prohibitively long.
\end{itemize}
Therefore, in the iterative fit procedure the \fastnlo framework for the recalculations of the NLO predictions for the jet production have been used. 

The \fastnlo~\cite{thesis:wobisch:2001,Kluge:2006,Wobisch:2011,Britzger:2012} approach is a based on the idea of calculating the LO and NLO weights, $\mathcal{C}_{n,a,i,j}$, which are independent of the \as and PDFs, on a 2d-grid in convolution variable $x$ and factorisation scale $\mu_F$. Thus, the cross section can be obtained from a simple sum:
\begin{equation}
\sigma = \sum_{n,a,i,j}{ \as^n\left( \mu_{F,j} \right) F_a\left( x_i, \mu_{F,j}\right)\mathcal{C}_{n,a,i,j} },
\end{equation}
where the magnitude of the strong coupling and PDFs can be chosen independently, without repeating the time-consuming MC integration. Because certain parton configurations lead to the identical final-state, only weights for the linear combinations of PDFs $a=\left( \Delta, \Sigma, g\right) $ are stored, where $\Delta=\sum_a{e_q^2\left(q_a\left(x\right)+\bar{q}_a\left(x\right)\right)}, \Sigma=\sum_a{q_a\left(x\right)+\bar{q}_a\left(x\right)}$ and $g=g\left(x\right)$ is the gluon PDF. The weights $\mathcal{C}_{n,a,i,j}$ are organised in form of a table and can be effectively processed leading to the jet cross sections recalculation time of the order $\mathcal{O}\left( ms\right)$ sufficient for the usage in the fitting procedure.

The predictions for the inclusive-jet cross sections were calculated using the \nlojet program with the \fastnlo interface. The setting for the NLO pQCD calculations were described in detail in Section~\ref{sec:nlopredictions} and are briefly summarised in the Table~\ref{tab:nlosettings}.
\begin{table}[h]
\centering
\begin{tabular}{l|c}
Parameter  & Default setup \\ 
\hline \hline proton PDF set & \herapdf1.5 (NLO) \\
\hline renormaliastion scale & $\mu_R^2=\qsq + \etjetb^2$ \\ 
\hline factorisation scale          & $\mu_F^2=\qsq $ \\ 
\hline number of active flavours    & $n_f = 5 $ \\ 
\end{tabular} 
\caption{Summary of the theory settings used for the calculations for the \as determination.}
\label{tab:nlosettings}
\end{table}
Unless otherwise stated, these setting define the, so-called ``central-fit''. The value of the strong coupling determined with this setup is used as a reference for the studies of the sensitivity of the extracted \asz to the variation of the theory parameters that will be discussed in a section devoted to the treatment of systematic uncertainties (see Section~\ref{subsec:assystematics}).

Withing the \herafitter program, the evolution of the parton densities according to the DGLAP-equations is performed using the \qcdnum code~\cite{Botje:2010ay}. In the \qcdnum approach the equations are solved numerically on a $n\times m$ lattice in $x$ and $\mu$ variables. The method is based on numerical spline interpolation of the parton densities on equidistant logarithmic grid in $x$ and logarithmic grid\footnote{(Non-)equidistant $\mu^2$-grid can be used as an option.} in $\mu^2$. Thus, the integral convolution is represented as a finite sum and $t \equiv \ln{\mu^2}$-integration is formulated as an interative process. The analytic form of the DGLAP-equations and specific choice of spline-type were successfully exploited to recast the continuous evolution into linear problem that admits very efficient numerical solution.

The NLO predictions were corrected for electroweak and non-perturbative effects using the MC models as described in Sections~\ref{electroweak hadronisation}.

\textcolor{blue}{The uncertainties due to experimental precision of the data are propagated to the determined parameters by means of ``Hesse''-method implemented in \minuit. The other details concerning the $\chi^2$-function definition can be found in the \herafitter manual~\cite{herafitter:2014:manual}.}