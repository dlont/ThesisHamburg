The proton PDFs are non-perturbative quantities, predictions for only certain moments of which are possible at the moment~\cite{Hagler:2009ni}. The most precise extraction of the proton PDFs nowadays is achieved in QCD analysis of experimental data.

The minimisation of the $\chi^2$ is performed employing the iterative \migrad algorithm from the \minuit package~\cite{James:1975dr}. The uncertainties due to experimental precision of the data are propagated to the determined parameters by means of ``Hesse''-method implemented in \minuit.

The minimisation procedure requires repeated recalculation of the $\chi^2$-function, hence the NLO pQCD predictions for the jet observables, if included in the fit, has to be calculated at each iteration in the QCD-fit. However, evaluation of the predictions for the jet production in an iterative fit using the techniques described in Section~\ref{sec:nlojet} has two main limitations: 
\begin{itemize}
 \item due to stochastic nature of the Monte Carlo approach for calculation of the phase space integrals, the $\chi^2$ minimisation procedure,  in general, may not converge;
 \item the computing time for a single iteration (i.e. evaluation of the $\chi^2$ function for a single fit parameter value) is prohibitively long.
\end{itemize}
Therefore, in the iterative fit procedure the \fastnlo framework for the recalculations of the NLO predictions for the jet production have been used. A brief summary of the \fastnlo approach is provided in the next section. 
