The \herafitter program~\cite{Aaron:2009aa,Aaron:2009kv} is a tool originally developed for least-square fits of the proton PDFs. Nowadays, it is an efficient code for the QCD analysis of high-energy collider data. In this analysis the \herafitter program package was used for extraction of \as as well as for simultaneous \as+ PDF fits.

The minimisation of the $\chi^2$-function is performed employing the iterative \migrad algorithm from the \minuit package~\cite{James:1975dr}. The minimisation procedure requires repeated recalculation of the $\chi^2$-function. Hence, the NLO pQCD predictions for the jet observables, when included in the fit, have to be calculated repeatedly in each iteration of the QCD fit. However, evaluation of the predictions for the jet production in an iterative procedure using the techniques described in Section~\ref{sec:nlojet} has two main limitations: 
\begin{itemize}
 \item due to the stochastic nature of the Monte Carlo approach for calculation of the phase space integrals in the NLO calculation, the $\chi^2$ minimisation procedure, in general, may not converge, because of fluctuations of the cross sections predictions;
 \item the computing time for a single iteration (i.e. evaluation of the $\chi^2$ function for given fit parameter values) is prohibitively long.
\end{itemize}
Therefore, for the iterative fit procedure, the \fastnlo framework is used for the recalculation of the NLO predictions for jet production. 

The \fastnlo~\cite{thesis:wobisch:2001,Kluge:2006,Wobisch:2011,Britzger:2012} approach is a based on the idea of calculating the LO and NLO weights, $\mathcal{C}_{n,a,i,j,k}$, which are independent of \as and the PDFs, on a 3d-grid in the convolution variable $x_i$ and the factorisation and renormalisation scales $\mu_{F,j}, \mu_{R,k}$; index $a$ runs over the number of contributing processes. The cross section can be obtained from a simple sum:
\begin{equation}
\sigma = \sum_{n,a,i,j,k}{ \as^n\left( \mu_{R,k} \right) F_a\left( x_i, \mu_{F,j}\right)\mathcal{C}_{n,a,i,j,k} },
\end{equation}
where the magnitude of the strong coupling and PDFs can be chosen independently, without repeating the time-consuming MC integration. Because certain parton configurations lead to the identical final-state, only weights for the linear combinations of PDFs $a=\left( \Delta, \Sigma, g\right) $ are stored, where $\Delta=\sum_a{e_q^2\left(q_a\left(x\right)+\bar{q}_a\left(x\right)\right)}$, $\Sigma=\sum_a{q_a\left(x\right)+\bar{q}_a\left(x\right)}$ and $g=g\left(x\right)$ is the gluon PDF. The weights $\mathcal{C}_{n,a,i,j,k}$ are organised in form of a table and can be efficiently processed leading to a jet cross sections calculation time $\mathcal{O}\left( ms\right)$, to be practical for the usage in the fitting procedure.

The predictions for the inclusive-jet cross sections were calculated using the \nlojet program with the \fastnlo interface. The setting for the NLO pQCD calculations were described in detail in Section~\ref{sec:nlopredictions} and are briefly summarised in Table~\ref{tab:nlosettings}.
\begin{table}[h]
\centering
\begin{tabular}{l|c}
Parameter  & Default setup \\ 
\hline \hline proton PDF set & \herapdf1.5 (NLO) \\
\hline renormalisation scale & $\mu_R^2=\qsq + \etjetb^2$ \\ 
\hline factorisation scale          & $\mu_F^2=\qsq $ \\ 
\hline number of active flavours    & $n_f = 5 $ \\ 
\end{tabular} 
\caption{Summary of the theory settings used for the calculations for the \as determination.}
\label{tab:nlosettings}
\end{table}
Unless otherwise stated, these settings define the so-called ``central-fit''. The value of the strong coupling determined with this setup is used as a reference for studies of the sensitivity of the extracted \asz to the variation of the theory parameters discussed in Section~\ref{subsec:assystematics}.

Within the \herafitter program, the evolution of the parton densities according to the DGLAP-equations is performed using the \qcdnum code~\cite{Botje:2010ay}. In the \qcdnum approach the equations are solved numerically on a $n\times m$ lattice in the $x$ and $\mu=\mu_F$ variables. The method is based on a numerical spline interpolation of the parton densities on an equally spaced grid in $\ln\left(x\right)$ and in $\ln\left(\mu^2\right)$\footnote{An unequally spaced $\mu^2$-grid can be set as an option.} . Thus, the convolution (see Eqn.~\ref{DGLAP}) is represented as a finite sum and the $t \equiv \ln{\mu^2}$ integration is formulated as an iterative process.
The analytic form of the DGLAP-equations and a specific choice of the type of interpolation were used to recast the continuous evolution into a linear problem that admits a very efficient numerical solution. For the sake of illustration, an example for the evolution of a non-singlet combination of parton distributions is considered. Let $\mathbf{h_i}=\hat{S}\mathbf{a_i}$ denote vectors of the non-singlet PDF values on the discrete $\x$ grid determined at different factorisation scales $t_0 < ... < t_i=\ln{\mu^2_i} < t_n$, where $t_0=\ln{Q^2_0}$ is the initial scale and $\mathbf{a_i}$ is the corresponding vector of spline coefficients, while the matrix $\hat{S}$ contains spline-specific constants. Then, for the initial scale $t_0$, the discretised version of the DGLAP equations has the following form:
\begin{equation}
 \frac{\mathrm{d}\mathbf{h_0}}{\mathrm{d}t} \equiv \frac{\mathrm{d}\hat{S}\mathbf{a_0}}{\mathrm{d}t} = \hat{W_0}\mathbf{a_0},
\end{equation}
where $\hat{W_0}$ represents the $\x$-space PDF convolution. The subscript of matrix $\hat{W_i}$ corresponds to the grid node $t_i$, while the r.h.s of the DGLAP equations includes $\as\left(t_i\right)$. Moving $\hat{S}$ to the r.h.s of the equation gives:
\begin{equation}
 \frac{\mathrm{d}\mathbf{a_0}}{\mathrm{d}t} = \hat{S}^{-1}\hat{W_0}\mathbf{a_0}.
\end{equation}
Approximating the infinitesimal differential $\mathrm{d}t$ by small $\Delta t$, the equation can be solved numerically for the scale $t_1=t_0+\Delta t$:
\begin{equation}
 \mathbf{a_1} = \hat{S}^{-1}\hat{W_0}\mathbf{a_0}\Delta t.
\end{equation}
The solution can be iterated to find PDFs for all scales $t_i>t_0$\footnote{In principle, backward evolution $t_i<t_0$ is possible.}. However, in the QCDNUM method, a more precise higher-order integration scheme was proposed~\cite{Botje:2010ay}.

Because the \nlojet program only provides predictions for the single-photon exchange, the NLO calculations were corrected for electroweak and non-perturbative effects using the MC models as described in Sections~\ref{subsec:z0corr,subsec:hadrcorr}, to match the definition of the cross section in the data.
