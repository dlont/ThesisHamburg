As a next step, the sensitivity of the measured double-differential cross sections to the value of the strong coupling was investigated. Similarly to the single-differential cross section case considered above, \as-values were determined from individual measurements and in different regions of \qsq. The results of the extraction are summarised in Figure~\ref{fig:asindividual_dd} and Table~\ref{tab:asdoublediff}. 
\begin{table}[h]
 \centering
 \begin{tabular}{|c|c|c|c|c|}
 \hline
 \qsq-range & $\asz \pm $ exp. unc. & $N_\text{DF}$ & $\chi^2/N_\text{DF}$ & Prob $\chi^2 > \chi^2_\text{obs}$ \\
 \hline
 \hline
 $125 < \qsq < 250\;\GeV^2$    & $0.1206 \pm 0.0034$ & 5 & 0.18 & 0.97\\
 $250 < \qsq < 500\;\GeV^2$    & $0.1189 \pm 0.0036$ & 5 & 0.60 & 0.7\\
 $500 < \qsq < 1000\;\GeV^2$   & $0.1224 \pm 0.0038$ & 5 & 0.56 & 0.73\\
 $1000 < \qsq < 2000\;\GeV^2$  & $0.1242 \pm 0.0039$ & 5 & 0.61 & 0.69\\
 $2000 < \qsq < 5000\;\GeV^2$  & $0.1248 \pm 0.0046$ & 5 & 1.44 & 0.17\\
 $5000 < \qsq < 20000\;\GeV^2$ & $0.1246 \pm 0.0073$ & 5 & 0.58 & 0.72\\
 \hline
 \end{tabular}
 \caption{Values of \asz obtained in a fit to the measured double-differential inclusive-jet cross sections. The experimental uncertainty includes statistical as well as correlated and uncorrelated systematic uncertainties.}
 \label{tab:asdoublediff}
\end{table}

Several observations can be made:
\begin{itemize}
 \item The $\chi^2/N_\text{DF}$ of the fits to \dsdetjetb in different regions of \qsq varies between 0.18 and 1.44, but is typically about 0.6. In general, such values can be considered as a sign of a reasonable fit quality.
 \item The minimal value of $\chi^2/N_\text{DF}$ arise from the lowest \qsq bin, which, in principle, is characterised by the smallest total experimental uncertainty. In fact, it may indicate an overestimation of experimental errors in this region.
 \item All values of \asz determined in different bins of \qsq are in good agreement within experimental uncertainties.
 \item The \as-values extracted in the four regions in the range $125 < \qsq < 2000\;\GeV$ have comparable experimental precision. The values determined from the measurements in the two last \qsq bins have larger uncertainty, which is due to the statistical precision of the data in this region.
\end{itemize}

The value of the strong coupling extracted from the simultaneous fit to all points of the measured double-differential cross section is
\begin{equation}
 \asz = 0.1218 \pm 0.0028 \left( \text{exp.} \right). %^{+0.0079}_{-0.0061} \left( \text{scales} \right) ^{+0.0028}_{-0.0010} \left( \text{pdf}\right)
 \label{eq:asdoubledifval}
\end{equation}
This is consistent with the extractions from the individual \qsq bins. The experimental uncertainty was obtained by propagating the uncertainties listed in the previous sections. The obtained fitted nuisance parameters $b_{\mu}$ for the correlated systematic uncertainties are $b_{JES}=-0.04$ and $b_{lumi}=-0.2$ for the absolute jet-energy calibration and the luminosity error, respectively. Because the parameters $b_{\mu}$ are introduced with a negative sign in Eq.~\eqref{eq:chi2corr}, it indicates that the \as-fit prefers the 'down-shift' of the measured cross sections. It is important to note that the extracted \as-values are very sensitive to the overall normalisation of the data, because, as it can be seen from perturbative predictions for the jet production, the absolute cross section value depends monotonically on \asz.

\begin{figure}[p]
 \centering
 \includegraphics[height=0.85\textheight]{Figures/alphas/alphas_individual_dd.pdf}
 % alphas_individual_dd.png: 796x1172 pixel, 72dpi, 28.08x41.35 cm, bb=0 0 796 1172
 \caption{Values of \asz extracted from the individual measurements of the \dsdetjetb cross sections in different regions of \qsq. The determined values are compared to the result of a simultaneous fit to all data points in the separate \qsq ranges (pink line and green band) and to all measured data points (orange line and dotted band). The error bars for individual points indicate the size of the total experimental uncertainty, while the green bands represents that for the simultaneous fits.}
 \label{fig:asindividual_dd}
\end{figure}

% 
% \begin{align}
% 	:\qquad \asz &=  \left( \text{exp.} \right) ^{+0.0033}_{-0.0044} \left( \text{th.} \right) \notag \\
% 	:\qquad \asz &= 0.1221 \pm 0.0027 \left( \text{exp.} \right) ^{+0.0033}_{-0.0044} \left( \text{th.} \right) \notag \\
% 	:\qquad \asz &= 0.1221 \pm 0.0027 \left( \text{exp.} \right) ^{+0.0033}_{-0.0044} \left( \text{th.} \right) \notag \\
% 	:\qquad \asz &= 0.1221 \pm 0.0027 \left( \text{exp.} \right) ^{+0.0033}_{-0.0044} \left( \text{th.} \right) \notag \\
% 	:\qquad \asz &= 0.1221 \pm 0.0027 \left( \text{exp.} \right) ^{+0.0033}_{-0.0044} \left( \text{th.} \right) \notag \\
% 	:\qquad \asz &= 0.1221 \pm 0.0027 \left( \text{exp.} \right) ^{+0.0033}_{-0.0044} \left( \text{th.} \right) 	
% \end{align}
