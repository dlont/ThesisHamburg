As a next step the sensitivity of the measured double-differential cross sections to the value of the strong coupling was investigated. Similar to single-differential cross section case, considered above, \as-values were determined from individual measurements and in different regions of \qsq. The results of the extraction are summarised in Figure~\ref{fig:asindividual_dd} and Table~\ref{tab:asdoublediff}. 
\begin{table}[h]
 \centering
 \begin{tabular}{|c|c|c|c|}
 \hline
 \qsq-range & $\asz \pm $ exp. unc. & $N_\text{DF}$ & $\chi^2/N_\text{DF}$ \\
 \hline
 \hline
 $125 < \qsq < 250\;\GeV^2$    & $0.1206 \pm 0.0034$ & 5 & 0.18 \\
 $250 < \qsq < 500\;\GeV^2$    & $0.1189 \pm 0.0036$ & 5 & 0.60 \\
 $500 < \qsq < 1000\;\GeV^2$   & $0.1224 \pm 0.0038$ & 5 & 0.56\\
 $1000 < \qsq < 2000\;\GeV^2$  & $0.1242 \pm 0.0039$ & 5 & 0.61\\
 $2000 < \qsq < 5000\;\GeV^2$  & $0.1248 \pm 0.0046$ & 5 & 1.44\\
 $5000 < \qsq < 20000\;\GeV^2$ & $0.1246 \pm 0.0073$ & 5 & 0.58\\
 \hline
 \end{tabular}
 \caption{Values of \asz obtained in a fit to the measured double-differential inclusive-jet cross sections. The experimetal uncertainty includes statistical as well as correlated and uncorrelated systematic uncertainties.}
 \label{tab:asdoublediff}
\end{table}

Several observations can be made:
\begin{itemize}
 \item The $\chi^2/N_\text{DF}$ of the fits to \dsdetjetb in different regions of \qsq varies between 0.18 and 1.44, but is typically about 0.6. In general, it can be considered as a good fit, however, extremal values may indicate some problems.
 The minimal value of $\chi^2/N_\text{DF}$ arise from the lowest \qsq bin, which, in principle, is characterised by the smallest total experimental uncertainty. In fact, it may indicate an overestimation of experimental errors in this region. On the other hand, the quite large value $\chi^2/N_\text{DF}=1.44$ in $2000 < \qsq < 5000\;\GeV$ bin was attributed to the data fluctualtion in the last statistical-uncertainty dominated bin and therefore was considered reasonable.
 \item All values of \asz determined in different bins of \qsq are in good agreement within experimental uncertainties, however the central values determined in range $125 < \qsq < 1000\;\GeV$ are somewhat lower than those determined from high-\qsq bins.
 \item The \as-values extracted in region $125 < \qsq < 2000\;\GeV$ have comparable experimental precision. The values determined from the measurements in two last \qsq bins have larger uncertainty, which is due to statistical precision of the data in this region.
\end{itemize}

The value of the strong coupling extracted from the simultaneous fit to all points of the measured double-differential cross section is
\begin{equation}
 \asz = 0.1218 \pm 0.0028 \left( \text{exp.} \right). %^{+0.0079}_{-0.0061} \left( \text{scales} \right) ^{+0.0028}_{-0.0010} \left( \text{pdf}\right)
 \label{eq:asdoubledifval}
\end{equation}
It is consistent with extractions from individual \qsq bins. The experimental uncertainty was obtained propagating the uncertainties listed in the previous sections. The nuisance parameters $b_{\mu}$ for the correlated systematic uncertainties are $b_{JES}=-0.04$ and $b_{lumi}=-0.2$ for absolute jet-energy calibration and luminosity error, respectively. Because the parameters $b_{\mu}$ are introduced with a negative sign in Eq.~\eqref{eq:chi2corr}, it indicates that the \as-fit prefers the 'down-shift' of the measured cross sections. It is important to note, that the extracted \as-values are very sensitive to the overall normalisation of the data, beacause, as it can be seen from perturbative predictions for the jet production, the absolute cross section value depends monotonicaly on the \asz.

\begin{figure}
 \centering
 \includegraphics[height=\textheight,bb=0 0 796 1172]{Figures/alphas/alphas_individual_dd}
 % alphas_individual_dd.png: 796x1172 pixel, 72dpi, 28.08x41.35 cm, bb=0 0 796 1172
 \caption{Values of \asz extracted from individual measurements of \dsdetjetb cross sections in different regions of \qsq. The determined values are compared to the result of simultaneous fit to all data points in separate \qsq ranges (pink line and green band) and to all measured data points (orange line and dotterd band). The error bars for individual points indicate the size of the total experimental uncertainty, while the green band represents that for the simultaneous fits.}
 \label{fig:asindividual_dd}
\end{figure}

% 
% \begin{align}
% 	:\qquad \asz &=  \left( \text{exp.} \right) ^{+0.0033}_{-0.0044} \left( \text{th.} \right) \notag \\
% 	:\qquad \asz &= 0.1221 \pm 0.0027 \left( \text{exp.} \right) ^{+0.0033}_{-0.0044} \left( \text{th.} \right) \notag \\
% 	:\qquad \asz &= 0.1221 \pm 0.0027 \left( \text{exp.} \right) ^{+0.0033}_{-0.0044} \left( \text{th.} \right) \notag \\
% 	:\qquad \asz &= 0.1221 \pm 0.0027 \left( \text{exp.} \right) ^{+0.0033}_{-0.0044} \left( \text{th.} \right) \notag \\
% 	:\qquad \asz &= 0.1221 \pm 0.0027 \left( \text{exp.} \right) ^{+0.0033}_{-0.0044} \left( \text{th.} \right) \notag \\
% 	:\qquad \asz &= 0.1221 \pm 0.0027 \left( \text{exp.} \right) ^{+0.0033}_{-0.0044} \left( \text{th.} \right) 	
% \end{align}
