As a first step \asz is determined in a fit to individual measurements of single-differential inclusive-jet cross section as a function of $\etjetb$. Because single parameter is extracted from exactly one point the minimum of the $\chi^2$-function is attained when $\chi^2=0$. The systematic shifts, represented by nuisance paramters, cannot be determined in this case. Thus, the only constraint, imposed by the penalty terms, ensures vanishing of the systematic shifts. The extracted values of \asz are illustrated in Figure~\ref{fig:alphassinglediffindividual}. Overall, all \as-values have comparable experimental uncertainty. However, the \as-value determined from the highest-\etjetb point is characterised by somewhat smaller experimental precision which is due to larger statistical uncertainty of the data in this region of phase space. In general, \as-values from individual measurements are in very good agreement withing the experimental uncertainties, which indicates the overall consistency of all mesurements.

The combined \asz-value determined from a simultaneous fit to all measured data point is
\begin{equation}
 \asz = 0.1195 \pm 0.0029 \left( \text{exp.} \right) %^{+0.0079}_{-0.0061} \left( \text{scales} \right) ^{+0.0028}_{-0.0010} \left( \text{pdf}\right)
\end{equation}
and is also demonstrated in the Figure~\ref{fig:alphassinglediffindividual}. The quality of combined fit is characterised by $\chi^2/N_\text{DF}=1.79/5$ which is below 1 and may indicate an overestimation of the experimental uncertainties.

\begin{figure}[tp]
 \begin{center}
 \includegraphics[width=\textwidth,bb=0 0 1196 772]{Figures/alphas/alphas_individual_sd.png}
 % alphas_individual_sd.png: 1196x772 pixel, 72dpi, 42.19x27.23 cm, bb=0 0 1196 772
\end{center}
 \caption{Values of \asz extracted from individual measurements of \dsdetjetb cross section. The determined values are compared to the result of simultaneous fit to all data points in the measured \etjetb range (pink line and green band). The error bars for individual points indicate the size of the total experimental uncertainty, while the green band represents that for the simultaneous fit.}
 \label{fig:alphassinglediffindividual}
\end{figure}


\begin{figure}[th!]
\centering
\includegraphics[width=0.99\linewidth,trim={0 70 0 0},clip]{./Figures/alphas/alphas_systematics_sd}
\caption{The results of the strong coupling $\asz$ extraction. Summary of individual systematic variations (see text).}
\label{fig:alphas_systematics}
\end{figure}


\begin{align}
	125 < \qsq < 250\;\GeV^2:\qquad \asz &= 0.1221 \pm 0.0027 \left( \text{exp.} \right) ^{+0.0033}_{-0.0044} \left( \text{th.} \right) \notag \\
	250 < \qsq < 500\;\GeV^2:\qquad \asz &= 0.1221 \pm 0.0027 \left( \text{exp.} \right) ^{+0.0033}_{-0.0044} \left( \text{th.} \right) \notag \\
	500 < \qsq < 1000\;\GeV^2:\qquad \asz &= 0.1221 \pm 0.0027 \left( \text{exp.} \right) ^{+0.0033}_{-0.0044} \left( \text{th.} \right) \notag \\
	1000 < \qsq < 2000\;\GeV^2:\qquad \asz &= 0.1221 \pm 0.0027 \left( \text{exp.} \right) ^{+0.0033}_{-0.0044} \left( \text{th.} \right) \notag \\
	2000 < \qsq < 5000\;\GeV^2:\qquad \asz &= 0.1221 \pm 0.0027 \left( \text{exp.} \right) ^{+0.0033}_{-0.0044} \left( \text{th.} \right) \notag \\
	5000 < \qsq < 20000\;\GeV^2:\qquad \asz &= 0.1221 \pm 0.0027 \left( \text{exp.} \right) ^{+0.0033}_{-0.0044} \left( \text{th.} \right) 	
\end{align}
