As a first step \asz is determined in a fit to individual measurements of single-differential inclusive-jet cross section as a function of $\etjetb$. Because single parameter is extracted from exactly one data point the minimum of the $\chi^2$-function is attained when $\chi^2=0$. The systematic shifts, represented by nuisance parameters, cannot be determined in this case. The only constraint, imposed by the penalty terms, ensures vanishing of the systematic shifts. The extracted values of \asz are illustrated in Figure~\ref{fig:alphassinglediffindividual}. Overall, all \as-values have comparable experimental uncertainty. However, the \as-value determined from the highest-\etjetb point is characterised by somewhat smaller experimental precision which is due to larger statistical uncertainty of the data in this region of phase space. In general, \as-values from individual measurements are in very good agreement withing the experimental uncertainties, which indicates the overall consistency of all measurements.

The combined \asz-value determined from a simultaneous fit to all measured data point is
\begin{equation}
 \asz = 0.1195 \pm 0.0029 \left( \text{exp.} \right) %^{+0.0079}_{-0.0061} \left( \text{scales} \right) ^{+0.0028}_{-0.0010} \left( \text{pdf}\right)
 \label{eq:assingledifval}
\end{equation}
and is also demonstrated in the Figure~\ref{fig:alphassinglediffindividual}. The quality of combined fit is characterised by $\chi^2/N_\text{DF}=1.79/5$ which is below 1 and may indicate an overestimation of the experimental uncertainties.

\begin{figure}[tp]
 \begin{center}
 \includegraphics[width=\textwidth,bb=0 0 1196 772,trim=0 100 0 20,clip]{Figures/alphas/alphas_individual_sd}
 % alphas_individual_sd.png: 1196x772 pixel, 72dpi, 42.19x27.23 cm, bb=0 0 1196 772
\end{center}
 \caption{Values of \asz extracted from individual measurements of \dsdetjetb cross section. The determined values are compared to the result of simultaneous fit to all data points in the measured \etjetb range (pink line and green band). The error bars for individual points indicate the size of the total experimental uncertainty, while the green band represents that for the simultaneous fit.}
 \label{fig:alphassinglediffindividual}
\end{figure}

\begin{figure}[t!]
 \centering
 \includegraphics[width=\textwidth,bb=0 0 796 772,trim=0 80 0 20,clip]{Figures/alphas/alphas_running}
 % alphas_running.png: 796x772 pixel, 72dpi, 28.08x27.23 cm, bb=0 0 796 772
 \caption{The \as values determined in each $\left<\etjetb\right>$ from the analysis of the measured \dsdetjetb cross section. The error bars represent the total experimental uncertainty. The solid-line curve represents the renormalisation-group prediction at two loops approximation obtained from the corresponding \asz value determined in this analysis (Eq.~\eqref{eq:assingledifval}).}
 \label{fig:asrunning}
\end{figure}
Besides \asz extraction, the energy-scale dependence of the strong coupling was investigated. The \as-values were determined in the QCD fit to the measured \dsdetjetb values. The predictions for individual \dsdetjetb data points were parametrised in terms of $\as\left(\left<\etjetb\right>\right)$ instead of \asz (see Eq.~\eqref{eq:fitnloqcdpredictions}), where $\left<\etjetb\right>$ is the average \etjetb of the data in particular cross section bin. For this determination the theoretical settings were modified in order to minimise sensitivity of the extraction procedure to assumed running of the strong coupling. By default, the calculations were performed for the renormalisation scale choice $\mu_R=\sqrt{\qsq+\etjetb}$. Therefore, to avoid running from $\mu=\left<\etjetb\right>$ to $\mu=\mu_R=\sqrt{\qsq+\etjetb}$ the renormalisation scale was set to $\mu_R=\etjetb$. Nevertheless, the residual dependence on RGE assumption indirectly persits in pPDF evolution, because the factorisation scale was chosen to be $\mu_F=Q$ and the evolution between two scales $\mu_R = \etjetb \longrightarrow \mu_F = Q$ has to be performed. Figure~\ref{fig:asrunning} shows the extracted values of \as. The data demonstrate the running of the strong coupling over a large range of \etjetb. The renormalisation-group equation QCD predictions determined with two-loop precision~\cite{Gross:1973id, Politzer:1973fx, Gross:1973ju, Politzer:1974fr} are in good agreement with the measured values. Although, the described \as-extraction procedure cannot be treated as an independent test of the coupling running, it illustrates the consistency of the \as determination approach.


% \begin{figure}[th!]
% \centering
% \includegraphics[width=0.99\linewidth,trim={0 70 0 0},clip]{./Figures/alphas/alphas_systematics_sd}
% \caption{The results of the strong coupling $\asz$ extraction. Summary of individual systematic variations (see text).}
% \label{fig:alphas_systematics}
% \end{figure}