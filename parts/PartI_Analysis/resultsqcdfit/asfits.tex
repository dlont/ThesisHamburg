As a first step, \asz is determined in a fit to individual measurements of the single-differential inclusive-jet cross section as a function of $\etjetb$. Since a single parameter is extracted from exactly one data point the minimum of the $\chi^2$-function is attained when $\chi^2=0$. The systematic shifts, represented by nuisance parameters, $\mathbf{b}$ (see~\eqref{eq:measmodel}), cannot be determined in this case and the only constraints, imposed by the penalty terms, ensure vanishing of the systematic shifts. The extracted values of \asz are illustrated in Figure~\ref{fig:alphassinglediffindividual}. Overall, all \as-values have comparable experimental uncertainty. However, the \as-value determined from the highest-\etjetb point is characterised by a somewhat smaller experimental precision which is due to the larger statistical uncertainty of the data in this region of phase space. In general, \as-values from individual measurements are in very good agreement within the experimental uncertainties, which indicates the overall consistency of the measurements.

The combined \asz-value determined from a simultaneous fit to all measured data points is
\begin{equation}
 \asz = 0.1195 \pm 0.0029 \left( \text{exp.} \right) %^{+0.0079}_{-0.0061} \left( \text{scales} \right) ^{+0.0028}_{-0.0010} \left( \text{pdf}\right)
 \label{eq:assingledifval}
\end{equation}
as shown in Figure~\ref{fig:alphassinglediffindividual}. The quality of the combined fit is characterised by $\chi^2/N_\text{DF}=1.79/5$ which is well below 1 indicating an overestimation of the experimental uncertainties.

\begin{figure}[t]
 \centering
 \includegraphics[width=0.7\textwidth,angle=-90]{Figures/alphas/alphas_individual_sd}
 % alphas_individual_sd.png: 1196x772 pixel, 72dpi, 42.19x27.23 cm, bb=0 0 1196 772
 \caption{Values of \asz extracted from the individual \dsdetjetb measurements. The determined values are compared to the result of a simultaneous fit to all data points in the measured \etjetb range (pink line and green band). The error bars for individual points indicate the size of the total experimental uncertainty, while the green band represents that for the simultaneous fit.}
 \label{fig:alphassinglediffindividual}
\end{figure}

\begin{figure}[t!]
 \centering
 \includegraphics[width=\textwidth,angle=-90]{Figures/alphas/alphas_running}
 % alphas_running.png: 796x772 pixel, 72dpi, 28.08x27.23 cm, bb=0 0 796 772
 \caption{The $\as\left(\left<\etjetb\right>\right)$ values determined from the analysis of the measured \dsdetjetb cross sections. The error bars represent the total experimental uncertainty. The solid line represents the renormalisation-group prediction at two-loops approximation obtained from the corresponding \asz value determined in this analysis (Eq.~\eqref{eq:assingledifval}).}
 \label{fig:asrunning}
\end{figure}
Besides the \asz extraction, the energy-scale dependence of the strong coupling was investigated. The \as-values were determined in the QCD fit to the measured \dsdetjetb values. The predictions for individual \dsdetjetb data points were parametrised in terms of $\as\left(\left<\etjetb\right>\right)$ instead of \asz (see Eq.~\eqref{eq:fitnloqcdpredictions}), where $\left<\etjetb\right>$ is the average \etjetb of the data in a particular cross section bin. For this determination the renormalisation and factorisation scales were set to $\mu_R=\etjetb$ and $\mu_F=\qsq$, respectively. Figure~\ref{fig:asrunning} shows the extracted values of \as. The data demonstrate the running of the strong coupling over a large range of \etjetb. The renormalisation-group equation predictions performed with two-loop accuracy~\cite{Gross:1973id, Politzer:1973fx, Gross:1973ju, Politzer:1974fr} are in good agreement with the measured values. Since, the jet production in NC DIS naturally involves two scale $\etjetb$ and $\qsq$, the process scale cannot be unambiguously defined for particular reaction, therefore the described \as-extraction procedure must no be treated as a test of the coupling running, but rather illustrates the self-consistency of the \as determination approach.


% \begin{figure}[th!]
% \centering
% \includegraphics[width=0.99\linewidth,trim={0 70 0 0},clip]{./Figures/alphas/alphas_systematics_sd}
% \caption{The results of the strong coupling $\asz$ extraction. Summary of individual systematic variations (see text).}
% \label{fig:alphas_systematics}
% \end{figure}