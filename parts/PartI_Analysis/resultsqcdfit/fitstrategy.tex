The simplest variant of the fit comprises determination of the strong coupling \asz from inclusive-jet data only, assuming a fixed PDF parametrisation. A more complex strategy includes simultaneous determination of the proton PDFs and \asz from inclusive DIS and inclusive-jet data. The different fit options are discussed below.
 
{\flushleft \textbf{Determination of \asz}}\newline
In this approach the PDFs are assumed to be fixed in independent studies and the theoretical predictions for inclusive jet production are parametrised as functions of a single parameter \asz. Similar analyses have been performed in a variety of environments. Examples are recent studies in \ep~\cite{}, $pp$~\cite{} and $p\bar{p}$~\cite{} collisions. This approach, however, has the following limitations:
\begin{itemize}
 \item the dominant contribution to the jet cross sections measured in this analysis comes from the high-$\x$ region $0.003<x<0.4?$ of the phase space. In this region the knowledge of PDFs from \hera inclusive data alone is limited by statistical precision of the measuremets;
 \item typically, in the global PDF analyses the value of \asz is an external parameter. Thus a particular assumption of \asz may bias the extraction of the strong coupling in the \as-fit, when the proton PDFs are fixed and treated as independent quantities. For consistent determinations of the strong coupling, the value assumed for the PDF and that extracted in the fit must coincide. Recent global analyses provide collection of proton PDF parametrisations extracted assuming different values of \asz. This information can be used to investigate of the influence of the \asz assumption in the used PDF on the \as-results obtained in the fit and to check the consistency of the fitting procedure;
 \item nearly all PDF fitting groups provide an error analysis for their PDF sets. Nevertheless, differences between predictions based on different PDFs are often larger than the uncertainties suggested by the error analyses. Different prescriptions for the estimation of uncertainties attributed to pPDFs exist. One of them, proposed by the PDF4LHC group~\cite{pdf4lhc:2011}, consists in taking an envelope of the predictions based on different PDF sets. Such an approach provides a rather conservative estimate. 
%  Therefore, the dependence of the results of \as-fit on PDF assumption has to be investigated in details.
\end{itemize}

The advantage of the described variant of the fit is its simplicity. The influence of different assumptions can be easily studied because every change in the input manifests itself in the extracted \asz value. 
% \textcolor{blue}{Following sections contain the discussion of the results of this approach.}

{\flushleft \textbf{Simultaneous determination of \asz and PDFs}}\newline
The aim of this approach is to fit \asz and gluon and quark densities simultaneously. For this purpose inclusive-jet measurements are fitted together with inclusive DIS data. Such a treatment takes into account correlations between \as and PDFs and leads to unbiased determination of both quantities. However, the extracted \asz-value will be statistically less precise, than in the case of the fixed PDF option, because of the larger number of parameters fitted to the same data.

To test various fitting options, the \herafitter program~\cite{herafitter} was used in this thesis. The main features of the fitting code are briefly outlined in the next section. 