The simplest variant of the fit comprises determination of the strong coupling from inclusive-jet data only assuming fixed PDF parameterisation. More complex strategy includes simultaneous determination of the proton PDFs and \asz from inclusive DIS and inclusive-jet data without additional external input. Different fit options distinguished by amount of external assumptions are discussed below.
 
{\flushleft \textbf{Determination of \asz}}\newline
In this approach the PDFs are assumed to be fixed in independent studies and the theoretical predictions are parametrised as functions of a single parameter \asz. Similar analyses have been performed in variety of environments. Examples are recent studies in \ep~\cite{}, $pp$~\cite{} and $p\bar{p}$~\cite{} collisions. This approach, however, has the following limitations:
\begin{itemize}
 \item the dominant contribution to the jet cross sections measured in this analysis comes from high-$\x$ region of the phase space. The proton PDFs obtained in global analyses however are dominated by the low-$x$ data from \hera. The PDF parametrisation in the high-$\x$ region are therefore only partially constrained by the data and often rely on extrapolation of restrictions imposed by sum rules.
 \item typically, in the global PDF analyses a value of \asz is an external parameter. Thus, particular assumption of \asz may bias the extraction of the strong coupling in the \as-fit, where the proton PDFs are fixed and treated as independent quantities. For consistent determinations of the strong coupling the assumed value and that extracted in the fit must coincide. Recent global analyses provide proton PDF parametrisation extracted assuming series of different values of \asz. This information can be used for investigation of the influence of \asz assumption on obtained results and to check consistency of the fitting procedure.
 \item nearly all PDF fitting groups provide an error analysis for theirs PDF sets. Nevertheless, differences between predictions based on different PDFs are often larger than the uncertainties suggested by mentioned error estimates. Different prescriptions for the estimation of uncertainties attributed to the indeterminacy of pPDFs exist. One of them, proposed by the PDF4LHC group~\cite{pdf4lhc:2011}, consists in taking an envelope of the predictions based on different PDF sets. However, such approach provides rather conservative estimate. Therefore, the dependence of the results of \as-fit on PDF assumption has to be investigated in details.
\end{itemize}

The advantage of the described variant of the fit is its simplicity. The influence of different assumptions can be easily studied because every change in the input manifest in the extracted \asz value. 
% \textcolor{blue}{Following sections contain the discussion of the results of this approach.}

{\flushleft \textbf{Simultaneous determination of \asz and PDFs}}\newline
The aim of this approach is to fit simultaneously \asz, gluon and quark densities independently of additional assumptions. For this purpose inclusive-jet measurements have to be fitted together with inclusive DIS data. Such treatment takes into account correlation between \as and PDF and leads to unbiased determination of both quantities. However, extracted values will be less precise, when fitted to the same data, than in case of fixed PDF option, because obviously larger number of parameters introduces more freedom for interpretation of the data.

To test various fitting options the \herafitter program~\cite{herafitter} was used in this thesis. The main features of the of the fitting code are briefly outlined in the next section. 