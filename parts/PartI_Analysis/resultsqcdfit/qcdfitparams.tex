In the QCD fit, the predictions $\mathbf{t}$ are usually treated as functions of the proton PDF parameters, $\mathbf{a}$, and the strong coupling constant, \asz. In the kinematic phase space of \hera the following parton species contribute effectively to the PDFs: the gluon ($g$), and five quark flavours: up ($u$), down ($d$), strange ($s$), charm ($c$), bottom ($b$) and corresponding anti-quarks; because of the large mass of the top quark its contribution can be neglected. The PDFs are functions of the parton longitudinal-momentum fraction, $x$. In principle, their shape is a calculable quantity in QCD. However, due to the non-perturbative nature of the PDFs, only predictions for a limited number of first moments 
\begin{equation}
 M_n=\int\mathrm{d}x\,x^n f\left(x\right), \qquad (n\ge1)
\end{equation}
are currently available~\cite{Hagler:2009ni}, and the most precise extraction of the proton PDFs is achieved in a QCD analysis of experimental data. The usual ansatz is to describe the PDFs at a starting scale $Q_0$ by a suitable parametrisation. Then PDFs can be evolved to any scale by using the DGLAP equations~\cite{Altarelli:1977zs,Gribov:1972,Dokshitzer:1977,Balitsky:1978}. Some assumptions concerning the shape of the distributions at the starting scale have to be made. In the \herapdf approach~\cite{Aaron:2009aa}, the generic functional form of the parton distribution functions is chosen to be:
\begin{equation}
 xf\left(x\right) = A_ix^{B_i}\left(1-x\right)^{C_i}\left(1+\epsilon_i\sqrt{x}+D_ix+E_ix^2\right).
 \label{eq:pdfansatz}
\end{equation}
In general, variables sets $\left(A_i,B_i,C_i,D_i,E_i,\epsilon_i\right)\subset\mathbf{a}$ for each individual parton type have to be considered as independent free parameters in the QCD fit, although the normalisation parameters $A_i$ are constrained by the quark number and momentum sum rules~\cite{Yndurain:2006lfa}, i.e. quantum numbers of total intrinsic strangeness, charm and beauty must vanish $S|uud\rangle=C|uud\rangle=B|uud\rangle=0$ and the total parton momentum has to sum up to the proton momentum. Including \asz into the fit and applying the aforementioned constraints results in 
\begin{equation}
2 \left( \text{quarks/anti-quarks}\right)\cdot 6\left(g,u,d,s,c,b\right)\cdot 6 + 1\left(\asz\right) - n_\text{constr} > 60
\end{equation}
free parameters in the fit. In addition, the contribution of different parton species cannot be distinguished in NC DIS processes, e.g. apart from their different masses, all $u$-type quarks behave similarly in hard \ep-scattering. Usually, in the predictions for high-energy reactions, the masses of the light and sometimes even heavy quarks are ignored, resulting in identical predictions for every quark species. This leads to an ill-defined problem that requires further assumptions about the parameter space to be introduced. 

Based on the above symmetry arguments for $u$- and $d$-type quarks, the number of free parameters can be reduced. For example, instead of fitting individual parton distributions for each parton type, linear combinations of PDFs can be introduced.  In \herapdf1.0~\cite{Aaron:2009aa}, the following PDF components are treated independently:
\begin{align}
 xg\left(x\right) &= A_gx^{B_g}\left(1-x\right)^{C_g}, \label{eq:gpar} \\
 xu_v\left(x\right) &= A_{u_v}x^{B_{u_v}}\left(1-x\right)^{C_{u_v}}\left(1+E_{u_v}x^2\right),\\
 xd_v\left(x\right) &= A_{d_v}x^{B_{d_v}}\left(1-x\right)^{C_{d_v}},\\
 x\bar{U}\left(x\right) &= A_{\bar{U}}x^{B_{\bar{U}}}\left(1-x\right)^{C_{\bar{U}}},\\
 x\bar{D}\left(x\right) &= A_{\bar{D}}x^{B_{\bar{D}}}\left(1-x\right)^{C_{\bar{D}}}, \label{eq:Dseapar}
\end{align}
where $xg$ corresponds to the gluon, $xu\left(d\right)_v$ represents $u\left(d\right)$-valence quark distributions and $x\bar{U}\left(x\right)$ and $x\bar{D}\left(x\right)$ are $u$-type and $d$-type anti-quark distributions, respectively. At the initial scale $Q^2_0$, the anti-quark sea distributions are $x\bar{U}\left(x\right)=x\bar u$ and $x\bar{D}\left(x\right)=x\bar d+x\bar s$. Apart from parameters in Eqs.\eqref{eq:gpar}--\eqref{eq:Dseapar}, all other parameters in the generic parametrisation Eq.~\eqref{eq:pdfansatz} are set to zero by default\footnote{Such a combination is called `central-fit`.} and released one at a time, when the sensitivity of the fit to the parametrisation ansatz is investigated. Further assumptions imposed on the fit parameters are:
\begin{itemize}
 \item at the initial scale the strange sea is expressed as a constant fraction of the $d$-type sea $x\bar s=f_s x \bar D$ with $f_s=0.31$~\cite{Martin:2009iq,Nadolsky:2008zw}; the contribution from the strange quarks is assumed to be suppressed due to the larger $s$-quark mass;
 \item in order to conform with the isospin symmetry hypothesis and ensure $x\bar u \rightarrow x\bar d$ as $x\rightarrow 0$, the constraints $A_{\bar U}=\left(1-f_s\right)A_{\bar D}$ and $B_{\bar{U}}=B_{\bar{D}}$ are imposed;
 \item $B_{u_v}=B_{d_v}$ is assumed for the central fit, however this requirement is omitted in the study of systematic effects.
\end{itemize}
These restrictions leave 10 or 11 free parameters for the proton PDFs, depending on the chosen fit configuration. The strong coupling constant, \asz, can be added as an additional parameter. The measured jet data alone cannot constrain these parameter sets because jet production populates a restricted region of phase space (mostly at medium and high $\xi$), moreover certain types of initial-states (i.e. incoming partons) are indistinguishable in NC jet reactions, since the jet algorithm used is insensitive to the parton flavour. In \herapdf1.0 the combined inclusive DIS data from \hone and \zeus were used for the proton PDF determination. In the following section, these additional data sets and the possible fit strategies are discussed.
