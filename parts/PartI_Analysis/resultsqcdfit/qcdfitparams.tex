In the QCD fit the predictions $\mathbf{t}\left( \mathbf{a},\asz\right)$ are usually treated as functions of the proton PDF parameters, $\mathbf{a}$, and the strong coupling constant \asz. In the kinematic phase space of \hera only limitted set of parton species contribute to the hadronic final-state. These are the gluon ($g$), and five quark flavours: up ($u$), down ($d$), strange ($s$), charm ($c$), bottom ($b$) and corresponding anti-quarks; because of large mass of the (anti-)top quark its contribution can be neglected. The PDFs are functions of the parton momentum fraction $x$. In principle, the shape of the proton parton density distributions is a calculable quantity in QCD. However, due to non-perturbative nature of the PDFs, the predictions for only certain moments 
\begin{equation}
 M_n=\int\mathrm{d}x\,x^n f\left(x\right)
\end{equation}
are avaliable at the moment~\cite{Hagler:2009ni}, and the most precise extraction of the proton PDFs is achieved in QCD analysis of experimental data. In the extraction procedures some assumtions concerning the shape of the distributions have to be made. In the HERAPDF approach~\cite{Aaron:2009aa} the generic functional form of the parton distribution functions at the input scale is choosen to be:
\begin{equation}
 xf\left(x\right) = A_ix^{B_i}\left(1-x\right)^{C_i}\left(1+\epsilon_i\sqrt{x}+D_ix+E_ix^2\right).
 \label{eq:pdfansatz}
\end{equation}
In general, variables sets $\left(A_i,B_i,C_i,D_i,E_i,\epsilon_i\right)\subset\mathbf{a}$ for each individual parton type have to be considered as independent free parameters in the QCD fit, although the normalisation parameters $A_i$ are constrained by the quark number and momentum sum rules, e.g. total intrinsic strangeness, charm and beauty must vanish $S=C=B=0$ and total parton momentum has to sum up to unity. Including \asz into the fit and applying mentioned constraints results in 
\begin{equation}
2 \left( \text{quarks/anti-quarks}\right)\cdot 6\left(g,u,d,s,c,b\right)\cdot 6 + 1\left(\asz\right) - n_\text{constr} > 60
\end{equation}
free parameters in the fit problem. Besides that, the contribution of different parton species can not be distinguished in NC DIS processes, e.g. apart from mass, all $u$-type quarks have behave similarly in hard \ep-scatering. 
This leads to ill-defined problem and requires further assumtions about the parameter space to be introduced. 

Taking into account relations between perturbative coefficients the number of free parameters can be reduced. For example, instead of fitting individual parton distributions for each parton type, linear combinations of PDFs can be introduced.  In HERAPDF1.0~\cite{Aaron:2009aa} the following PDFs are treated independently:
\begin{align}
 xg\left(x\right) &= A_gx^{B_g}\left(1-x\right)^{C_g},\\
 xu_v\left(x\right) &= A_{u_v}x^{B_{u_v}}\left(1-x\right)^{C_{u_v}}\left(1+E_{u_v}x^2\right),\\
 xd_v\left(x\right) &= A_{d_v}x^{B_{d_v}}\left(1-x\right)^{C_{d_v}},\\
 x\bar{U}\left(x\right) &= A_{\bar{U}}x^{B_{\bar{U}}}\left(1-x\right)^{C_{\bar{U}}},\\
 x\bar{D}\left(x\right) &= A_{\bar{D}}x^{B_{\bar{D}}}\left(1-x\right)^{C_{\bar{D}}},
\end{align}
where $xg$ corresponds to the gluon, $xu\left(d\right)_v$ represent $u\left(d\right)$ valence quark distributions and $x\bar{U}\left(x\right)$ and $x\bar{D}\left(x\right)$ are $u$-type and $d$-type anti-quark distributions, respectively. At the initial scale $Q^2_0$ the anti-quark sea distributions are $x\bar{U}\left(x\right)=x\bar u$ and $x\bar{D}\left(x\right)=x\bar d+x\bar s$. The other parameters in Eq.~\eqref{eq:pdfansatz} are set to zero by default\footnote{Such combination is called `central-fit`.} and released one at a time, when the sensetivity of the fit to the parametrisation ansatz in investigated. Further assumtions imposed on the fit parameters are:
\begin{itemize}
 \item $x\bar{U}$ and $x\bar{D}$ have single $B$ parameter i.e. $B_{\bar{U}}=B_{\bar{D}}$;
 \item at the intial scale the strange sea is expressed as a constant fraction of the $d$-type sea $x\bar s=f_s x \bar D$ with $f_s=0.31$~\cite{Martin:2009iq,Nadolsky:2008zw};
 \item in order to ensure $x\bar u \rightarrow x\bar d$ as $x\rightarrow 0$ additional constraints $A_{\bar U}=\left(1-f_s\right)A_{\bar D}$ and $B_{\bar U}=B_{\bar D}$ were imposed;
 \item $B_{u_v}=B_{d_v}$ were assumed for the central fit, however this requirement was droped in the study of systematic effects.
\end{itemize}
In summary, these restriction leave 10 or 11, depending on the choosen fit configuration, free parameters for the proton PDFs. Besides that the strong coupling constant \asz can be added as additional parameter. Obviously the measured jet data alone cannot constrain these parameter set. In HERAPDF1.0 the combined inclusive DIS data from \hone and \zeus were used in the QCD analysis. In the following sections additional data set and the possible fit strategies are described.
