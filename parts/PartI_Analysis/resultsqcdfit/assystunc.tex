The influence of systematic uncertainties on the extracted \asz value was investigated and summarised in Table~\ref{tab:asuncbreakdown}.
\begin{table}
\centering
\begin{tabular}{|l|c|c|}
 \hline
 Uncertainty source & $\delta\as$ & $\delta\as$ uncor. treatment\\
 \hline\hline
 Statistical       & 0.2\% & ---     \\
 Tot. uncor. syst. & 0.4\% & ---     \\
 JES               & 0.9\% & 0.3\%   \\
 Luminosity        & 0.8\% & 0.2\%   \\
 \hline
\end{tabular}
\caption{Contribution of different sources of experimental uncertainty to the total experimental uncertainty on \asz determined from the fit to double-differential cross sections. The considered sources of experimental error on \asz-value are statistical, combined uncorrelated systematic uncertainty, jet-energy scale and luminosity uncertainties. The second column contains the breakdown of uncertainty contributions attributed to different sources and treated as described in Section~\ref{subsec:sysunctreatment}. The last column presents the contribution of correlated errors source, when treated as uncorrelated.}
\label{tab:asuncbreakdown}
\end{table}

The dominant contribution to the total experimental uncertainty on extracted \as-value arises due to absolute jet-energy scale and luminosity errors on the jet cross sections and amounts to about 1\%. The other contributions due to limited statistics and due to uncorrelated sources of systematic uncertainties have a subdominant effect. 

For comparison the last column in Table~\ref{tab:asuncbreakdown} illustrates the effect of treating jet-energy scale and luminosity errors as uncorrelated. It can be seen, that, although, such an aproach can be inconsistent