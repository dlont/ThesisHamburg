A value of the strong coupling constant have been determined from the double-differential inclusive-jet cross sections measured as functions of \etjetb in different regions of \qsq. The extraction refers to a wide interval of scales, which spans approximately $19 < \mu_R < 150$~\GeV~range. The sensitivity of the extraction procedure to the theoretical assumptions e.g. choice of the pPDF or renormalisation or factorisation scale was investigated in detail. The extracted value of \asz is:
\begin{equation}
 \asz = 0.1218 \pm 0.0028\left(\text{exp.}\right)^{+0.0066}_{-0.0053}\left(\text{scales}\right)^{+0.0029}_{-0.0018}\left(\text{PDF}\right)\pm{0.003}\left(\text{hadr.}\right),
\end{equation}
where individual uncertainties attributed to various sources are indicated separately. 

The dominant source of experimental uncertainty is due to the precision of the calibration of the absolute jet-energy-scale, while the theoretical uncertainty is dominated by that due to missing higher order in pQCD predictions.
Different approaches to the treatment of experimental and theoretical uncertainties were investigated. It was demonstrated that inclusion of uncertainties correlations is important. 

To check the consistency of the data, the \asz-values were determined from individual cross section bins and were found in good agreement within the estimated experimental uncertainties. However, the quality of the combined \as-fit, assessed by the value of $\chi^2/n_\text{DF}$, indicates that this estimate can be of conservative nature.

Apart from that, the running of the strong coupling was illustrated. The RGE predictions calculated with two-loop precision are in a good agreement with observed evolution of \as in a wide range of scales.
 
Comparing to other determinations at HERA this result is in a good agreement with previous similar analyses~\cite{HERAIjets} and more recent extraction~\cite{britzger}. When compared to HERAI results, this determination features increased statistical precision, especially in the high-\etjetb and high-\qsq region. However, the main limitation of the precision on \asz is due to unknown contribution of the truncated terms of perturbative series and therefore, high-order predictions are necessary in order to profit from the experimental precision of the data.