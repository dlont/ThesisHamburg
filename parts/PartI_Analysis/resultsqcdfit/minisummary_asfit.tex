A value of the strong coupling constant has been determined from the double-differential inclusive-jet cross sections measured as functions of \etjetb in different regions of \qsq. The extraction refers to a wide interval of hard scales corresponding to $19 < \mu_R < 150$~\GeV. The sensitivity of the extraction procedure to theoretical assumptions e.g. the choice of the PDFs or renormalisation or factorisation scales was investigated in detail. The extracted value of \asz is:
\begin{equation}
 \asz = 0.1218 \pm 0.0028\left(\text{exp.}\right)^{+0.0066}_{-0.0053}\left(\text{scales}\right)^{+0.0029}_{-0.0018}\left(\text{PDF}\right),
\end{equation}
where individual uncertainties attributed to various sources are indicated separately. 

The dominant source of experimental uncertainty is due to the precision of the calibration of the absolute jet-energy-scale, while the theoretical uncertainty is dominated by missing higher orders in the pQCD predictions.
Different approaches to the treatment of experimental and theoretical uncertainties were investigated. It was demonstrated that the results of the fit can significantly differ when the systematic uncertainties are treated as correlated or uncorrelated. 

To check the consistency of the data, the \asz-values were determined from individual cross section bins and were found in good agreement within the estimated experimental uncertainties.

The running of the strong coupling was also illustrated. The renormalisation-group-equation predictions calculated with two-loop precision were found in a good agreement with the observed evolution of \as over a wide range of scales.