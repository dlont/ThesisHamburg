A value of the strong coupling constant has been determined from the double-differential inclusive-jet cross sections measured as functions of \etjetb in different regions of \qsq. The extraction refers to a wide interval of hard scales, which spans approximatelyroton $19 < \mu_R < 150$~\GeV. The sensitivity of the extraction procedure to theoretical assumptions e.g. the choice of the PDFs or renormalisation or factorisation scales was investigated in detail. The extracted value of \asz is:
\begin{equation}
 \asz = 0.1218 \pm 0.0028\left(\text{exp.}\right)^{+0.0066}_{-0.0053}\left(\text{scales}\right)^{+0.0029}_{-0.0018}\left(\text{PDF}\right)\pm{0.003}\left(\text{hadr.}\right),
\end{equation}
where individual uncertainties attributed to various sources are separately indicated. 

The dominant source of experimental uncertainty is due to the precision of the calibration of the absolute jet-energy-scale, while the theoretical uncertainty is dominated by that missing higher orders in pQCD predictions.
Different approaches to the treatment of experimental and theoretical uncertainties were investigated. It was demonstrated that inclusion of correlated uncertainties is important. 

To check the consistency of the data, the \asz-values were determined from individual cross section bins and were found in good agreement within the estimated experimental uncertainties.

The running of the strong coupling was also illustrated. The renormalisation-group-equation predictions calculated with two-loop precision are in a good agreement with the observed evolution of \as over a wide range of scales.
 
This result is in good agreement with previous similar analyses~\cite{HERAIjets} and more recent extractions at HERA~\cite{britzger}. When compared to \hera I results~\cite{}, this determination features increased statistical precision, especially in the high-\etjetb and high-\qsq region. However, the main limitation of the precision on \asz is due to the unknown contribution of the truncated terms of the perturbative series and therefore higher-order predictions are necessary in order to profit from the high experimental precision of the data.