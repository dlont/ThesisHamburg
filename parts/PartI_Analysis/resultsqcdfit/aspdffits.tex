As demonstrated in Chapter~\ref{ch:resultscs}, the measured inclusive-jet cross sections are very well described by the predictions based on the proton PDFs and \asz-value extracted in global fits to the inclusive DIS data from \hera. Assuming the validity of NLO pQCD predictions for inclusive-jet production in NC DIS, the accurate jet data can be exploited to constrain possible values of input parameters used in the calculations.
\begin{figure}[ht]
 \centering
 \includegraphics[width=0.8\textwidth,angle=-90]{Figures/alphas/cs_etjet_different_as_values}
 % cs_etjet_different_as_values.eps: 0x0 pixel, 300dpi, 0.00x0.00 cm, bb= 0 0 567 515
 \caption{Comparison of the measured inclusive-jet cross sections to NLO pQCD predictions calculated assuming different values of \asz. \textcolor{blue}{to be updated}}
 \label{fig:etjetdifferentas}
\end{figure}

According to the factorisation ansatz, explained in Chapter~\ref{ch:theory}, the perturbative QCD predictions for inclusive-jet cross sections can be expressed as a convolution of the hard-scattering matrix elements and the proton PDFs. The parton level NLO predictions, e.g. for \dsdetjetb, have the following formal representation:
\begin{equation}
	\begin{split}
\sigma\left( \mu_R, \mu_F\right)  &= \as^1\left( \mu_R\right) \cdot \left[ c_{1,i}\left( x, \mu_R, \mu_F\right) \overset{x}{\bigotimes } F_i\left( x, \mu_f, \mu_R \right) \right] \\
&+ \as^2\left( \mu_R\right) \cdot \left[ c_{2,i}\left( x, \mu_R, \mu_F\right) \overset{x}{\bigotimes } F_i\left( x, \mu_f, \mu_R \right) \right],
	\end{split}
	\label{eq:fitnloqcdpredictions}
\end{equation}
where $c_{n,i}$ are the perturbative coefficients for the jet-production subprocess $i$, e.g. QCD Compton scattering or boson-gluon fusion at LO, and $F_i\left( x, \mu_f, \mu_R \right)$ are the corresponding linear combinations of the proton PDFs depending on the fraction of the proton momentum, $x$, renormalisation and factorisation scales $\mu_R, \mu_F$ and implicitly (through the DGLAP evolution equations) \as. The convolution symbol $\overset{x}{\bigotimes}$ corresponds to the $x$-integration over available phase space interval. Thus, jet rates in DIS are sensitive to the value of the strong coupling and the proton PDFs. The comparison of the predictions based on different proton PDFs set to the measured cross sections was presented in Chapter~\ref{ch:resultscs}; the predictions calculated assuming different values of \asz in the matrix elements and PDFs\footnote{In the HERAPDF1.5 analysis alternative sets of PDFs were determined for different values of \asz assumed in the fitting procedure.} are illustrated in Figure~\ref{fig:etjetdifferentas}. It can be seen that the calculations with lower (higher) value of the strong coupling provide lower (higher) jet cross sections. Hence, by comparing theoretical predictions evaluated assuming different values of \asz to the data, an optimal value of the strong coupling can be identified. The inclusive-jet measurements obtained in this thesis can thus be utilised for constraining \as and the proton PDF distributions. The details of this procedure, also called the \emph{'QCD fit'}, are described below.
% The values of input parameters providing the best description of the measurements can be determined in a fit to the data if the predictions are parametrised as functions of \asz and the proton PDF parameters. 



%In the following the explicit dependence of the strong coupling on the scale is suppressed and the reference scale at which \as is determined is set to the mass of the $\zn$ boson. The \as denotes \asz, whenever otherwise stated.
