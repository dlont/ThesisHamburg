In this thesis single-differential and double-differential inclusive-jet cross sections in neutral current deep-inelastic scattering has been measured with the \zeus detector at \hera. Jets were reconstructed with the \kt-clustering algorithm in longitudinally-invariant inclusive mode in the Breit-frame with $\etjetb > 8$~\GeV~and $-1<\etajetlab<2.5$. The measurement refers to the kinematic range $125 < \qsq < 20000$~\GeV~and $0.2<y<0.6$. An extensive study of the systematic uncertainties has been performed. In particular, multidimensional reweighting of the MC spectra made possible reliable estimate of the corresponding uncertainty. When compared to other similar analyses from \zeus this measurements make use of large statistics sample collected during the \hera II running phase. The total integrated luminosity of the data amounts to 295~\invpb, which is almost three time large than in the previous published measurement~\cite{inclusivejetDIS}. The NLO pQCD predictions based on different pPDF sets were compared with the measured jet cross sections. In general, theoretical calculations provide a good description of the data within the experimental and theoretical uncertainties. The measurements demonstated high sensitivity to the proton parton distributions used in the predictions.

From the measured data a value of the strong coupling at the scale of the $\zn$-boson mass has been determined and the running of \as over a wide range of scales was demonstrated. The detailed analysis of the uncertainties on \asz-values was carried out and importance of the treatment of systematic errors was emphasised. The extracted \asz value is:
\begin{equation}
 \asz = 0.1218 \pm 0.0028\left(\text{exp.}\right)^{+0.0066}_{-0.0053}\left(\text{scales}\right)^{+0.0029}_{-0.0018}\left(\text{PDF}\right)\pm{0.003}\left(\text{hadr.}\right).
\end{equation}
Nevertheless, the uncertainty on theoretical predictions is limited by the unknown size of missing terms beyond NLO, therefore in order take an advantage of the data higher-order calculations are vital.
 
The potential of the jet measurements to constrain the proton PDFs was demonstrated when new data were included into the QCD-fit together with inclusive DIS data. Such an approach provided an unbiased way for determination of the strong coupling taking into account correlations between parton densities and \as.

These data contribute to the ultimate legacy of \hera. An improvement of the precision can be achieved in future if these measurements will be combined together with those from \hone experiment. Further step can be made when the \hera data will be used simultaneously with the jet measurements from other high-energy colliders \lhc and \tevatron. Such extensive data sample will have a strong impact on the precision of the future PDF and \as extractions.

% \section*{Comparison to Other Measurements}
% \section*{Possible Improvements in the Future}
% \section*{Potentials to improve \as and PDFs at Future Colliders}