In this thesis single-differential and double-differential inclusive-jet cross sections in neutral current deep-inelastic scattering have been measured with the \zeus detector at \hera. Jets were reconstructed with the \kt-clustering algorithm in the longitudinally-invariant inclusive mode in the Breit-frame with $\etjetb > 8$~\GeV~and $-1<\etajetlab<2.5$. The measurement refers to the kinematic range $125 < \qsq < 20000$~$\GeV^2$~and $0.2<y<0.6$. An extensive study of the systematic uncertainties has been performed. In particular, multidimensional re\-weight\-ing of the MC spectra made possible reliable estimate of the corresponding uncertainty. When compared to other similar analyses from \zeus, this measurements make use of large statistics sample collected during the \hera II running phase. The total integrated luminosity of the data amounts to 295~\invpb, which is almost three time large than in the previous published measurement~\cite{Abramowicz:2010ke}. The NLO pQCD predictions based on different pPDF sets were compared with the measured jet cross sections. In general, theoretical calculations provide a good description of the data within the experimental and theoretical uncertainties. The measurements demonstrated high sensitivity to the proton parton distributions used in the predictions.

Several potential directions can be considered for further improvement of the precision of the jet cross sections measurements. For instance, as was mentioned, the jets with low transverse energy $3 < \etjetlab < 10$~\GeV~are characterised by larger energy-scale uncertainty than those with $\etjetlab > 10$~\GeV. In order to improve the precision of the jet-energy measurement, especially in low-\etjetlab region, the combined information from tracking and calorimeter systems can be used. In particular, an improvement of the jet energy-scale uncertainty is expected in the measurements based on the so-called \zufos clusters~\cite{upub:Tuning:zn01021}. Moreover, as was shown, the systematic uncertainty due to the variation of MC models is substantial in this analysis, therefore further improvement of the results can be achieved when more advanced MC models, like those based on higher-order perturbative QCD predictions, are used.%(e.g. MC@NLO~\cite{})
Furthermore, an improvement of the precision can be achieved in future if these measurements will be combined together with those from the \hone experiment.

From the measured data a value of the strong coupling at the scale of the $\zn$-boson mass has been determined and the running of \as over a wide range of scales was demonstrated. A detailed analysis of the uncertainties on \asz-values was carried out and the importance of the treatment of systematic errors was emphasised. The extracted \asz value is:
\begin{equation}
 \asz = 0.1218 \pm 0.0028\left(\text{exp.}\right)^{+0.0066}_{-0.0053}\left(\text{scales}\right)^{+0.0029}_{-0.0018}\left(\text{PDF}\right).
\end{equation}
The uncertainty is dominated by the scale uncertainty of the predictions which is an estimate of the effect of missing higher order terms beyond NLO, therefore in order take advantage of the data, higher-order calculations are vital. 

These data contribute to the ultimate legacy of \hera. Further steps in the improvement of the \asz precision can be made when the \hera data will be used simultaneously with the jet measurements from other high-energy colliders \lhc and \tevatron. Such extensive data sample will have a strong impact on the precision of future \asz extractions as well as on proton PDF.

Looking into far future, the ongoing studies for the lepton-nucleon collider \lhec demonstrate a possibility for an accesses to unprecedented kinematic range for lepton-nucleon scattering. \lhec will open a new window to unexplored regions of electroweak and strong sectors of the Standard Model, where jet studies will play very significant role.

%The potential of the jet measurements to constrain the proton PDFs was demonstrated when the new data were included into the QCD-fit together with inclusive DIS data. Such an approach provided an unbiased way for determination of the strong coupling taking into account correlations between parton densities and \as.


 %\section*{Comparison to Other Measurements}
 %\st{\section*{Possible Improvements in the Future}}
 %\section*{Potentials to improve \as and PDFs at Future Colliders}