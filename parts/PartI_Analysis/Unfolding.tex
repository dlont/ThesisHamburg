The data are subject to various detector effects including non-linear response, finite resolution, limited acceptance and reconstruction inefficiencies. Such distortions in the case of vanishing background contributions are described by the Fredholm integral equation:
\begin{equation}
 \int_\Omega K\left(s,t\right)\,f\left(t\right)\,\mathrm{d}t = g\left(s\right),
 \label{eq:fredholm}
\end{equation}
where $g\left(s\right)$ is the measured distribution, $f\left(t\right)$ the underlying true distribution and the kernel $K\left(s,t\right)$ describes the response of the detector. Thus the determination of $f\left(t\right)$ requires the solution of this equation.
The process of extracting of the true distribution from the measured one is called \emph{unfolding}. Various techniques were proposed in the literature to solve equation~\ref{eq:fredholm} (see for example~\cite{statistics book page 187} and references therein). In this thesis the so-called \emph{method of correction factors (or bin-by-bin method)} was employed.

In this chapter the procedure used for the determination of inclusive-jet cross sections is described. In Section~\ref{sec:acccor} the bin-by-bin method is presented. Afterwards in Section~\ref{sec:bindef} the binning definition for the cross sections measurement is discussed. The chapter concludes with the description of polarisation corrections.

\section{Acceptance Correction}
\label{sec:acccor}
The differential cross section for inclusive-jet production in some kinematic bin is determined according to:
\begin{equation}
 \left.\frac{\mathrm{d}\sigma}{\mathrm{d}X}\right|_{\mathrm{bin},i} = \frac{N_{\mathrm{bin},i}}{\mathcal{L} \cdot \Delta_{\mathrm{bin},i}} \cdot \mathcal{A}_{\mathrm{bin},i} \cdot \mathcal{C}_{\mathrm{bin},i},
 \label{eq:csdef}
\end{equation}
where $N_{\mathrm{bin},i}$ is the number of jets reconstructed in the data in bin $i$, $\mathcal{L}$ the integrated luminosity of the data sample, $\Delta_{\mathrm{bin},i}$ the bin width and $\mathcal{A}_{\mathrm{bin},i}$ the acceptance correction factor described below. The effects from higher-order QED processes or those related to the polarisation of the lepton beam need to be included into the definition of the cross section in order to obtain an observable consistent with that provided by existing NLO QCD codes (see Chapter~\ref{sec:nloqcd}). These effects are combined in the additional multiplicative term $\mathcal{C}_{\mathrm{bin},i}$.

The acceptance correction factors $\mathcal{A}_{\mathrm{bin},i}$ are applied to the data in order to correct for detector effects. They are determined using MC samples from the number of jets generated in some kinematic bin, $N_{\mathrm{bin},i}^{\mathrm{gen}}$, and the number of jets reconstructed in the same bin, $N_{\mathrm{bin},i}^{\mathrm{rec}}$:
\begin{equation}
 \mathcal{A}=\frac{N_{\mathrm{bin},i}^{\mathrm{gen}}}{N_{\mathrm{bin},i}^{\mathrm{rec}}}.
 \label{eq:accdef}
\end{equation}
In order to ensure the validity of this bin-by-bin multiplicative correction, the migrations across the neighbouring bins have to be sufficiently small and MC simulations have to provide a good description of the shape of the measured distributions. Two additional variables can be defined in order to quantify the detector effects,  namely the \emph{purity}
\begin{equation}
 \mathcal{P}=\frac{N_{\mathrm{bin},i}^{\mathrm{rec \wedge gen}}}{N_{\mathrm{bin},i}^{\mathrm{rec}}}
\label{eq:puritydef}
\end{equation}
and the \emph{efficiency}
\begin{equation}
 \mathcal{E}=\frac{N_{\mathrm{bin},i}^{\mathrm{rec \wedge gen}}}{N_{\mathrm{bin},i}^{\mathrm{gen}}}.
\label{eq:efficiencydef}
\end{equation}
In these definitions $N_{\mathrm{bin},i}^{\mathrm{rec \wedge gen}}$ represents the number of jets generated and reconstructed in some particular bin. The purity of a bin is the fraction of jets that migrated into this bin while originating from the other bin. This can happen, for example, due to the finite resolution of the detector. Unlike purity, the efficiency is an estimate of jet loss due to migrations outside the measurement bin or due to reconstruction inefficiency or cut requirements. Using purity and efficiency the acceptance can be re-expressed as:
\begin{equation}
  \mathcal{A} = \frac{N_{\mathrm{bin},i}^{\mathrm{rec}}}{N_{\mathrm{bin},i}^{\mathrm{gen}}} = \frac{\mathcal{E}}{\mathcal{P}}.
\end{equation}

The terms appearing in the definitions~\eqref{eq:accdef}--\eqref{eq:efficiencydef} are not statistically independent, therefore a correlation between different factors has to be taken into account when statistical uncertainty attributed to $\mathcal{A},\, \mathcal{P}$ or $\mathcal{E}$ factors is needed. However, they can be expressed in terms of statistically independent quantities 
\begin{equation}
 \mathcal{P}=\frac{N_{\mathrm{bin},i}^{\mathrm{rec \wedge gen}}}{N_{\mathrm{bin},i}^{\mathrm{rec \wedge gen}}+N_{\mathrm{bin},i}^{\mathrm{rec \wedge \ulcorner gen}}},\quad \mathcal{E}=\frac{N_{\mathrm{bin},i}^{\mathrm{rec \wedge gen}}}{N_{\mathrm{bin},i}^{\mathrm{rec \wedge gen}}+N_{\mathrm{bin},i}^{\mathrm{\ulcorner rec \wedge gen}}},\quad 
 \mathcal{A}=\frac{N_{\mathrm{bin},i}^{\mathrm{rec \wedge gen}}+N_{\mathrm{bin},i}^{\mathrm{\ulcorner rec \wedge gen}}}{N_{\mathrm{bin},i}^{\mathrm{rec \wedge gen}}+N_{\mathrm{bin},i}^{\mathrm{rec \wedge \ulcorner gen}}},
\end{equation}
where $N_{\mathrm{bin},i}^{\mathrm{\ulcorner rec \wedge gen}}$ and $N_{\mathrm{bin},i}^{\mathrm{rec \wedge \ulcorner gen}}$ are the number of jets generated but not reconstructed in bin $i$ and reconstructed but not generated in that bin, respectively.

The acceptance correction factors, efficiency and purity for the relevant inclusive-jet cross sections determined using either \lepto or \ariadne MC samples are shown in Figures~\ref{fig:epa}~\subref{fig:epa_subfig1}--\subref{fig:epa_subfig3}. In general, the purity is typically above 40\% for all kinematic observables, while the efficiency is typicaly within 30\% and 65\%. The decrease of efficiency in the region $250\,\GeV^2<\qsq<500\,GeV^2$ was observed before~\cite{joerg hanno trevor januschek} and was attributed to the reduced electron identification capabilities in the transition region between the RCAL and BCAL. The acceptance correction factors never exceed 1.6 and are typically below 1.4.
\begin{figure}[pht]
\begin{center}
\begin{subfloat}{\includegraphics[width=\linewidth,trim={0 0 0 0},clip] {./Figures/epa/h_etjetb_CS_d_epa}
   \label{fig:epa_subfig1}
 }%
\end{subfloat}
\newline
 \begin{subfloat}{\includegraphics[width=\linewidth,trim={0 0 0 0},clip]{./Figures/epa/h_etajetb_CS_d_epa}
   \label{fig:epa_subfig2}
 }%
\end{subfloat}
\newline
\begin{subfloat}{\includegraphics[width=\linewidth,trim={0 0 0 0},clip] {./Figures/epa/h_q2_CS_d_epa}
   \label{fig:epa_subfig3}
 }%
\end{subfloat}
\end{center}
\caption{Acceptance correction factors, efficiency and purity for inclusive-jet cross sections as functions of \etjetb, \etajetb and \qsq.}
\label{fig:epa}
\end{figure}

A major limitation of the described method is its possibly high sensitivity to the MC true level distribution~\cite{cowan note}. Hence this effect was investigated by using alternative event generators and was taken into account in the systematic uncertainty assessment (see Chapter~\ref{systematics}).

\section{Polarisation Correction}
\label{sec:polcor}
\begin{figure}[h]
 \begin{center}
 \includegraphics[width=0.33\textwidth,bb= 0 0 567 544]{./Figures/polrew/05e}
 % 05e.eps: 0x0 pixel, 300dpi, 0.00x0.00 cm, bb= 0 0 567 544
\end{center}
\caption{The polarisation reweighting factors for 2004/2005$e^-$ data taking period determined using the HECTOR program. The red curve represents the spline interpolation.}
\label{fig:polcor05e}
\end{figure} 
The MC samples used in this analysis were generated assuming vanishing polarisation of the lepton beam, $P_e = 0$. In order to take non-zero polarisation of the electrons into account, the MC samples were reweighted using theoretical predictions. For this purpose the HECTOR program~\cite{HECTOR} interfaced to BASES~\cite{bases} with the CTEQ5D PDFs~\cite{CTEQ5} was used. The reweighting factors were determined from the ratio of predictions for the unpolarised inclusive DIS cross sections and those for the lepton beam polarisation corresponding to particular data-taking period. The polarisation correction was implemented as weight assigned to each MC event according to the \qsq~of the scattering process:
\begin{equation}
 w_p\left(\qsq\right) = w_p = \frac{\sigma_\mathrm{pol}}{\sigma_\mathrm{unpol}}.
\end{equation}

The average polarisation for different data-taking periods is summarised in Table~\ref{tab:polvalues}.
\begin{table}[h]
 \centering
 \begin{tabular}{lc}
 Data-taking period & Average polarisation, $P_e$ \\
\hline
 2004/2005$e^-$   & -0.06184 \\
 2006$e^-$   & 0.09386  \\
 2006/2007$e^+$ & -0.06857
\end{tabular} 
\caption{The average polarisation values for the data samples used in the analysis.}
\label{tab:polvalues}
\end{table}
The obtained correction factors as a function of \qsq~and a spline interpolation is illustrated in Figure~\ref{fig:polcor05e}. The size of the correction increases with increasing \qsq~but nowhere exceeds 3\% and typicaly is below 1\%. The sign of the correction depends on the helicity of the lepton beam.

Besides the polarisation correction applied to MC events, an inverse of determined factors was applied to the data in order to obtain jet cross sections corresponding to unpolarised lepton scattering.
 
\section{QED Corrections}
The theoretical predictions for the measured cross sections obtained using the NLOJET++ program include only the leading order QED contribution, while the measurements were influenced by the higher-order processes like running of the electromagnetic coupling, initial-state and final-state EM radiation etc. To maintain the consistency between the data and theoretical predictions the measured cross sections were corrected to the Born level using the MC predictions. A multiplicative factor applied to the data was determined using two \lepto MC samples with higher-order QED processes switched on and off. The correction factor is equal the the ratio of the corresponding jet cross sections:
\begin{equation}
 \mathcal{C}^\text{QED}_i = \frac{\sigma_i^\text{BORN}}{\sigma_i^\text{QED}}.
 \label{eq:eqdcorr}
\end{equation}

Figures~\ref{fig:qedcorr}~\subref{fig:z0corr_subfig1}--\subref{fig:z0corr_subfig3} illustrate the determined QED corrections in different kinematic bins. In general, the correction
 
 \begin{figure}[ht]
\begin{center}
\begin{subfloat}{\includegraphics[width=0.45\linewidth,trim={0 0 0 0},clip] {./Figures/qedcorr/h_etjetb_CS_h_h2pqed_corr_fac}
   \label{fig:qedcorr_subfig1}
 }%
\end{subfloat}
 \begin{subfloat}{\includegraphics[width=0.45\linewidth,trim={0 0 0 0},clip]{./Figures/qedcorr/h_etajetb_CS_h_h2pqed_corr_fac}
   \label{fig:qedcorr_subfig2}
 }%
\end{subfloat}
\begin{subfloat}{\includegraphics[width=0.45\linewidth,trim={0 0 0 0},clip] {./Figures/qedcorr/h_q2_CS_h_h2pqed_corr_fac.pdf}
   \label{fig:qedcorr_subfig3}
 }%
\end{subfloat}
\end{center}
\caption{QED multiplicative correction factors for inclusive-jet cross sections as functions of \etjetb, \etajetb and \qsq applied to pQCD predictions.}
\label{fig:qedcorr}
\end{figure}

\section{Bin Definition}
\label{sec:bindef}
The choice of the size of the bin width is limited by two factors. Naturally, in order to obtain maximum information from the measurements, the bin width has to be as small as possible. However reduction of the bin size leads to a decrease of the number of entries. Therefore a width must be large enough to obtain a statistically significant number of entries per bin. In addition, because of finite resolution and possibly non-linear response of the detector, the migration effects can be substantial in case the bin width is much smaller than the detector resolution. Besides that it is convenient to define the binning for inclusive-jet analysis to be consistent with that for dijet and trijet analysis performed at \zeus~\cite{joerg Makarenko}.

The binning definition for the jet cross sections measured in this analysis is outlined below.
\subsection*{Bin definition for ${\mathrm{d}\sigma}/{\mathrm{d}Q^2}$}
\label{subsec:bindefq2}
For the single differential cross section ${\mathrm{d}\sigma}/{\mathrm{d}Q^2}$ six bins were defined spanning the measurement phase space from 125 \GeV to 20000 \GeV. Because the inclusive NC DIS cross section scales as $1/Q^4$ the size of the bins increases with increasing \qsq~in order to obtain a statistically significant jet sample in each bin.
\subsection*{Bin definition for ${\mathrm{d}\sigma}/{\mathrm{d}\etjetb}$}
\label{subsec:bindefet}
For the single differential inclusive jet cross section as a function of \etjetb~also six bins were chosen with varying width. Although the cross section is large at low \etjetb~the size of the bin width was limited by the purity at low jet transverse energies. At low \etjetb the measurement is bounded by the phase space restriction $\etjetb > 8$ \GeV while the upper limit was chosen as 100 \GeV~became only few jets in the whole sample have larger jet \etjetb value.
\subsection*{Bin definition for ${\mathrm{d}\sigma}/{\mathrm{d}\etajetlab}$}
\label{subsec:bindefeta}
The cross section ${\mathrm{d}\sigma}/{\mathrm{d}\etajetlab}$ spans the region $-2<\etajetb<1.8$ and has 5 bins. The sizes of the bins were chosen in order to optimise the purity and at the same time keep an appropriate number of bins. As can be seen in Figure~\ref{fig:epa}~\subref{fig:epa_subfig2}, a stable and reasonably high purity was achieved.
\subsection*{Bin definition for ${\mathrm{d}\sigma}/{\mathrm{d}\etjetb}$ in different regions of \qsq}
\label{subsec:bindefetinq2}
The single differential cross sections ${\mathrm{d}\sigma}/{\mathrm{d}\etjetb}$ were measured in six regions of \qsq. The size of the \qsq~bins corresponds to that of the ${\mathrm{d}\sigma}/{\mathrm{d}Q^2}$ and the \etjetb binning scheme was the same as for the integrated ${\mathrm{d}\sigma}/{\mathrm{d}\etjetb}$ cross section.


