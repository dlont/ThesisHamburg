The following sources of uncertainty in the theoretical predictions were investigated:
\begin{itemize}
 \item the uncertainty on the predicted inclusive-jet cross section due to that on the value of \asz was estimated by repeating the calculations assuming different values of \asz=0.1156 and 0.1196\footnote{These values correspond to the standard deviation of \as, that was determined in the HERAPDF fit} and using the corresponding proton PDF sets from the \herapdf1.5 bundle. The resulting uncertainty was typically about $\pm2\%$;
 \item PDFPDFPDF
 \item the uncertainty on the NLO QCD calculations arising from higher-order terms, estimated by varying the renormalisation scale up and down by a factor of 2. It was about $\pm10\%$ in the low-\qsq, low-\etjetb region but decreasing to about $\pm5\%$ at high \qsq and high \etjetb;
 \item the uncertainty of the calculations originating from the residual dependence on factorisation scale was estimated by repeating the calculations with $\mu_F$ scaled up and down by factors 0.5 and 2, resulting in a difference between the predictions $\lesssim 3\%$;
 \item the uncertainty due to modelling of the parton shower was typically about 1\% in most of the phase space.
\end{itemize}
\begin{figure}[t!]
\begin{center}
\begin{subfloat}{\includegraphics[width=0.48\linewidth,trim={0 0 100 0},clip] {./Figures/theorunc/h_q2_CS_unc}
   \label{fig:z0corr_subfig3}
 }%
\end{subfloat}
\begin{subfloat}{\includegraphics[width=0.48\linewidth,trim={0 0 100 0},clip]{./Figures/theorunc/h_etjetb_CS_unc}
   \label{fig:z0corr_subfig2}
 }%
\end{subfloat}
\end{center}
\caption{Estimated theoretical uncertainties on NLO pQCD predictions as functions of \etjetb and \qsq.}
\label{fig:z0corr}
\end{figure}
The total theoretical uncertainty was calculated by summing the individual contributions in quadrature. The break-down of the theoretical uncertainties as functions of various kinematic variables are shown in Figures~\ref{fig:z0corr}~\subref{fig:thunc_subfig1}--\subref{fig:thunc_subfig3}.
