Jet rates in DIS are sensitive to the value of the strong coupling. The inclusive-jet cross sections determined in this thesis can be utilised for the \as determination if used in the QCD fit with \as as a free parameter. This section outlines the results of the strong coupling determination performed in this work.

According to factorisation property, explained in Chapter~\ref{ch:theory}, the perturbative QCD predictions for inclusive-jet cross sections can be expressed a convolution of the hard-scattering matrix elements and the proton PDFs. The NLO predictions for \dsdetjetb,~thus, have the following formal representation:
\begin{equation}
	\begin{split}
\dsdetjetb\left( \mu_R, \mu_F\right) &= \as^1\left( \mu_R\right) \cdot \left[ c_{1,i}\left( x, \mu_R, \mu_F\right) \overset{x}{\bigotimes } F_i\left( x, \mu_f, \mu_R \right) \right] \\
&+ \as^2\left( \mu_R\right) \cdot \left[ c_{2,i}\left( x, \mu_R, \mu_F\right) \overset{x}{\bigotimes } F_i\left( x, \mu_f, \mu_R \right) \right],
	\end{split}
\end{equation}
where $c_{n,i}$ is the perturbative coefficient for the jet-production subprocess $i$, e.g. QCD Compton scattering, and $F_i\left( x, \mu_f, \mu_R \right)$ is the corresponding linear combination of the proton PDFs depending on the fraction of the proton momentum, $x$, renormalisation and factorisation scales and implicitly on \as through the DGLAP evolution equations.

The value of the strong coupling is determined in a fit to the data parametrising the predictions as a function of \as. The \herafitter~\cite{Aaron:2009aa,Aaron:2009kv} program package was used for the \as extraction. The main features of the \herafitter framework are outlined below.

%In the following the explicit dependence of the strong coupling on the scale is suppressed and the reference scale at which \as is determined is set to the mass of the $\zn$ boson. The \as denotes \asz, whenever otherwise stated.
