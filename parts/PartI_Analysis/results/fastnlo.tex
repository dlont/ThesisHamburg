The \fastnlo~\cite{fastnlo1,fastnlo2,fastnlo3,fastnlo4} approach is a based on the idea of calculating the LO and NLO weights, $\mathcal{C}_{n,a,i,j}$, which are independent of the \as and PDFs, on a 2d-grid in convolution variable $x$ and factorisation scale $\mu_F$. The cross section can be, then, obtained from a simple sum:
\begin{equation}
\sigma = \sum_{n,a,i,j}{ \as^n\left( \mu_{F,j} \right) F_a\left( x_i, \mu_{F,j}\right)\mathcal{C}_{n,a,i,j} },
\end{equation}
where the magnitude of the strong coupling and PDFs can be chosen independently, without repeating the time-consuming MC integration. Because certain parton configurations lead to the identical final-state, only weights for the linear combinations of PDFs $a=\left( \Delta, \Sigma, g\right) $ are stored. The weights $\mathcal{C}_{n,a,i,j}$ are organised in form of a table and can be effectively processed leading to the jet cross sections recalculation time of the order $\mathcal{O}\left( ms\right)$ sufficient for the usage in the fitting procedure.

