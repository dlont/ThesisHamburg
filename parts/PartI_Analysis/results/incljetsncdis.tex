In this chapter the results of the inclusive-jet cross sections measurements in neutral current deep-inelastic scattering are presented. The measurements include single and double-differential jet production cross sections. The later were used for determination of the value of the strong coupling, \asz. 

The cross sections were measured in the high-\qsq~region of phase space defined by $125 < \qsq < 20000$~\GeV~and $0.2<y<0.6$. The measurements refer to each jet of hadrons reconstructed using the \kt\, algorithm in the longitudinally invariant inclusive mode in the Breit reference frame with transverse energies $\etjetb > 8$~\GeV~and pseudorapidity $-1<\etajetlab<2.5$. The total integrated luminosity of the data amounts to $\mathcal{L}=295$ pb$^{-1}$.

The results presented in this section are being prepared for submission to the physics journal. The absolute values of cross sections, statistical and systematic uncertainties as well as values for QED and $\zn$ cross sections are summarised in Appendix~\ref{appendixZ}.

\section{Observables}
The single-differential jet cross sections measured as functions of \qsq,~\etjetb~and~\etajetlab are shown in Figures~\ref{fig:inclusivesingledif_q2},~\ref{fig:inclusivesingledif_et},~\ref{fig:inclusivesingledif_eta}. In addition, the inclusive-jet cross sections $\dsdetjetb$ and the relative difference of the cross sections with respect to the next-to-leading-order predictions, measured in different regions of \qsq, are shown in Figures~\ref{fig:inclusive_doubledif} and~\ref{fig:inclusive_doubledif_rel}, respectively.

The measurements were corrected for detector effects like limited detector acceptance, resolution and efficiency as well as for higher-order QED radiation and running of the fine-structure constant, $\aem$, using Monte Carlo simulations (see Chapter~\ref{ch:mccor}). Perturbative NLO QCD predictions calculated using the NLOjet++ program based on HERAPDF1.5~\cite{herapdf} proton PDFs sets were compared to the data. The NLOjet++ predictions refer to jets of partons, while the measurements refer to those of hadrons, therefore the theoretical predictions were corrected for hadronisation effects (see Section~\ref{ch:hadrcor}). These predictions do not include contributions from $\zn$ exchange and $\gamma\zn$ interference effects for which the calculations have been corrected as described in Section~\ref{sec:z0exchange}. The renormalisation and factorisation scales were set to $\mu_R^2 = \qsq + \etjetb^2$ and $\mu_F^2 = \qsq$ in the calculations, respectively.

The measured cross sections presented in this chapter are shown as dots in the figures. The inner error bars represent the statistical uncertainties. The statistical and systematic uncertainties not associated with the uncertainty in the absolute energy scale of the jets, added in quadrature are shown as the outer error bars. The total theoretical uncertainty and the uncertainty on the measured cross sections due to the absolute energy scale of the jets are displayed as hatched and shaded bands, respectively. The uncertainty due to jet energy scale is highly correlated across individual bins and therefore is indicated separately.

\section{Single-Differential cross sections}
\subsubsection*{Inclusive-jet \dsdqsq~cross section}
The inclusive-jet production cross section\\ \dsdetjetb~is presented in the Figure~\ref{fig:inclusivesingledif_q2}. The cross sections are steeply-falling functions decreasing more than three orders of magnitude in the measured phase space.

The size of the statistical uncertainty varies within the 1--3\% range depending on the value of \qsq. The uncorrelated part of the systematic uncertainty amounts to 1--5\% increasing toward large values of \qsq. The uncertainty due to that of the absolute jet-energy scale is about 5\% at low \qsq~and decreases to 3--4\% in the high-\qsq~region. The dominant sources of systematic uncertainties are those due to modelling of the parton shower and to the jet energy scale.

The total relative uncertainty on the theoretical predictions is less than 11?\% and typically within 5?\%. The uncertainty is largest at low values of \qsq and decreases with increasing momentum transfer. Overall, the NLO QCD predictions describe data very well in shape and normalisation within the experimental and theoretical uncertainties.
\begin{figure}[p]
	\centering
		\includegraphics[width=\textwidth]{./Figures/cs/inclusivesingledif_q2}
	\caption{The measured single-differential cross section $d\sigma/d\qsq$ based on the \kt~jet algorithm for inclusive-jet NC DIS reconstructed in the Breit frame with $\etjetb > 8$~\GeV~and $-1<\etajetlab<2.5$ (dots), in the kinematic region given by $0.2<y<0.6$ and $125 < \qsq < 20000$~\GeV. The NLO QCD calculation (solid line), corrected for hadronisation effects and $\zn$ exchange and using the HERAPDF1.5 parametrisations of the proton PDFs, is also shown. The lower part of the figure shows the relative difference between the measured $d\sigma/d\qsq$ and the NLO QCD calculations (dots). In both panels, the inner error bars represent the statistical uncertainties; the outer error bars show the statistical and systematic uncertainty not associated with the uncertainty in the absolute energy scale of the jets, added in quadrature; the shaded band displays the uncertainty due to the absolute energy scale of the jets and the hatched band displays the total theoretical uncertainty. In some bins, the error bars on the data points are smaller than the marker size and are therefore not visible.} 
	\label{fig:inclusivesingledif_q2}
\end{figure}

\subsubsection*{Inclusive-jet \dsdetjetb~cross section}
In Figure~\ref{fig:inclusivesingledif_et} the single-differential inclusive-jet \dsdetjetb~cross section is presented. The cross sections decrease as the transverse energy of the jet increases, falling by more than two orders of magnitude in the measured range. The statistical uncertainty is less than 1\% at low \etjetb~and reaches about 4\% at large values of \etjetb. The total uncorrelated systematic uncertainty is less than 10\% and typically within 5\%. It is dominated by the uncertainty due to model variation for the acceptance correction. The correlated part of the systematic uncertainty amounts to 5(7)\%  in the regions of low (high) values of \etjetb.

The predictions describe the measured cross sections very well. The total theoretical uncertainty, dominated by the uncertainty due to terms beyond NLO, is about 7--5?\% at low and high \etjetb, respectively. 
\begin{figure}[p]
	\centering
		\includegraphics[width=\textwidth]{./Figures/cs/inclusivesingledif_et}
	\caption{The measured single-differential cross section $d\sigma/d\etjetb$. The other comments are as to Figure~\ref{fig:inclusivesingledif_q2}}
	\label{fig:inclusivesingledif_et}
\end{figure}

\subsubsection*{Inclusive-jet \dsdetajetb~cross section}
The measured inclusive-jet single differential \dsdetajetb~cross section is presented in Figure~\ref{fig:inclusivesingledif_eta}. The measured range spans $-2 < \etajetb < 2$ region. In the region  $-2 < \etajetb < 1$ the cross section rapidly increases by more than one order of magnitude, reaching a maximum value at $\etajetb \approx 0.75$. At higher values of the jet pseudorapidity the cross section decreases. The relative statistical uncertainty is about 2.5\%--1\% in the measured range. The dominant source of systematic uncertainty is due to the jet energy-scale, which amounts to typically 5\%. The uncertainty due to modelling of the parton shower reaches 10\% in low \etajetb region, where MC statistic was limited and less than 1\% in $-0.25 < \etajetb < 2$ region. The pQCD prediction describes the measurements very well in shape and normalisation within the theoretical and experimental errors.
\begin{figure}[p]
	\centering
		\includegraphics[width=\textwidth]{./Figures/cs/inclusivesingledif_eta}
	\caption{The measured single-differential cross section $d\sigma/d\etajetb$. The other comments are as to Figure~\ref{fig:inclusivesingledif_q2}}
	\label{fig:inclusivesingledif_eta}
\end{figure}

\section{Double-Differential cross sections}
In order to investigate inclusive-jet production in more detail, single-differential cross sections as functions of \etjetb were measured in different regions of \qsq. The obtained cross sections differ from the double-differential measurements by a constant factor equal to the \qsq~bin width. Figures~\ref{fig:inclusive_doubledif},~\ref{fig:inclusive_doubledif_rel} show the measured cross sections and the relative difference with respect to theoretical predictions. The cross sections exhibit the same features as the single-differential cross section \dsdetjetb. The transverse energy spectrum becomes harder as the virtuality of exchange-boson increases.

The statistical uncertainty increases with increasing \etjetb~and \qsq, ranging from 1(8)\% in the first \qsq~bin at low (high) \etjetb~to 7(16)\% in the last \qsq~bin at low(high) \etjetb. The correlated uncertainty due to the absolute jet-energy scale is approximately independent of \etjetb~and is less than about 7\% in the measured phase space. The uncertainty is somewhat smaller in the low \etjetb~region and has a tendency to decrease with increasing \qsq.
\begin{figure}[p]
	\centering
		\includegraphics[width=\textwidth]{./Figures/cs/inclusive_doubledif}
	\caption{\textcolor{blue}{The other comments are as to} Figure~\ref{fig:inclusivesingledif_q2}}
	\label{fig:inclusive_doubledif}
\end{figure}


\begin{figure}[p]
	\centering
		\includegraphics[width=\textwidth]{./Figures/cs/inclusive_doubledif_pdfs}
  \caption{The other comments are as to Figure~\ref{fig:inclusivesingledif_q2}}
	\label{fig:inclusive_doubledif_rel}
\end{figure}

