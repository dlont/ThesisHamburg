The proton PDFs are non-perturbative quantities, predictions for only certain moments of which are possible at the moment~\cite{Hagler:2009ni}. The most precise extraction of the proton PDFs nowadays is achieved in QCD analysis of experimental data. The \herafitter program is a tool originally developed for least-square fits of the proton PDFs. In \herafitter the agreement between data and theory is estimated using the following ansatz for the $\chi^2$:
%\begin{equation}
% \chi^2\left( \mathbf{m},\mathbf{b}\right) = \sum_{i}{\frac{\left[ m^i-\Sigma_j{\gamma^i_jm^ib_j-\mu^i}\right]^2 }{\left( \delta_{i,\text{stat}}\mu^i\right)^2 + \left( \delta_{i,\text{uncor}}\mu^i\right)^2 }} + \sum_{j}{b_j^2}
% \label{eq:chi2uncor}
%\end{equation}
\begin{equation}
 \chi^2\left( \mathbf{m},\mathbf{b}\right) = \sum_{ij}{\left( m^i -\sum_{l}{\Gamma_l^i\left( m^i\right)b_l - \mu^i }\right) C^{-1}_{ij,\text{stat}} \left( m^j -\sum_{l}{\Gamma_l^j\left( m^j\right)b_l - \mu^j }\right) } + \sum_{j}{b_j^2},
 \label{eq:chi2corr}
\end{equation}
where $\mu^i$ is the measured central value of the cross section in bin $i$, the $\Gamma_i^l$ quantifies the sensitivity of the measurement $i$ to the correlated systematic source of uncertainty $l$, the $C^{-1}_{ij,\text{stat}}$ is covariance matrix for the measured data points. The parameter vectors $\mathbf{m}$ and $\mathbf{b}$ represent the theoretical predictions for the point $i$ and systematic uncertainties, respectively. The predictions $\mathbf{m}\left( \mathbf{a},\asz\right)$ are usually treated as functions of the proton PDF parameters, $\mathbf{a}$,  and \asz. The determination procedure of the parametrisation of the pPDFs and \as that best fit the data is based on the numerical minimisation of the $\chi^2$ function with respect to the free parameters. The nuisance parameters $\mathbf{b}$ introduced in the $\chi^2$ definition are fitted together with other parameters in order to consistently take into account systematic uncertainties correlated across measurement points. The last term in Eq.~\eqref{eq:chi2corr} represents the penalty for these parameters. The other details concerning the $\chi^2$-function definition can be found in the \herafitter manual~\cite{herafitter:2014:manual}.

The minimisation of the $\chi^2$ is performed employing the iterative \migrad algorithm from the \minuit package~\cite{James:1975dr}. The uncertainties due to experimental precision of the data are propagated to the determined parameters by means of ``Hesse''-method implemented in \minuit.

The minimisation procedure requires repeated recalculation of the $\chi^2$-function, hence the NLO pQCD predictions for the jet observables, if included in the fit, has to be calculated at each iteration in the QCD-fit. However, evaluation of the predictions for the jet production in an iterative fit using the techniques described in Section~\ref{sec:nlojet} has two main limitations: 
\begin{itemize}
 \item due to stochastic nature of the Monte Carlo approach for calculation of the phase space integrals, the $\chi^2$ minimisation procedure,  in general, may not converge;
 \item the computing time for a single iteration (i.e. evaluation of the $\chi^2$ function for a single fit parameter value) is prohibitively long.
\end{itemize}
Therefore, in the iterative fit procedure the \fastnlo framework for the recalculations of the NLO predictions for the jet production have been used. A brief summary of the \fastnlo approach is provided in the next section. 
