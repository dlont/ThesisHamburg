Systematic uncertainties arise due to various effects such as, for example, incomplete understanding of the detector response or deficiencies in the modelling of the \ep~physics. The systematic uncertainties from identified sources were quantified in the the following way. For each source of uncertainty attributed to imperfect knowledge of some continuous parameter, a variation by its estimated uncertainty was performed and the cross sections were re-evaluated. In the case when the source of uncertainty is attributed to a specific reconstruction technique, the difference with respect to the cross section calculated using an alternative reconstruction method is taken as an approximation of the systematic uncertainty. The positive, $\delta^+_i$, (negative, $\delta^-_i$) variations of the cross sections were added in quadrature in order to obtain the positive (negative) total systematic uncertainties
\[
	\begin{aligned}
		\delta_{\left( +\right)}^{2} = \sum_i{\mbox{$\delta_i^{\left( +\right)}$}^2},\\
		\delta_{\left( -\right)}^{2} = \sum_i{\mbox{$\delta_i^{\left( -\right)}$}^2}.
	\end{aligned}
\]
%In general, the total uncertainty of the measurements presented in this thesis is dominated by the systematic contribution and only in some regions of phase space were limited accuracy from insufficiently large data samples an issue. The details of the determination of the statistical correlations between various measurement bins are provided in the next subsection.

In what follows a detailed description of the sources of systematic uncertainties is presented.
\begin{itemize}
	\item \textbf{Hadronic energy scale} \\
		As demonstrated in Section~\ref{subsec:jetenescale}, the precision to which the absolute jet-energy scale can be calibrated in the data is down to $\pm1\%$ for jets with $\etjetlab > 10\;\GeV$ and, as was found in other jet analysis at \zeus, $\pm3\%$ for jets with $3\,< \etjetlab < 10\,\GeV$. Therefore, to assess the effect of this uncertainty on the cross sections, the jet transverse energy was varied in the MC simulations up and down within the specified limits. Each variation resulted in a single-sided\footnote{As the result of the variation, the measured cross sections in \textit{all} bins increase or decrease simultaneously.} change of the cross sections in all bins with respect to the nominal values. This uncertainty was monotonically decreasing as a function of \qsq, ranging from 5\% at low \qsq~to 3\% in the highest \qsq~bin. As a function of \etjetb, it was approximately constant and did not exceed 7\%.
 \item \textbf{Jet energy resolution} \\
     In study~\cite{thesis:behr:2010} the description of the jet energy resolution in data and MC simulations was investigated. For this purpose single-jet events were selected and two quantities:
		\begin{align}
		\sigma_\mathrm{rel}^\mathrm{DATA} &= \frac{\etjetlab-E_{T,DA}}{E_{T,DA}},\label{eq:dataetres}\\
		\sigma_\mathrm{rel}^\mathrm{MC} &= \frac{E_{T,lab}^{det,jet}-E_{T,lab}^{had,jet}}{E_{T,lab}^{had,jet}} \label{eq:mcetres}.
		\end{align}
		where $\etjetlab$ is the jet transverse energy in the laboratory frame and $E_{T,DA}$ is the transverse energy of the scattered electron (see Eq.~\eqref{eq:eeda}), while $E_{T,lab}^{det,jet}(E_{T,lab}^{had,jet})$ is the jet transverse energy determined at the reconstructed (generated) level in MC, were examined. Two variables $\sigma_\mathrm{rel}^\mathrm{DATA}$ and $\sigma_\mathrm{rel}^\mathrm{MC}$ represent the measure of the jet-energy resolution in data and simulations, respectively. In that study, the agreement between these two quantities was found to be sufficient for not attributing the systematic uncertainty to this source.
	\item \textbf{Acceptance correction} \\
		The dependence of the acceptance correction factors on the modelling of the parton-shower process was taken into account in the systematic uncertainty. The cross sections were re-evaluated using the acceptance corrections obtained from the \ariadne sample instead of \lepto. In order to symmetrize the uncertainty, half of the absolute difference between the obtained cross sections was assigned to positive and negative components of the systematic error. The effect of the change of MC program was below 3\% for \qsq < 1000 $\GeV^2$ but increased up to 5\% in the largest \qsq~bin. As a function of \etjetb~it was typically below 5\% but reached 10\% in the high-\etjetb~regions.
In principle, in order to reduce this sensitivity a composition of generated event samples can be used for the determination of the acceptance-correction factors. Nevertheless, in this analysis, the \lepto sample was used as a nominal event generator, as it initially provided a better description of the data.
			
	\item \textbf{Dependence on the electron finder}\\
		The \textsc{SINISTRA} electron finder (see Section~\ref{sec:eleid}) was used for the determination of the nominal cross-section values. In order to estimate the size of the systematic uncertainty attributed to the electron identification, the \textsc{EM} electron finder~\cite{epj:c11:427,upub:Straub:url} was used as an alternative\footnote{By such a variation the effects due to the characteristics of a particular identification algorithm (SINISTRA is based on neural-network pattern recognition, while EM models directly the probability of a shower to originate from the DIS electron candidate) are taken into account.}. The resulting uncertainty was typically about 0.5\% but could reach 2\% in regions of high \qsq.
	
	\item \textbf{Electromagnetic energy scale}\\
		The uncertainty on the electromagnetic energy scale was determined using a technique similar to that for the calibration of the hadronic energy scale. Exploiting the approximate independence of the energy of the electron reconstructed using the double-angle method, a calibration of the absolute electromagnetic energy scale of the calorimeter can be performed. As was demonstrated in Section~\ref{sec:eleenescale}, the residual discrepancy between data and MC simulations was less than 2\%. This value was interpreted as an uncertainty on the electron energy scale and was propagated to the cross sections by varying the scattered electron energy up and down by $\pm$ 2\% in the Monte Carlo simulation. Therefore, this variation also covers the uncertainty due to the determination of the Lorentz boost to the Breit reference frame. 
	
	\item \textbf{Track-veto efficiency correction}\\
		In order to estimate the size of the effect on the cross sections due to the MC track-veto correction introduced in Section~\ref{sec:trkvetoeff}, the procedure was changed slightly. Instead of the parametrisation of the track-veto efficiency as a function of $y_{DA}$, the CTD-FLT vertex-track multiplicity was used as an alternative. The effect on the cross section was typically about 0.5\%.
	
	\item \textbf{Photoproduction background}\\
		The contribution to the high-\qsq NC DIS jet cross sections from photoproduction processes can be estimated using the dedicated Monte Carlo event generators e.g. \pythia and \herwig. As was demonstrated in Section~\ref{sec:eventsampletab}, the photoproduction background can be neglected in the phase space studied here.
	
	\item \textbf{Selection cuts variation} \\
		The stability of the results with respect to the choice of particular cut values, introduced in Chapter~\ref{ch:selection}, was investigated. For this purpose the cut thresholds were changed in the data and MC according to the resolution of the considered quantity. As mentioned in Section~\ref{sec:eventsampletab}, the description of the $\left(E-p_Z\right)$-distribution in the data by simulations was inaccurate and requires correction. Hence, for 
		\begin{itemize}
			\item $\left( E-p_Z\right) $ cut: the variation by $\pm6\%$ was performed and resulted in the change of the cross sections by $\lesssim \pm 1\%$
		\end{itemize}
	\item \textbf{Luminosity} \\
		The luminosity and its uncertainty were determined using dedicated detector components, as described in Section~\ref{sec:lumimeas}. The overall normalisation of the measurements depends on the integrated luminosity of the data sample that was determined with $\pm1.8\%$ for the given data-taking period.
\end{itemize}
All described systematic studies are summarised in Table~\ref{tab:sysunc}. The total uncorrelated component of the experimental systematic uncertainty was obtained by adding corresponding individual contributions in quadrature. The Figure~\ref{fig:systunc} illustrates the individual components of the statistical and systematic uncertainties as functions of various kinematic observables.
\begin{table}[htpb!]
\centering
{\footnotesize
\begin{tabular}{l|l|c|c}
\textbf{Source} & \textbf{Variation} & \textbf{Applied to} & \textbf{Correlation}\\ 
\hline jet energy scale & \begin{tabular}[x]{@{}c@{}}$\pm3\%$ for $3 < \etjetlab < 10 \GeV$,\\$\pm1\%$ for $\etjetlab > 10 \GeV$\end{tabular} & MC & \pbox{2.5cm}{bin-to-bin correlated}\\ 
\hline electron energy scale & $\pm2\%$ & MC & \pbox{2.5cm}{bin-to-bin correlated} \\
\hline \hline \multicolumn{3}{l}{\textbf{selection cuts variation}}  \\
\hline $E-p_Z$ cut & $\pm 6\%$ & data and MC & uncorrelated \\
\hline \hline \multicolumn{3}{l}{\textbf{Additional}}  \\
\hline Acceptance correction & \ariadne & data & \pbox{2.5cm}{bin-to-bin correlated}\\
\hline Electron identification & EM-algorithm & data and MC & uncorrelated\\
\hline \pbox{3.5cm}{track veto efficiency correction} & \pbox{3.5cm}{as function of CTD-FLT track multiplicity} & MC & uncorrelated \\
\hline
\end{tabular} 
}
\caption{Summary of variations investigated for the systematic-uncertainty estimation.}
\label{tab:sysunc}
\end{table}

\newgeometry{textwidth=12cm,textheight=22cm}
\begin{landscape}
\begin{figure}[p!]
\begin{center}
\begin{subfloat}[]{\includegraphics[height=\textheight] {./Figures/systunc/h_q2_CS_systBreakdown}
   \label{fig:z0corr_subfig1}
 }%
\end{subfloat}
\begin{subfloat}[]{\includegraphics[height=\textheight]{./Figures/systunc/h_etjetb_CS_systBreakdown}
   \label{fig:z0corr_subfig2}
 }%
\end{subfloat}
\begin{subfloat}[]{\includegraphics[height=\textheight]{./Figures/systunc/h_etajetb_CS_systBreakdown}
   \label{fig:z0corr_subfig3}
 }%
\end{subfloat}
\end{center}
\caption{Estimated systematic uncertainties on inclusive-jet cross sections as functions of \qsq (a), \etjetb (b) and \etajetb (c).}
\label{fig:systunc}
\end{figure}
\end{landscape}
\restoregeometry