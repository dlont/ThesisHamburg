Systematic uncertainties arise due to incomplete understanding of the detector response, utilised reconstruction method or possible bias in calibrations and corrections. The systematic uncertainties from identified sources were quantified in the the following way. For each source of uncertainty attributed to imperfect knowledge of some continuous parameter a variation by its estimated uncertainty is performed and the cross section is re-evaluated. In case when the source of uncertainty is attributed to a reconstruction technique, the difference with respect to the cross section calculated using an alternative reconstruction method is taken as an approximation to the systematic uncertainty. The positive (negative) variations of the cross sections are added in quadrature in order to obtain positive (negative) total systematic uncertainty
\[
	\left.
		\begin{aligned}
			\text{if $\left(\sigma_i > \sigma\right)$},& \; \delta_+^2\\
			\text{if $\left(\sigma_i < \sigma\right)$},& \; \delta_-^2
		\end{aligned}
	\right\} 
	= \sum_i{\delta_i^2}.
\]
In this case the uncertainty can be asymmetric. Jet-energy-scale variation result in correlated change of the measured cross section across all bins.

In the following a detailed description of considered sources of systematic uncertainties is presented.
\begin{itemize}
	\item \textbf{Dependence of the electron finder}\\
	The \textsc{SINISTRA} electron finder (see Section~\ref{sec:eleid}) was used for the determination of the nominal cross section values. In order to estimate the size of the systematic uncertainty attributed to the electron identification the \textsc{EM} electron finder was used as an alternative. The resulting uncertainty was typically about 0.5\% but could reach 2\% in the regions of high \qsq.
	\item \textbf{Electromagnetic energy scale}\\
	The uncertainty on electromagnetic energy scale was determined using the technique similar to that for calibration of hadronic energy scale. Exploiting approximate independence of the energy of the electron reconstructed using double-angle method, calibration of the absolute electromagnetic energy scale of the calorimeter can be performed. As was demonstrated in Section~\ref{subsec:eleenescale}, the residual discrepancy between data and MC simulations was less than 2\%. This value was interpreted as an uncertainty on the electron energy scale and was propagated to the cross sections by varying the scattered electron energy up and down by $\pm$ 2\% in Monte Carlo. In addition, this variation covers the uncertainty due to determination of the Lorentz boost to the Breit reference frame. 
	
	The resulting uncertainty on the jet cross sections 
\end{itemize}