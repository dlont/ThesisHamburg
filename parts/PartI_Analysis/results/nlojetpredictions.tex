The next-to-leading-order QCD $ \left( \mathcal{O}\left( \as^2\right) \right) $ predictions for the observables measured in this thesis were calculated using the \nlojet~\cite{nlojet++} program. The predictions were performed in the $\overline{\text{MS}}$ renormalisation and factorisation scheme and the dipole-subtraction method was applied to cancel the singularities arising in intermediate calculations from infrared and collinear phase-space regions. The number of active flavours was set to five and renormalisation $\left( \mu_R \right) $ and factorisation $\left( \mu_F \right) $ scales were set to $\sqrt{Q^2+\etjetb^2}$ and $\qsq$, respectively. The strong coupling evolution was calculated at two loops with $\asz=0.1176$. The \herapdf1.5 proton PDF parametrisation was used in the calculations, however alternative set were also investigated. The parton-level predictions for the jet cross sections were obtained by applying the $\kt$ algorithm to the partons generated by the program. In order to obtain predictions for jets of hadrons the calculations were corrected to the hadron level using the MC models, as described in~\ref{sec:hadrcor}. The predictions do not include the contributions neither from $\gamma Z$-interference or $\znought$-exchange so that the calculations were corrected for these effects; the details of the correction procedure are given in section~\ref{sec:z0cor}. 

The uncertainty on theoretical predictions was investigated and is presented in the next section.