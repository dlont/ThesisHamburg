A precise theoretical description of the final state of $\ep$ scattering from first principles is currently an intractable problem. It requires calculations in regions of phase-space where perturbative techniques are not applicable or have to be performed to high orders.
\subsection{QCD Parton Showers}
The parton-shower approach is used to simulate higher-order perturbative QCD contributions when a complete calculation is unfeasible or unknown. For example, the DGLAP approach can be utilised to describe initial-state and final-state radiation. The probability for a branching, $\mathcal{P}_{a\rightarrow bc}$, during the evolution is governed by the equation:
\begin{equation}
\frac{\mathrm{d}\mathcal{P}_{a\rightarrow bc}}{\mathrm{d}t} = \int_0^1{\mathrm{d}z\frac{\as\left(\qsq\right)}{2\pi}P_{_{a\rightarrow bc}}}\left(z\right),
\end{equation}
where $P_{a\rightarrow bc}\left(z\right)$ are the Altareli-Parisi splitting kernels.

Such an approximation is usually used in general event generators where the successive radiation is simulated until the evolution parameter reaches some low energy scale, $E_0$, of the order $\mathcal{O}\left(1\;\GeV\right)$. At this point the showering process is stopped and partons are recombined into colourless hadrons.

In order to improve the leading logarithmic accuracy of the parton shower approach, hard emissions are described using complete matrix elements. In such cases an intermediate scale is introduced at which regions dominated by parton shower or hard-scattering dynamics are matched. Nowadays most of the event generators are based on LO matrix elements. However, NLO calculations with matched parton showers are starting to appear~\cite{powheg, mcatnlo}.

Another approximation for QCD radiation that is widely used to describe DIS related processes is the \emph{colour dipole model} (CDM)~\cite{cdm}. Here it is assumed the quark--anti-quark pairs form colour dipoles with corresponding radiation of gluons. The gluons themselves are interpreted as pairs of colour charges that also build colour dipoles and so on. The corresponding schematic illustration is depicted in Figure~\ref{fig:cdm}. 
\begin{figure}[t]
	\centering
		\includegraphics[width=0.65\textwidth,angle=90,trim={500 0 0 0},clip]{./Figures/source/MEPSradiation}
	\caption{Schematic demonstration of the matrix element + parton shower approach.}
\label{fig:meps}
\end{figure}
\begin{figure}[t]%
\centering
\includegraphics[width=0.95\textwidth,trim={20 550 20 0},clip]{./Figures/source/CDMradiation}%
\caption{The colour dipole model radiation pattern.}%
\label{fig:cdm}%
\end{figure}
The radiation from each dipole is assumed to be independent. It proceeds iteratively until some stopping criterion is reached, for example the invariant mass of a dipole falls below some cut-off  value. The CDM is based on leading-order matrix elements in the soft gluon approximation for gluon radiation with transverse momentum $p_T$ and rapidity $y$
\begin{equation}
\mathrm{d}\sigma = \frac{n_c\as}{2\pi}\frac{\mathrm{d}p_T^2}{p_T^2}\mathrm{d}y.
\end{equation}
In contrast to the leading-logarithm DGLAP-based parton-shower algorithm there is no $k_T$ ordering for the gluon radiation. The partons are rather uniformely distributed in $k_T$. Thus the CDM approach is somewhat similar to BFKL evolution.

Another important issue in the simulation of the parton showers is quantum mechanical interference of the initial-state and final-state radiation or the interference between the partons emitted either in the initial or final state. These effects are naturally taken into account in the complete perturbative calculations, however special care must be taken in the resummed calculations like those based on DGLAP evolution, because it 

\section{Fragmentation }
In order to be able to compare pQCD predictions to experimental result the calculations have to be defined in term of experimentally observable quantities. Because the quarks and gluons do not appear as asymptotic states, the transition from the patronic quantities to those defined in terms of hadrons has to be modelled. The formation of colorless objects from partons, called \emph{hadronisation}, is essentialy non-perturbative process and the first-principle calculations are very difficult. 
\begin{figure}[t]
	\centering
	\begin{subfloat}[]{
		\includegraphics[width=0.45\linewidth,angle=0,trim={50 500 0 0},clip]{./Figures/source/LundString}
		\label{fig:lund}
	 }%
	\end{subfloat}
	\begin{subfloat}[]{
		\includegraphics[width=0.45\linewidth,trim={50 500 0 0},clip]{./Figures/source/ClusterModel}
		\label{fig:cluster}
	}%
	\end{subfloat}
	\caption{Schematic illustration of the string fragmentation (a) and cluster fragmentation (b) model}
\label{fig:fragmentationmodels}
\end{figure}
\subsection{String fragmentation model}
%aaa
\subsection{Cluster fragmentation model}
%aaa
\subsection{Monte Carlo Event Generators}
%aa
\subsubsection{LEPTO}
The \lepto event generator~\cite{lepto} combines the leading-order QCD matrix elements (ME) for the hard-scattering process together with the DGLAP parton shower (PS) for the soft-gluon emission. In order to ensure colour coherence during the showering process angular ordering is imposed. The Lund string model~\cite{lund} as implemented in JETSET~\cite{jetset} is used to simulate the hadronisation process. This generator also includes LO electroweak processes necessary for the description of high-\qsq~DIS. The higher-order QED effects are obtained through the interface to the HERACLES~\cite{heracles} program. The \lepto generator is also oftenly called MEPS and is used as a reference MC generator in this analysis.
\subsubsection{ARIADNE}
The colour-dipole pattern for QCD radiation is implemented in the \ariadne event generator~\cite{ariadne}. Since this model naturally includes only the QCD Compton scattering diagram, the BGF graph contribution was introduced by hand. The hadronisation is performed using the same JETSET interface. This event generator was used in the analysis mainly to estimate systematic effects attributed to the choice of the parton-shower model.
%\newpage
%\input{parts//PartI_Analysis//theory//inna}