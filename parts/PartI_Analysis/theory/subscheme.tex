The NLO $n$-jet partonic cross section is a sum:
\begin{equation}
\sigma = \sigma_\text{LO} + \sigma_\text{NLO} = \int_n{\mathrm{d}\sigma^B} + \left[ \int_{n+1}{\mathrm{d}\sigma^R} + \int_n{\mathrm{d}\sigma^V} \right],
\label{eq:sigmanjet}
\end{equation}
where $\sigma^B$ is the Born-level cross section, $\sigma^R$ is the real-radiation correction and $\sigma^V$ is the virtual correction. In order to remove explicit divergences from the the real and virtual parts, specially constructed counter-terms are added and subtracted from Eq.~\eqref{eq:sigmanjet}. The counter-term is an approximation to the real-radiation contribution in the region of the phase-space containing a singularity and has the same point-wise singular behaviour. Each singular parton configuration requires a corresponding counter term\footnote{The infrared structure persists also at higher orders in perturbative expansion. Partial $\mathcal{O}\left(\as^3\right)$ corrections were worked out in~\cite{Ridder:2013mf}}. The real-radiation contribution with subtracted counter-term, $\sigma^A$, becomes a regular function that can be integrated in $D=4$ dimensions:
\begin{equation}
\int_{n+1}{\mathrm{d}\sigma^R} \rightarrow \left[ \int_{n+1}{\mathrm{d}\sigma^R} - \int_{n+1}{\mathrm{d}\sigma^A} \right].
%\label{eq:sigmanjet}
\end{equation}
The virtual contribution term is modified as follows:
\begin{equation}
	\begin{split}
		\int_{n}{\mathrm{d}\sigma^V} \rightarrow& \left[ \int_{n}{\mathrm{d}\sigma^V} + \int_{n+1}{\mathrm{d}\sigma^A} \right] = \\
                                                      & \left[ \int_{n}{\mathrm{d}\sigma^V + \int_{1}{\mathrm{d}\sigma^A}} \right]
	\end{split}
%\label{eq:sigmanjet}
\end{equation}
The divergence in the virtual contribution appears as a pole in $\epsilon$ but this pole is exactly cancelled by that resulting from one-parton phase-space analytic integration of the counter term. After the cancellation, the integration of the virtual part can be carried out numerically in physical $D=4$ dimensions. Since the net effect of adding and subtracting counter-terms is zero, this scheme results only in reshuffling of the divergences.
