Deep inelastic scattering is a process in which a high-energy lepton ($l$) scatters on a nucleon\footnote{The proton in this thesis.} or a nucleus ($h$) with large momentum transfer. The formal equation for this reactions reads:
\[
l\left( k \right) + h\left( P \right) \rightarrow l'\left( k' \right) + X\left( P' \right),
%\label{eq:DISreactions}
\]
where $X$ denotes the hadronic final state. The leading-order Feymnan diagram for this process is illustrated in Figure~\ref{fig:DISgraph}.
\begin{figure}
	\centering
		\includegraphics[width=\textwidth]{./Figures/DISgraph.png}
	\caption{The leading order Feynman diagram for the deep inelastic scattering process.}
	\label{fig:DISgraph}
\end{figure}
At leading order, the interaction between lepton and proton is mediated by electroweak bosons. For virtual $\gamma$ or $\zn$ exchange, the process is called Neutral Current (NC) DIS, while for $W^\pm$ exchange the process is called Charged Current (CC) DIS. At \hera the CC DIS process is characterised by the transformation of the initial-state electron (positron) into a final-state (anti-) neutrino.

Because of the fixed centre-of-mass energy, $\sqrt{s}$, only two independent variables are sufficient to describe the basic scattering process at \hera. The following Lorentz-invariant quantities\footnote{In the equations the masses of initial-state lepton and hadron are ignored} are typically used:
\begin{align}
\qsq &= -q^2 = -\left( k - k' \right)^2,\\
     x &= \frac{\qsq}{2p\cdot q},\\
		 y &= \frac{p\cdot q}{p\cdot k} = \frac{\qsq}{sx},
\label{eq:vardefinit}
\end{align}
where \qsq\, is the negative square of the four-momentum transfer or the virtuality of the exchange boson. Two kinematic regions were formally distinguished: $\qsq < 1\GeV^2$, typically $\qsq \approx 0$, called the photoproduction region; $\qsq > 1 \GeV^2$, called the deep inelastic scattering regime. The scaling variable, $x$, introduced by Bjorken~\cite{bjorken}, in the Quark-Parton Model can be interpreted as the longitudinal momentum fraction of the proton taking part in the hard scattering. The variable $y$ represents the fraction of the lepton energy transferred to the proton in the hadron rest frame. The following equation relating the introduced variables holds:
\begin{equation}
\qsq = sxy.
\label{eq:qsxy}
\end{equation}

Choosing $\qsq$ and $x$ as independent variables, the deep inelastic scattering cross section can be written in terms of the proton structure functions $F_i\left(x,\qsq\right)$:
\begin{equation}
\frac{d^2\sigma\left( e^\pm p \right) }{dxd\qsq} = \frac{4\pi\alpha^2}{x\qsq^4}\left[ Y_+F_2\left(x,\qsq\right) - y^2F_L\left(x,\qsq\right) \mp Y_-xF_3\left(x,\qsq\right) \right],
\label{eq:ddifDIS}
\end{equation}
where $\alpha$ is the fine-structure constant and $Y_\pm = 1 \pm \left( 1 - y \right)^2$. The dominant contribution to the scattering cross section is given by $F_2\left(x,\qsq\right)$, which in the Quark-Parton model is directly related to the quark content of the proton:
\begin{equation}
F_2\left(x\right) = \sum_i{e_i^2xf_i\left(x\right)}.
\label{eq:f2pdf}
\end{equation}
In this equation $e_i$ is a fractional charge of the quark and $f_i\left(x\right)$ is the proton parton density function (PDF) describing the density of quarks of flavour $i$ in the nucleon. The longitudinal structure function, $F_L\left(x,\qsq\right)$, has significant contribution to the cross section only at high values of $y$ and can be related to the cross section, $\sigma_L$, for the absorption of the longitudinaly polarised virtual photons:
\begin{equation}
F_L = \frac{Q^2}{4\pi^2\alpha}\cdot \sigma_L.
\label{eq:sigmal}
\end{equation}
The structure function $F_3\left(x,\qsq\right)$ arises from \zn-exchange and $\gamma\zn$-interference and has significant size only for $\qsq \gg M^2_Z$.

According to the factorisation theorem~\ref{sec:factorisation}, the proton PDFs are universal and independent of the process under consideration. Currently, PDFs cannot be predicted reliably from first principles and have to be determined by experiment. The measurement of the inclusive DIS cross section at \hera provides direct access to the proton PDFs. Investigation of various sub-processes contributing to the inclusive DIS cross section can help to understand the details of the hard scattering process.
