In this analysis, the jet search was performed in the so-called Breit\footnote{Also called the brick-wall reference frame.} frame~\cite{feynman:1972:photon,zfp:c2:237}. It is defined such that the exchanged boson collides with a proton without transverse momentum transfer. In this frame the momenta of the proton, $P$, and exchange boson, $q$, satisfy the equation:
\begin{equation}
2x\vec{P} + \vec{q} = 0.
\label{eq:breitframe}
\end{equation}
In this frame the boson momentum is aligned along the positive $Z$-direction and has only one space-like component \textit{i.e.} $q=\left( 0, 0, 0, -Q\right)$. The Breit frame is constructed in such a way that it corresponds in the QPM process to the back-scattering of the struck quark, maintaining the absolute value of its momentum $\left|xP\right|$. The schematic illustration of the QPM and QCD Compton processes in the Breit frame is demonstrated in Figure~\ref{fig:breitframe}. The presence of non-zero transverse momentum in the Breit frame is a distinct feature of a QCD process that can be easily identified experimentally. As a result, the requirement of a jet in the Breit frame with sufficiently high transverse energy is related to the generation of a parton in the lowest-order QCD hard process at order $\mathcal{O}\left(\alpha\as\right)$ .
\begin{figure}
	\centering
	\begin{subfloat}[]{
		\includegraphics[width=0.9\linewidth,trim={0 550 0 50},clip]{./Figures/source/BreitFrame}
		\label{fig:breitframeqpm}
	}%
	\end{subfloat}
	\begin{subfloat}[]{
		\includegraphics[width=0.9\linewidth,trim={0 550 0 50},clip]{./Figures/source/BreitFrameQCD}
		\label{fig:breitframeqcd}
	}%
	\end{subfloat}
	\begin{subfloat}[]{
		\includegraphics[width=0.9\linewidth,trim={0 550 0 50},clip]{./Figures/source/BreitFrameBGF}
		\label{fig:breitframeqcd}
	}%
	\end{subfloat}
	\caption{Schematic illustration of the Born (a); QCD Compton (b); boson-gluon fusion (c) processes in the Breit frame in the ($p_T$, $p_Z$)-plane. In the quark-parton model process, the incoming exchange boson and parton have collinear momenta. The contribution from QCD processes results in non-zero outgoing parton transverse momentum.}
\label{fig:breitframe}
\end{figure}