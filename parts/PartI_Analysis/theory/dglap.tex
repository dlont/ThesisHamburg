The factorisation scale dependence of the PDFs is governed by the DGLAP evolution equation~\ref{DGLAP}:
\begin{equation}
\frac{\mathrm{d}}{\mathrm{d}\log{\mu_F^2}}
 \begin{pmatrix}
	\mathbf{f_{q_i}}\left(x,\mu_F\right) \\
	f_g\left(x,\mu_F\right)
 \end{pmatrix} = 
\sum_j^{2n_f}{\int_x^1{\frac{\mathrm{d}z}{z}
 \begin{pmatrix}
  \mathbf{P_{q_i \leftarrow q_j}}\left(z\right) & \mathbf{P_{q_i \leftarrow  g}}\left(z\right) \\
  P_{g \leftarrow q_j}\left(z\right) & P_{g \leftarrow g}\left(z\right) \\
 \end{pmatrix}
 \begin{pmatrix}
	\mathbf{f_{q_j}}\left(x/z,\mu_F\right) \\
	f_g\left(x/z,\mu_F\right)
 \end{pmatrix}
}}
\end{equation}
In this equation, the summation runs over the number of active quark and antiquark flavours. The kernels of these equations are splitting functions $P_{ab}$ representing the probability of one parton splitting into several partons and can be calculated from the collinear singularity of any hard-scattering process as a power series in $\as$:
\begin{equation}
P\left(z,\as\left(\mu_f\right)\right) = \as\left(\mu_f\right) P_0\left(z\right) + \as^2\left(\mu_f\right)P_1\left(z\right) + \as^3\left(\mu_f\right)P_2\left(z\right) + \ldots
\label{eq:splittingfunc}
\end{equation}
At the moment the splitting functions are known to next-to-next-to-leading order~\cite{nnlosplittingfinctions}. Intuitively, this system of equations states how sensitive the probe becomes to the low momentum partons as the resolution scale $\mu_f$ increases. %The universality of the splitting kernels allows determination of the PDFs from the global fits to experimental data. 

Evolution of the scale-dependent parton distributions according to the DGLAP equations effectively resums the Feynman diagrams with parton emission strongly ordered in transverse momentum $ \mu_{F,0} \ll \ldots \ll k_{T,i} \ll k_{T,i+1} \ll \ldots \ll \mu_F$ as in Figure~\ref{fig:DGLAPLadder}. Each parton emission in such an approximation is accompanied by a term $\as\cdot\ln{\mu^2_F/\mu^2_{F,0}}$ in the matrix element, therefore such resummation is also called the ``leading log approximation''.

