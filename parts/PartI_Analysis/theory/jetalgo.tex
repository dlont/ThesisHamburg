%Jets are collimated ``spays'' of particles emerging from the hard interaction of partons. Large kinematic boost of the scattered partons results in focusing of QCD radiation in a narrow cone around the parton momentum vector.In order to determine the momentum of the parent-parton an appropriate assignment of particles to a jet has to be performed. 
As was mentioned, bare partons do not appear as free particles because of the nature of the strong interaction. However high-energy quarks and gluons manifest themselves as collections of hadrons with approximately collinear momenta. Such hadronic final states localised in the kinematic phase space are called jets. Investigation of jet production provides access to the details of the underlying hard interaction as well as to the parton dynamics and the mechanism of parton showering and hadronisation. Provided the kinematics of the final-state jets, important quantities describing the kinematics of the hard scattering can be estimated. For example the longitudinal momentum fraction of the struck parton, $\xi$, can be calculated using:
\begin{equation}
\xi = x\left(1+\frac{M^2}{\qsq}\right),
\label{eq:xidef}
\end{equation}
where $x$ is the Bjorken scaling variable defined in the Eq.~\eqref{eq:vardefinitxbj} and $M$ is the invariant mass of two or more identified jets. Jets are important objects with which the test of perturbative QCD predictions is possible (see Section~\ref{subsec:nlojetcalc}).

In the leading-order picture, jets correspond to individual partons emerging in high-energy collisions. An example of the basic diagrams contributing to the jet production in DIS is demonstrated in Figure~\ref{fig:LOFeynmandiags}. Since the flavour of the struck parton cannot be distinguished in NC DIS reactions, formally two types of processes contributing at leading order in the strong coupling can be distinguished, namely, the boson-gluon fusion (BGF) Figure~\ref{fig:LOFeynmandiags}\subref{subfig:lobgf} and QCD Compton (QCDC) scattering Figure~\ref{fig:LOFeynmandiags}\subref{subfig:loqcdc}, with gluons and quarks in the initial state, respectively.
\begin{figure}
	\centering
	\begin{subfloat}[]{
		\includegraphics[width=0.45\textwidth]{Figures/source/jetsfeynmab/BGF}
		\label{subfig:lobgf}
		}%
		\end{subfloat}
		\begin{subfloat}[]{
		\includegraphics[width=0.45\textwidth]{Figures/source/jetsfeynmab/QCDC}
		\label{subfig:loqcdc}
		}%
		\end{subfloat}
	\caption{Leading-order Feynman diagrams contributing to the jet production cross section in NC DIS. (a) Boson-gluon fusion; (b) QCD-Compton scatering processes.}
	\label{fig:LOFeynmandiags}
\end{figure}

The interplay of these two processes allows the effects attributed to the strong coupling and various PDF components to be disentangled, a value of \asz to be extracted and the proton PDFs to be constrained.

In order to give a rigorous definition of the jet, an algorithm for assignment of the particles to a jet must be provided. The proper combination of the particles has to fulfil the following general conditions:
\begin{itemize}
	\item infrared and collinear safety (see Section~\ref{subsec:nlojetcalc});
	\item conservation of factorisation properties of the hard and soft processes;
	\item little sensitivity to the hadronisation effects;
	\item relative insensitivity to the soft interactions of the hadron remnant;
	\item invariance under longitudinal Lorentz boosts;
	\item easy implementation at the particle level in experimental analyses as well as at the parton/hadron level in perturbative theoretical calculations.
\end{itemize}
Among others the recombination-type generalised \kt-algorithm satisfies all mentioned requirements and is defined by the following iterative procedure\footnote{The input objects may refer to the energy deposits in the calorimeter cells; the set of partons in MC or fixed-order predictions or the set of stable hadrons appearing at the hadron level of MC simulations.} (see Figure~\ref{fig:jetcombinationalgorithm}).
\begin{enumerate}
	\item A distance measure, $d_{ij}$, quantifying the phase-space separation of two objects $i$ and $j$, is defined for each pair of particles:
	\begin{equation}
	  d_{ij} = \mathrm{\text{min}} \left( E_{\text{T},i}^{2n}, E_{\text{T},j}^{2n} \right) \dfrac{\Delta R_{ij}^2}{R_0^2},
		\label{eq:dij}
	\end{equation}
	where $\Delta R_{ij}^2 = \left( \eta_{i} - \eta_{j} \right)^2 + \left( \phi_{i} - \phi_{j} \right)^2$ is the angular separation between objects. The dimensionless parameter $R_0$ determines the jet radius.
	\item A quantity, $d_i$, defining the distance to the beam-axis is calculated for each object $i$:
		\begin{equation}
	  d_{i} = E_{\text{T},i}^{2n}.
		\label{eq:di}
	\end{equation}
	\item Two objects $i$ and $j$ are merged according to the Snowmass~\cite{proc:snowmass:1990:134} convention\footnote{Other conventions exist. The Snowmass prescription results in massless jets.}, whenever some $d_{ij}$ is minimal among all $d_{ij}$ and $d_{i}$:
	\begin{align}
		E_\text{T} = E_{\text{T},i} + E_{\text{T},j} & \qquad \eta = \frac{\eta_iE_{\text{T},i} + \eta_jE_{\text{T},j}}{E_\text{T}} & \qquad \phi = \frac{\phi_iE_{\text{T},i} + \phi_jE_{\text{T},j}}{E_\text{T}}.			 \label{eq:snowmass}
	\end{align}
	When $d_i$ is the smallest, the object is called jet and removed from the list.
	\item The algorithm is repeated until no objects remain in the list.
\end{enumerate}

The parameter $n$ in Eq.~\eqref{eq:dij} defines three types of algorithm:
\begin{itemize}
	\item \textsl{$n$ = -1}: the inclusive anti-\kt~algorithm~\cite{Cacciari:2008gp}, which is now extensively used at the LHC. This algorithm results in jets of circular shape. The recombination process is characterised by first assigning particles with largest $E_\text{T}$ to the jets;
	\item \textsl{$n$ = 0}: the Cambridge-Aachen~\cite{Dokshitzer:1997in} algorithm, which takes into account only angular separations between objects, was mostly used in $e^+e^-$ collider experiments;
	\item \textsl{$n$ = 1}: the inclusive \kt~algorithm~\cite{Catani:1993hr}, which produces jets of irregular shape and, in contrast to anti-\kt, recombines particles with small $E_\text{T}$ first.
\end{itemize}
 It has been shown that the \kt~and anti-\kt~have similar performance in photoproduction~\cite{np:b864:1} and DIS~\cite{Abramowicz:2010ke}. The study~\cite{pl:b649:12} has demonstrated that $R=1$ is the optimal choice of the radius parameter at \hera. Taking this into account, the choice of the \kt~algorithm with $R=1$ was adopted in this thesis. Taking advantage of the longitudinal invariance of the algorithm, the jet search was performed in the Breit frame, which is described below.

\begin{figure}
	\centering
		\includegraphics[width=\linewidth]{./Figures/jetcombinationalgorithm.png}
	\caption{Recombination-type jet algorithm flow char.}
	\label{fig:jetcombinationalgorithm}
\end{figure}
