Recombination-type generalised \kt-algorithm is defined by the following iterative procedure applied to the input list of objects\footnote{The input objects may refer to the energy deposits in the calorimeter cells; the set of partons in MC or fixed-order predictions or the set of stable hadrons appearing at the hadron level of MC simulations.} (see Fig.~\ref{fig:jetcombinationalgorithm}).
\begin{enumerate}
	\item A distance measure, $d_{ij}$, quantifying the phase-space separation of two objects $i$ and $j$, is defined for each pair of particles:
	\begin{equation}
	  d_{ij} = \mathrm{\text{min}} \left( E_{\text{T},i}^{2n}, E_{\text{T},j}^{2n} \right) \dfrac{\Delta R_{ij}^2}{R_0^2},
		\label{eq:dij}
	\end{equation}
	where $\Delta R_{ij}^2 = \left( \eta_{i} - \eta_{j} \right)^2 + \left( \phi_{i} - \phi_{j} \right)^2$ is the angular separation between objects. Dimensionless parameter $R_0$ determines the jet radius.
	\item A quantity, $d_i$, defining the distance to the beam-axis is calculated for each object $i$:
		\begin{equation}
	  d_{i} = E_{\text{T},i}^{2n}.
		\label{eq:di}
	\end{equation}
	\item Two objects $i$ and $j$ are merged according to the Snowmass~\cite{proc:snowmass:1990:134} convention\footnote{Other conventions exist. The Snowmass prescription results in massless jets.}, whenever some $d_{ij}$ is minimal among all $d_{ij}$ and $d_{i}$:
	\begin{align}
		E_\text{T} = E_{\text{T},i} + E_{\text{T},j} & \qquad \eta = \frac{\eta_iE_{\text{T},i} + \eta_jE_{\text{T},j}}{E_\text{T}} & \qquad \phi = \frac{\phi_iE_{\text{T},i} + \phi_jE_{\text{T},j}}{E_\text{T}}.			 \label{eq:snowmass}
	\end{align}
	In case $d_i$ is the smallest, object is called jet and removed from the list.
	\item The algorithm is repeated until no objects remain in the list.
\end{enumerate}
Parameter $n$ in equation~\ref{eq:dij} defines three types of algorithm:
\begin{itemize}
	\item \textsl{$n$ = -1}: the inclusive anti-\kt~algorithm~\cite{pub:antikt}, which is now extensively used at the LHC. This algorithm results in jets of circular shape and combines particles with large $E_\text{T}$ at the first place;
	\item \textsl{$n$ = 0}: the Cambrdige-Aachen~\cite{pub:cambidgeaachen} algorithm, which takes into account only angular separation between two object.  This algorithm was mostly used in the $e^+e^-$ collider experiments.
	\item \textsl{$n$ = 1}: the inclusive \kt~algorithm~\cite{pub:kt}. It produces jets of irregular shape and combines soft particles first.
\end{itemize}
 It has been shown that in photoproduction~\cite{pub:incljetphp} and DIS~\cite{pub:claudia} the \kt~and anti-\kt~have similar performance. The study~\cite{pub:jetradius} has demonstrated that $R=1$ is the optimal choice of the radius parameter. Taking this into account, the choice of the \kt~algorithm with $R=1$ was adopted in this thesis. Taking the advantage of longitudinal invariance of the algorithm, jet search is performed in the Breit frame.

\begin{figure}
	\centering
		\includegraphics[width=\linewidth]{./Figures/jetcombinationalgorithm.png}
	\caption{Recombination-type jet algorithm flow char.}
	\label{fig:jetcombinationalgorithm}
\end{figure}
