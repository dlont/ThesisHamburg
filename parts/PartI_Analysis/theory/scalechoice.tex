
The size of the unknown higher-order terms in the perturbative series~\ref{eq:pertseries} is usually one of the dominant sources of uncertainty in theoretical predictions. These contributions can be estimated from the dependence of the perturbative expansion on the renormalisation and factorisation scales. Using the renormalisation group equation it can be demonstrated that the scale dependence of higher-order coefficients accompanied by logarithmic terms are fully determined by lower-order coefficients~\cite{DataAnalysisBehnke}. For example, in the case of a dimensionless observable with perturbative expansion of the form~\ref{eq:pertseries}, the following expressions can be obtained:
\begin{align}
	f_1\left(\frac{\mu}{Q}\right) &= f_1\left(1\right) - k\beta_0f_0\log{\frac{\mu}{Q}}, \\
	f_2\left(\frac{\mu}{Q}\right) &= f_2\left(1\right) - \left[\left(k+1\right)\beta_0f_1\left(1\right) + k\beta_1f_0\right]\log{\frac{\mu}{Q}}, \\
																						&+ \frac{k\left(k+1\right)}{2}\beta_0^2f_0\log^2{\frac{\mu}{Q}},\\
																						& \ldots 
\end{align}
Thus, the $\mu$ variation in the $\mathcal{O}\left(\as^n\right)$ expression corresponds to the higher-order terms of the form:
\begin{equation}
 \as^{n+1}\left(\mu\right) \sum_{i=1}^{n+1-k}{\text{(know part)}\cdot\log^i{\frac{\mu}{Q}}} + \mathcal{O}\left(\as^{n+1}\right).
\label{eq:scalevariationerror}
\end{equation}
However, terms that are not accompanied by the logarithms e.g. $f_2\left(1\right)$, require explicit calculation. Therefore the reliability of an estimate of the size of the truncated terms depends on whether $f_i\left(1\right), i \ge 1$ are of similar order as $f_0$. Notably,
the leading-order coefficient $f_0$ is independent of the renormalisation scale, therefore the scale dependence of the LO approximation is completely governed by the scale dependence of the strong coupling. Therefore a realistic estimate of the size of unknown terms is possible at least to NLO.

Besides that, the sensitivity of perturbative predictions to the $\mu_f$ scale variation has to be taken into account. Although formally the DGLAP equations perform all-orders resummation of the ladder diagrams, the residual dependence on the factorisation scale of order $\mathcal{O}\left(\as^{k+1}\right)$ persists in the pQCD calculations. Similarly to the renormalisation-scale dependence, the behaviour of the perturbative coefficients on $\mu_f$ can be recovered~\cite{markusdiehl}:
\begin{equation}
 f_1\left(\frac{\mu_f}{Q},\frac{\mu_r}{Q}\right) = f_1\left(1,\frac{\mu_r}{Q}\right) + P_0 \otimes f_0 \log{\frac{Q^2}{\mu_f^2}},
\label{eq:factorisationscaledep}
\end{equation}
where $P_0$ is the LO splitting function. For the higher-order coefficients, a pattern similar to the renormalisation case is obtained.

Moreover, the DGLAP equations involve the strong coupling evaluated at the factorisation scale $\mu_f$. Whenever this scale differs from $\mu_r$, the RGE evolution from factorisation to renormalisation point has to be performed. Since the RGE is determined by incomplete series for the $\beta$-function, the strong coupling evolution can be unreliable if transition over a wide interval of scales is required. Mathematically this can be formulated as follows. Considering a Taylor expansion of $\as\left(\mu_f\right)$ around $\as\left(\mu_r\right)$ and substituting the RGE expression for the derivatives of the strong coupling $\mathrm{d}\as/\mathrm{d}\log{\frac{\mu_f^2}{\mu_r^2}}$ gives~\ref{botjeqcdnum}:
{\small
\begin{align}
 \as\left(\mu_f\right) &= \as\left(\mu_r\right) - \beta_0\log{\frac{\mu_f^2}{\mu_r^2}}\as^2\left(\mu_r\right) - \left(\beta_1\log{\frac{\mu_f^2}{\mu_r^2}} - \beta_2^2\log^2{\frac{\mu_f^2}{\mu_r^2}}\right)\as^3\left(\mu_r\right) + \mathcal{O}\left(\as^4\right)\\ \nonumber
 \as^2\left(\mu_f\right) & = \as^2\left(\mu_r\right) - 2\beta_0\log{\frac{\mu_f^2}{\mu_r^2}}\as^3\left(\mu_r\right) + \mathcal{O}\left(\as^4\right)\\
\as^3\left(\mu_f\right) & = \as^3\left(\mu_r\right) + \mathcal{O}\left(\as^4\right).
\end{align}
}
Thus, to ensure convergence of such an expansion, the factorisation scale must be closely related to the renormalisation scale. In addition, as in the case of renormalisation, the $\mu_f$ scale must be much larger than $\Lambda_\mathrm{QCD}$ to justify the applicability of perturbation theory results for the PDF evolution.

There is no commonly accepted way for the estimation of the size of the contribution from missing terms in perturbative series. However, it is widely accepted that the corresponding uncertainty can be estimated from the variation of the renormalisation and factorisation scales up and down by a factor of two. Additionally, the resulting variation of the observable depends on the central values $\mu_{f,0},\,\mu_{r,0}$ around which the variation is performed. It is desirable to choose central values such that the difference between the nominal result and the one with scaled values of $\mu_f$ and $\mu_r$ is minimised i.e. $\partial\mathcal{S}/\partial\mu=0$. This method is called the ``principle of minimum sensitivity'' (PMS)~\cite{pmsprinciple}. However, straightforward application of this method can result in a very large inclusive-jet cross section at the lowest \qsq~and \etjet~values~\cite{britzer}. Alternative variants can be found in~\cite{ioffelipatovfadin}. The proposed methods emphasise different aspects of the perturbative expansion. However, it should be noted that all are related to the behaviour of logarithmically enhanced terms.

In this analysis, the traditional prescription of choosing the scale corresponding to the typical energy scale of the process was adopted. \textcolor{blue}{Theoretical setting used in this work to perform the calculations} for the jet production cross sections are described in Chapter~\ref{ch:results} in details.
