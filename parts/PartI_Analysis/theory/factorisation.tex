Another remarkable feature of QCD is factorisation of short- and long-time-scale processes. For example, the cross section for jet production in DIS can be represented in factorised form~\cite{Collins:1989gx} as:
\begin{equation}
 \mathrm{d}\sigma_{\mathrm{jet}} = \int{f_a\left(x,\mu_f\right)}\, \mathrm{d}\sigma_{\mathrm{part}}\left(x,\mu_r, \mu_f, \alpha_s\left(\mu_r\right) \right).
\label{eq:disfactorisation}
\end{equation} 
In this expression $f\left(x,\mu_f\right)$ represents the nonperturbative proton parton distribution function and $\mathrm{d}\sigma_{\mathrm{part}}$ is the hard-scattering partonic cross section that is calculated in perturbation theory. A schematic illustration representing this equation is depicted in Figure~\ref{fig:Factorisation}. Technically, in the calculation an additional factorisation scale, $\mu_F$ is introduced. It defines approximately the virtuality of the intermediate states that contribute to the hard scattering while the long-distance physics is absorbed in universal nonperturbative parameters. Factorisation leads to the calculations being usually performed in two steps. The perturbative part can be evaluated as a series expansion in the small coupling constant, described above, while parton distributions have to be determined experimentally. In this procedure, singularities attributed to the long-distance processes e.g. soft or collinear radiation of partons, are absorbed into nonperturbative terms. The factorisation scale, $\mu_f$, serves as a reference point at which the subtraction of the singularities is performed. The subtraction scheme defines the prescription for reshuffling of finite terms between partonic cross section and PDFs. The employed factorisation scheme must be consistent with that used for renormalisation. In this analysis the modified variant of the minimal subtraction, $\overline{\mathrm{MS}}$, scheme~\cite{Bardeen:1978yd} was used.
\begin{figure}[t]
	\centering
	\begin{subfloat}[]{
		\includegraphics[width=0.45\linewidth]{./Figures/source/Factorisation.png}
		\label{fig:Factorisation}
	 }%
	\end{subfloat}
	\begin{subfloat}[]{
		\includegraphics[width=0.45\linewidth]{./Figures/source/DGLAPLadder.png}
		\label{fig:DGLAPLadder}
	}%
	\end{subfloat}
	\caption{Schematic illustration of the factorisation of the hard \ep-process into non-perturbative proton PDFs and hard-scattering partonic cross section (a). An example diagram with strongly ordered parton emission contributing to the NC DIS process (b).}
	\label{fig:factorisationdglapladder}
\end{figure}
%\begin{figure}[t]
	%\centering
		%\includegraphics[width=0.5\textwidth]{./Figures/source/Factorisation.png}
		%\includegraphics[width=0.5\textwidth]{./Figures/source/DGLAPLadder.png}
	%\caption{Factorisation}
	%\label{fig:Factorisation}
%\end{figure}
