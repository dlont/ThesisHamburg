\documentclass[notitlepage,a4paper,10pt]{scrreprt}

\usepackage{amsmath}
\usepackage{amssymb}
\usepackage{amsbsy}

\usepackage[normalem]{ulem} % either use this (simple) or
\usepackage{xspace}

\usepackage{color}
\usepackage{graphicx}
\usepackage{caption}
\usepackage{subfig}
% \usepackage{subcaption}
\usepackage{wrapfig}
\usepackage{pdflscape}
\newcommand*\rfrac[2]{{}^{#1}\!/_{#2}}

\newcommand{\hera}{\mbox{\textsf{HERA}}\xspace}
\newcommand{\zeus}{\mbox{\textsf{ZEUS}}\xspace}
\newcommand{\lepto}{\mbox{\textsf{LEPTO}}\xspace}
\newcommand{\ariadne}{\mbox{\textsf{ARIADNE}}\xspace}

%%%%%%%%%%%%%%%%%%%%%%%%%%%%%%%%%%%%%%%%%%%%%%%%%%%%%%%
%%
%%
%%    						Definitions
%%
%%
%%%%%%%%%%%%%%%%%%%%%%%%%%%%%%%%%%%%%%%%%%%%%%%%%%%%%%%

%\theoremstyle{plain}
%\newtheorem{acknowledgement}{Acknowledgement}
%\newtheorem{algorithm}{Algorithm}
%\newtheorem{axiom}{Axiom}
%\newtheorem{case}{Case}
%\newtheorem{claim}{Claim}
%\newtheorem{conclusion}{Conclusion}
%\newtheorem{condition}{Condition}
%\newtheorem{conjecture}{Conjecture}
%\newtheorem{corollary}{Corollary}
%\newtheorem{criterion}{Criterion}
%\newtheorem{definition}{Definition}
%\newtheorem{example}{Example}
%\newtheorem{exercise}{Exercise}
%\newtheorem{lemma}{Lemma}
%\newtheorem{notation}{Notation}
%\newtheorem{problem}{Problem}
%\newtheorem{proposition}{Proposition}
%\newtheorem{remark}{Remark}
%\newtheorem{solution}{Solution}
%\newtheorem{summary}{Summary}
%\newtheorem{theorem}{Theorem}
%\numberwithin{equation}{section}


\newcommand{\hera}{\mbox{\textsf{HERA}}\xspace}
\newcommand{\desy}{\mbox{\textsf{DESY}}\xspace}
\newcommand{\linac}{\mbox{\textsf{LINAC}}\xspace}
\newcommand{\lhc}{\mbox{\textsf{LHC}}\xspace}
\newcommand{\lhec}{\mbox{\textsf{LHeC}}\xspace}
\newcommand{\tevatron}{\mbox{\textsf{TEVATRON}}\xspace}
\newcommand{\petra}{\mbox{\textsf{PETRA}}\xspace}
\newcommand{\zeus}{\mbox{\textsf{ZEUS}}\xspace}
\newcommand{\hone}{\mbox{\textsf{H1}}\xspace}
\newcommand{\hermes}{\mbox{\textsf{HERMES}}\xspace}
\newcommand{\herab}{\mbox{\textsf{HERA-B}}\xspace}
\newcommand{\atlas}{\mbox{\textsf{ATLAS}}\xspace}
\newcommand{\cms}{\mbox{\textsf{CMS}}\xspace}
\newcommand{\lepto}{\mbox{\textsf{LEPTO}}\xspace}
\newcommand{\pythia}{\mbox{\textsf{PYTHIA}}\xspace}
\newcommand{\herwig}{\mbox{\textsf{HERWIG}}\xspace}
\newcommand{\ariadne}{\mbox{\textsf{ARIADNE}}\xspace}
\newcommand{\nlojet}{\mbox{\textsf{NLOJET++}}\xspace}
\newcommand{\herapdf}{\mbox{\textsf{HERAPDF}}\xspace}
\newcommand{\fastnlo}{\mbox{\textsf{FastNLO}}\xspace}
\newcommand{\qcdnum}{\mbox{\textsf{QCDNUM}}\xspace}
\newcommand{\herafitter}{\mbox{\textsf{HERAFitter}}\xspace}
\newcommand{\migrad}{\mbox{\textsf{MIGRAD}}\xspace}
\newcommand{\minuit}{\mbox{\textsf{MINUIT}}\xspace}
\newcommand{\heracles}{\mbox{\textsf{HERACLES}}\xspace}
\newcommand{\jetset}{\mbox{\textsf{JETSET}}\xspace}
\newcommand{\djangoh}{\mbox{\textsf{DJANGOH}}\xspace}
\newcommand{\cteqfived}{\mbox{\textsf{CTEQ5D}}\xspace}

\newcommand{\amadeus}{\mbox{\textsf{AMADEUS}}\xspace}
\newcommand{\mozart}{\mbox{\textsf{MOZART}}\xspace}
\newcommand{\zgana}{\mbox{\textsf{ZGANA}}\xspace}
\newcommand{\zephyr}{\mbox{\textsf{ZEPHYR}}\xspace}
\newcommand{\orange}{\mbox{\textsf{ORANGE}}\xspace}
\newcommand{\PHANTOM}{\mbox{\textsf{PHANTOM}}\xspace}
\newcommand{\paw}{\mbox{\textsf{PAW}}\xspace}
\newcommand{\rootpaw}{\mbox{\textsf{ROOT}}\xspace}
\newcommand{\geant}{\mbox{\textsf{GEANT}}\xspace}
\newcommand{\adamo}{\mbox{\textsf{ADAMO}}\xspace}
\newcommand{\gaf}{\mbox{\textsf{GAF}}\xspace}
\newcommand{\zufos}{\mbox{\textsf{ZUFOS}}\xspace}

\newcommand{\aem}{\mbox{$\alpha_{em}$}\xspace}
\newcommand{\as}{\mbox{$\alpha_{s}$}\xspace}
\newcommand{\asz}{\mbox{$\alpha_{s}\left(M_{Z}\right)$}\xspace}
\newcommand{\etjet}{\mbox{$E_T^{jet}$}\xspace}
\newcommand{\etjetlab}{\mbox{$E_{T,lab}^{jet}$}\xspace}
\newcommand{\etjetb}{\mbox{$E_{T,B}^{jet}$}\xspace}
\newcommand{\etjetbhad}{\mbox{$E_{T,B}^{jet,had}$}\xspace}
\newcommand{\etjetbdet}{\mbox{$E_{T,B}^{jet,det}$}\xspace}
\newcommand{\etjetav}{\mbox{$\overline{E_T^{jet}}$}\xspace}
\newcommand{\etjetbav}{\mbox{$\overline{E_{T,B}^{jet}}$}\xspace}
\newcommand{\etajet}{\mbox{$\eta^{jet}$}\xspace}
\newcommand{\etajetlab}{\mbox{$\eta^{jet}_{lab}$}\xspace}
\newcommand{\etajetb}{\mbox{$\eta_B^{jet}$}\xspace}

\newcommand{\mjj}{\mbox{$M_{jj}$}\xspace}
\newcommand{\mjjj}{\mbox{$M_{jjj}$}\xspace}
\newcommand{\dsdetdeta}{$\frac{d\sigma}{dE_{T}d\eta^{jet}}$\xspace}
\newcommand{\kt}{\mbox{$k_{T}$}\xspace}
\newcommand{\ep}        {\mbox{$e p$}\xspace}
\newcommand{\znought}        {\mbox{$Z^0$}\xspace}

\newcommand{\epini}{\mbox{$E_p$}\xspace}
\newcommand{\eeini}{\mbox{$E_e$}\xspace}
\newcommand{\eefin}{\mbox{$E_e'$}\xspace}
\newcommand{\thetae}{\mbox{$\theta_e$}\xspace}
\newcommand{\gamha}{\mbox{$\gamma_\text{had}$}\xspace}

%\newcommand{\GeVs}      {\mbox{${\,\rm \text{ГеВ}^2}$}}
%\newcommand{\GeV}       {\mbox{${\,\rm \text{ГеВ}}$}}
%\newcommand{\MeV}       {\mbox{${\,\rm \text{МеВ}}$}}
%
\newcommand{\sqs}       {\mbox{$\sqrt{s}$}\xspace}
\newcommand{\qsq}       {\mbox{${Q^2}$}\xspace}
\newcommand{\y}       {\mbox{${y}$}\xspace}
\newcommand{\x}       {\mbox{${x}$}\xspace}
%
\newcommand{\dsdqsq}       {\mbox{$d\sigma/dQ^2$}\xspace}
\newcommand{\dsdetjetb}       {\mbox{$d\sigma/dE_{T,B}^{jet}$}\xspace}
\newcommand{\dsdetajetb}       {\mbox{$d\sigma/\eta_{B}^{jet}$}\xspace}
%
\newcommand{\sig}       {\mbox{$\sigma$}\xspace}
\newcommand{\sigb}      {\mbox{$\sigma\cdot B$}\xspace}
%
\newcommand{\gcc}       {\mbox{$\Gamma_{c\bar c}$}\xspace}
\newcommand{\gha}       {\mbox{$\Gamma_{had}$}\xspace}
\newcommand{\gch}       {\mbox{$\frac{\Gamma_{c\bar c}}{\Gamma_{had}}$}\xspace}
\newcommand{\gchfb}     {\mbox{$\gch \cdot f(c\ra D,\Lambda) \cdot B$}\xspace}
%
\newcommand{\ga}        {\mbox{$\gamma$}\xspace}
\newcommand{\zn}        {\mbox{$\rm{Z}^0$}\xspace}
%
\newcommand{\gp}        {\mbox{$\gamma p$}\xspace}
%
\newcommand{\gaa}        {\mbox{$\gamma^{\ast}$}\xspace}
%
\newcommand{\gs}        {\mbox{$\gamma_s$}\xspace}
%
\newcommand{\Pom}       {\mbox{I$\!$P}\xspace}
%
\newcommand{\lum}{\mbox{$\cal L$}\xspace}
%
\newcommand{\bran}{\mbox{$\cal B$}\xspace}
%
\newcommand{\fr}{\mbox{$\cal F$}\xspace}
%
\newcommand{\rat}{\mbox{$\cal R$}\xspace}
%
\newcommand{\acc}{\mbox{$\cal A$}\xspace}

%%%%%%%%%%%%%%%%%%%%%%%%%%%%%%%%%%%%%%%%%%%%%%%%%%%%%%%%%
%%%%%%%%%%%%%%%%%%%%%%%%%%%%%%%%%%%%%%%%%%%%%%%%%%%%%%%%%
%%%%%%%%%%%%%%%%%%%%%%%%%%%%%%%%%%%%%%%%%%%%%%%%%%%%%%%%%

%\newcommand{\etal}{et al.}
% \hoffset = -10pt \voffset = -45pt \textwidth = 450pt \textheight = 642pt
% \hoffset = -45pt \voffset = -45pt \textwidth = 450pt \textheight = 642pt

\makeatletter
\newcommand*{\toccontents}{\@starttoc{toc}}
\makeatother

\begin{document}
% Title Page
\title{}
\author{Denys Lontkovskyi}
\maketitle

\toccontents
\chapter{Introduction}
Since ancient times humanity spent an enormous effort trying to identify the basic building blocks of Nature and find guiding principles that govern all observed phenomena. Beginning from early 20th century scattering experiments play a majour role in revealing microscopic structure of matter. Thus, for example, pioneering studies of the scattering of $\alpha$-particles on Gold led Rutherford~\cite{rutherford} to the discovery of atomic nucleus. Soon it was realised that the nucleus is composed of protons and neutrons, generally called nucleons. Approximately half a century after Rutherford's experiment the investigation of high energy inelastic electron-nucleon scattering in  the  series of SLAC experiments~\cite{slac} provided key evidence for the nucleon substructure. The studies of internal structure of matter culminated in high-precision experiments conducted at HERA.

In this attempt all experimental and theoretical findings accumulated over the centuries were unified in modern concept of the \emph{Standard Model} of elementary particles.

% \chapter{Corrections and reweightings}
% In many high energy physics analyses the estimation of detector effects is often based on the Monte Carlo simulations. However, as it will be discussed in Chapter~\ref{ch:cs} a reliable estimate of the influence of the detector responce is only possible if the MC simulations accurately describe all relevant distributions. %\sout{Therefore a detailed description of the observed data distributions is one of the crucial steps in any analysis.} 
% The discrepancy between data and MC may originate from two basic sources: inadequacy of the modelling of the physics process or the particle transport through the detector volume.
% 
% The improvement in the simulations is provided by using more accurate predictions for the physical observables and employing better simulation algorithms. However, obtaining the detailed description of the physical process can be a formidable task, therefore, often, a more feasible approach is used. An improvement of the description of the data distribution is obtained by assigning weights to the MC events (\emph{reweighting method}). The weights are usually the functions of the kinematic variables and the sum of the reweighting factors are adjusted in such a way that the simulations reasonably describe the data. The size of the weights is usually determined from the empirical fit to the data. The reweighting procedure is required to be independent of the reconstruction therefore it must be based on \emph{true} level information. In case when several quantities are reweighted, the final weight factor applied to the MC event is a product of individual weights, $w = \prod_i w_i$.
% 
% This chapter describes the corrections applied to the MC simulations. At the beginning, the details of the longitudinal vertex position reweighting are presented. Later the MC track-veto correction is described. 
% 
% \section{Longitudinal vertex position reweighting}
% The detection and reconstruction of the scattered electron depend on the longitudinal position of the primary vertex, $Z_\text{vtx}$. In particular the detector and trigger acceptances vary with $Z_\text{vtx}$. The shape of the distribution of the primary vertex position is determined by the length\footnote{The space-charge distribution within a bunch typically has a Gaussian shape.} of the interacting bunches and thus depends on machine conditions. In order to suppress non-$ep$ background, restrictions were applied on the primary vertex position (see Chapter~\ref{ch:selection}). Therefore, since the luminosity measurements refer to the whole $ep$ interaction region, any cuts on $Z_\text{vtx}$ have an effect on the overall normalisation. 

An accurate simulation of the $Z_\text{vtx}$ distribution in the MC samples is therefore very important. The $Z_\text{vtx}$-reweighting procedure adopted here was developed in~\cite{thesis:behr:2010} and consisted of the following steps:
\begin{itemize}
 \item in order to avoid a possible bias from the jet selection as well as from tracking restrictions at the trigger level, the FLT30 bit was required instead of the FLT40, 43, 50 bits that were used as standard in the analysis;
 \item selection cuts on the longitudinal position of the interaction vertex were removed;
 \item the $Z_\text{vtx}$ distributions in the data and MC were fitted to the sum of four Gaussian functions, 
\begin{equation}
f\left(\vec a\right)=\sum_{i=1}^4{a_N^{\left(i\right)}\exp{\left[-\left(Z_\text{vtx}-a_{\mu}^{\left(i\right)}\right)^2/\left(a_\sigma^{\left(i\right)}\right)^2\right]}}.
\label{eq:fourgauss}
\end{equation}
 The reweighting factors, $w_{Z}$, were determined from the fit of the ratio of the normalised data and MC distributions to the function 
\begin{equation}
w=f\left(\vec a_1\right)/f\left(\vec a_2\right)
\label{eq:zvtxweght}
\end{equation}
 using the parameters of individual fits $\vec a_{\mathrm{Data}}, \vec a_{\mathrm{MC}}$ as seed values;
 \item the weights were determined from the detector-level distributions as function of the reconstructed position, $Z_\text{vtx}$. They were then applied to the MC events as a function of the true position, $Z_\text{vtx}^\text{true}$. This substitution $Z_\text{vtx} \mapsto Z_\text{vtx}^\text{true}$ can be made because the migration effects for the $Z_\text{vtx}$ distribution were found to be very small and can be neglected.
\end{itemize}

The existing MC samples for the 2004--2005~$e^-$ and 2006~$e^-$ data taking periods describe the data very well in the whole interaction region. Only for the 2006/2007~$e^+$ running period did the MC not reproduce the data distribution reasonably\footnote{The primary vertex distribution for the \hera II running period was measured in a dedicated un-biased study of low-\qsq NC DIS events~\cite{upub:oliver:zn07008} and was implemented in the MC production software}. In particular, disagreement between the measured and simulated distributions in the satellite-bunch and transition regions ($\left|Z_{\text{vtx}}\right|>30\,\cm$) was observed. Therefore, a reweighting of the longitudinal position of the primary vertex was implemented only for this MC sample. The comparison between data and MC distributions for the 2006/2007~$e^+$ data taking period before the reweighting is shown in Figure~\ref{fig:zvtxrew}. Although, individual fits have relatively large $\chi^2/N_\text{df}$ values, which, in principle, may introduce a bias, the final fit to the ratio of the normalised distributions has $\chi^2/N_\text{df}\approx 1$, and justifies using the fit results in the reweighting. After correcting the MC distribution (see Figure~\ref{fig:zvtxrewaf}) good agreement between data and simulations was observed.\marginpar{OB:unclear which true distribution of zvtx was generated in MC?\\DL:added a footnote.}
\begin{figure}[t]
\begin{center}
 \hspace{-35pt}\includegraphics[width=1.1\textwidth]{./Figures/zvtxrew/h_Zvtxfit_ratio_p_lepto_fix}%
\end{center}
\caption{The $Z_\text{vtx}$ distributions in the data and \lepto MC before the reweighting (top left panel) together with the individual fits of the data (top right panel) and MC (bottom right panel) to the function Eq.~\eqref{eq:fourgauss}. The ratio DATA/MC and the fit to Eq.~\eqref{eq:zvtxweght} is also shown (bottom left panel).} 
\label{fig:zvtxrew}
\end{figure}

\begin{figure}[p]
\begin{center}
 \includegraphics[height=0.9\textheight]{./Figures/zvtxrew/h_Zvtxratio_p_lepto}
\end{center}
\caption{Comparison of the $Z_\text{vtx}$ in data and \lepto MC distribution after reweighting (top panel) and ratio of the two distributions (bottom panel).} 
\label{fig:zvtxrewaf}
\end{figure}
% 
% \section{Track veto efficiency correction}
% An accurate description of the trigger efficiency is an important ingredient in this analysis.

As it was described in Chapter~\ref{ch:detector} the \zeus trigger was used to reject non-$ep$ background and to enhance the acceptance for physics-related processes. At the FLT most of the trigger bits were utilising CTD information to veto the events characterised by specific combinations of all and vertex-fitted tracks. The definitions for the corresponding track classes is illustrated in Figure~\ref{fig:trackvetodefinition}. According to this information two track-veto types, relevant for this analysis, were identified (see Table~\ref{tab:trackveto}). 
\begin{table}[htpb]
 \centering
 \begin{tabular}{lc}
 \multicolumn{2}{c}{track-veto requirement} \\
  \hline
 ``semi-loose'' & track-class $\le$ 2 or (track class = 8 and track multiplicity $\ge$ 26) \\
 ``tight''      & track-class $\le$ 2 \\
 \end{tabular} 
\caption{The track-veto condition used in the first level trigger.}
\label{tab:trackveto}
\end{table} 
In order to check the description of the track-veto in MC simulations monitor trigger bit was used. The FLT30 was requiring an isolated electromagnetic cluster in the RCAL and therefore was independent of the CTD information. The track-veto efficiency, expressed as a ratio
\begin{equation}
 \epsilon_\mathrm{trk} = \frac{N\left(\text{track veto} \wedge \text{FLT30}\right)}{N\left(\text{FLT30}\right)},
\end{equation}
where $N\left(\text{FLT30}\right)$ is a number of events triggered by the FLT30 and $N\left(\text{track veto} \wedge \text{FLT30}\right)$ is a number of events in a subset satisfying additional track-veto requirements, was studied separately in data and MC for different data taking periods. To determine $N\left(\text{track veto} \wedge \text{FLT30}\right)$ the track-veto was simulated offline by imposing additional restrictions on track quantities available at the FLT.

The efficiency was investigated as a function of $y_{DA}$. This variable was strongly correlated with the size of hadronic activity and thus with track multiplicity. The corresponding ratios in the data and MC are shown in Figures~\ref{fig:tveffdatamc}~\subref{fig:tveffdatamc_subfig1}--\subref{fig:tveffdatamc_subfig4}. The discrepancy between the data and MC simulations was attributed to the bad description of the track-class distribution in MC. In order to compensate for higher efficiency in MC an additional correction was implemented. The ratio of efficiencies in the data and MC was fitted to the first order polynomial
\begin{equation} 
 f\left(y_{DA}\right)=a_0 + a_1 \cdot y_{DA}.
\end{equation}
Because the efficiency observed in MC was higher than that in the data it can be corrected by rejecting overabundant MC events. Therefore, for each MC event a uniformly distributed random number, $r$, was generated and the event was rejected if $r > f\left(y_{DA}\right)$. The correction was implemented in \lepto and \ariadne samples and for different data-taking periods separately. The size of the correction depends on the value of $y_{DA}$ and was typically less than 0.5\% for 2004--2005$e^-$ and 2006$e^-$ and less than 3\% for 2007$e^+$. It was observed that for the ``semi-loose'' track-veto the same correction as for ``tight'' track-veto can be applied. The comparison of the track-veto efficiencies in the data and MC after applying the correction is illustrated in Figures~\ref{fig:aftveffdatamc}~\subref{fig:aftveffdatamc_subfig1}--\subref{fig:aftveffdatamc_subfig4}. After the correction the data efficiency was very well described by MC. 

The systematic effects attributed to the MC track-veto correction was examined investigating trigger efficiency as a function of CTD-FLT track multiplicity. The results of these studies are detailed in Chapter~\ref{systematics}.
\begin{figure}[t]
  \begin{center}
    \includegraphics[width=0.65\textwidth,trim={0 120 0 120},clip]{../thesis/Figures/classes96}
  \end{center}
  \caption{The definition of track veto classes (taken from~\cite{Yamazaki site}).}
  \label{fig:trackvetodefinition}
\end{figure}

\begin{figure}[ht!]
\begin{center}
\begin{subfloat}[]{\includegraphics[width=.45\linewidth,trim={0 0 280 0},clip] {../thesis/Figures/tvrew/tvrew_lep07p_yda_ltv}
   \label{fig:tveffdatamc_subfig1}
 }%
\end{subfloat}
 \begin{subfloat}[]{\includegraphics[width=.45\linewidth,trim={0 0 280 0},clip]{../thesis/Figures/tvrew/tvrew_ari07p_yda_ltv}
   \label{fig:tveffdatamc_subfig2}
 }%
\end{subfloat}
\newline
\begin{subfloat}[]{\includegraphics[width=.45\linewidth,trim={0 0 280 0},clip] {../thesis/Figures/tvrew/ratio_tvrew_lep07p_yda_ltv}
   \label{fig:tveffdatamc_subfig3}
 }%
\end{subfloat}
 \begin{subfloat}[]{\includegraphics[width=.45\linewidth,trim={0 0 280 0},clip]{../thesis/Figures/tvrew/ratio_tvrew_ari07p_yda_ltv}
   \label{fig:tveffdatamc_subfig4}
 }%
\end{subfloat}
\end{center}
\caption{Loose track-veto efficiency as a function of $y_{DA}$ in data and \lepto (a), \ariadne (b) MC. The ratio of the track-veto efficiency in data and \lepto (c), \ariadne (d) MC and the result of the fit.}
\label{fig:tveffdatamc}
\end{figure}

%After correction
\begin{figure}[pht]
\begin{center}
\begin{subfloat}[]{\includegraphics[width=.45\linewidth,trim={0 0 280 0},clip] {../thesis/Figures/tvrew/checktvrew_lep07p_yda_ltv}
   \label{fig:aftveffdatamc_subfig1}
 }%
\end{subfloat}
 \begin{subfloat}[]{\includegraphics[width=.45\linewidth,trim={0 0 280 0},clip]{../thesis/Figures/tvrew/checktvrew_lep07p_yda_sltv}
   \label{fig:aftveffdatamc_subfig2}
 }%
\end{subfloat}
\newline
\begin{subfloat}[]{\includegraphics[width=.45\linewidth,trim={0 0 280 0},clip] {../thesis/Figures/tvrew/checktvrew_ari07p_yda_ltv}
   \label{fig:aftveffdatamc_subfig3}
 }%
\end{subfloat}
 \begin{subfloat}[]{\includegraphics[width=.45\linewidth,trim={0 0 280 0},clip]{../thesis/Figures/tvrew/checktvrew_ari07p_yda_sltv}
   \label{fig:aftveffdatamc_subfig4}
 }%
\end{subfloat}
\end{center}
\caption{Comparison between data and MC of the loose (a,c) and semi-loose (b,d) track-veto after applying the track veto correction.}
\label{fig:aftveffdatamc}
\end{figure}


% 
% \chapter{Cross section determination}
% The data are subject to various random effects including non-linear detector responce, finite resolution, limited acceptance, reconstruction inefficiencies etc. Such distortions in case of vanishing background contribution are described by the Fredholm integral equation:
\begin{equation}
 \int_\Omega K\left(s,t\right)\,f\left(t\right)\,\mathrm{d}t = g\left(s\right),
 \label{eq:fredholm}
\end{equation}
where $g\left(s\right)$ is the measured distribution corresponding to underlying true distribution $f\left(t\right)$ and kernel $K\left(s,t\right)$ describes the response of the detector. Thus the determination of $f\left(t\right)$ requires solving this equation.
In particle physics the process of extraction of the true distribution from the measured one is called \emph{unfolding}. Various techniques were proposed in the literature to solve the equation~\ref{eq:fredholm} (see for example~\cite{statistics book page 187} and references therein). In this thesis the so-called \emph{method of correction factors (or bin-by-bin method)} was employed.

In this chapter the procedure used for the determination of inclusive-jet cross sections is described. In Section~\ref{sec:acccor} the bin-by-bin method is presented. Afterwards in Section~\ref{sec:bindef} binning definition for the cross sections measurement is discussed. The chapter finishes with the description of polarisation corrections.

\section{Acceptance correction}
\label{sec:acccor}
The cross section for inclusive-jet production in some kinematic bin is determined according to:
\begin{equation}
 \left.\frac{\mathrm{d}\sigma}{\mathrm{d}X}\right|_{\mathrm{bin},i} = \frac{N_{\mathrm{bin},i}}{\mathcal{L} \cdot \Delta_{\mathrm{bin},i}} \cdot \mathcal{A}_{\mathrm{bin},i} \cdot \mathcal{C}_{\mathrm{bin},i},
\end{equation}
where $N_{\mathrm{bin},i}$ number of jets reconstructed in data in bin $i$, $\mathcal{L}$ is the integrated luminosity of the data sample, $\Delta_{\mathrm{bin},i}$ is the bin width and $\mathcal{A}_{\mathrm{bin},i}$ is the acceptance correction factor described below. The effects from higher-order QED processes or those related to the polarisation of the lepton beam need to be included into the definition of the cross section in order to obtain an observable consistent with that provided by existing NLO QCD codes (see Chapter~\ref{sec:nloqcd}). These effects are combined in additional multiplicative term $\mathcal{C}_{\mathrm{bin},i}$.

The acceptance correction factors applied to the data in order to correct for detector effects are determined using MC samples from the number of jets generated in some kinematic bin, $N_{\mathrm{bin},i}^{\mathrm{gen}}$, and the number of jets reconstructed in that bin, $N_{\mathrm{bin},i}^{\mathrm{rec}}$:
\begin{equation}
 \mathcal{A}=\frac{N_{\mathrm{bin},i}^{\mathrm{gen}}}{N_{\mathrm{bin},i}^{\mathrm{rec}}}.
 \label{eq:accdef}
\end{equation}
In order to ensure the validity of this bin-by-bin multiplicative correction, the migrations across the neighboring bins have to be sufficiently small and MC simulations have to provide a good description of the shape of the measured distributions. Two additional variables can be defined in order to quantify the detector effects,  namely \emph{purity}
\begin{equation}
 \mathcal{P}=\frac{N_{\mathrm{bin},i}^{\mathrm{rec \wedge gen}}}{N_{\mathrm{bin},i}^{\mathrm{rec}}}
\label{eq:puritydef}
\end{equation}
and \emph{efficiency}
\begin{equation}
 \mathcal{E}=\frac{N_{\mathrm{bin},i}^{\mathrm{rec \wedge gen}}}{N_{\mathrm{bin},i}^{\mathrm{gen}}}.
\label{eq:efficiencydef}
\end{equation}
In these definitions $N_{\mathrm{bin},i}^{\mathrm{rec \wedge gen}}$ represents the number of jets generated and reconstructed in some particular bin. The purity of a bin determines the fraction of jets that migrated into this bin while originating from the other bin. This can happen, for example, due to finite resolution of the detector. Unlike purity, the efficiency is an estimate of jets loss due to migrations outside the measurement bin or due reconstruction inefficiency or cut requirements. Acceptance, purity and efficiency have the following relation:
\begin{equation}
  \mathcal{A} = \frac{\mathcal{E}}{\mathcal{P}}.
\end{equation}

The terms appearing in the definitions~\eqref{eq:accdef}--\eqref{eq:efficiencydef} are not statistically independent, therefore a correlation between different factors has to be taken into account when statistical uncertainty attributed to $\mathcal{A},\, \mathcal{P}$ or $\mathcal{E}$ factors is needed. However, they can be expressed in terms of statistically independent quantities 
\begin{equation}
 \mathcal{P}=\frac{N_{\mathrm{bin},i}^{\mathrm{rec \wedge gen}}}{N_{\mathrm{bin},i}^{\mathrm{rec \wedge gen}}+N_{\mathrm{bin},i}^{\mathrm{rec \wedge \ulcorner gen}}},\quad \mathcal{E}=\frac{N_{\mathrm{bin},i}^{\mathrm{rec \wedge gen}}}{N_{\mathrm{bin},i}^{\mathrm{rec \wedge gen}}+N_{\mathrm{bin},i}^{\mathrm{\ulcorner rec \wedge gen}}},\quad 
 \mathcal{A}=\frac{N_{\mathrm{bin},i}^{\mathrm{rec \wedge gen}}+N_{\mathrm{bin},i}^{\mathrm{\ulcorner rec \wedge gen}}}{N_{\mathrm{bin},i}^{\mathrm{rec \wedge gen}}+N_{\mathrm{bin},i}^{\mathrm{rec \wedge \ulcorner gen}}},
\end{equation}
where $N_{\mathrm{bin},i}^{\mathrm{\ulcorner rec \wedge gen}}$ and $N_{\mathrm{bin},i}^{\mathrm{rec \wedge \ulcorner gen}}$ are the number of jets generated but not reconstructed in bin $i$ and reconstructed but not generated in that bin, respectively.

The acceptance correction factors, efficiency and purity for the relevant inclusive-jet cross sections determined using either \lepto or \ariadne MC samples are shown in Figures~\ref{fig:epa}~\subref{fig:epa_subfig1}--\subref{fig:epa_subfig3}. In general, the purity is typically above 40\% for all kinematic observables, while the efficiency is typicaly within 30\% and 65\%. The decrease of efficency in the region $250GeV<\qsq<500GeV$ was observed before~\cite{joerg hanno trevor januschek} and was attributed to the reduced electron identification capabilities in the transition region between the RCAL and BCAL. The acceptance correction factors never exceed 1.6 and are typicaly below 1.4.
\begin{figure}[pht]
\begin{center}
\begin{subfloat}{\includegraphics[width=\linewidth,trim={0 0 0 0},clip] {../thesis/Figures/epa/h_etjetb_CS_d_epa}
   \label{fig:epa_subfig1}
 }%
\end{subfloat}
\newline
 \begin{subfloat}{\includegraphics[width=\linewidth,trim={0 0 0 0},clip]{../thesis/Figures/epa/h_etajetb_CS_d_epa}
   \label{fig:epa_subfig2}
 }%
\end{subfloat}
\newline
\begin{subfloat}{\includegraphics[width=\linewidth,trim={0 0 0 0},clip] {../thesis/Figures/epa/h_q2_CS_d_epa}
   \label{fig:epa_subfig3}
 }%
\end{subfloat}
\end{center}
\caption{Acceptance correction factors, efficiency and purity for inclusive-jet cross sections as functions of \etjetb, \etajetb and \qsq.}
\label{fig:epa}
\end{figure}

A major limitation of described method is its possibly high sensitivity to the MC true level distribution~\cite{cowan note}. Hence this effect was investigated by using alternative event generators and was taken into account in systematic uncertainty (see Chapter~\ref{systematics}).
\section{Bin definition}
\label{sec:bindef}
The choice of the size of the bin width is limited by two factors. Naturally, in order to obtain maximum information from the measurements, when the bin width has to be as small as possible. However reduction of the bin size leads to decrease of the number of entries. Therefore the width must be large enough to obtain statistically significant number of entries per bin. In addition, because of finite resolution and possibly non-linear responce of the detector the migration effects can be substantial in case the bin width is much smaller than detector resolution. Besides that it was convenient to define the binning for inclusive-jet analysis to be consistent with that for dijet and trijet analysis performed at \zeus~\cite{joerg Makarenko}.

The binning definition for the jet cross sections measured in this analysis is outlined below.
\subsection{Bin definition for ${\mathrm{d}\sigma}/{\mathrm{d}Q^2}$}
\label{subsec:bindefq2}
For the single differential cross section ${\mathrm{d}\sigma}/{\mathrm{d}Q^2}$ 6 bins were defined spanning the measurement phase space from 125 GeV to 20000 GeV. Because the inclusive NC DIS cross section scales as $1/Q^4$ the size of the bins increases with increasing \qsq~in order to obtain statistically significant jet sample in each bin.
\subsection{Bin definition for ${\mathrm{d}\sigma}/{\mathrm{d}\etjetb}$}
\label{subsec:bindefet}
The single differential inclusive jet cross section as a function of \etjetb~also has 6 bins of varying width. Although the cross section is large at low \etjetb~the size of the bin width was limited by the purity at low jet transverse energies. At low \etjetb the measurement is bounded by the phase space restriction $\etjetb > 8$ GeV while the upper limit extends upto 100 GeV.
\subsection{Bin definition for ${\mathrm{d}\sigma}/{\mathrm{d}\etajetlab}$}
\label{subsec:bindefeta}
The cross section ${\mathrm{d}\sigma}/{\mathrm{d}\etajetlab}$ spans the region $-2<\etajetb<1.8$ and has 5 bins. The sizes of the bins were chosen in order to optimise purity values and at the same time keep an appropriate number of bins. As can be seen in Figure~\ref{fig:epa}~\subref{fig:epa_subfig2} stable and reasonably high purity was achieved.
\subsection{Bin definition for ${\mathrm{d}\sigma}/{\mathrm{d}\etjetb}$ in different regions of \qsq}
\label{subsec:bindefetinq2}
The single differential cross sections ${\mathrm{d}\sigma}/{\mathrm{d}\etjetb}$ were measured in six regions of \qsq. The size of the \qsq~bins corresponds to that of the ${\mathrm{d}\sigma}/{\mathrm{d}Q^2}$ and the \etjetb binning scheme was the same as for the integrated ${\mathrm{d}\sigma}/{\mathrm{d}\etjetb}$ cross section.



% 
% \section{Polarisation correction}
% \label{sec:polcor}
\begin{figure}[h]
 \begin{center}
 \includegraphics[width=0.7\textwidth,bb= 0 0 567 544]{./Figures/polrew/05e}
 % 05e.eps: 0x0 pixel, 300dpi, 0.00x0.00 cm, bb= 0 0 567 544
\end{center}
\caption{The polarisation reweighting factors for the 2004/2005~$e^-$ data taking period determined using the HECTOR program. The red curve represents the spline interpolation.}
\label{fig:polcor05e}
\end{figure} 
The MC samples used in this analysis were generated assuming vanishing polarisation of the lepton beam, $P_e = 0$. In order to take non-zero polarisation of the electrons into account, the MC samples were reweighted using theoretical predictions. For this purpose the HECTOR program~\cite{cpc:94:128} interfaced to BASES~\cite{upub:Nagano:url} with the CTEQ5D PDFs~\cite{pr:d51:4763} was used. The reweighting factors were determined from the ratio of predictions for the unpolarised inclusive DIS cross sections and those for the lepton beam polarisation corresponding to the particular data-taking period. The polarisation correction was implemented as a weight assigned to each MC event according to the \qsq~of the scattering process:
\begin{equation}
 w_p\left(\qsq\right) = w_p = \frac{\sigma_\mathrm{pol}}{\sigma_\mathrm{unpol}}.
\end{equation}

The average polarisation for different data-taking periods is summarised in Table~\ref{tab:polvalues}.
\begin{table}[h]
 \centering
 \begin{tabular}{lc}
 Data-taking period & Average polarisation, $P_e$ \\
\hline
 2004-2005~$e^-$   & -0.06184 \\
 2006~$e^-$   & 0.09386  \\
 2006-2007~$e^+$ & -0.06857
\end{tabular} 
\caption{The average polarisation values for the data samples used in the analysis.}
\label{tab:polvalues}
\end{table}
The obtained correction factors for the 2005 $e^-$ sample as a function of \qsq~and using a spline interpolation are illustrated in Figure~\ref{fig:polcor05e}. The size of the correction increases with increasing \qsq~but nowhere exceeds 3\% and typicaly is below 1\%. The sign of the correction depends on the helicity of the lepton beam.

Besides the polarisation correction applied to MC events, an inverse of determined factors, $w_p^{-1}\left(\qsq\right)$, were applied to the data in order to obtain jet cross sections corresponding to unpolarised lepton scattering.
% 
% \section{QED corrections}
% The theoretical predictions for the measured cross sections obtained using the NLOJET++ program include only the leading order QED contribution, while the measurements were influenced by the higher-order processes like running of the electromagnetic coupling, initial-state and final-state EM radiation etc. To maintain the consistency between the data and theoretical predictions the measured cross sections were corrected to the Born level using the MC predictions. A multiplicative factor applied to the data was determined using two \lepto MC samples with higher-order QED processes switched on and off. The correction factor is equal the the ratio of the corresponding jet cross sections:
\begin{equation}
 \mathcal{C}^\text{QED}_i = \frac{\sigma_i^\text{BORN}}{\sigma_i^\text{QED}}.
 \label{eq:eqdcorr}
\end{equation}

Figures~\ref{fig:qedcorr}~\subref{fig:z0corr_subfig1}--\subref{fig:z0corr_subfig3} illustrate the determined QED corrections in different kinematic bins. In general, the correction
 
 \begin{figure}[ht]
\begin{center}
\begin{subfloat}{\includegraphics[width=0.45\linewidth,trim={0 0 0 0},clip] {./Figures/qedcorr/h_etjetb_CS_h_h2pqed_corr_fac}
   \label{fig:qedcorr_subfig1}
 }%
\end{subfloat}
 \begin{subfloat}{\includegraphics[width=0.45\linewidth,trim={0 0 0 0},clip]{./Figures/qedcorr/h_etajetb_CS_h_h2pqed_corr_fac}
   \label{fig:qedcorr_subfig2}
 }%
\end{subfloat}
\begin{subfloat}{\includegraphics[width=0.45\linewidth,trim={0 0 0 0},clip] {./Figures/qedcorr/h_q2_CS_h_h2pqed_corr_fac.pdf}
   \label{fig:qedcorr_subfig3}
 }%
\end{subfloat}
\end{center}
\caption{QED multiplicative correction factors for inclusive-jet cross sections as functions of \etjetb, \etajetb and \qsq applied to pQCD predictions.}
\label{fig:qedcorr}
\end{figure}
% % 
% \section{Hadronisation correction}
% In order to compare directly the NLO QCD predictions with the data, the calculations has to be corrected for hadronisation effects, because the measurements refer to the jets of hadrons while the predictions to that of partons. To estimated the influence of the parton-shower modelling and hadronisation process on the jet production the prediction from \ariadne and \lepto event generators were utilised. The hadronisation correction was determined from the ratio 
\begin{equation}
 \mathcal{C}^\text{hadr}_i = \frac{\sigma_i^\text{hadr}}{\sigma_i^\text{part}}
 \label{eq:hadrcor}
\end{equation}
of the jet cross sections at the hadron, $\sigma_i^\text{hadr}$, and parton $\sigma_i^\text{part}$ levels, respectively. The parton level cross section was determined using the partons available as an input to hadronisation model after the parton shower simulation step. The hadron level refer to the 'stable'\footnote{According to the \zeus convention, all particles with the lifetime $\tau > 10$ ns.} particles available in the MC event record. An average of the correction factors determined from \ariadne and \lepto was used to correct for hadronisation effectss.

Figures~\ref{fig:hadrcorr} illustrate the hadronisation corrections as functions of \etjetb, \etajetb and \qsq. In general, the correction

\begin{figure}[htp!]
\begin{center}
\begin{subfloat}[]{\includegraphics[width=.48\textwidth,trim={5 0 50 0},clip] {./Figures/hadrcorr/h_etajetb_CS_h_h2phadr_corr_fac}
   \label{fig:hadrcor_subfig1}
 }%
\end{subfloat}
 \begin{subfloat}[]{\includegraphics[width=.48\textwidth,trim={5 0 50 0},clip]{./Figures/hadrcorr/h_etjetb_CS_h_h2phadr_corr_fac}
   \label{fig:hadrcor_subfig2}
 }%
\end{subfloat}
\newline
\begin{subfloat}[]{\includegraphics[width=.48\textwidth,trim={5 0 50 0},clip] {./Figures/hadrcorr/h_q2_CS_h_h2phadr_corr_fac}
   \label{fig:hadrcor_subfig3}
 }%
\end{subfloat}
\label{fig:hadrcor}
\end{center}
\end{figure}



% % 
% \section{Electroweak corrections}
% As mentioned earlier, the fixed-order pQCD predictions from the NLOJET++ include only the single-photon exchange component. However the contribution from $\gamma Z$-interference and $Z^0$-exchange become significant in the kinematic region of with virtualities around $M_{Z^0}^{2}$. The \lepto generator was utilised for the estimation of the size of these effects. The ratio of the jet production cross sections calculated including and excusing electroweak effects was used for correction:
\begin{equation}
 \mathcal{C}^\text{$Z^0$}_i = \frac{\sigma_i^\text{$Z^0$}}{\sigma_i^\text{no $Z^0$}}.
 \label{eq:z0corr}
\end{equation}
However, the cross-section predictions depend on the charge of the lepton beam, thus the luminosity-weighted average of the correction factors for the $e^+p$ and $e^-p$ scattering was applied to the pQCD predictions.

The correction factors are shown in Figures~\ref{fig:z0corr}~\subref{fig:z0corr_subfig1}--\subref{fig:z0corr_subfig3}. Overall ...

%\begin{figure}[htp!]
%\begin{center}
%\begin{subfloat}{\includegraphics[width=0.45\linewidth,trim={0 0 0 0},clip] {./Figures/Z0corr/h_etjetb_CS_h_h2pZ0_corr_fac_avcs}
   %\label{fig:z0corr_subfig1}
 %}%
%\end{subfloat}
 %\begin{subfloat}{\includegraphics[width=0.45\linewidth,trim={0 0 0 0},clip]{./Figures/Z0corr/h_etajetb_CS_h_h2pZ0_corr_fac_avcs}
   %\label{fig:z0corr_subfig2}
 %}%
%\end{subfloat}
%\begin{subfloat}{\includegraphics[width=0.45\linewidth,trim={0 0 0 0},clip] {./Figures/Z0corr/h_q2_CS_h_h2pZ0_corr_fac_avcs}
   %\label{fig:z0corr_subfig3}
 %}%
%\end{subfloat}
%\end{center}
%\caption{Electroweak multiplicative correction factors for inclusive-jet cross sections as functions of \etjetb, \etajetb and \qsq~applied to pQCD predictions.}
%\label{fig:z0corr}
%\end{figure}

% % % % % % % % % % % % % % % % % % % % HADRONISATION CORRECTIONS % % % % % % % % % % % % % % % % % %

\begin{figure}[ht!]
\begin{center}
\begin{subfloat}[]{\includegraphics[width=.32\textwidth ] {./Figures/Z0corr/h_etjetbinq2_CS_h_h2p0Z0_corr_fac_avcs}
   \label{fig:z0corr_subfig1}																																							
 }%
\end{subfloat}
 \begin{subfloat}[]{\includegraphics[width=.32\textwidth ]{./Figures/Z0corr/h_etjetbinq2_CS_h_h2p1Z0_corr_fac_avcs}
   \label{fig:z0corr_subfig2}
 }%
\end{subfloat}
\begin{subfloat}[]{\includegraphics[width=.32\textwidth ] {./Figures/Z0corr/h_etjetbinq2_CS_h_h2p2Z0_corr_fac_avcs}
   \label{fig:z0corr_subfig3}
 }%
\end{subfloat}
\newline
 \begin{subfloat}[]{\includegraphics[width=.32\textwidth ]{./Figures/Z0corr/h_etjetbinq2_CS_h_h2p3Z0_corr_fac_avcs}
   \label{fig:z0corr_subfig4}
 }%
\end{subfloat}
 \begin{subfloat}[]{\includegraphics[width=.32\textwidth ]{./Figures/Z0corr/h_etjetbinq2_CS_h_h2p4Z0_corr_fac_avcs}
   \label{fig:z0corr_subfig5}
 }%
\end{subfloat}
 \begin{subfloat}[]{\includegraphics[width=.32\textwidth ]{./Figures/Z0corr/h_etjetbinq2_CS_h_h2p5Z0_corr_fac_avcs}
   \label{fig:z0corr_subfig6}
 }%
\end{subfloat}
\caption{Electroweak multiplicative correction factors for inclusive-jet cross sections as function of \etjetb in bins of \qsq,~applied to NLO pQCD predictions. The correction factors were determined using dedicated \lepto MC samples.}
\label{fig:z0corr}
\end{center}
\end{figure}

% 
% \newpage
% \section{Jet corrections}
% \subsection{Hadronic energy scale}
% \subsection{Jet energy correction}
% The energy of hadronic jets reconstructed from energy deposits in the calorimeter is influenced by various effects e.g. particle absorption in uninstrumented material between the production point and the calorimeter, inhomogeneities of the detector, noise etc. Hadron jets in the \zeus detector typically loose 5-15\% of their energy in inactive media (superconducting solenoid, support structures etc.) in front of the calorimeter. This effect may lead to systematic migrations of jets to cross section bins with lower energy. In principle, such effects must be taken into account in the unfolding procedure (see Chapter~\ref{ch:unfolding}). Nevertheless, in order to minimise migrations and to avoid a possible bias from the energy loss in inactive detector media, a dedicated jet-energy correction was employed in this analysis.

Two approaches for correcting the jet energy loss exist:
\begin{itemize}
 \item the \emph{bottom-up} approach consists of correcting the energy of the input objects (i.e. calorimeter cells in this analysis) to compensate for the energy loss and then using the corrected objects as an input to the jet algorithm;
 \item in the \emph{top-down} approach the energy of identified jets is corrected directly.
\end{itemize}
In principle, with the former approach more precise correction can be achieved, because individual jet details can be take into account, while in the later, only an average correction is achieved. Nevertheless, the \emph{top-down} approach is much more simple and therefore was used in this thesis.

Monte Carlo simulations were used to estimate the energy loss because they provide the detailed information about hadron propagation in the detector volume. The measured jet energy, $E_T^{jet,det}$, depends approximately linearly on the 'true', $E_T^{jet}$, value, but the size of the energy loss depends on the thickness of the traversed material and therefore on the jet pseudorapidity in the laboratory frame. For this reason, the complete measurement region $-1<\etajetlab<2.5$ was divided into 14 equal size regions and the correction was determined separately in each $\etajetlab$ bin.

The procedure was as follows:
\begin{itemize}
 \item in the MC events accepted at the generated and reconstructed levels, a pair of jets was identified according to the distance between jets in the $\eta-\phi$ plane. A hadron-level jet was matched to the reconstructed jet if the distance 
\begin{equation}
r=\sqrt{\left[\left(\eta_{had}-\eta_{det}\right)^2 + \left(\phi_{had}-\phi_{det}\right)^2\right]}
\end{equation}
between the two was small, $r<0.7$, and no further jets were reconstructed within the cone\marginpar{OB:Would that be possible (Rcut <1).\\DL:I think, yes, because jets have irregular shape and R=1 is a maximum radius.}. In order to avoid any bias on the correction procedure introduced by the boundaries of the jet phase space, the cuts on the jet transverse energy were relaxed to $\etjetbhad>6\;\GeV$ and $\etjetbdet>3\;\GeV$ at the hadron and detector levels, respectively;
 \item in each bin $i$ of $\etajetlab$, a linear fit $\left\langle\etjetbdet\right\rangle = a_i + b_i\cdot\left\langle\etjetbhad\right\rangle$ was performed, where $\left\langle\etjetbdet\right\rangle$ and $\left\langle\etjetbhad\right\rangle$ were determined using the set of matched jets; the fits for the 2004--2005~$e^-$ running period are illustrated in Figure~\ref{fig:05e_lepto_coeffitients};
 \item the corrections were determined using the \ariadne and \lepto sample for each data-taking period separately. 
\end{itemize}
\begin{figure}[p]
\centering
\includegraphics[height=0.9\textheight]{./Figures/etcorr/new/coeff/05e_lepto_coeffitients}
\caption{The average measured detector-level jet transverse energy $\left\langle\etjetbdet\right\rangle$ as a function of $\left\langle\etjetbhad\right\rangle$ and the corresponding straight-line fits in different regions of $\etajetlab$ for the data-taking period 2004--2005~$e^-$ in the \lepto MC sample.}
\label{fig:05e_lepto_coeffitients}
\end{figure}

Given the extracted fit parameters, the components of the jet four-momentum were scaled such that the following relation was obtained:
 \begin{equation}
  E_{T,B}^{jet,det} \mapsto E_{T,B}^{jet,corr} = \frac{E_{T,B}^{jet,det} - a_i}{b_i}.
 \end{equation}
As can be seen in Figure~\ref{fig:05e_lepto_coeffitients}, the fractional energy loss, represented by the slope of the fitted straight line, varies as a function of jet pseudorapidity. The size of the correction decreases towards the forward region of the detector. Such a behaviour was attributed to the variation of the amount of material in front of the calorimeter, in particular, the presence of superconducting solenoid surrounding the tracking system. 

Since the analysis was performed in the Breit frame, the jet pseudorapidity in the laboratory frame was recalculated and the corresponding correction factors were applied. Assuming a valid description of the detector effects in the simulations, the correction was applied to both the data and MC jets. In the simulations the correction was applied on top of that introduced in Section~\ref{subsec:jetenescale}.

%thus introducing a dependence on the parton-shower simulation. \textcolor{blue}{Therefore in the unfolding procedure the respective data and MC samples were utilised.} 

% \subsection{Conclusion}
\end{document}
